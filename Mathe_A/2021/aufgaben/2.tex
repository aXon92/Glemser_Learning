\fancyhead[C]{\normalsize\textbf{$\qquad$ Teil II: Multiple-Choice-Fragen}}
\section*{Aufgabe 2 (34 Punkte)}
\vspace{0.4cm}
\subsection*{\frage{1}{3}}
$ A $ und $ B $ seien zwei Aussagen. Welche der folgenden zwei zusammengesetzten Aussagen sind äquivalent? 
\renewcommand{\labelenumi}{(\alph{enumi})}
\begin{enumerate}
	\item $ A \Rightarrow   \neg B $ und $\neg (A \wedge B)$.
	\item $ A \wedge B $ und $\neg (A \wedge B)$.
	\item $ (A \Rightarrow \neg B) $ und $A \wedge B$
	\item Keine der obigen Antworten ist richtig.
\end{enumerate}
\ \\
\textbf{Lösung:}
\begin{mdframed}
\underline{\textbf{Vorgehensweise:}}
\renewcommand{\labelenumi}{\theenumi.}
\begin{enumerate}
\item Löse die Aufgabe mithilfe einer Wahrheitstabelle.
\end{enumerate}
\end{mdframed}

\underline{1. Löse die Aufgabe mithilfe einer Wahrheitstabelle}\\
Eine Aussage kann nur die Wahrheitswerte wahr ($ W $) oder falsch ($ F $) annehmen.
Dementsprechend gibt es bei zwei Aussagen $ A $, $ B $ genau vier Kombinationsmöglichkeiten der Wahrheitswerte.
Aus diesen Kombinationen ergeben sich dann die Wahrheitswerte der verknüpften Aussagen.
\begin{center}
	\begin{tabular}{cllll}
		\hline
		\multicolumn{1}{c|}{$A$} & \multicolumn{4}{l}{$W$ $W$ $F$ $F$} \\
		\multicolumn{1}{c|}{$B$} & \multicolumn{4}{l}{$W$ $F$ $W$ $F$} \\ 
		\multicolumn{1}{c|}{$\neg A$} & \multicolumn{4}{l}{$F$ $F$ $W$ $W$} \\
		\multicolumn{1}{c|}{$\neg B$} & \multicolumn{4}{l}{$F$ $W$ $F$ $W$} \\
		\hline
		\multicolumn{1}{c|}{(a) \ \ $ A \vee B$} & \multicolumn{4}{l}{$W$ $W$ $W$ $F$} \\ 
		\multicolumn{1}{c|}{(b) \ \ $ A \wedge B$} & \multicolumn{4}{l}{$W$ $F$ $F$ $F$} \\ 
		\multicolumn{1}{c|}{(c) \ \ $ (\neg A) \vee (\neg B)$} & \multicolumn{4}{l}{$F$ $W$ $W$ $W$} \\
		\multicolumn{1}{c|}{(d) \ \ $ (\neg A) \wedge (\neg B)$} & \multicolumn{4}{l}{$F$ $F$ $F$ $W$} \\
		\hline
		\multicolumn{1}{c|}{$ A \Rightarrow B$} & \multicolumn{4}{l}{$W$ $F$ $W$ $W$} \\ 
		\multicolumn{1}{c|}{$ A \Rightarrow  \neg B$} & \multicolumn{4}{l}{$F$ $W$ $W$ $W$} \\ 
		\multicolumn{1}{c|}{$ \neg(A \Rightarrow  \neg B)$} & \multicolumn{4}{l}{$W$ $F$ $F$ $F$} \\ 
		\hline
	\end{tabular}
\end{center}
Damit ist Antwort (b) korrekt. \\
\\
Alternativ lässt sich die Antwort auch anders herleiten. Durch Umformungen erhalten wir:
\begin{align*}
	\neg (A \ \Rightarrow \ \neg B)
	&\Leftrightarrow
	\neg ((\neg A) \vee (\neg B))
	\Leftrightarrow
	\neg (\neg A) \wedge \neg(\neg B)
	\Leftrightarrow
	A \wedge B.
\end{align*}


\newpage

\subsection*{\frage{2}{4}}
Welche der folgenden zusammengesetzten Aussagen ist \textit{keine} Tautologie? 
\renewcommand{\labelenumi}{(\alph{enumi})}
\begin{enumerate}
	\item $( \neg A \vee \neg B ) \Leftrightarrow (\neg  (A \wedge B))$.
	\item $ A \Rightarrow ( B \Rightarrow A)$.
	\item $ ((A \Rightarrow B) \wedge \neg  B) \Rightarrow \neg A$
	\item $ \neg ( \neg A \wedge \neg  B ) \vee B$.
\end{enumerate}
\ \\
\textbf{Lösung:}
\begin{mdframed}
	\underline{\textbf{Vorgehensweise:}}
	\renewcommand{\labelenumi}{\theenumi.}
	\begin{enumerate}
		\item .
	\end{enumerate}
\end{mdframed}
\underline{1. }\\

 

\newpage
\subsection*{\frage{3}{4}}
Seien $\{a_k\}_{k \in \mathbb{N}}$ und $\{b_k\}_{k \in \mathbb{N}}$ zwei geometrische Folgen mit $a_1 = 1$, $a_2= 1.05$ und $b_1 = 2$, $b_2 = 1.9$. Seien $\{s_n^a \}_{n \in \mathbb{N}}$ und $\{s_n^b \}_{n \in \mathbb{N}}$ die entsprechenden Reihen.\\
\\
Welche der folgenden Aussagen ist korrekt?
\renewcommand{\labelenumi}{(\alph{enumi})}
\begin{enumerate}
	\item 
	Es gibt genau ein $n^\star \in \N$, sodass $s^b_{n^\star} = s^a_{n^\star} $.
	\item 
	Es gibt genau ein $n^\star \in \N$, sodass $s^b_n \geq s_n^a$ für alle $n \geq n^\star$ und $s_n^b < s_n^a$ für alle $n < n^\star$.
	\item 
	$s_n^b < s_n^a$ für alle $n \in \N$.
	\item
	Es gibt genau ein $n^\star \in \N$, sodass $s_n^b \geq s_n^a$ für alle $n \leq n^\star$ und $s_n^b < s_n^a$ für alle $n > n^\star$.
	\item 
	$s_n^b > s_n^a$ für alle $n \in \N$.
\end{enumerate}
\ \\
\textbf{Lösung:}
\begin{mdframed}
\underline{\textbf{Vorgehensweise:}}
\renewcommand{\labelenumi}{\theenumi.}
\begin{enumerate}
\item 
\end{enumerate}
\end{mdframed}

\underline{1. }\\




\newpage

\subsection*{\frage{4}{3}}
Bei einer stetigen Verzinsung mit jährlichem Zinssatz $i$ ist der effektive Jahreszins $i_{\mathrm{eff}}$ gegeben durch:
\renewcommand{\labelenumi}{(\alph{enumi})}
\begin{enumerate}
	\item 
	$i_{\mathrm{eff}} = e^i$.
	\item
	$i_{\mathrm{eff}} = \ln(1+ i)$.
	\item
	$i_{\mathrm{eff}} = e^i -1$.
	\item
	$i_{\mathrm{eff}} = i$.
\end{enumerate}
\ \\
\textbf{Lösung:}
\begin{mdframed}
\underline{\textbf{Vorgehensweise:}}
\renewcommand{\labelenumi}{\theenumi.}
\begin{enumerate}
\item .
\end{enumerate}
\end{mdframed}

\underline{1. }\\


\newpage
\subsection*{\frage{5}{3}}
Ein Projekt erfordert eine Anfangsinvestition von $2'000$ Schweizer Franken und generiert Erträge in Höhe von $50$ Schweizer Franken am Ende jedes Jahres für $10$ Jahre, sowie zusätzlich die Anfangsinvestition von $2'000$ Schweizer Franken am Ende des zehnten Jahres.\\
\\
Der interne Zinssatz des Projekts ist:
\renewcommand{\labelenumi}{(\alph{enumi})}
\begin{enumerate}
	\item 
	Echt grösser als $2.5 \%$.
	\item 
	Gleich $2.5 \%$.
	\item
	Echt kleiner als $2.5 \%$.
	\item
	Es hängt davon ab, welcher Zinssatz $i$ gilt, wenn der Anfangsbetrag von $2'000$ Schweizer Franken auf einem Bankkonto angelegt wird, anstatt in das Projekt investiert werden.
\end{enumerate}
\ \\
\textbf{Lösung:}
\begin{mdframed}
\underline{\textbf{Vorgehensweise:}}
\renewcommand{\labelenumi}{\theenumi.}
\begin{enumerate}
\item 
\end{enumerate}
\end{mdframed}

\underline{1. }\\



 \newpage

\subsection*{\frage{6}{3}}
Beim Ableiten besagt die Kettenregel:
\renewcommand{\labelenumi}{(\alph{enumi})}
\begin{enumerate}
	\item 
	$ (f \circ g)^\prime(x) = f^\prime(g(x)) g^\prime(x)$.
	\item 
	$ (f \circ g)^\prime(x) = f(x) g^\prime(x)$.
	\item
	$ (f \circ g)^\prime(x) = f^\prime(x) g(x)$.
	\item
	$ (f \circ g)^\prime(x) = f^\prime(g(x)) g(x)$.
	\item
	$ (f \circ g)^\prime(x) = f^\prime(x) g^\prime(x)$.
\end{enumerate}
\ \\
\textbf{Lösung:}
\begin{mdframed}
\underline{\textbf{Vorgehensweise:}}
\renewcommand{\labelenumi}{\theenumi.}
\begin{enumerate}
\item Verwende die Kettenregel.
\end{enumerate}
\end{mdframed}

\underline{1. Verwende die Kettenregel}\\

 


\newpage
\subsection*{\frage{7}{3}}
Eine invertierbare Funktion $f$ hat zwei Fixpunkte in ihrem Definitionsbereich, also $x_1,x_2 \in D_f$, sodass $f(x_1) = x_1$ und $f(x_2) = x_2$ gilt.\\
Es folgt, dass die Funktion $g$, definiert als $g(x) = f(x) - x$ für $x \in D_f$,
\renewcommand{\labelenumi}{(\alph{enumi})}
\begin{enumerate}
	\item 
	die Umkehrfunktion $g^{-1}(x) = f^{-1}(x) - \frac{1}{x} $ besitzt.
	\item
	die Umkehrfunktion $g^{-1}(x) = f^{-1}\left(\frac{1}{x}\right) $ besitzt.
	\item
	die Umkehrfunktion $g^{-1}(x) = f^{-1}(x) - x $ besitzt.
	\item
	eine Umkehrfunktion besitzt, jedoch kann ein allgemeiner Ausdruck für $g^{-1} $ Kenntnis von $f$ nicht hergeleitet werden.
	\item
	nicht invertierbar ist.
\end{enumerate}
\ \\
\textbf{Lösung:}
\begin{mdframed}
\underline{\textbf{Vorgehensweise:}}
\renewcommand{\labelenumi}{\theenumi.}
\begin{enumerate}
\item 
\end{enumerate}
\end{mdframed}

\underline{1. }\\




\newpage

\subsection*{\frage{8}{4}}
Sei $f$ eine an der Stelle $x_0 \in I $ differenzierbare Funktion einer reellen Variablen und $I$ ein im Definitionsbereich von $f$ enthaltenes Intervall. Die erste Ableitung $f^\prime(x_0)$ approximiert: 
\renewcommand{\labelenumi}{(\alph{enumi})}
\begin{enumerate}
	\item 
	die absolute Änderung $f(x_0 + \Delta x) - f(x_0)$, wenn $\Delta x$ nahe $0$ ist.
	\item
	die durchschnittliche Änderung $\frac{f(x_0 + \Delta x) - f(x_0)}{\Delta x}$, wenn $\Delta x$ nahe $0$ ist.
	\item
	die relative Änderung $\frac{f(x_0 + \Delta x) - f(x_0)}{f(x_0)}$, wenn $\Delta x$ nahe $0$ ist.
	\item
	die relative durchschnittliche Änderung $\frac{1}{x_0} \frac{f(x_0 + \Delta x) - f(x_0)}{\Delta x}$, wenn $\Delta x$ nahe $0$ ist.
\end{enumerate}
\ \\
\textbf{Lösung:}
\begin{mdframed}
\underline{\textbf{Vorgehensweise:}}
\renewcommand{\labelenumi}{\theenumi.}
\begin{enumerate}
\item 
\end{enumerate}
\end{mdframed}

\underline{1. }\\



\newpage
\subsection*{\frage{9}{3}}
Sei $f$ eine an der Stelle $x_0 \in I$ differenzierbare Funktion einer reellen Variablen und $I$ ein im Definitionsbereich von $f$ enthaltenes Intervall.
Ausserdem gelte, dass $f(x_0)  \neq 0, \ x_0 > 0$, und $\varepsilon_f(x_0) > 1$, wobei $\varepsilon_f(x_0)$ die Elastizität von $f$ an der Stelle $x_0$ bezeichnet.\\
\\
Welche der folgenden Aussagen ist korrekt?
\renewcommand{\labelenumi}{(\alph{enumi})}
\begin{enumerate}
	\item 
	In $x_0$ gilt, dass wenn das Argument um einen kleinen Betrag $\Delta x$ erhöht wird, der Anstieg in $f$ relativ betrachtet grösser ist als der Anstieg in $x$.
	\item
	In $x_0$ gilt, dass wenn das Argument um einen kleinen Betrag $\Delta x$ erhöht wird, der Anstieg in $f$ absolut betrachtet grösser ist als der Anstieg in $x$.
	\item
	In $x_0$ gilt, dass wenn das Argument um einen kleinen Betrag $\Delta x$ erhöht wird, der Anstieg in $f$ absolut betrachtet kleiner ist als der Anstieg in $x$
	\item
	Es lässt sich mit den gegebenen Informationen nicht sagen, wie sich $f$ in $x_0$ ändert, wenn das Argument um einen kleinen Betrag $\Delta x$ erhöht wird.
\end{enumerate}
\ \\
\textbf{Lösung:}
\begin{mdframed}
	\underline{\textbf{Vorgehensweise:}}
	\renewcommand{\labelenumi}{\theenumi.}
	\begin{enumerate}
		\item 
	\end{enumerate}
\end{mdframed}

\underline{1. }\\



\newpage
\subsection*{\frage{10}{4}}
Sei $f$ eine auf dem Intervall $I \subset D_f$ streng konkave Funktion und $x_0 \in I$. Sei $P_2 $ das Taylor-Polynom zweiter Ordnung von $f$ in $x_0$, dann gilt:
\renewcommand{\labelenumi}{(\alph{enumi})}
\begin{enumerate}
	\item 
	Der Graph von $P_2$ ist eine nach oben geöffnete Parabel.
	\item
	Der Graph von $P_2$ ist eine gerade Linie.
	\item
	Der Graph von $P_2$ ist eine nach unten geöffnete Parabel.
	\item
	Abhängig von $x_0$ sind alle oben genannten Fälle möglich.
\end{enumerate}
\ \\
\textbf{Lösung:}
\begin{mdframed}
	\underline{\textbf{Vorgehensweise:}}
	\renewcommand{\labelenumi}{\theenumi.}
	\begin{enumerate}
		\item 
	\end{enumerate}
\end{mdframed}

\underline{1. }\\


