\fancyhead[C]{\normalsize\textbf{$\qquad$ Teil II: Multiple-Choice-Fragen}}
\section*{Aufgabe 2 (34 Punkte)}
\vspace{0.4cm}
\subsection*{\frage{1}{3}}
$ A $ und $ B $ seien zwei Aussagen. Welche der folgenden zwei zusammengesetzten Aussagen sind äquivalent? 
\renewcommand{\labelenumi}{(\alph{enumi})}
\begin{enumerate}
	\item $ A \Rightarrow   \neg B $ und $\neg (A \wedge B)$.
	\item $ A \wedge B $ und $\neg (A \wedge B)$.
	\item $ (A \Rightarrow \neg B) $ und $A \wedge B$
	\item Keine der obigen Antworten ist richtig.
\end{enumerate}
\ \\
\textbf{Lösung:}
\begin{mdframed}
\underline{\textbf{Vorgehensweise:}}
\renewcommand{\labelenumi}{\theenumi.}
\begin{enumerate}
\item Löse die Aufgabe mithilfe einer Wahrheitstabelle.
\end{enumerate}
\end{mdframed}

\underline{1. Löse die Aufgabe mithilfe einer Wahrheitstabelle}\\
Eine Aussage kann nur die Wahrheitswerte wahr ($ W $) oder falsch ($ F $) annehmen.
Dementsprechend gibt es bei zwei Aussagen $ A $, $ B $ genau vier Kombinationsmöglichkeiten der Wahrheitswerte.
Aus diesen Kombinationen ergeben sich dann die Wahrheitswerte der verknüpften Aussagen.
\begin{center}
	\begin{tabular}{cllll}
		\hline
		\multicolumn{1}{c|}{$A$} & \multicolumn{4}{l}{$W$ $W$ $F$ $F$} \\
		\multicolumn{1}{c|}{$B$} & \multicolumn{4}{l}{$W$ $F$ $W$ $F$} \\ 
		\hline
		\multicolumn{1}{c|}{$\neg B$} & \multicolumn{4}{l}{$F$ $W$ $F$ $W$} \\
		\hline
		\multicolumn{1}{c|}{ $ A \wedge B$} & \multicolumn{4}{l}{$W$ $F$ $F$ $F$} \\
		\multicolumn{1}{c|}{$ \neg (A \wedge B)$} & \multicolumn{4}{l}{$F$ $W$ $W$ $W$} \\
		\hline
		\multicolumn{1}{c|}{$ A \Rightarrow B$} & \multicolumn{4}{l}{$W$ $F$ $W$ $W$} \\
		\multicolumn{1}{c|}{$ A \Rightarrow  \neg B$} & \multicolumn{4}{l}{$F$ $W$ $W$ $W$} \\  	
		\hline	
	\end{tabular}
\end{center}
Damit ist Antwort (a) korrekt. \\
\\
Alternativ lässt sich die Antwort auch anders herleiten.
Die Implikation ist definiert durch:
\begin{align*}
	A \Rightarrow B \ :\Leftrightarrow \ \neg A \vee B.
\end{align*}
Damit erhalten wir mit
\begin{align*}
	A \Rightarrow \neg B 
	&\Leftrightarrow
	\neg A \vee \neg B
	\Leftrightarrow
	\neg (A \wedge B).
\end{align*}
die korrekte Antwort.


\newpage

\subsection*{\frage{2}{4}}
Welche der folgenden zusammengesetzten Aussagen ist \textit{keine} Tautologie? 
\renewcommand{\labelenumi}{(\alph{enumi})}
\begin{enumerate}
	\item $( \neg A \vee \neg B ) \Leftrightarrow (\neg  (A \wedge B))$.
	\item $ A \Rightarrow ( B \Rightarrow A)$.
	\item $ ((A \Rightarrow B) \wedge \neg  B) \Rightarrow \neg A$
	\item $ \neg ( \neg A \wedge \neg  B ) \vee B$.
\end{enumerate}
\ \\
\textbf{Lösung:}
\begin{mdframed}
	\underline{\textbf{Vorgehensweise:}}
	\renewcommand{\labelenumi}{\theenumi.}
	\begin{enumerate}
		\item Verwende logische Umformungen.
	\end{enumerate}
\end{mdframed}
\underline{1. Verwende logische Umformungen}\\
Die Aussage (a) entspricht der de Morganschen Regel der Aussagenlogik und ist somit eine Tautologie (Aussage welche immer wahr ist). Für die Aussage (b) gilt:
\begin{align*}
	A \Rightarrow (B \Rightarrow A)
	\ \Leftrightarrow \
	\neg A \vee (B \Rightarrow A)
	\ \Leftrightarrow \
	\neg A \vee ( \neg B \vee A)
	\ \Leftrightarrow \ 
	\neg A \vee A \vee B.
\end{align*}
Wegen $\neg A \vee A $ ist diese Aussage immer wahr. Die Aussage (c) lässt sich durch
\begin{align*}
	((A \Rightarrow B ) \wedge \neg B) \Rightarrow \neg A
	&\ \Leftrightarrow \
	(\neg A \vee B ) \wedge \neg B ) \Rightarrow \neg A\\
	&\ \Leftrightarrow \
	(\neg A \wedge \neg B \vee B \wedge \neg B) \Rightarrow \neg A\\
	&\ \Leftrightarrow \
	(\neg A \wedge \neg B ) \Rightarrow \neg A\\
	&\ \Leftrightarrow \
	 \neg (\neg A \wedge \neg B ) \vee \neg A\\
	&\ \Leftrightarrow  \
	A \vee \neg A \vee B
\end{align*}
umformen. Auch diese ist wegen $A \vee \neg A$ immer wahr. Damit bleibt nur die Aussage (d) übrig. Für diese gilt:
\begin{align*}
	\neg ( \neg A \wedge \neg  B ) \vee B
	\ \Leftrightarrow \
	A \vee B \vee B 
	\ \Leftrightarrow \ 
	A \vee B
\end{align*}
Diese Aussage(ODER) ist offensichtlich nicht immer wahr($A =F $ und $B = F$).\\
\\
Alternativ lässt sich diese Aufgabe natürlich auch über eine Wahrheitstafel lösen:
\begin{center}
	\begin{tabular}{cllll}
		\hline
		\multicolumn{1}{c|}{$A$} & \multicolumn{4}{l}{$W$ $W$ $F$ $F$} \\
		\multicolumn{1}{c|}{$B$} & \multicolumn{4}{l}{$W$ $F$ $W$ $F$} \\ 
		\hline
		\multicolumn{1}{c|}{$ A \wedge B$} & \multicolumn{4}{l}{$W$ $F$ $F$ $F$} \\
		\multicolumn{1}{c|}{$ A \Rightarrow B$} & \multicolumn{4}{l}{$W$ $F$ $W$ $W$} \\
		\multicolumn{1}{c|}{$ B \Rightarrow A$} & \multicolumn{4}{l}{$W$ $W$ $F$ $W$} \\
		\multicolumn{1}{c|}{$ \neg A$} & \multicolumn{4}{l}{$F$ $F$ $W$ $W$} \\
		\multicolumn{1}{c|}{$\neg B$} & \multicolumn{4}{l}{$F$ $W$ $F$ $W$} \\
		\hline
		\multicolumn{1}{c|}{ $ \neg A \wedge  \neg B$} & \multicolumn{4}{l}{$F$ $F$ $F$ $W$} \\
		\multicolumn{1}{c|}{ $ \neg A \vee  \neg B$} & \multicolumn{4}{l}{$F$ $W$ $W$ $W$} \\
		\multicolumn{1}{c|}{$ \neg (A \wedge B)$} & \multicolumn{4}{l}{$F$ $W$ $W$ $W$} \\
		\multicolumn{1}{c|}{$ \neg (\neg A \wedge \neg B)$} & \multicolumn{4}{l}{$W$ $W$ $W$ $F$} \\
		\multicolumn{1}{c|}{$(A \Rightarrow B) \wedge \neg B $} & \multicolumn{4}{l}{$F$ $F$ $F$ $W$}\\
		\hline
		\multicolumn{1}{c|}{(a) \ $ ( \neg A \vee \neg B ) \Leftrightarrow (\neg  (A \wedge B))$} & \multicolumn{4}{l}{$W$ $W$ $W$ $W$} \\
		\multicolumn{1}{c|}{(b) \ $ A \Rightarrow ( B \Rightarrow A)$} & \multicolumn{4}{l}{$W$ $W$ $W$ $W$} \\  
		\multicolumn{1}{c|}{(c) \ $ ((A \Rightarrow B) \wedge \neg  B) \Rightarrow \neg A$} & \multicolumn{4}{l}{$W$ $W$ $W$ $W$} \\ 	
		\multicolumn{1}{c|}{(d) \ $ \neg ( \neg A \wedge \neg  B ) \vee B$} & \multicolumn{4}{l}{$W$ $W$ $W$ $F$} \\ 
		\hline	
	\end{tabular}
\end{center}

 

\newpage
\subsection*{\frage{3}{4}}
Seien $\{a_k\}_{k \in \mathbb{N}}$ und $\{b_k\}_{k \in \mathbb{N}}$ zwei geometrische Folgen mit $a_1 = 1$, $a_2= 1.05$ und $b_1 = 2$, $b_2 = 1.9$. Seien $\{s_n^a \}_{n \in \mathbb{N}}$ und $\{s_n^b \}_{n \in \mathbb{N}}$ die entsprechenden Reihen.\\
\\
Welche der folgenden Aussagen ist korrekt?
\renewcommand{\labelenumi}{(\alph{enumi})}
\begin{enumerate}
	\item 
	Es gibt genau ein $n^\star \in \N$, sodass $s^b_{n^\star} = s^a_{n^\star} $.
	\item 
	Es gibt genau ein $n^\star \in \N$, sodass $s^b_n \geq s_n^a$ für alle $n \geq n^\star$ und $s_n^b < s_n^a$ für alle $n < n^\star$.
	\item 
	$s_n^b < s_n^a$ für alle $n \in \N$.
	\item
	Es gibt genau ein $n^\star \in \N$, sodass $s_n^b \geq s_n^a$ für alle $n \leq n^\star$ und $s_n^b < s_n^a$ für alle $n > n^\star$.
	\item 
	$s_n^b > s_n^a$ für alle $n \in \N$.
\end{enumerate}
\ \\
\textbf{Lösung:}
\begin{mdframed}
\underline{\textbf{Vorgehensweise:}}
\renewcommand{\labelenumi}{\theenumi.}
\begin{enumerate}
\item Stelle die zugehörigen Reihen auf.
\item Schließe falsche Antworten aus.
\end{enumerate}
\end{mdframed}

\underline{1. Stelle die zugehörigen Reihen auf}\\
Für $\{a_k\}_{k \in \mathbb{N}}$ und $\{b_k\}_{k \in \mathbb{N}}$ gilt
\begin{align*}
	q^a = \frac{a_2}{a_1} &= \frac{1.05}{100} = 1.05.\\
	q^b = \frac{b_2}{b_1} &= \frac{1.9}{2} = \frac{190}{200} = \frac{95}{100} = 0.95.
\end{align*}
Die zugehörigen Reihen sind gegeben durch:
\begin{align*}
	&s_n^a = a_1 \frac{1 - 1.05^n}{1- 1.05} = \frac{1- 1.05^n}{-\frac{5}{100}}
	= 20 (1.05^n - 1)\\
	&s_n^b = b_1 \frac{1 - 0.95^n}{1- 0.95} = 2 \frac{1-0.95^n}{0.05} = 40 (1- 0.95^n).
\end{align*}
Beide Partialsummenfolgen sind streng monoton wachsend. Weiter gilt
\begin{align*}
	\lim \limits_{n \to \infty} s_n^a = \infty \ \textrm{und} \
	\lim \limits_{n \to \infty} s_n^a = 40.
\end{align*}
\ \\
\underline{2. Schließe falsche Antworten aus}\\
Die Gleichung
\begin{align*}
	s_n^a = s_n^b 
	\ \Leftrightarrow \
	20 (1.05^n - 1) = 40 (1- 0.95^n)
	\ \Leftrightarrow \ 
	1.05^n - 1 = 2 - 2 \cdot 0.95^n
	\ \Leftrightarrow \
	1.05^n + 2 \cdot  0.95^n = 3
\end{align*}
besitzt keine Lösung für $n \in \N$. Damit ist die Antwort (a) falsch.
Wegen
\begin{align*}
	s_1^b  = a_1 =  2 > 1 = b_1 = s_1^b
\end{align*}
können die Antworten (b) und (c) nicht korrekt sein. Die Antwort (d) ist korrekt, da
beide Partialsummenfolgen streng mononton steigend und die Aussagen gelten:
\begin{align*}
	&s_n^a \to \infty, \ s_n^b \to 40\\
	&s_1^b  > s_1^a. 
\end{align*}
Damit kann insbesondere Antwort (e) nicht stimmen.\\
\\
Die Antwort (d) ist korrekt.
\newpage

\subsection*{\frage{4}{3}}
Bei einer stetigen Verzinsung mit jährlichem Zinssatz $i$ ist der effektive Jahreszins $i_{\mathrm{eff}}$ gegeben durch:
\renewcommand{\labelenumi}{(\alph{enumi})}
\begin{enumerate}
	\item 
	$i_{\mathrm{eff}} = e^i$.
	\item
	$i_{\mathrm{eff}} = \ln(1+ i)$.
	\item
	$i_{\mathrm{eff}} = e^i -1$.
	\item
	$i_{\mathrm{eff}} = i$.
\end{enumerate}
\ \\
\textbf{Lösung:}
\begin{mdframed}
\underline{\textbf{Vorgehensweise:}}
\renewcommand{\labelenumi}{\theenumi.}
\begin{enumerate}
\item Überlege, was für den effektiven Jahreszins bei stetiger Verzinsung gilt.
\end{enumerate}
\end{mdframed}

\underline{1. Überlege, was für den effektiven Jahreszins bei stetiger Verzinsung gilt}\\
Der effektive Jahreszins $i_{\mathrm{eff}}$ ist der Zinssatz, sodass der Einheitsbeitrag $1$ nach $t$ Zeiteinheiten auf $(1+ i_{\mathrm{eff}})^t$ anwächst.
Nach einem Jahr muss mit stetiger Verzinsung gelten:
\begin{align*}
	1+ i_{\mathrm{eff}} = e^i
	\ \Leftrightarrow \
	 i_{\mathrm{eff}} = e^i -1.
\end{align*}
Damit ist die Antwort (c) korrekt.



\newpage
\subsection*{\frage{5}{3}}
Ein Projekt erfordert eine Anfangsinvestition von $2'000$ Schweizer Franken und generiert Erträge in Höhe von $50$ Schweizer Franken am Ende jedes Jahres für $10$ Jahre, sowie zusätzlich die Anfangsinvestition von $2'000$ Schweizer Franken am Ende des zehnten Jahres.\\
\\
Der interne Zinssatz des Projekts ist:
\renewcommand{\labelenumi}{(\alph{enumi})}
\begin{enumerate}
	\item 
	Echt grösser als $2.5 \%$.
	\item 
	Gleich $2.5 \%$.
	\item
	Echt kleiner als $2.5 \%$.
	\item
	Es hängt davon ab, welcher Zinssatz $i$ gilt, wenn der Anfangsbetrag von $2'000$ Schweizer Franken auf einem Bankkonto angelegt wird, anstatt in das Projekt investiert werden.
\end{enumerate}
\ \\
\textbf{Lösung:}
\begin{mdframed}
\underline{\textbf{Vorgehensweise:}}
\renewcommand{\labelenumi}{\theenumi.}
\begin{enumerate}
\item Berechne den internen Zinssatz.
\end{enumerate}
\end{mdframed}

\underline{1. Berechne den internen Zinssatz}\\
Der interne Zinssatz ist wie folgt definiert:
Der Nettebarwert des Projekts ist $0$ bei diesem Zinssatz.\\
\\
Formal lässt sich dies durch folgenden Zahlungsstrom zeigen:
\begin{align*}
	-&2000 + \sum \limits_{t = 1}^{10} \frac{50}{(1+i)^i} + 2000 \cdot \frac{1}{(1+i)^{10}}
	=
	-2000 + \sum \limits_{t = 1}^{10} \frac{2000 \cdot i}{(1+i)^i} + 2000 \cdot \frac{1}{(1+i)^{10}}\\
	=
	-&2000 \left(-1 + i \sum \limits_{t = 1}^{10 }\frac{1}{(1+i)^t} + \frac{1}{(1+i)^{10}} \right) 
	=
	-2000 \left(-1 + i \left(\frac{1 - \left(\frac{1}{1+i}\right)^{11} }{1 - \frac{1}{1+i}} -1 \right)+ \frac{1}{(1+i)^{10}} \right)\\
	=
	-&2000 \left(-1 + i \left(\frac{1 - \left(\frac{1}{1+i}\right)^{11} }{ \frac{i}{1+i}}  -1 \right)+ \frac{1}{(1+i)^{10}} \right)\\
	=
	-&2000 \left(-1 + i \left( (1+i)\frac{1 - \left(\frac{1}{1+i}\right)^{11} }{ i} -1 \right)+ \frac{1}{(1+i)^{10}} \right )\\
	=
	-&2000 \left(-1 +  1+i  - \left(\frac{1}{1+i}\right)^{10}  -i + \frac{1}{(1+i)^{10}} \right )
	= 0.
\end{align*}
Hierbei ergibt sich $i = 2.5 \%$.\\
\\
Damit ist die Antwort (b) korrekt.


 \newpage

\subsection*{\frage{6}{3}}
Beim Ableiten besagt die Kettenregel:
\renewcommand{\labelenumi}{(\alph{enumi})}
\begin{enumerate}
	\item 
	$ (f \circ g)^\prime(x) = f^\prime(g(x)) g^\prime(x)$.
	\item 
	$ (f \circ g)^\prime(x) = f(x) g^\prime(x)$.
	\item
	$ (f \circ g)^\prime(x) = f^\prime(x) g(x)$.
	\item
	$ (f \circ g)^\prime(x) = f^\prime(g(x)) g(x)$.
	\item
	$ (f \circ g)^\prime(x) = f^\prime(x) g^\prime(x)$.
\end{enumerate}
\ \\
\textbf{Lösung:}
\begin{mdframed}
\underline{\textbf{Vorgehensweise:}}
\renewcommand{\labelenumi}{\theenumi.}
\begin{enumerate}
\item Gebe die Kettenregel wieder.
\end{enumerate}
\end{mdframed}

\underline{1. Gebe die Kettenregel wieder}\\
Die Kettenregel der Differentialrechnung ist gegeben durch:
\begin{align*}
	(f \circ g)^\prime (x) = f^\prime(g(x))  \cdot g^\prime(x).
\end{align*}
Damit ist Antwort (a) korrekt.
 


\newpage
\subsection*{\frage{7}{3}}
Eine invertierbare Funktion $f$ hat zwei Fixpunkte in ihrem Definitionsbereich, also $x_1,x_2 \in D_f$, sodass $f(x_1) = x_1$ und $f(x_2) = x_2$ gilt.\\
Es folgt, dass die Funktion $g$, definiert als $g(x) = f(x) - x$ für $x \in D_f$,
\renewcommand{\labelenumi}{(\alph{enumi})}
\begin{enumerate}
	\item 
	die Umkehrfunktion $g^{-1}(x) = f^{-1}(x) - \frac{1}{x} $ besitzt.
	\item
	die Umkehrfunktion $g^{-1}(x) = f^{-1}\left(\frac{1}{x}\right) $ besitzt.
	\item
	die Umkehrfunktion $g^{-1}(x) = f^{-1}(x) - x $ besitzt.
	\item
	eine Umkehrfunktion besitzt, jedoch kann ein allgemeiner Ausdruck für $g^{-1} $ Kenntnis von $f$ nicht hergeleitet werden.
	\item
	nicht invertierbar ist.
\end{enumerate}
\ \\
\textbf{Lösung:}
\begin{mdframed}
\underline{\textbf{Vorgehensweise:}}
\renewcommand{\labelenumi}{\theenumi.}
\begin{enumerate}
\item Überlege, ob $g $ invertierbar ist.
\end{enumerate}
\end{mdframed}

\underline{1. Überlege, ob $g $ invertierbar ist.}\\
Die Fixpunkte der Funktion $f$ sind Nullstellen von $g$, d.h. es gilt:
\begin{align*}
	g(x_1 ) &= f(x_1) -x_1 = x_1 - x_1 = 0\\
	g(x_2 ) &= f(x_2) -x_2 = x_2 - x_2 = 0
\end{align*}
Die Funktion $g$ ist somit nicht injektiv, womit $g$ nicht invertierbar ist.


\newpage

\subsection*{\frage{8}{4}}
Sei $f$ eine an der Stelle $x_0 \in I $ differenzierbare Funktion einer reellen Variablen und $I$ ein im Definitionsbereich von $f$ enthaltenes Intervall. Die erste Ableitung $f^\prime(x_0)$ approximiert: 
\renewcommand{\labelenumi}{(\alph{enumi})}
\begin{enumerate}
	\item 
	die absolute Änderung $f(x_0 + \Delta x) - f(x_0)$, wenn $\Delta x$ nahe $0$ ist.
	\item
	die durchschnittliche Änderung $\frac{f(x_0 + \Delta x) - f(x_0)}{\Delta x}$, wenn $\Delta x$ nahe $0$ ist.
	\item
	die relative Änderung $\frac{f(x_0 + \Delta x) - f(x_0)}{f(x_0)}$, wenn $\Delta x$ nahe $0$ ist.
	\item
	die relative durchschnittliche Änderung $\frac{1}{x_0} \frac{f(x_0 + \Delta x) - f(x_0)}{\Delta x}$, wenn $\Delta x$ nahe $0$ ist.
\end{enumerate}
\ \\
\textbf{Lösung:}
\begin{mdframed}
\underline{\textbf{Vorgehensweise:}}
\renewcommand{\labelenumi}{\theenumi.}
\begin{enumerate}
\item Verwende die Definition der Ableitung.
\end{enumerate}
\end{mdframed}

\underline{1. Verwende die Definition der Ableitung}\\
Sei $f$ eine differenzierbare Funktion. Die Ableitung in $x_0 \in D_f$ ist gegeben durch
\begin{align*}
	f^\prime(x_0) = \lim \limits_{x \to x_0} \frac{f(x) - f(x_0)}{x - x_0}
	= \lim \limits_{\Delta x \to 0 } \frac{f(x_0 + \Delta x)- f(x_0)}{\Delta x}.
\end{align*} 
Hierbei wird die Substitution $\Delta x = x - x_0$ verwendet.\\
\\
Damit ist die Antwort (b) korrekt. 

\newpage
\subsection*{\frage{9}{3}}
Sei $f$ eine an der Stelle $x_0 \in I$ differenzierbare Funktion einer reellen Variablen und $I$ ein im Definitionsbereich von $f$ enthaltenes Intervall.
Ausserdem gelte, dass $f(x_0)  \neq 0, \ x_0 > 0$, und $\varepsilon_f(x_0) > 1$, wobei $\varepsilon_f(x_0)$ die Elastizität von $f$ an der Stelle $x_0$ bezeichnet.\\
\\
Welche der folgenden Aussagen ist korrekt?
\renewcommand{\labelenumi}{(\alph{enumi})}
\begin{enumerate}
	\item 
	In $x_0$ gilt, dass wenn das Argument um einen kleinen Betrag $\Delta x$ erhöht wird, der Anstieg in $f$ relativ betrachtet grösser ist als der Anstieg in $x$.
	\item
	In $x_0$ gilt, dass wenn das Argument um einen kleinen Betrag $\Delta x$ erhöht wird, der Anstieg in $f$ absolut betrachtet grösser ist als der Anstieg in $x$.
	\item
	In $x_0$ gilt, dass wenn das Argument um einen kleinen Betrag $\Delta x$ erhöht wird, der Anstieg in $f$ absolut betrachtet kleiner ist als der Anstieg in $x$
	\item
	Es lässt sich mit den gegebenen Informationen nicht sagen, wie sich $f$ in $x_0$ ändert, wenn das Argument um einen kleinen Betrag $\Delta x$ erhöht wird.
\end{enumerate}
\ \\
\textbf{Lösung:}
\begin{mdframed}
	\underline{\textbf{Vorgehensweise:}}
	\renewcommand{\labelenumi}{\theenumi.}
	\begin{enumerate}
		\item Verwende die Approximation der Elastizität für $\Delta x $ nahe $0$.
	\end{enumerate}
\end{mdframed}

\underline{1. Verwende die Approximation der Elastizität für $\Delta x $ nahe $0$}\\
Die Elastizität $\varepsilon_f(x_0)$ kann für $\Delta x$ nahe $0$ durch
\begin{align*}
	\frac{\frac{\Delta f}{f(x_0)}}{\frac{\Delta x}{x_0}}
	= 
	\frac{\Delta f}{f(x_0)} \cdot \frac{x_0}{\Delta x} \approx \varepsilon_f(x_0)
 	\ \Leftrightarrow \
 	\frac{\Delta f}{f(x_0)} \approx \varepsilon_f(x_0 )\frac{\Delta x}{x_0}
\end{align*} 
approximiert werden. Wegen $\varepsilon_f(x_0) > 1$ gilt:
\begin{align*}
	\frac{\Delta f}{f(x_0)} \approx \varepsilon_f(x_0 )\frac{\Delta x}{x_0}
	> 
	\frac{\Delta x}{x_0}.
\end{align*} 
Also ist der Anstieg in $f$ relativ gesehen größer als der Anstieg in $x$. Somit ist die Antwort (a) korrekt.

\newpage
\subsection*{\frage{10}{4}}
Sei $f$ eine auf dem Intervall $I \subset D_f$ streng konkave Funktion und $x_0 \in I$. Sei $P_2 $ das Taylor-Polynom zweiter Ordnung von $f$ in $x_0$, dann gilt:
\renewcommand{\labelenumi}{(\alph{enumi})}
\begin{enumerate}
	\item 
	Der Graph von $P_2$ ist eine nach oben geöffnete Parabel.
	\item
	Der Graph von $P_2$ ist eine gerade Linie.
	\item
	Der Graph von $P_2$ ist eine nach unten geöffnete Parabel.
	\item
	Abhängig von $x_0$ sind alle oben genannten Fälle möglich.
\end{enumerate}
\ \\
\textbf{Lösung:}
\begin{mdframed}
	\underline{\textbf{Vorgehensweise:}}
	\renewcommand{\labelenumi}{\theenumi.}
	\begin{enumerate}
		\item Verwende die zweite Ableitung.
	\end{enumerate}
\end{mdframed}

\underline{1. Verwende die zweite Ableitung}\\
Eine Funktion $f$ ist genau dann streng konkav auf $I$, wenn
\begin{align*}
f^{\prime \prime}(x) < 0
\end{align*}
für alle $x \in I$ gilt. Das Taylorpolynom zweiter Ordnung ist gegeben durch:
\begin{align*}
	P_2(x) 
	= \sum \limits_{k=0}^2\frac{f^{(k)}(x_0)}{k!} (x-x_0)^k
	=
	\frac{f^{\prime \prime }(x_0)}{2} (x-x_0)^2 + f^{\prime} (x_0) (x- x_0) + f(x_0)
\end{align*}
Da die zweite Ableitung für $x_0 \in I$ negativ ist, ist der Faktor vor dem quadratischen Term negativ. Damit ist die Parabel nach unten geöffnet.\\
\\
Also ist die Antwort (c) korrekt.