\fancyhead[C]{\normalsize\textbf{$\qquad$ Teil II: Multiple-Choice}}
\section*{Aufgabe 2 (34 Punkte)}
\vspace{0.4cm}
\subsection*{\frage{1}{3}}
$ A $ und $ B $ seien zwei Aussagen. Welche der folgenden zwei zusammengesetzten Aussagen sind äquivalent? 
\renewcommand{\labelenumi}{(\alph{enumi})}
\begin{enumerate}
	\item $ A \Rightarrow  B $ und $A \vee B$.
	\item $ A \vee B $ und $\neg (A \vee B)$.
	\item $ (A \Rightarrow \neg B) $ und $\neg ( A \vee B )$
	\item Keine der obigen Antworten ist richtig.
\end{enumerate}\ \\
\textbf{Lösung:}
\begin{mdframed}
\underline{\textbf{Vorgehensweise:}}
\renewcommand{\labelenumi}{\theenumi.}
\begin{enumerate}
\item Löse die Aufgabe mithilfe einer Wahrheitstabelle.
\end{enumerate}
\end{mdframed}

\underline{1. Löse die Aufgabe mithilfe einer Wahrheitstabelle}\\
Eine Aussage kann nur die Wahrheitswerte wahr ($ W $) oder falsch ($ F $) annehmen.
Dementsprechend gibt es bei zwei Aussagen $ A $, $ B $ genau vier Kombinationsmöglichkeiten der Wahrheitswerte.
Aus diesen Kombinationen ergeben sich dann die Wahrheitswerte der verknüpften Aussagen.
\begin{center}
	\begin{tabular}{cllll}
		\hline
		\multicolumn{1}{c|}{$A$} & \multicolumn{4}{l}{$W$ $W$ $F$ $F$} \\
		\multicolumn{1}{c|}{$B$} & \multicolumn{4}{l}{$W$ $F$ $W$ $F$} \\ 
		\multicolumn{1}{c|}{$\neg A$} & \multicolumn{4}{l}{$F$ $F$ $W$ $W$} \\
		\multicolumn{1}{c|}{$\neg B$} & \multicolumn{4}{l}{$F$ $W$ $F$ $W$} \\
		\hline
		\multicolumn{1}{c|}{(a) \ \ $ A \vee B$} & \multicolumn{4}{l}{$W$ $W$ $W$ $F$} \\ 
		\multicolumn{1}{c|}{(b) \ \ $ A \wedge B$} & \multicolumn{4}{l}{$W$ $F$ $F$ $F$} \\ 
		\multicolumn{1}{c|}{(c) \ \ $ (\neg A) \vee (\neg B)$} & \multicolumn{4}{l}{$F$ $W$ $W$ $W$} \\
		\multicolumn{1}{c|}{(d) \ \ $ (\neg A) \wedge (\neg B)$} & \multicolumn{4}{l}{$F$ $F$ $F$ $W$} \\
		\hline
		\multicolumn{1}{c|}{$ A \Rightarrow B$} & \multicolumn{4}{l}{$W$ $F$ $W$ $W$} \\ 
		\multicolumn{1}{c|}{$ A \Rightarrow  \neg B$} & \multicolumn{4}{l}{$F$ $W$ $W$ $W$} \\ 
		\multicolumn{1}{c|}{$ \neg(A \Rightarrow  \neg B)$} & \multicolumn{4}{l}{$W$ $F$ $F$ $F$} \\ 
		\hline
	\end{tabular}
\end{center}
Damit ist Antwort (b) korrekt. \\
\\
Alternativ lässt sich die Antwort auch anders herleiten. Durch Umformungen erhalten wir:
\begin{align*}
	\neg (A \ \Rightarrow \ \neg B)
	&\Leftrightarrow
	\neg ((\neg A) \vee (\neg B))
	\Leftrightarrow
	\neg (\neg A) \wedge \neg(\neg B)
	\Leftrightarrow
	A \wedge B.
\end{align*}


\newpage

\subsection*{\frage{2}{4}}
Welche der folgenden zusammengesetzten Aussagen ist eine Tautologie? 
\renewcommand{\labelenumi}{(\alph{enumi})}
\begin{enumerate}
	\item $(A \Rightarrow B ) \wedge (B \Rightarrow A)$.
	\item $ ((A \wedge B) \Rightarrow B) \vee A$.
	\item $ ((A \vee B) \Rightarrow B) \vee B$
	\item $(A \Rightarrow B ) \vee \neg A \vee B$.
\end{enumerate}
\ \\
\textbf{Lösung:}
\begin{mdframed}
	\underline{\textbf{Vorgehensweise:}}
	\renewcommand{\labelenumi}{\theenumi.}
	\begin{enumerate}
		\item .
	\end{enumerate}
\end{mdframed}
\underline{1. }\\

 

\newpage
\subsection*{\frage{3}{3}}
Seien $\{a_k\}_{k \in \mathbb{N}}$ und $\{b_k\}_{k \in \mathbb{N}}$ zwei geometrische Folgen mit $a_1 = 1$, $a_2= 1.01$ und $b_1 = 3$, $b_2 = -2$.\\
\\
Welche der folgenden Aussagen ist korrekt? 
\renewcommand{\labelenumi}{(\alph{enumi})}
\begin{enumerate}
	\item 
	$\{a_k\}_{k \in \mathbb{N}}$ und $\{b_k\}_{k \in \mathbb{N}}$ konvergieren beide.
	\item 
	Nur $\{a_k\}_{k \in \mathbb{N}}$ konvergiert.
	\item 
	Nur $\{b_k\}_{k \in \mathbb{N}}$ konvergiert.
	\item
	Weder $\{a_k\}_{k \in \mathbb{N}}$ noch $\{b_k\}_{k \in \mathbb{N}}$ konvergieren.
\end{enumerate}
\ \\
\textbf{Lösung:}
\begin{mdframed}
\underline{\textbf{Vorgehensweise:}}
\renewcommand{\labelenumi}{\theenumi.}
\begin{enumerate}
\item 
\end{enumerate}
\end{mdframed}

\underline{1. }\\




\newpage

\subsection*{\frage{4}{4}}
Seien $\{a_k\}_{k \in \mathbb{N}}$ und $\{b_k\}_{k \in \mathbb{N}}$ zwei geometrische Folgen mit $a_1 = 1$, $a_2= 1.1$ und $b_1 = 3$, $b_2 = 3.1$.
Seien $\{s_n^a \}_{n \in \mathbb{N}}$ und $\{s_n^b \}_{n \in \mathbb{N}}$ die entsprechenden Reihen.\\
\\
Welche der folgenden Aussagen ist korrekt? 
\renewcommand{\labelenumi}{(\alph{enumi})}
\begin{enumerate}
	\item 
	$\lim_{n \to \infty} \frac{s_n^a}{s_n^b} = 3$.
	\item
	$\lim_{n \to \infty} \frac{s_n^a}{s_n^b} = \frac{1}{3}$.
	\item
	$\lim_{n \to \infty} \frac{s_n^a}{s_n^b} = \infty$.
	\item
	$\lim_{n \to \infty} \frac{s_n^a}{s_n^b} = 0$.
\end{enumerate}
\ \\
\textbf{Lösung:}
\begin{mdframed}
\underline{\textbf{Vorgehensweise:}}
\renewcommand{\labelenumi}{\theenumi.}
\begin{enumerate}
\item .
\end{enumerate}
\end{mdframed}

\underline{1. }\\


\newpage
\subsection*{\frage{5}{3}}
Ein Projekt erfordert eine Anfangsinvestition von $10'000$ Schweizer Franken und generiert Erträge in Höhe von $500$ Schweizer Franken am Ende jedes Jahres für $10$ Jahre, sowie zusätzliche $12'000$ Schweizer Franken am Ende des zehnten Jahres.\\
\\
Der interne Zinssatz des Projekts ist:
\renewcommand{\labelenumi}{(\alph{enumi})}
\begin{enumerate}
	\item 
	Echt grösser als $5 \%$.
	\item 
	Gleich $5 \%$.
	\item
	Echt kleiner als $5 \%$.
	\item
	Die Frage kann nicht beantwortet werden, ohne zu wissen, welcher Zinssatz $i$ gilt.
\end{enumerate}
\ \\
\textbf{Lösung:}
\begin{mdframed}
\underline{\textbf{Vorgehensweise:}}
\renewcommand{\labelenumi}{\theenumi.}
\begin{enumerate}
\item 
\end{enumerate}
\end{mdframed}

\underline{1. }\\



 \newpage

\subsection*{\frage{6}{3}}
Seien $h$, $f$ und $g$ differenzierbare Funktionen einer reellen Variable. Welche der folgenden Formeln ist korrekt?
\renewcommand{\labelenumi}{(\alph{enumi})}
\begin{enumerate}
	\item 
	$ \frac{d}{dx}h(f(x) + g(x)) = h^\prime(x) ( f^\prime(x) + g^\prime(x))$.
	\item 
	$ \frac{d}{dx}h(f(x) + g(x)) = h^\prime(x) ( f(x) + g(x))^\prime$.
	\item
	$ \frac{d}{dx}h(f(x) + g(x)) = h^\prime(f(x))  f^\prime(x) + h^\prime(g(x)) g^\prime(x)$.
	\item
	$ \frac{d}{dx}h(f(x) + g(x)) = h^\prime(f(x) + g(x))  (f^\prime(x) +  g^\prime(x) )$.
	\item
	$ \frac{d}{dx}h(f(x) + g(x)) = h^\prime(f(x) + g(x))  f^\prime(x)  g^\prime(x)$.
\end{enumerate}
\ \\
\textbf{Lösung:}
\begin{mdframed}
\underline{\textbf{Vorgehensweise:}}
\renewcommand{\labelenumi}{\theenumi.}
\begin{enumerate}
\item Verwende die Kettenregel.
\end{enumerate}
\end{mdframed}

\underline{1. Verwende die Kettenregel}\\

 


\newpage
\subsection*{\frage{7}{4}}
Welche der folgenden Aussagen ist korrekt für eine differenzierbare Funktion $f$ auf einem Intervall $[a,b]$ und $x_0 \in (a,b)$?
\renewcommand{\labelenumi}{(\alph{enumi})}
\begin{enumerate}
	\item 
	$x_0$ ist genau dann ein stationärer Punkt, wenn er ein Extrempunkt ist.
	\item
	Wenn $x_0$ ein Wendepunkt ist, dann ist er auch ein stationärer Punkt.
	\item
	Wenn $x_0$ der einzige lokale Extrempunkt in $(a,b)$ ist, dann ist er auch ein globaler Extrempunkt in $[a,b]$.
	\item
	Wenn $f^\prime(x_0) = 0$, dann ist $x_0$ ein Extrempunkt.
	\item
	Keine der obigen Antworten ist richtig.
\end{enumerate}
\ \\
\textbf{Lösung:}
\begin{mdframed}
\underline{\textbf{Vorgehensweise:}}
\renewcommand{\labelenumi}{\theenumi.}
\begin{enumerate}
\item 
\end{enumerate}
\end{mdframed}

\underline{1. }\\




\newpage

\subsection*{\frage{8}{3}}
Seien $f$ und $g$ konkave, zweimal differenzierbare Funktionen. Welche der folgenden Aussagen ist korrekt?
\renewcommand{\labelenumi}{(\alph{enumi})}
\begin{enumerate}
	\item 
	$f \ g$ ist konkav.
	\item
	$f \ g$ ist konvex.
	\item
	$f \circ g $ ist konkav.
	\item
	$f +g $ ist konvex.
	\item 
	$f+g$ ist konkav.
\end{enumerate}
\ \\
\textbf{Lösung:}
\begin{mdframed}
\underline{\textbf{Vorgehensweise:}}
\renewcommand{\labelenumi}{\theenumi.}
\begin{enumerate}
\item 
\end{enumerate}
\end{mdframed}

\underline{1. }\\



\newpage
\subsection*{\frage{9}{3}}
Sei $f$ eine differenzierbare Funktion. Welcher der folgenden Ausdrücke approximiert 
$x_0 \frac{f(x_0 + \Delta x) - f(x_0)}{f(x_0) \Delta x}$, wenn $\Delta x$ klein ist?
\renewcommand{\labelenumi}{(\alph{enumi})}
\begin{enumerate}
	\item 
	Die Ableitung $f^\prime(x_0)$.
	\item
	Die Änderungsrate $\rho_f(x_0)$.	
	\item
	Das Differential $df$.
	\item
	Die Elastizität $\varepsilon_f(x_0)$.
\end{enumerate}
\ \\
\textbf{Lösung:}
\begin{mdframed}
	\underline{\textbf{Vorgehensweise:}}
	\renewcommand{\labelenumi}{\theenumi.}
	\begin{enumerate}
		\item 
	\end{enumerate}
\end{mdframed}

\underline{1. }\\



\newpage
\subsection*{\frage{10}{4}}
Sei $f$ eine Funktion und $x_0 \in D_f$ so, dass $f(x_0) = 1$ und $f^{(k)}(x_0) = 1$ für $k = 1,2,3$ gilt. Seien $P_2$ und $P_3$ jeweils das Taylor-Polynom zweiter und dritter Ordnung von $f$ in $x_0$. Es folgt, dass:
\renewcommand{\labelenumi}{(\alph{enumi})}
\begin{enumerate}
	\item 
	$P_3(x) - P_2(x) = 0 $ für alle $x$.
	\item
	$P_3(x) - P_2(x) = (x - x_0)^2 $ für alle $x$.
	\item
	$P_3(x) - P_2(x) = \frac{1}{6} (x - x_0)^2 $ für alle $x$.
	\item
	Es ist nicht möglich die Frage zu beantworten, ohne die Funktionsvorschrift von $f$ zu kennen.
\end{enumerate}
\ \\
\textbf{Lösung:}
\begin{mdframed}
	\underline{\textbf{Vorgehensweise:}}
	\renewcommand{\labelenumi}{\theenumi.}
	\begin{enumerate}
		\item 
	\end{enumerate}
\end{mdframed}

\underline{1. }\\


