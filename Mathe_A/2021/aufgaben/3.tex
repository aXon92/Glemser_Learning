\section*{Aufgabe 3 (28 Punkte)}
\vspace{0.4cm}
\subsection*{\frage{1}{3}}
Welche der folgenden Folgen konvergiert \textit{nicht} gegen $0$?
\renewcommand{\labelenumi}{(\alph{enumi})}
\begin{enumerate}
	\item 
	$ \{a_n\}_{n \in \mathbb{N}} $ mit $a_n = \frac{\sin(n) - \cos(n)}{n}$.
	\item
	$ \{b_n\}_{n \in \mathbb{N}} $ mit $b_n = \sqrt{n^2 + 1 }  - n$.
	\item
	$ \{c_n\}_{n \in \mathbb{N}} $ mit $c_n = \frac{n}{n^2 - 3n +1 }$.
	\item
	$ \{d_n\}_{n \in \mathbb{N}} $ mit $d_n = \frac{n^2}{e^n}$.
	\item
	$ \{e_n\}_{n \in \mathbb{N}} $ mit $e_n = \frac{\ln(n)}{n}$.
	\item
	$ \{f_n\}_{n \in \mathbb{N}} $ mit $f_n = \frac{(n+2) (n^2 + 3n -1 )}{5n^3 + 4}$.
\end{enumerate}
\ \\
\textbf{Lösung:}
\begin{mdframed}
\underline{\textbf{Vorgehensweise:}}
\renewcommand{\labelenumi}{\theenumi.}
\begin{enumerate}
\item Löse die Aufgabe durch geschicktes Ausschließen.
\item Löse die Aufgabe durch Bestimmen der Grenzwerte.
\end{enumerate}
\end{mdframed}

\underline{1. Löse die Aufgabe durch geschicktes Ausschließen} \\
Wir suchen nach der Folge, welche \textbf{nicht} gegen $0$ konvergiert.
Bei Folgen, welche als Quotienten vorliegen, lässt sich Konvergenz gegen $0$ (meist) leicht bestimmen. Wenn der Nenner schneller als der Zähler wächst, ist dies ein Indiz für eine Nullfolge.
In (a) ist der Zähler als Summe beschränkter Funktionen beschränkt. Damit gilt
\begin{align*}
	|a_n | = \frac{|\sin(n) - \cos(n)| }{n} 
	\leq C \frac{1}{n} \to 0
\end{align*}
für $n \to \infty$. Für (c) - (d) ist schnell ersichtlich, dass der Nenner schneller als der Zähler wächst. Kniffliger ist die (b), jedoch nur auf den ersten Blick:
\begin{align*}
	b_n = \sqrt{n^2 +1} - n
	= \frac{n^2 +1  - n^2}{\sqrt{n^2 +1 } + n}
	=
	\frac{1}{\sqrt{n^2 +1 } + n}.
\end{align*}
Hierbei haben wir die dritte Binomische Formel angewendet. 
Übrig bleibt noch die (f). Bei Folgen dieser Art ist es nur notwendig die führenden Potenzen im Zähler und Nenner zu vergleichen:
\begin{align*}
	f_n = \frac{(n+2) (n^2 + 3n -1 )}{5n^3 + 4} 
	= 
	\frac{n^3 \left(1+ \frac{3}{n} + \cdots  \right)}{5n^3 \left(1 + \frac{4}{5 n^3} \right)}.
\end{align*}
Dies führt dazu, dass $f_n $ gegen $\frac{1}{5}$ konvergiert. Damit haben wir die Folge gefunden, welche nicht gegen $0$ konvergiert.\\
\\
\underline{2. Löse die Aufgabe durch Bestimmen der Grenzwerte} \\
Für die Folge in (a) gilt:
\begin{align*}
	|a_n| =
	\left|
	\frac{\sin(n) - \cos(n)}{n}
	\right|
	\leq
	\left|
	\frac{\sin(n)}{n}
	\right|
	+
	\left|
	\frac{\cos(n)}{n}
	\right|
	\leq 
	\frac{1}{n} + \frac{1}{n} \to 0 \ \textrm{für } n \to \infty.
\end{align*}
Hierbei haben wir die Dreiecksungleichung angewendet. Mit der dritten Binomischen Formel gilt für die (b)
\begin{align*}
	b_n = \sqrt{n^2 +1} - n
	= \frac{\sqrt{n^2 +1}^2  - n^2}{\sqrt{n^2 +1 } + n}
	= \frac{n^2 +1  - n^2}{\sqrt{n^2 +1 } + n}
	=
	\frac{1}{\sqrt{n^2 +1 } + n} \to 0 \textit{ für } n \to \infty.
\end{align*}
Für (c) erhalten wir:
\begin{align*}
	  c_n = \frac{n}{n^2 - 3n +1 }
	  = \frac{n}{n^2 \cdot \left( 1+ \frac{3}{n} + \frac{1}{n^2}\right)}
	  =
	  \frac{\frac{1}{n}}{   1 + \frac{3}{n} + \frac{1}{n^2}}
	  \to \frac{0}{1} \ \textrm{für } n \to \infty.
\end{align*}
Bei (d) und (e) gilt:
\begin{align*}
	&d_n = \frac{n^2}{ e^n} \to 0  \ \textrm{für } n \to \infty\\
	&e_n = \frac{\ln(n)}{n} \to 0 \ \textrm{für } n \to \infty.
\end{align*}
Damit bleibt (f) übrig, wofür gilt:
\begin{align*}
	f_n = \frac{(n+2) (n^2 + 3n -1 )}{5n^3 + 4}
	&=
	\frac{n^3 + 5n^2 + 5n - 2}{5n^3 + 4}
	=
	\frac{n^3 \left(1 + \frac{5}{n} + \frac{5}{n^2}  - \frac{2}{n^3}\right)}{n^3 \left(n^3 ( 5 + \frac{4}{n^3}) \right)}\\
	&=
	\frac{1 + \frac{5}{n} + \frac{5}{n^2}  - \frac{2}{n^3}}{(n^3 ( 5 + \frac{4}{n^3})}
	\to 0 \ \textrm{für } n \to \infty.
\end{align*}
Somit ist Antwort (f) korrekt.
\newpage

\subsection*{\frage{2}{4}}
Ein neues FinTech Unternehmen schätzt, dass der durchschnittliche investierte Betrag ihrer Kunden $20'000$ Schweizer Franken pro Kunde beträgt.
Die Anzahl ihrer Kunden steigt monatlich um $10 \% $. Zu Beginn hat das Unternehmen $1'000$ Kunden und es muss ein Vermögen im Wert von $500$ Millionen Schweizer Franken verwalten, um alle Kosten decken zu können und Gewinn zu machen.
Unter den gegebenen Informationen wird das Unternehmen Gewinne machen beginnend nach dem
\renewcommand{\labelenumi}{(\alph{enumi})}
\begin{enumerate}
	\item 
	$33$. Monat.
	\item
	$17$. Monat.
	\item
	$24$. Monat.
	\item
	$13$. Monat.
	\item
	$17$. Monat.
	\item
	$42$. Monat.
	\item
	$37$. Monat.
\end{enumerate}
\ \\
\textbf{Lösung:}
\begin{mdframed}
\underline{\textbf{Vorgehensweise:}}
\renewcommand{\labelenumi}{\theenumi.}
\begin{enumerate}
\item Verwende eine geometrische Folge zur Lösung des Problems.
\end{enumerate}
\end{mdframed}

\underline{1. Verwende eine geometrische Folge zur Lösung des Problems}\\
Die Kundenanzahl im Verlauf der Monate entspricht der geometrischen Folge $\{ c_n \}_{n \in \N}$ mit 
\begin{align*}
	c_{n+1} = 1.1 c_n
\end{align*} 
um das monatliche Wachstum von $10 \%$ abzubilden. Zu Beginn hat das Unternehmen $1'000$ Kunden. Demnach ist $c_0 = 1'000$. Damit hat das Unternehmen nach $n$ Monaten
\begin{align*}
	c_n =  1'000 \cdot (1.1)^n
\end{align*}
Kunden. Mit der Annahme, dass jeder Kunde im Schnitt $20'000 $ Schweizer Franken investiert hat, erhalten wir
\begin{align*}
	20'000 c_n \geq 500 \cdot 10^6,
\end{align*}
damit das Unternehmen Gewinne macht. Durch Auflösen der Gleichung erhalten wir:
\begin{align*}
	20'000 c_n \geq 500 \cdot 10^6
	\ \Leftrightarrow \
	c_n = 1'000 \cdot (1.1)^n \geq 25'000
	\ \Leftrightarrow \
	(1.1)^n \geq 25
	\ \Leftrightarrow \
	n \geq \frac{\ln(25)}{\ln(1.1)}
	\approx
	33.77.
\end{align*}
Somit macht das Unternehmen ab dem $33$. Monat Gewinne und die Antwort (a) ist korrekt.


\newpage
\subsection*{\frage{3}{4}}
Gegeben sei die Funktion
\begin{align*}
	f(x) = x^{\ln(x^2)}.
\end{align*}
Die Ableitung von $f$ in $x_0 = e $ ist:
\renewcommand{\labelenumi}{(\alph{enumi})}
\begin{enumerate}
	\item 
	$ \frac{4}{e^2}$.
	\item
	$ 4 e^2 $.
	\item
	$ 4 e $.
	\item
	$ \frac{4}{e} $.
	\item
	$f$ ist nicht differenzierbar in $x_0 = e$.
\end{enumerate}
\ \\
\textbf{Lösung:}
\begin{mdframed}
\underline{\textbf{Vorgehensweise:}}
\renewcommand{\labelenumi}{\theenumi.}
\begin{enumerate}
\item Berechne die Ableitung von $f$.
\end{enumerate}
\end{mdframed}
%\allowdisplaybreaks
\underline{1. Berechne die Ableitung von $f$}\\
Wir verwenden die Exponentialfunktion um $f$ umzuschreiben:
\begin{align*}
	f(x) = x^{\ln(x^2)} = 
	e^{\ln \left(x^{\ln(x^2)} \right)  }
	=
	e^{\ln(x^2) \ln(x)}
	= 
	e^{2 \ln(x)^2}.
\end{align*}
Mithilfe zweifacher Anwendung der Kettenregel erhalten wir dann:
\begin{align*}
	f^\prime(x) =
	e^{2 \ln(x)^2} \cdot \left(2 \cdot 2 \ln(x) \frac{1}{x}\right)
	=4 e^{2 \ln(x)^2} \ln(x) \frac{1}{x}.
\end{align*}
Damit folgt:
\begin{align*}
	f^\prime(x_0)
	=
	4 e^{2 \ln(e)^2} \ln(e) \frac{1}{e}
	=
	4 e^{2} \frac{1}{e}
	= 4 e.
\end{align*}
Also ist die Antwort (c) korrekt.

\newpage
\subsection*{\frage{4}{4}}
Die Funktion $f$ sei definiert als $f(x) = e^{3x} (x^3 +2x +1)$. 
Das Taylor-Polynom dritter Ordnung von $f$ an der Entwicklungsstelle $x_0 = 0$ ist:
\renewcommand{\labelenumi}{(\alph{enumi})}
\begin{enumerate}
	\item 
	$P_3(x) = 1 + 5x + 12.5 x^2 + 15.5 x^3 $.
	\item 
	$P_3(x) = 1 + 5x + 10.5 x^2 + 14.5 x^3 $.
	\item
	$P_3(x) = 2 + 5x + 10.5 x^2 + 14.5 x^3 $.
	\item
	$P_3(x) = 1 + 5x + 14.5 x^2 + 10.5 x^3 $.
	\item
	$P_3(x) = 1 + 6x + 10.5 x^2 + 14.5 x^3 $.
\end{enumerate}
\ \\
\textbf{Lösung:}
\begin{mdframed}
\underline{\textbf{Vorgehensweise:}}
\renewcommand{\labelenumi}{\theenumi.}
\begin{enumerate}
\item Gebe das Taylorpolynom dritter Ordnung allgemein an.
\item Bestimme die entsprechenden Ableitungen und werte diese aus.
\end{enumerate}
\end{mdframed}

\underline{1. Gebe das Taylorpolynom dritter Ordnung allgemein an}\\
Das Taylorpolynom dritter Ordnung an der Entwicklungstelle $x_0 = 0$ ist gegeben durch:
\begin{align*}
	P_3(x)
	=
	f(x_0)
	+
	f^{\prime}(x_0)x
	+
	\frac{f^{\prime \prime}(x_0)}{2} x^2
	+
	\frac{f^{\prime \prime \prime(x_0) }}{6} x^3.	
\end{align*}
\ \\
\underline{2.Bestimme die entsprechenden Ableitungen und werte diese aus}\\
Die ersten drei Ableitungen sind mithilfe der Produktregel gegeben durch:
\begin{align*}
	f^\prime(x) 
	&= 
	3e^{x} (x^3 + 2x +1 ) + e^{3x}( 3x^2 + 2 )
	= 
	e^3x (3x^3 +3x^2 + 6x  + 5)\\
	f^{\prime \prime }(x)
	&= 
	3 e^{3x} (3x^3 + 3x^2 + 6x +5 )
	+
	e^{3x}(9x^2 + 6x + 6)
	=
	e^{3x}(9x^3 + 18 x^2 + 24 x +21)\\
	f^{\prime \prime \prime} (x)
	&=
	3 e^{3x} ( 9x^3 +18 x^2 +24 x +21)
	+
	e^{3x}(27 x^2 + 36 x +24)
	=
	e^{3x} (27 x^3 + 81 x^2 + 108 x+ 87).
\end{align*}
Damit folgt:
\begin{align*}
	f(0) = 1, \quad
	f^\prime(0) = 5, \quad
	f^{\prime \prime}(0) = 21, \quad
	f^{\prime \prime \prime} ( 0) = 87.
\end{align*}
Insgesamt erhalten wir:
\begin{align*}
	P_3(x)
	=
	1 + 5 x + 
	\frac{21}{2} x^2
	+
	\frac{87}{6} x^3
	=
	1 + 5x +10.5 x^2 + 14.5 x^3.
\end{align*}
Somit ist die Antwort (b) korrekt.

\newpage

\subsection*{\frage{5}{3}}
Sei $f$ die Funktion zweier reeller Variablen definiert durch $f(x,y) = \frac{1}{2} \sin(x^2 +y^2).$\\
\\
Die partiellen Ableitungen von $f$ in $(x_0, y_0) = \left(\sqrt{\frac{\pi}{2}}, 0 \right)$ sind:
\renewcommand{\labelenumi}{(\alph{enumi})}
\begin{enumerate}
	\item 
	$f_x(x_0,y_0) = \frac{\pi}{2}$ und $f_y(x_0,y_0) = 0$.
	\item
	$f_x(x_0,y_0) =  \sqrt{\frac{\pi}{2}} $ und $f_y(x_0,y_0) = 0$.
	\item
	$f_x(x_0,y_0) = \sqrt{\frac{\pi}{2}} \sin(\pi^2) $ und $f_y(x_0,y_0) = \sin(\sqrt{\pi})$.
	\item
	$f_x(x_0,y_0) = 0 $ und $f_y(x_0,y_0) = 0$.
\end{enumerate}
\ \\
\textbf{Lösung:}
\begin{mdframed}
\underline{\textbf{Vorgehensweise:}}
\renewcommand{\labelenumi}{\theenumi.}
\begin{enumerate}
\item 
\end{enumerate}
\end{mdframed}

\underline{1. }\\



\newpage

\subsection*{\frage{6}{4}}
Sei 
\begin{align*}
	U(c_1,c_2) = c_1^{0.25} c_2^{0.75}
\end{align*}
eine Nutzenfunktion, wobei $c_1 > 0$ und $c_2 > 0$ die jeweiligen Einheiten von Gut $1$ und Gut $2$ darstellen.
Das Konsumbündel $(c_1^\star, c_2^\star)$ hat einen Nutzen von $1$ und die Grenzrate der Substitution in diesem Punkt entspricht
$\frac{d c_2}{d c_1} = - \frac{1}{3}$.
Es gilt:
\renewcommand{\labelenumi}{(\alph{enumi})}
\begin{enumerate}
	\item 
	$ (c_1^\star,c_2^\star) = (8,0.5) $.
	\item
	$ (c_1^\star,c_2^\star) = (8,1) $.
	\item
	$ (c_1^\star,c_2^\star) = (64,0.25) $.
	\item
	$ (c_1^\star,c_2^\star) = (1,2) $.
	\item
	$ (c_1^\star,c_2^\star) = (1,1) $.
	\item
	$ (c_1^\star,c_2^\star) = (2,2) $.
\end{enumerate}
\ \\
\textbf{Lösung:}
\begin{mdframed}
\underline{\textbf{Vorgehensweise:}}
\renewcommand{\labelenumi}{\theenumi.}
\begin{enumerate}
\item 
\end{enumerate}
\end{mdframed}

\underline{1. }\\





\newpage

\subsection*{\frage{7}{2}}
Gegeben ist die Funktion
\begin{align*}
	f(x,y) 
	=
	x^2 e^{\frac{ 2x+y}{x - 3y }}
	-
	xy e^{\frac{x+y}{x-y}}
	+
	x^2 \ln \left( \frac{x }{2y} \right)
	\quad \textrm{für } x>0,y>0.
\end{align*}
Welche der folgenden Aussagen ist korrekt?
\renewcommand{\labelenumi}{(\alph{enumi})}
\begin{enumerate}
	\item
	$ f  $ ist homogen vom Grad $ -1 $.
	\item
	$ f  $ ist homogen vom Grad $ 2 $.
	\item
	$f $ ist nicht homogen.
	\item 
	$ f  $ ist homogen vom Grad $ 1$.
	\item
	$ f  $ ist homogen vom Grad $ -2 $.
\end{enumerate}\ \\
\textbf{Lösung:}
\begin{mdframed}
\underline{\textbf{Vorgehensweise:}}
\renewcommand{\labelenumi}{\theenumi.}
\begin{enumerate}
\item Überlege, ob $ f $ homogen sein kann.
\end{enumerate}
\end{mdframed}

\underline{1. Überlege, ob $ f $ homogen sein kann}\\
Wir betrachten die einzelnen Summanden von $ f $ und untersuchen diese auf Homogenität:
\begin{align*}
	f_1(x,y)
	&=
	x^2 e^{\frac{x+y}{x}}\\
	f_2(x,y)
	&=
	-
	xy e^{\frac{x+y}{x-y}}\\
	f_3(x,y)
	&=
	x \ln \left( \frac{x}{y} \right).
\end{align*} 
Wir untersuchen nun jeweils die Homogenität.
Sei $ \lambda \in \R $ beliebig. Dann gilt:
\begin{align*}
	f_1(\lambda x, \lambda y)
	&=
	(\lambda x)^2 e^{\frac{\lambda x+ \lambda y} { \lambda x}}
	=
	\lambda^2 x^2 e^{\frac{\lambda( x+  y) } { \lambda x}}
	=
	\lambda^2 x^2 e^{\frac{x+y}{x}} 
	= \lambda^2 f_1(x,y)
	\\
	f_2(\lambda x, \lambda y)
	&=
	-
	\lambda x \lambda y e^{\frac{\lambda x+ \lambda y}{ \lambda x- \lambda y}}
	=
	-
	\lambda^2 xy e^{\frac{\lambda (x+  y)}{ \lambda (x-  y)}}
	=
	-
	\lambda^2 xy e^{\frac{x+  y}{  x-  y}}
	=
	\lambda^2 f_2(x,y)\\
	f_3(\lambda x, \lambda y)
	&=
	\lambda x \ln \left( \frac{\lambda x}{ \lambda y} \right)
	=
	\lambda x \ln \left( \frac{ x}{ y} \right)
	=
	\lambda f_3(x,y).
\end{align*}
Damit sind die ersten beiden Summanden homogen vom Grad $ 2 $.
Der dritte Summand ist homogen vom Grad $ 1 $. Aufgrund des Distributivgesetz müssten alle Summanden von $ f $ dieselbe Homogenität besitzen, damit $ f $ homogen ist. Also ist $ f $ nicht homogen.\\
\\
Damit ist Antwort (e) korrekt.


\newpage

\subsection*{\frage{8}{4}}
Gegeben ist die Funktion
\begin{align*}
	f(x,y)
	=
	\frac{x^{3a} y^b}{x^3 +y^3}
	- 
	\frac{1}{x^3 y^{3b} + x y^{2 + 3b}},
\end{align*}
wobei $ x > 0, y > 0 $ und $ a,b \in \R $.\\
\\
Für welche Werte von $ a $ und $ b $ gilt
\begin{align*}
	\varepsilon_x(x,y) + \varepsilon_y(x,y) = 1 \ \textrm{für alle } x>0, y>0\textrm{?}
\end{align*}
\renewcommand{\labelenumi}{(\alph{enumi})}
\begin{enumerate}
	\item 
	$a = \frac{16}{9}$ und $ b=-\frac{4}{3} $.
	\item
	$a = \frac{16}{9}$ und $ b=-\frac{2}{3} $.
	\item
	$a = \frac{11}{9}$ und $ b=-\frac{1}{3} $.
	\item
	$a = \frac{15}{9}$ und $ b=-\frac{2}{3} $.
	\item
	$a = \frac{19}{9}$ und $ b=-\frac{4}{3} $.
	\item
	Es gibt keine Werte $ a $ und $ b $, die die Bedingung erfüllen.
\end{enumerate}
\ \\
\textbf{Lösung:}
\begin{mdframed}
\underline{\textbf{Vorgehensweise:}}
\renewcommand{\labelenumi}{\theenumi.}
\begin{enumerate}
\item Verwende die Eulersche Relation.
\end{enumerate}
\end{mdframed}

\underline{1. Verwende die Eulersche Relation}\\
Falls $ f $ homogen vom Grad $ \kappa $ ist, besagt die Eulersche Relation
\begin{align*}
	\varepsilon_{f,x}(x,y) + \varepsilon_{f,y}(x,y) 
	= x \frac{f_x(x,y)}{f(x,y)} + y \frac{f_y(x,y)}{f(x,y)}
	= \kappa.
\end{align*}
Dies ist äquivalent zu:
\begin{align*}
	x f_x(x,y) + f_y(x,y) = \kappa \cdot f(x,y).
\end{align*}
Wie müssen nun überprüfen, für welche $ a, b\in \R $ eine Homogenität von $ 0 $ vorliegt:
\begin{align*}
	f(\lambda x, \lambda x)
	&=
	\frac{(\lambda x)^{2a} (\lambda y)^b}{(\lambda x)^3 +(\lambda y)^3}
	- 
	\frac{1}{(\lambda x)^3 (\lambda y)^{3b} + (\lambda x) (\lambda y)^{2 + 3b}}
	=
	\frac{\lambda^{2a +b} ( x^{2a}  y^b)}{\lambda^3 (x^3 + y^3 )}
	- 
	\frac{1}{\lambda^{3b + 3} x^3  y^{3b} + \lambda^{3b + 3} x  y^{2 + 3b}}\\
	&=
	\frac{\lambda^{2a +b}}{\lambda^3}\frac{ ( x^{2a}  y^b)}{ (x^3 + y^3 )}
	- 
	\frac{1}{\lambda^{3b + 3}}\frac{1}{ x^3  y^{3b} +  x  y^{2 + 3b}}
	=
		\lambda^{2a +b - 3}\frac{ ( x^{2a}  y^b)}{ (x^3 + y^3 )}
	- 
	\lambda^{-(3b + 3)}\frac{1}{ x^3  y^{3b} +  x  y^{2 + 3b}}.
\end{align*}
Damit $ f $ homogen vom Grad $ 0 $ ist, muss 
\begin{align*}
	2a + b - 3 &= 0\\
	-3b - 3 &= 0
\end{align*}
gelten. Für die zweite Gleichung gilt:
\begin{align*}
	-3b - 3 = 0 
	\ \Leftrightarrow \
	-3b = 3
	\ \Leftrightarrow \
	b = -1.
\end{align*}
Wenn wir dies in die erste Gleichung einsetzen, erhalten wir:
\begin{align*}
	2a - 1 - 3 = 0 
	\ \Leftrightarrow \
	2a = 4
	\ \Leftrightarrow \
	a = 2.
\end{align*}
Somit ist $ f $ für $ a = 2 $ und $ b= -1 $ homogen vom Grad $ 0 $ und die erwünschte Gleichung ist erfüllt.\\
\\
Also ist Antwort (a) korrekt.


