\section*{Aufgabe 3 (28 Punkte)}
\vspace{0.4cm}
\subsection*{\frage{1}{3}}
Welche der folgenden Folgen konvergiert gegen $1$?
\renewcommand{\labelenumi}{(\alph{enumi})}
\begin{enumerate}
	\item 
	$ \{a_n\}_{n \in \mathbb{N}} $ mit $a_n = \frac{\sin(n) - \cos(n)}{n}$.
	\item
	$ \{b_n\}_{n \in \mathbb{N}} $ mit $b_n = \frac{n^2}{2n^2 - 3n +1 }$.
	\item
	$ \{c_n\}_{n \in \mathbb{N}} $ mit $c_n = \sqrt{n^2 + 1 }  - n$.
	\item
	$ \{d_n\}_{n \in \mathbb{N}} $ mit $d_n = \frac{n^2}{e^n}$.
	\item
	$ \{e_n\}_{n \in \mathbb{N}} $ mit $e_n = \frac{(n+2)(n^2 + 3n -1 )}{n^3 + 4 }$.
	\item
	$ \{f_n\}_{n \in \mathbb{N}} $ mit $f_n = \frac{\ln(n)}{n}$.
\end{enumerate}
\ \\
\textbf{Lösung:}
\begin{mdframed}
\underline{\textbf{Vorgehensweise:}}
\renewcommand{\labelenumi}{\theenumi.}
\begin{enumerate}
\item .
\end{enumerate}
\end{mdframed}

\underline{1. Löse die Summe auf}\\


\newpage

\subsection*{\frage{2}{4}}
Welche der folgenden Konditionen ist am besten?
\renewcommand{\labelenumi}{(\alph{enumi})}
\begin{enumerate}
	\item 
	Jährlicher Zinssatz von $5 \%$ mit jährlicher Verzinsung.
	\item
	Jährlicher Zinssatz von $4.95 \%$ mit halbjährlicher Verzinsung.
	\item
	Jährlicher Zinssatz von $4.9 \%$ mit vierteljährlicher Verzinsung.
	\item
	Jährlicher Zinssatz von $4.85 \%$ mit monatlicher Verzinsung.
	\item
	Jährlicher Zinssatz von $4.8 \%$ mit stetiger Verzinsung.	
\end{enumerate}
\ \\
\textbf{Lösung:}
\begin{mdframed}
\underline{\textbf{Vorgehensweise:}}
\renewcommand{\labelenumi}{\theenumi.}
\begin{enumerate}
\item 
\end{enumerate}
\end{mdframed}

\underline{1. }\\



\newpage
\subsection*{\frage{3}{4}}
Gegeben ist die Funktion
\begin{align*}
	f(x) = \sin(\ln(x^2 + x + 1)).
\end{align*}
Die Ableitung von $f$ in $x_0 = 0 $ ist:
\renewcommand{\labelenumi}{(\alph{enumi})}
\begin{enumerate}
	\item 
	$ 0$.
	\item
	$ 1 $.
	\item
	$ 2 $.
	\item
	$ \pi $.
	\item
	$f$ ist nicht differenzierbar in $x_0 = 0$.
\end{enumerate}
\ \\
\textbf{Lösung:}
\begin{mdframed}
\underline{\textbf{Vorgehensweise:}}
\renewcommand{\labelenumi}{\theenumi.}
\begin{enumerate}
\item 
\end{enumerate}
\end{mdframed}
%\allowdisplaybreaks
\underline{1. }\\

\newpage
\subsection*{\frage{4}{4}}
Sei $P_3$ das Taylor-Polynom dritter Ordnung von $f(x) = \ln(1+ x) $ in $x_0 = 0$.\\
\\
Es gilt, dass:
\renewcommand{\labelenumi}{(\alph{enumi})}
\begin{enumerate}
	\item 
	$P_3(1) = 1 $.
	\item 
	$P_3(1) = \frac{1 }{2}$.
	\item
	$P_3(1) =  0$.
	\item
	$P_3(1) = \frac{1 }{6}$.
	\item
	$P_3(1) = \frac{5 }{6}$.
\end{enumerate}
\ \\
\textbf{Lösung:}
\begin{mdframed}
\underline{\textbf{Vorgehensweise:}}
\renewcommand{\labelenumi}{\theenumi.}
\begin{enumerate}
\item 

\end{enumerate}
\end{mdframed}

\underline{1. }\\

\newpage

\subsection*{\frage{5}{3}}
Sei $f$ die Funktion zweier reeller Variablen definiert durch $f(x,y) = \cos(x^2 +y^2 + 2xy).$\\
\\
Die partiellen Ableitungen von $f$ in $(x_0, y_0) = \left(\sqrt{\frac{\pi}{2}}, \sqrt{\frac{\pi}{2}}\right)$ sind:
\renewcommand{\labelenumi}{(\alph{enumi})}
\begin{enumerate}
	\item 
	$f_x(x_0,y_0) = 0$ und $f_y(x_0,y_0) = 0$.
	\item
	$f_x(x_0,y_0) = 4 \sqrt{\frac{\pi}{2}} $ und $f_y(x_0,y_0) = 0$.
	\item
	$f_x(x_0,y_0) = 0 $ und $f_y(x_0,y_0) = 4 \sqrt{\frac{\pi}{2}}$.
	\item
	$f_x(x_0,y_0) = 4 \sqrt{\frac{\pi}{2}} $ und $f_y(x_0,y_0) = 4 \sqrt{\frac{\pi}{2}}$.
\end{enumerate}
\ \\
\textbf{Lösung:}
\begin{mdframed}
\underline{\textbf{Vorgehensweise:}}
\renewcommand{\labelenumi}{\theenumi.}
\begin{enumerate}
\item 
\end{enumerate}
\end{mdframed}

\underline{1. }\\



\newpage

\subsection*{\frage{6}{4}}
Wir betrachten die Gleichung
\begin{align*}
	\ln(x) + x y^2 + \frac{2x}{x + 2y} - 2 = 0,
\end{align*}
die für $(x_0,y_0) = (1,0)$ erfüllt ist.\\
\\
Die marginale Änderung in $y$, wenn $x$ ausgehend von $x_0$ marginal geändert wird und die Gleichung weiterhin erfüllt ist, ist:
\renewcommand{\labelenumi}{(\alph{enumi})}
\begin{enumerate}
	\item 
	$ 0 $.
	\item
	$ \frac{1}{4}$.
	\item
	$ -\frac{1}{4}$.
	\item
	$ 1$.
	\item
	Keine der obigen Antworten ist korrekt.
\end{enumerate}
\ \\
\textbf{Lösung:}
\begin{mdframed}
\underline{\textbf{Vorgehensweise:}}
\renewcommand{\labelenumi}{\theenumi.}
\begin{enumerate}
\item 
\end{enumerate}
\end{mdframed}

\underline{1. }\\





\newpage

\subsection*{\frage{7}{2}}
Gegeben ist die Funktion
\begin{align*}
	f(x,y) 
	=
	2 x^3 e^{\frac{ 2x^2+y^2}{x^2 - 3y^2 }}
	-
	xy^2 e^{\frac{x+y}{x-y}}
	+
	x y^3 \ln \left( \frac{x+y }{2y} \right)
	\quad \textrm{für } x>0,y>0.
\end{align*}
Welche der folgenden Aussagen ist korrekt?
\renewcommand{\labelenumi}{(\alph{enumi})}
\begin{enumerate}
	\item
	$ f  $ ist homogen vom Grad $ 0 $.
	\item
	$ f  $ ist homogen vom Grad $ 2 $.
	\item
	$ f  $ ist homogen vom Grad $ 2 $..	
	\item 
	$ f  $ ist homogen vom Grad $ 4 $.
	\item
	$ f $ ist nicht homogen.
\end{enumerate}
\ \\
\textbf{Lösung:}
\begin{mdframed}
\underline{\textbf{Vorgehensweise:}}
\renewcommand{\labelenumi}{\theenumi.}
\begin{enumerate}
\item Überlege, ob $ f $ homogen sein kann.
\end{enumerate}
\end{mdframed}

\underline{1. Überlege, ob $ f $ homogen sein kann}\\
Wir betrachten die einzelnen Summanden von $ f $ und untersuchen diese auf Homogenität:
\begin{align*}
	f_1(x,y)
	&=
	x^2 e^{\frac{x+y}{x}}\\
	f_2(x,y)
	&=
	-
	xy e^{\frac{x+y}{x-y}}\\
	f_3(x,y)
	&=
	x \ln \left( \frac{x}{y} \right).
\end{align*} 
Wir untersuchen nun jeweils die Homogenität.
Sei $ \lambda \in \R $ beliebig. Dann gilt:
\begin{align*}
	f_1(\lambda x, \lambda y)
	&=
	(\lambda x)^2 e^{\frac{\lambda x+ \lambda y} { \lambda x}}
	=
	\lambda^2 x^2 e^{\frac{\lambda( x+  y) } { \lambda x}}
	=
	\lambda^2 x^2 e^{\frac{x+y}{x}} 
	= \lambda^2 f_1(x,y)
	\\
	f_2(\lambda x, \lambda y)
	&=
	-
	\lambda x \lambda y e^{\frac{\lambda x+ \lambda y}{ \lambda x- \lambda y}}
	=
	-
	\lambda^2 xy e^{\frac{\lambda (x+  y)}{ \lambda (x-  y)}}
	=
	-
	\lambda^2 xy e^{\frac{x+  y}{  x-  y}}
	=
	\lambda^2 f_2(x,y)\\
	f_3(\lambda x, \lambda y)
	&=
	\lambda x \ln \left( \frac{\lambda x}{ \lambda y} \right)
	=
	\lambda x \ln \left( \frac{ x}{ y} \right)
	=
	\lambda f_3(x,y).
\end{align*}
Damit sind die ersten beiden Summanden homogen vom Grad $ 2 $.
Der dritte Summand ist homogen vom Grad $ 1 $. Aufgrund des Distributivgesetz müssten alle Summanden von $ f $ dieselbe Homogenität besitzen, damit $ f $ homogen ist. Also ist $ f $ nicht homogen.\\
\\
Damit ist Antwort (e) korrekt.


\newpage

\subsection*{\frage{8}{4}}
Gegeben ist die Funktion
\begin{align*}
	f(x,y)
	=
	\frac{x^{3a} y^b}{x^3 +y^3} e^{\frac{x^a y}{y^{a+1}}}
	- 
	\frac{1}{x^3 y^{3b} + x y^{2 + 3b}},
\end{align*}
wobei $ x > 0, y > 0 $ und $ a,b \in \R $.\\
\\
Für welche Werte von $ a $ und $ b $ gilt
\begin{align*}
	\varepsilon_x(x,y) + \varepsilon_y(x,y) = 2 \ \textrm{für alle } x>0, y>0\textrm{?}
\end{align*}
\renewcommand{\labelenumi}{(\alph{enumi})}
\begin{enumerate}
	\item 
	$a = \frac{10}{9}$ und $ b=-\frac{5}{3} $.
	\item
	$a = \frac{17}{9}$ und $ b=-\frac{2}{3} $.
	\item
	$a = \frac{20}{9}$ und $ b=-\frac{5}{3} $.
	\item
	$a = \frac{20}{9}$ und $ b=-\frac{2}{3} $.
	\item
	$a = \frac{19}{9}$ und $ b=-\frac{5}{3} $.
	\item
	Es gibt keine Kombination $ a $ und $ b $, die die Bedingung erfüllt.
\end{enumerate}
\ \\
\textbf{Lösung:}
\begin{mdframed}
\underline{\textbf{Vorgehensweise:}}
\renewcommand{\labelenumi}{\theenumi.}
\begin{enumerate}
\item Verwende die Eulersche Relation.
\end{enumerate}
\end{mdframed}

\underline{1. Verwende die Eulersche Relation}\\
Falls $ f $ homogen vom Grad $ \kappa $ ist, besagt die Eulersche Relation
\begin{align*}
	\varepsilon_{f,x}(x,y) + \varepsilon_{f,y}(x,y) 
	= x \frac{f_x(x,y)}{f(x,y)} + y \frac{f_y(x,y)}{f(x,y)}
	= \kappa.
\end{align*}
Dies ist äquivalent zu:
\begin{align*}
	x f_x(x,y) + f_y(x,y) = \kappa \cdot f(x,y).
\end{align*}
Wie müssen nun überprüfen, für welche $ a, b\in \R $ eine Homogenität von $ 0 $ vorliegt:
\begin{align*}
	f(\lambda x, \lambda x)
	&=
	\frac{(\lambda x)^{2a} (\lambda y)^b}{(\lambda x)^3 +(\lambda y)^3}
	- 
	\frac{1}{(\lambda x)^3 (\lambda y)^{3b} + (\lambda x) (\lambda y)^{2 + 3b}}
	=
	\frac{\lambda^{2a +b} ( x^{2a}  y^b)}{\lambda^3 (x^3 + y^3 )}
	- 
	\frac{1}{\lambda^{3b + 3} x^3  y^{3b} + \lambda^{3b + 3} x  y^{2 + 3b}}\\
	&=
	\frac{\lambda^{2a +b}}{\lambda^3}\frac{ ( x^{2a}  y^b)}{ (x^3 + y^3 )}
	- 
	\frac{1}{\lambda^{3b + 3}}\frac{1}{ x^3  y^{3b} +  x  y^{2 + 3b}}
	=
		\lambda^{2a +b - 3}\frac{ ( x^{2a}  y^b)}{ (x^3 + y^3 )}
	- 
	\lambda^{-(3b + 3)}\frac{1}{ x^3  y^{3b} +  x  y^{2 + 3b}}.
\end{align*}
Damit $ f $ homogen vom Grad $ 0 $ ist, muss 
\begin{align*}
	2a + b - 3 &= 0\\
	-3b - 3 &= 0
\end{align*}
gelten. Für die zweite Gleichung gilt:
\begin{align*}
	-3b - 3 = 0 
	\ \Leftrightarrow \
	-3b = 3
	\ \Leftrightarrow \
	b = -1.
\end{align*}
Wenn wir dies in die erste Gleichung einsetzen, erhalten wir:
\begin{align*}
	2a - 1 - 3 = 0 
	\ \Leftrightarrow \
	2a = 4
	\ \Leftrightarrow \
	a = 2.
\end{align*}
Somit ist $ f $ für $ a = 2 $ und $ b= -1 $ homogen vom Grad $ 0 $ und die erwünschte Gleichung ist erfüllt.\\
\\
Also ist Antwort (a) korrekt.


