\fancyhead[C]{\normalsize\textbf{$\qquad$ Teil II: Multiple-Choice}}
\section*{Aufgabe 2 (34 Punkte)}
\vspace{0.4cm}
\subsection*{\frage{1}{3}}
$ A $ und $ B $ seien zwei Aussagen. Die zusammengesetzte Aussage
\begin{align*}
	\neg (A \ \Rightarrow \ \neg B)
\end{align*}
ist äquivalent zu 
\renewcommand{\labelenumi}{(\alph{enumi})}
\begin{enumerate}
	\item $ A \vee B $.
	\item $ A \wedge B $.
	\item $ (\neg A) \vee (\neg B)$
	\item $ (\neg A) \wedge (\neg B)$.
\end{enumerate}\ \\
\textbf{Lösung:}
\begin{mdframed}
\underline{\textbf{Vorgehensweise:}}
\renewcommand{\labelenumi}{\theenumi.}
\begin{enumerate}
\item Löse die Aufgabe mithilfe einer Wahrheitstabelle.
\end{enumerate}
\end{mdframed}

\underline{1. Löse die Aufgabe mithilfe einer Wahrheitstabelle}\\
Eine Aussage kann nur die Wahrheitswerte wahr ($ W $) oder falsch ($ F $) annehmen.
Dementsprechend gibt es bei zwei Aussagen $ A $, $ B $ genau vier Kombinationsmöglichkeiten der Wahrheitswerte.
Aus diesen Kombinationen ergeben sich dann die Wahrheitswerte der verknüpften Aussagen.
\begin{center}
	\begin{tabular}{cllll}
		\hline
		\multicolumn{1}{c|}{$A$} & \multicolumn{4}{l}{$W$ $W$ $F$ $F$} \\
		\multicolumn{1}{c|}{$B$} & \multicolumn{4}{l}{$W$ $F$ $W$ $F$} \\ 
		\multicolumn{1}{c|}{$\neg A$} & \multicolumn{4}{l}{$F$ $F$ $W$ $W$} \\
		\multicolumn{1}{c|}{$\neg B$} & \multicolumn{4}{l}{$F$ $W$ $F$ $W$} \\
		\hline
		\multicolumn{1}{c|}{(a) \ \ $ A \vee B$} & \multicolumn{4}{l}{$W$ $W$ $W$ $F$} \\ 
		\multicolumn{1}{c|}{(b) \ \ $ A \wedge B$} & \multicolumn{4}{l}{$W$ $F$ $F$ $F$} \\ 
		\multicolumn{1}{c|}{(c) \ \ $ (\neg A) \vee (\neg B)$} & \multicolumn{4}{l}{$F$ $W$ $W$ $W$} \\
		\multicolumn{1}{c|}{(d) \ \ $ (\neg A) \wedge (\neg B)$} & \multicolumn{4}{l}{$F$ $F$ $F$ $W$} \\
		\hline
		\multicolumn{1}{c|}{$ A \Rightarrow B$} & \multicolumn{4}{l}{$W$ $F$ $W$ $W$} \\ 
		\multicolumn{1}{c|}{$ A \Rightarrow  \neg B$} & \multicolumn{4}{l}{$F$ $W$ $W$ $W$} \\ 
		\multicolumn{1}{c|}{$ \neg(A \Rightarrow  \neg B)$} & \multicolumn{4}{l}{$W$ $F$ $F$ $F$} \\ 
		\hline
	\end{tabular}
\end{center}
Damit ist die Antwort (d) korrekt. \\
\\
Alternativ lässt sich die Antwort auch anders herleiten. Durch Umformungen erhalten wir:
\begin{align*}
	\neg (A \ \Rightarrow \ \neg B)
	&\Leftrightarrow
	\neg ((\neg A) \vee (\neg B))
	\Leftrightarrow
	\neg (\neg A) \wedge \neg(\neg B)
	\Leftrightarrow
	A \wedge B.
\end{align*}


\newpage

\subsection*{\frage{2}{2}}
Die Folge $ \lbrace a_n \rbrace_{n \in \N} $ ist monoton wachsend und konvergiert mit $ \lim_{n \to \infty} a_n = -2 $.
Die Folge $ \lbrace b_n \rbrace_{n \in \N} $ ist definiert durch $ b_n = (-3) \cdot a_n $.\\
\\
Dann folgt:
\renewcommand{\labelenumi}{(\alph{enumi})}
\begin{enumerate}
	\item $ \lbrace b_n \rbrace_{n \in \N} $ ist monoton wachsend und divergent.
	\item $ \lbrace b_n \rbrace_{n \in \N} $ ist monoton wachsend und konvergent.
	\item $ \lbrace b_n \rbrace_{n \in \N} $ ist monoton fallend und divergent.
	\item $ \lbrace b_n \rbrace_{n \in \N} $ ist monoton fallend und konvergent.
	\item $ \lbrace b_n \rbrace_{n \in \N} $ ist nicht monoton und divergent.
\end{enumerate}
\ \\
\textbf{Lösung:}
\begin{mdframed}
	\underline{\textbf{Vorgehensweise:}}
	\renewcommand{\labelenumi}{\theenumi.}
	\begin{enumerate}
		\item Leite die korrekte Antwort her.
	\end{enumerate}
\end{mdframed}
\underline{1. Leite die korrekte Antwort her}\\
Die konstante Folge $ (-3) $ ist konvergent. Demnach ist $ \lbrace b_n \rbrace_{n \in \N} $ als Produkt konvergenter Folgen konvergent.
Deswegen sind die Anworten (a),(c) und (d) falsch.
Die Folge $\lbrace a_n \rbrace_{n \in \N}$ ist monoton wachsend.
Damit folgt:
\begin{align*}
	a_n \leq a_{n+1}
	\ \Leftrightarrow \
	(-3) \cdot a_n = b_n \geq (-3 ) a_{n+1} = b_{n+1}.
\end{align*} 
Hierbei geht ein, dass sich die Richtung der Ungleichung bei negativem Vorzeichen ändert.\\
\\
Also ist $ \lbrace b_n \rbrace_{n \in \N} $ monoton fallend und somit Antwort (d) korrekt.
 

\newpage
\subsection*{\frage{3}{3}}
Eine Möbelfirma wirbt:
\begin{center}
	\glqq Wir schenken Ihnen die Mehrwertsteuer auf Ihren Möbelkauf\grqq
\end{center}
Die Mehrwertsteuer auf Möbel beträgt in der Schweiz $ 7.7 \% $.\\
Wie viel Prozent Rabatt gewährt also die Möbelfirma?
\renewcommand{\labelenumi}{(\alph{enumi})}
\begin{enumerate}
	\item 
	Es sind etwa $ 7.15 \% $.
	\item 
	Es sind etwa $ 7.56 \% $.
	\item 
	Es sind genau $ 7.7 \% $.
	\item
	Es sind etwa $ 7.78 \% $.
	\item
	Es sind etwa $ 8.34 \% $.
\end{enumerate}
\ \\
\textbf{Lösung:}
\begin{mdframed}
\underline{\textbf{Vorgehensweise:}}
\renewcommand{\labelenumi}{\theenumi.}
\begin{enumerate}
\item Bestimme den gewährten Rabatt.
\end{enumerate}
\end{mdframed}

\underline{1. Bestimme den gewährten Rabatt}\\
Angenommen wir kaufen ein Möbelstück im Wert von $ 100 $ (CHF).
Dann ist der Preis inklusive Mehrwertsteuer
\begin{align*}
	100 \cdot 1.077 = 107.7 \ \textrm{(CHF)}.
\end{align*} 
Den Rabatt erhalten wir durch:
\begin{align*}
	p \% = \frac{7.7}{107.7} 
	\approx 
	0.0715.
\end{align*}
Damit gewährt das Möbelhaus etwa einen Rabatt von $ 7.15 \% $.\\
\\
Also ist die Antwort (a) korrekt.



\newpage

\subsection*{\frage{4}{3}}
Sei $ f $ eine Funktion einer reellen Variablen und $ x_0 \in D_f $.
$ x_0  $ heißt \textit{Fixpunkt von} $ f $, wenn gilt: $ f(x_0) = x_0 $.\\
\\
Welche der folgenden Aussagen ist wahr? 
\renewcommand{\labelenumi}{(\alph{enumi})}
\begin{enumerate}
	\item 
	Wenn $ f $ invertierbar ist, dann hat $ f $ keinen Fixpunkt.
	\item
	Wenn $ f $ invertierbar ist und $ f^{-1} $ mehrere Fixpunkte hat, dann hat $ f $ höchstens einen Fixpunkt.
	\item
	Wenn $ f $ invertierbar ist, dann hat $ f $ mindestens einen Fixpunkt.
	\item
	Wenn $ f  $ einen Fixpunkt hat, dann ist $ f $ invertierbar.
	\item
	Wenn $ f $ invertierbar ist und einen Fixpunkt hat, dann hat auch $ f^{-1} $ einen Fixpunkt.
\end{enumerate}
\ \\
\textbf{Lösung:}
\begin{mdframed}
\underline{\textbf{Vorgehensweise:}}
\renewcommand{\labelenumi}{\theenumi.}
\begin{enumerate}
\item Schließe falsche Antworten aus.
\end{enumerate}
\end{mdframed}

\underline{1. Schließe falsche Antworten aus}\\
Wir setzen für alle Beispiele $ D_f = \R $.
Wir betrachten die Funktion $ f(x) = x $. Diese Funktion hat unendlich viele Fixpunkte und ist invertierbar mit $ f^{-1}(x) = x =f(x) $. 
Deswegen sind die Antworten (a) und (b) falsch.\\
\\
Nun setzen $ f(x) = x + 1 $. Diese Funktion ist invertierbar und besitzt keinen Fixpunkt.
Deshalb ist die Antwort (c) falsch.\\
\\
Die Funktion $ f(x) = |x| $ ist nicht invertierbar wegen $ f(-1) = f(1) $ und alle $ x \geq 0  $ sind Fixpunkte.
Damit ist die Antwort (d) falsch.\\
\\
Übrig bleibt die Antwort (e). Wenn $ f  $ invertierbar und einen Fixpunkt hat, gilt: 
\begin{align*}
	f(x_0) = x_0 
	\ \Leftrightarrow \
	x_0 = f^{-1}(x_0).
\end{align*}
Hiermit können wir die korrekte Antwort (e) auch direkt bestimmen.

\newpage
\subsection*{\frage{5}{3}}
Seien $ a,b >0, \ b \neq 1 , \ n \in \N  $.\\
Welche der folgenden Identitäten ist allgemein gültig?
\renewcommand{\labelenumi}{(\alph{enumi})}
\begin{enumerate}
	\item 
	$ \log_{b^n}(a^n) = \frac{\log_{b}(a)}{n} $.
	\item 
	$ \log_{b^n}(a^n) = \sqrt[n]{\log_{b}(a)} $.
	\item
	$ \log_{b^n}(a^n) = \log_{b}(a) $.
	\item
	$ \log_{b^n}(a^n) = n \ \log_{b}(a) $.
	\item
	Keine der vorangehenden Aussagen ist im Allgemeinen gültig.
\end{enumerate}
\ \\
\textbf{Lösung:}
\begin{mdframed}
\underline{\textbf{Vorgehensweise:}}
\renewcommand{\labelenumi}{\theenumi.}
\begin{enumerate}
\item 
\end{enumerate}
\end{mdframed}

\underline{1. }\\


 \newpage

\subsection*{\frage{6}{3}}
Ein Bergsteiger startet bei seinem Auto bei Sonnenaufgang um $ 5 $ Uhr mit dem Aufstieg und erreicht auf direktem Weg ohne Pause die Hütte um $ 13 $ Uhr.
Am anderen Tag geht er den genau gleichen Weg zurück; er startet um $ 8 $ Uhr und geht ohne anzuhalten. So erreicht er sein Auto um $ 12:50 $ Uhr.\\
\\
Gibt es einen Tageszeitpunkt, zu dem er auf dem Weg aufwärts bzw. auf dem Weg abwärts an der gleichen Stelle ist? 
\renewcommand{\labelenumi}{(\alph{enumi})}
\begin{enumerate}
	\item 
	Ja, es gibt genau einen solchen Zeitpunkt.
	\item 
	Nein.
	\item
	Es kann sein, muss aber nicht sein.
	\item
	Es kann auch mehrere Zeitpunkte geben.
\end{enumerate}
\ \\
\textbf{Lösung:}
\begin{mdframed}
\underline{\textbf{Vorgehensweise:}}
\renewcommand{\labelenumi}{\theenumi.}
\begin{enumerate}
\item

\end{enumerate}
\end{mdframed}

\underline{1. }\\

\newpage
\subsection*{\frage{7}{3}}
Gegeben ist die Funktion $ f $ definiert durch
\begin{align*}
	f(x) = \sin(x) + \cos(x)
\end{align*}
mit Definitionsgebiet $ D_f \in \R $. Für den Wertebereich $ W_f $ von $ f $ gilt:
\renewcommand{\labelenumi}{(\alph{enumi})}
\begin{enumerate}
	\item 
	$ W_f = [-1,1] $.
	\item
	$ W_f = [-\sqrt{2},\sqrt{2}] $.
	\item
	$ W_f = [-2,2] $.
	\item
	$ W_f = [0,2] $.
	\item
	$ W_f = [-\nicefrac{\pi }{2}, \nicefrac{\pi }{2}] $.
\end{enumerate}
\ \\
\textbf{Lösung:}
\begin{mdframed}
\underline{\textbf{Vorgehensweise:}}
\renewcommand{\labelenumi}{\theenumi.}
\begin{enumerate}
\item 
\end{enumerate}
\end{mdframed}

\underline{1. }\\

\newpage

\subsection*{\frage{8}{3}}
$ f $ und $ g $ seien ungerade, differenzierbare Funktionen. Dann gilt:
\renewcommand{\labelenumi}{(\alph{enumi})}
\begin{enumerate}
	\item 
	$ f^\prime + g^\prime $ ist ungerade.
	\item
	$ f^\prime + g^\prime $ ist gerade.
	\item
	$ f^\prime \cdot g^\prime $ ist ungerade.
	\item
	Keine der obigen Antworten ist im Allgemeinen richtig.
\end{enumerate}
\ \\
\textbf{Lösung:}
\begin{mdframed}
\underline{\textbf{Vorgehensweise:}}
\renewcommand{\labelenumi}{\theenumi.}
\begin{enumerate}
\item Leite die Antwort mithilfe der Kettenregel her.
\end{enumerate}
\end{mdframed}

\underline{1. Leite die Antwort mithilfe der Kettenregel her}\\
Sei $ h  $ eine ungerade, differenzierbare Funktion.
Dann gilt
\begin{align*}
	h(x) = -h(-x)
	\ \Rightarrow \
	h^\prime (x) = (-1) \cdot (-1) \cdot h^\prime(-x) = h^\prime(-x)
\end{align*}
für alle $ x \in D_h $. Das bedeutet, dass die Ableitung einer ungeraden Funktion gerade ist.\\
\\
\textit{Frage: Was gilt dann für die Ableitung einer geraden Funktion?}\\
\\
Wegen 
\begin{align*}
	(f+g)(x) = f(x) +g(x) =-f(-x) - g(-x)
	=-(f(-x) + g(-x))= -(f+g)(-x)
\end{align*}
ist auch die Summe zweier ungerader Funktion ungerade. Aufgrund von 
\begin{align*}
	(f+g)^\prime = f^\prime + g^\prime
\end{align*}
ist $ f^\prime + g^\prime $ gerade (Weil $ f+g $ ungerade ist).\\
\\
Damit ist Antwort (b) korrekt.


\newpage
\subsection*{\frage{9}{2}}
Wir betrachten die differenzierbaren Funktionen $ f(x) $ und $ g(x) $, wobei $ g(x) > 0 $ für $ x \in \R $ gelte. Die Ableitung der Funktion
\begin{align*}
	k(x)
	=
	\frac{f(g(x))}{g(f(x))}
\end{align*}
ist dann
\renewcommand{\labelenumi}{(\alph{enumi})}
\begin{enumerate}
	\item 
	$ k^\prime(x) = \frac{f^\prime(g(x))g(f(x)) + f(g(x))g^\prime(f(x))}{(g(f(x)))^2} $.
	\item
	$ k^\prime(x) = \frac{f^\prime(g(x))g^\prime(x) }{g^\prime(f(x))f^\prime(x)} $.
	
	\item
	$ k^\prime(x) = 
	\frac{f^\prime(g(x))g^\prime(x)g(f(x)) - f(g(x)) g^\prime(f(x)) f^\prime(x) }
	{(g(f(x)))^2} $.
	\item
	$ k^\prime(x) = \frac{f^\prime(g(x))g(f(x)) - f(g(x))g^\prime(f(x))}{(g(f(x)))^2} $.
\end{enumerate}
\ \\
\textbf{Lösung:}
\begin{mdframed}
	\underline{\textbf{Vorgehensweise:}}
	\renewcommand{\labelenumi}{\theenumi.}
	\begin{enumerate}
		\item Verwende die Quotientenregel.
	\end{enumerate}
\end{mdframed}

\underline{1. Verwende die Quotientenregel}\\
Nach der Quotientenregel gilt:
\begin{align*}
	k^\prime(x)
	=
	\frac{(f(g(x)))^\prime g(f(x)) - f(g(x)) (g(f(x)))^\prime}{(g(f(x)))^2}
	=
	\frac{f^\prime(g(x)) \cdot g^\prime(x) \cdot g(f(x))
		- 
		f(g(x)) g^\prime(f(x)) \cdot f^\prime(x)
		}{(g(f(x)))^2}.
\end{align*}
Damit ist Antwort (c) korrekt.


\newpage
\subsection*{\frage{10}{3}}
Gegeben ist die Funktion
\begin{align*}
	f \ : \ \R \rightarrow \R, \ x \mapsto y = x \cdot |x| 
\end{align*}
Dann folgt:
\renewcommand{\labelenumi}{(\alph{enumi})}
\begin{enumerate}
	\item 
	$ f $ ist überall stetig und differenzierbar.
	\item
	$ f $ ist stetig in $ x_0 = 0 $, aber nicht differenzierbar in $ x_0 = 0 $.
	\item
	$ f $ ist differenzierbar in $ x_0 = 0 $, aber nicht stetig in $ x_0 = 0 $.
	\item
    $ f $ ist nicht stetig und nicht differenzierbar in $ x_0 = 0 $.
\end{enumerate}
\ \\
\textbf{Lösung:}
\begin{mdframed}
	\underline{\textbf{Vorgehensweise:}}
	\renewcommand{\labelenumi}{\theenumi.}
	\begin{enumerate}
		\item 
	\end{enumerate}
\end{mdframed}

\underline{1. }\\

\newpage
\subsection*{\frage{11}{3}}
Das Taylorpolynom $ 4 $. Ordnung in $ x_0 = 0 $ der Funktion
\begin{align*}
	f(x) = e^{x^3}
\end{align*}
lautet:
\renewcommand{\labelenumi}{(\alph{enumi})}
\begin{enumerate}
	\item 
	$ P_4(x) = \frac{x^4}{4} + \frac{x^3}{3} + \frac{x^2}{2} + x $. 
	\item
	$ P_4(x) = \frac{x^4}{4} + \frac{x^3}{3} + \frac{x^2}{2} - x $. 
	\item
	$ P_4(x) = x^3 + x^2 + x + 2$. 
	\item
	$ P_4(x) = \frac{x^4}{4} + \frac{x^3}{3} + \frac{x^2}{2} + x +1 $. 
	\item
	$ P_4(x) = x^3 + 1$. 
	\item
	$ P_4(x) = \frac{x^4}{4} - \frac{x^3}{3} + \frac{x^2}{2} - x + 2 $. 
\end{enumerate}
\ \\
\textbf{Lösung:}
\begin{mdframed}
	\underline{\textbf{Vorgehensweise:}}
	\renewcommand{\labelenumi}{\theenumi.}
	\begin{enumerate}
		\item 
	\end{enumerate}
\end{mdframed}

\underline{1. }\\

\newpage

\subsection*{\frage{12}{3}}
Die Funktion zweier Variablen $ f $ ist homogen vom Grad $ 3 $ und die Funktion zweier Variablen ist homogen vom Grade $ -3 $.\\
\\
Die Funktion $ h $ ist definiert durch
\begin{align*}
	h(x,y) = f(g(x,y),g(x,y)).
\end{align*}
Dann gilt:
\renewcommand{\labelenumi}{(\alph{enumi})}
\begin{enumerate}
	\item 
	$ h $ ist homogen vom Grad $ -1 $. 
	\item
	$ h $ ist homogen vom Grad $ 0 $. 
	\item
	$ h $ ist homogen vom Grad $ 6 $. 
	\item
	$ h $ ist homogen vom Grad $ -9 $.  
	\item
	$ h $ ist nicht homogen.
\end{enumerate}
\ \\
\textbf{Lösung:}
\begin{mdframed}
	\underline{\textbf{Vorgehensweise:}}
	\renewcommand{\labelenumi}{\theenumi.}
	\begin{enumerate}
		\item 
	\end{enumerate}
\end{mdframed}

\underline{1. }\\