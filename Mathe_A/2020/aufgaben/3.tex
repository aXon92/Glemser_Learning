\section*{Aufgabe 3 (28 Punkte)}
\vspace{0.4cm}
\subsection*{\frage{1}{4}}
Der rechtsseitige Grenzwert
\begin{align*}
	\lim\limits_{x \to 1 +}
	\left(
	\frac{1}{x-1} - \frac{1}{\ln(x)}
	\right)
\end{align*}
ist gleich:
\renewcommand{\labelenumi}{(\alph{enumi})}
\begin{enumerate}
	\item 
	$ -\frac{1}{2}$.
	\item
	$ 0$.
	\item
	$ \frac{1}{2}$.
	\item
	$ \infty$.
\end{enumerate}
\ \\
\textbf{Lösung:}
\begin{mdframed}
\underline{\textbf{Vorgehensweise:}}
\renewcommand{\labelenumi}{\theenumi.}
\begin{enumerate}
\item Wende die Regel von l'H\^{o}pital an.
\end{enumerate}
\end{mdframed}

\underline{1. Wende die Regel von l'H\^{o}pital an}\\
Wegen 
\begin{align*}
	\frac{1}{x-1}
	- \frac{1}{\ln(x)}
	=
	\frac{\ln(x) - (x-1)}{(x-1) \ln (x)}
\end{align*}
erhalten wir den l'H\^{o}pital-Fall \glqq$ \frac{0}{0} $\grqq.
Mit zweifacher Anwendung der Regel von l'H\^{o}pital gilt somit (Der Fall \glqq$ \frac{0}{0} $\grqq~tritt ein weiteres Mal auf):
\begin{align*}
	\lim\limits_{x \to 1 +}
	\frac{\ln(x) - (x-1)}{(x-1) \ln (x)}
	=
	\lim\limits_{x \to 1 +}
	\frac{\frac{1}{x} - 1 }{(x-1) \frac{1}{x} + \ln(x)}
	=
	\lim\limits_{x \to 1 +}
	\frac{\frac{-1}{x^2} }{ \frac{1}{x^2} + \frac{1}{x}}
	=
	-
	\frac{1}{2}
\end{align*}

\newpage

\subsection*{\frage{2}{4}}
Der rechtsseitige Grenzwert
\begin{align*}
	\lim\limits_{x \to 0 +}
	x^x
\end{align*}
ist gleich:
\renewcommand{\labelenumi}{(\alph{enumi})}
\begin{enumerate}
	\item 
	$0$.
	\item
	$1$.
	\item
	$e$.
	\item
	$e^e$.	
\end{enumerate}
\ \\
\textbf{Lösung:}
\begin{mdframed}
\underline{\textbf{Vorgehensweise:}}
\renewcommand{\labelenumi}{\theenumi.}
\begin{enumerate}
\item Nutze die Stetigkeit der Exponentialfunktion aus.
\item Wende die Regel von l'H\^{o}pital an.
\end{enumerate}
\end{mdframed}

\underline{1. Nutze die Stetigkeit der Exponentialfunktion aus}\\
Zunächst formen wir um:
\begin{align*}
	x^x = e^{\ln\left(x^x\right)} = e^{x \ln(x)}.
\end{align*}
Nun gilt wegen der Stetigkeit der Exponentialfunktion:
\begin{align*}
	\lim \limits_{x \to 0 +}
	x^x 
	=
	\lim \limits_{x \to 0 +}
	e^{x \ln(x)}
	=
	e^{\lim_{x \to 0 +} x \ln(x)}.
\end{align*}
Damit genügt es $ \lim_{x \to 0 +} x \ln(x) $ zu bestimmen.\\
\\
\underline{2. Wende die Regel von l'H\^{o}pital an}\\
Mit 
\begin{align*}
	x \ln(x)
	=
	\frac{\ln(x)}{\frac{1}{x}}
\end{align*}
erhalten wir den l'H\^{o}pital-Fall \glqq$ \frac{0}{0} $\grqq.
Durch die Regel von l'H\^{o}pital erhalten wir 
\begin{align*}
	\lim \limits_{x \to 0 +} x \ln(x)
	=
	\lim \limits_{x \to 0 +} \frac{\ln(x)}{\frac{1}{x}}
	=
	\lim \limits_{x \to 0 +} \frac{\frac{1}{x}}{-\frac{1}{x^2}}
	=
	\lim \limits_{x \to 0 +} -x = 0.
\end{align*}
Insgesamt erhalten wir:
\begin{align*}
	\lim \limits_{x \to 0 +} x^x
	=
	e^{\lim_{x \to 0 +} x \ln(x)}
	=
	e^0
	= 
	1.
\end{align*}
Damit ist die Antwort (b) korrekt.




\newpage
\subsection*{\frage{3}{4}}
Ein Blatt Papier sei $ 0.5 $ Millimeter dick. Man faltet das Blatt nun in der Hälfte, so dass es eine neue Dicke von $ 1 $ Millimeter hat.
Anschliessend faltet man es solange erneut, bis die Dicke des gefalteten Papiers $ 400'000 $ Kilometer erreicht (ungefähr die Distanz zwischen Erde und Mond).\\
Wie oft muss das Blatt Papier gefaltet werden ? 
\renewcommand{\labelenumi}{(\alph{enumi})}
\begin{enumerate}
	\item 
	Ungefähr $ 20 $ Mal.
	\item
	Ungefähr $ 30 $ Mal.
	\item
	Ungefähr $ 40 $ Mal.
	\item
	Ungefähr $ 50 $ Mal.
\end{enumerate}
\ \\
\textbf{Lösung:}
\begin{mdframed}
\underline{\textbf{Vorgehensweise:}}
\renewcommand{\labelenumi}{\theenumi.}
\begin{enumerate}
\item Übersetze die Aufgabenstellung in eine geeignete geometrische Folge.
\item Bestimme die Anzahl der Faltungen.
\end{enumerate}
\end{mdframed}
%\allowdisplaybreaks
\underline{1. Übersetze die Aufgabenstellung in eine geeignete geometrische Folge}\\
Mit $ a_n $ bezeichnen wir die Dicke nach $ n $ Faltungen. Da die Dicke in jeder Faltung verdoppelt wird, erhalten wir die rekursive Darstellung
\begin{align*}
	a_n = 2 a_{n-1}
\end{align*}
mit dem Anfangswert $ a_0 = 0.5 \textrm{ (mm)} = 0.5 \cdot 10^{-6} \textrm{ (km)} $.
Also liegt eine geometrische Folge $ \{a_n\}_{n \in \N} $ mit expliziter Vorschrift
\begin{align*}
	a_n = q^n a_0 = 2^n \cdot  0.5 \cdot  10^{-6}
\end{align*}
mit $ q = 2 $ und $ a_0 = 0.5 \cdot  10^{-6} \textrm{ (km)} $ vor.\\
\\
\underline{2. Bestimme die Anzahl der Faltungen}\\
Die Anzahl der Faltungen für $ 400'000 $ Kilometer ergibt durch lösen der Ungleichung $ a_n \geq 400'000 $.
Durch Umformungen erhalten wir:
\begin{align*}
	a_n \geq 400'000 = 4 \cdot 10^5
	\ &\Leftrightarrow \
	2^n \cdot 0.5 \cdot 10^{-6}  \geq 4 \cdot 10^5
	\ \Leftrightarrow \
	2^n \cdot 0.5 \geq 4 \cdot 10^{11}
	\ \Leftrightarrow \
	2^n \geq 8 \cdot 10^{11}\\
	\ &\Leftrightarrow \
	\ln(2^n) \geq \ln\left(8 \cdot 10^{11}\right)
	\ \Leftrightarrow \
	n  \geq \frac{\ln\left(8 \cdot 10^{11}\right)}{\ln(2)} \approx 39.54.
\end{align*}
Also muss das Papier ungefähr $ 40 $ Mal gefaltet sein.\\
\\
Damit ist die Antwort (c) korrekt.


\newpage
\subsection*{\frage{4}{4}}
Ein Projekt benötigt eine Anfangsinvestition von $ 10'000 $ CHF.
Nach einem Jahr generiert das Projekt eine Auszahlung in Höhe von $ 5'000  $ CHF und nach zwei Jahren eine Auszahlung in Höhe von $ 10'000 $ CHF.\\
Der interne Zinssatz des Projekts ist ungefähr
\renewcommand{\labelenumi}{(\alph{enumi})}
\begin{enumerate}
	\item 
	$ 5 \% $.
	\item 
	$ 10 \% $.
	\item
	$ 20 \% $.
	\item
	$ 30 \% $.
\end{enumerate}
\ \\
\textbf{Lösung:}
\begin{mdframed}
\underline{\textbf{Vorgehensweise:}}
\renewcommand{\labelenumi}{\theenumi.}
\begin{enumerate}
\item Verwende den Zusammenhang von Nettobarwert und internem Zinssatz.

\end{enumerate}
\end{mdframed}

\underline{1. Verwende den Zusammenhang von Nettobarwert und internem Zinssatz}\\
Der Nettobarwert des Projekts muss für den internen Zinssatz $ r $ gleich null sein.
Es muss also
\begin{align*}
	-10'000 + \frac{5'000}{1+ r} + \frac{10'000}{(1+r)^2} = 0
\end{align*}
gelten. Durch Umformen erhalten wir hieraus eine quadratische Gleichung:
\begin{align*}
	-10'000 + \frac{5'000}{1+ r} + \frac{10'000}{(1+r)^2} = 0
	\ &\Leftrightarrow \
	\left(
	-10'000 + \frac{5'000}{1+ r} + \frac{10'000}{(1+r)^2}
	\right)
	\cdot 
	\frac{(1+r)^2}{5'000}
	= 
	0\\
	\ &\Leftrightarrow \
	-2 (1+ r)^2 + (1+r) + 2
	= 
	0.
\end{align*}
Die Lösungen hiervon sind durch
\begin{align*}
	1+ r =
	\frac{-1 \pm \sqrt{1^2 + 16}}{-4}
	=
	\frac{-1 \pm \sqrt{17}}{-4}
\end{align*}
gegeben. Für uns ist nur die positive Lösung relevant:
\begin{align*}
	1+ r 
	= \frac{-1 -  \sqrt{17}}{-4} =
	\frac{1 +  \sqrt{17}}{4} 
	\approx 1.3
	\ \Rightarrow \
	r \approx 30 \%.
\end{align*}
Damit ist die Antwort (d) korrekt.
\newpage

\subsection*{\frage{5}{3}}
Gegeben sei die Funktion $ f :D_f \to \R, (x,y) \mapsto z = f(x,y) = \frac{\ln(x^2 +y^2 -1)}{\sqrt{4 - x^2 - y^2}} $.
\renewcommand{\labelenumi}{(\alph{enumi})}
\begin{enumerate}
	\item 
	Der Definitionsbereich von $ f  $ ist das Innere (ohne Rand) des Kreises mit Mittelpunkt $ (0,0) $ und Radius $ 1 $.
	\item
	Der Definitionsbereich von $ f  $ ist das Innere (ohne Rand) des Kreises mit Mittelpunkt $ (0,0) $ und Radius $ 2 $.
	\item
	Der Definitionsbereich von $ f  $ ist das Innere (ohne Rand) des Kreises mit Mittelpunkt $ (0,0) $ und Radius $ 2 $, ohne das Innere (mit Rand) des Kreises mit Mittelpunkt $ (0,0) $ und Radius $ 1 $.
	\item
	Der Definitionsbereich von $ f  $ ist die leere Menge.
\end{enumerate}
\ \\
\textbf{Lösung:}
\begin{mdframed}
\underline{\textbf{Vorgehensweise:}}
\renewcommand{\labelenumi}{\theenumi.}
\begin{enumerate}
\item Verwende die Definitionsbereiche der Logarithmus-und Wurzelfunktion.
\end{enumerate}
\end{mdframed}

\underline{1. Verwende die Definitionsbereiche der Logarithmus-und Wurzelfunktion}\\
Der Definitionsbereich der Logarithmusfunktion liefert die Bedingung
\begin{align*}
	x^2 +y^2 -1 > 0
	\ \Leftrightarrow \
	x^2 + y^2 > 1^2 
\end{align*}
und mit dem der Wurzelfunktion erhalten wir
\begin{align*}
	4 - x^2 - y^2 > 0 
	\ \Leftrightarrow \
	x^2 + y^2 < 4 = 2^2.
\end{align*} 
Zusammengefasst gilt:
\begin{align*}
	(x,y) \in D_f
	\ \Leftrightarrow \
	\underbrace{x^2 + y^2 > 1^2}_{\textrm{(i)}} \  \wedge  \ \underbrace{x^2 + y^2 <  2^2}_{\textrm{(ii)}}.
\end{align*}
Durch (i) schneiden wir die Kreisscheibe mit Radius 1 und Rand aus der durch (ii) gegebenen Kreisscheibe mit Radius 2 und ohne Rand heraus.\\
\\
Damit ist die Antwort (c) korrekt.
\newpage

\subsection*{\frage{6}{3}}
Sei $ f $ eine Funktion zweier reellen Variablen gegeben durch:
\begin{align*}
	f: \R_{++}^2 \to \R, \ (x,y) \mapsto z = f(x,y) = 10 x^\alpha y^{1- \alpha},
\end{align*}
wobei $ \alpha \in (0,1) $.\\
Die Steigung der Tangente an die Niveaulinie von $ f $ am Punkt $ (x_0,y_0) = (1,1) $ ist gleich $ -0.5 $.\\
Dann folgt:
\renewcommand{\labelenumi}{(\alph{enumi})}
\begin{enumerate}
	\item 
	$ \alpha = \frac{1}{6}$.
	\item
	$ \alpha = \frac{1}{3}$.
	\item
	$ \alpha = \frac{1}{2}$.
	\item
	$ \alpha = \frac{2}{3}$.
\end{enumerate}
\ \\
\textbf{Lösung:}
\begin{mdframed}
\underline{\textbf{Vorgehensweise:}}
\renewcommand{\labelenumi}{\theenumi.}
\begin{enumerate}
\item Verwende den Satz über implizite Funktionen.
\end{enumerate}
\end{mdframed}

\underline{1. Verwende den Satz über implizite Funktionen}\\
Mit dem Satz über implizite Funktionen können wir die Steigung der Tangente an der Niveaulinie von $ f $ bei $ (x_0,y_0) = (1,1) $ direkt berechnen.
Zuerst benötigen wir die partiellen Ableitung:
\begin{align*}
	f_x(x,y) &= 10 \alpha x^{\alpha - 1} y^{1- \alpha}\\
	f_y(x,y) &= 10 (1- \alpha ) x^\alpha y^{- \alpha}.
\end{align*}
Mit diesen liefert der Satz über implizite Funktionen
\begin{align*}
	-0.5 = - 
	\frac{f_x(x_0 ,y_0)}{f_y(x_0,y_0)}
	=
	-
	\frac{ 10 \alpha x^{\alpha - 1} y^{1- \alpha}}{10 (1- \alpha ) x^\alpha y^{- \alpha}}
	- \frac{\alpha}{1- \alpha}
\end{align*}
mit $ (x_0,y_0) = (1,1) $. Mit 
\begin{align*}
	- \frac{\alpha}{1- \alpha} = -0.5
	\ \Leftrightarrow \
	\alpha = 0.5 (1- \alpha ) = \frac{1}{2} - \frac{\alpha}{2}
	\ \Leftrightarrow \
	\frac{3 \alpha } {2} = \frac{1}{2}
	\ \Leftrightarrow \
	3 \alpha = 1 
	\ \Leftrightarrow \
	\alpha = \frac{1}{3}
\end{align*}
erhalten wir die passenden Antwort.\\
\\
Damit ist die Antwort (b) korrekt.



\newpage



\subsection*{\frage{7}{3}}
Gegeben ist die Funktion
\begin{align*}
	f(x,y) 
	=
	\ln\left(
	x^3 \sqrt[3]{y^4} + \sqrt[6]{x^{11} y^{15}}
	\right)
	- \frac{13}{3} \ln(x) \quad \textrm{für } x>0,y>0.
\end{align*}
Welche der folgenden Aussagen ist richtig?
\renewcommand{\labelenumi}{(\alph{enumi})}
\begin{enumerate}
	\item
	$ f  $ ist homogen von Grad $ 5 $.
	\item
	$ f $ ist linear homogen.	
	\item 
	$ f  $ ist homogen von Grad $ 0 $.
	\item
	$ f $ ist nicht homogen.
\end{enumerate}
\ \\
\textbf{Lösung:}
\begin{mdframed}
\underline{\textbf{Vorgehensweise:}}
\renewcommand{\labelenumi}{\theenumi.}
\begin{enumerate}
\item Rechne die Homogenitätsbedingung nach.
\end{enumerate}
\end{mdframed}

\underline{1. Rechne die Homogenitätsbedingung nach}\\
Für $ \lambda >0  $ erhalten wir:
\begin{align*}
	f(\lambda x, \lambda y)
	&=
	\ln\left(
	(\lambda x)^3 \sqrt[3]{(\lambda y)^4}
	+
	\sqrt[6]{(\lambda x)^{11} (\lambda y)^{15}}
	\right)
	-
	\frac{13}{3} \ln(\lambda x)\\
	&=
	\ln\left(
	\lambda^3 x^3 \sqrt[3]{\lambda^4 y^4}
	+
	\sqrt[6]{\lambda^{11} x^{11} \lambda^{15} y^{15}}
	\right)
	-
	\frac{13}{3} \ln(\lambda) - \frac{13}{3 }\ln(x)\\
	&=
	\ln\left(
	\lambda^3 x^3 \lambda^{\frac{4}{3}} \sqrt[3]{ y^4}
	+
	\sqrt[6]{\lambda^{26} x^{11}  y^{15}}
	\right)
	-
	\frac{13}{3} \ln(\lambda) - \frac{13}{3 }\ln(x)\\
	&=
	\ln\left(
	\lambda^{\frac{9}{3} + \frac{4}{3}} x^3  \sqrt[3]{ y^4}
	+
	\lambda^{\frac{13}{3}}
	\sqrt[6]{x^{11}  y^{15}}
	\right)
	-
	\frac{13}{3} \ln(\lambda) - \frac{13}{3 }\ln(x)\\
	&=
	\ln\left(
	\lambda^{\frac{13}{3} } \left(
	 x^3  \sqrt[3]{ y^4}
	+
	\sqrt[6]{x^{11}  y^{15}}
	\right)
	\right)
	-
	\frac{13}{3} \ln(\lambda) - \frac{13}{3 }\ln(x)\\
	&=
	\ln\left(
	\lambda^{\frac{13}{3} }\right) + \ln  \left(
	x^3  \sqrt[3]{ y^4}
	+
	\sqrt[6]{x^{11}  y^{15}}
	\right)
	-
	\frac{13}{3} \ln(\lambda) - \frac{13}{3 }\ln(x)\\
	&=
	\frac{13}{3} \ln\left(
	\lambda\right) + \ln  \left(
	x^3  \sqrt[3]{ y^4}
	+
	\sqrt[6]{x^{11}  y^{15}}
	\right)
	-
	\frac{13}{3} \ln(\lambda) - \frac{13}{3 }\ln(x)\\
	&=
	\frac{13}{3} \ln\left(
	\lambda\right) + \ln  \left(
	x^3  \sqrt[3]{ y^4}
	+
	\sqrt[6]{x^{11}  y^{15}}
	\right)
	-
	\frac{13}{3} \ln(\lambda) - \frac{13}{3 }\ln(x)\\
	&=
	\ln  \left(
	x^3  \sqrt[3]{ y^4}
	+
	\sqrt[6]{x^{11}  y^{15}}
	\right)
	-
	\frac{13}{3 }\ln(x)
	= \lambda^0 f(x,y).
\end{align*}
Also ist $ f $ homogen von Grad $ 0 $.\\
\\
Damit ist Antwort (c) korrekt.

\newpage

\subsection*{\frage{8}{3}}
Sei $ f $ eine Funktion einer reellen Variablen, definiert als $ f(x) = x^\alpha $ für $ \alpha > 0 $, und sei $ g $ eine Funktion zweier reellen Variablen, die strikt positiv und homogen von Grad $ \kappa $ ist.\\
Wir definieren die Funktion $ h $ als $ h(x,y)  = f(g(x,y))$.\\
Welche der folgenden Aussagen ist richtig?
\renewcommand{\labelenumi}{(\alph{enumi})}
\begin{enumerate}
	\item 
	$ h $ ist homogen von Grad $ 1 $, falls $ \alpha =0.5 $ und $ \kappa = 1 $.
	\item
	$ h $ ist homogen von Grad $ 2 $, falls $ \alpha =0.5 $ und $ \kappa = 4 $.
	\item
	$ h $ ist homogen von Grad $ 3 $, falls $ \alpha =0.5 $ und $ \kappa = 7 $.
	\item
	$ h $ ist homogen von Grad $ 5 $, falls $ \alpha =0.5 $ und $ \kappa = 11 $.
\end{enumerate}
\ \\
\textbf{Lösung:}
\begin{mdframed}
\underline{\textbf{Vorgehensweise:}}
\renewcommand{\labelenumi}{\theenumi.}
\begin{enumerate}
\item Verwende die Definition einer homogenen Funktion.
\end{enumerate}
\end{mdframed}

\underline{1. Verwende die Definition einer homogenen Funktion}\\
Sei $ \lambda > 0 $.
Dann erhalten wir mit der Homogenität von $ g $:
\begin{align*}
	h(\lambda x, \lambda y)
	&=
	f ( g(\lambda x, \lambda y))
	=
	f ( \lambda^\kappa g(x,y))
	=
	( \lambda^\kappa g(x,y))^\alpha\\
	&=
	\lambda^{\alpha \kappa} (g(x,y))^\alpha
	=
	\lambda^{\alpha \kappa} f((g(x,y))\\
	&=
	\lambda^{\alpha \kappa} h(x,y).
\end{align*}
Also ist $ h $ homogen vom Grad $ \alpha \kappa $.
Wir müssen noch überprüfen, welche Antwort passt.\\
Die Multiplikation von $ \alpha $ und $  \kappa $ ergibt in (a), (c) und (d) keine ganze Zahl. Damit können wir diese Möglichkeiten ausschließen.
Für $ \alpha = 0.5 $ und $ \kappa = 4 $ ergibt sich $ \alpha \cdot \kappa = 2 $.\\
\\
Damit ist Antwort (b) korrekt.


