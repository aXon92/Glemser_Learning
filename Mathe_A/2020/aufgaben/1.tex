\vspace{1cm}
\fancyhead[C]{\normalsize\textbf{$\qquad$ Teil I: Offene Aufgaben}}
\renewcommand{\labelenumi}{\theenumi.}
\section*{Aufgabe 1 (36 Punkte)}
\vspace{0.4cm}
%\titleformat{\subsection}[runin]
%{\normalfont\large\bfseries}{\thesubsection}{1em}{}
\subsection*{\aufgabe{a1}{6}} 
Thomas Robert Malthus (1766-1834), ein britischer Ökonom und Pfarrer, veröffentlichte
1798 seinen vielbeachteten \textit{Essay on the Principle of Population}.
Er formulierte das Axiom, dass die Weltbevölkerung exponentiell wachse, die Nahrung dagegen nur linear. 
Das bedeutet, dass die Funktion $ n(t) $, welche die Anzahl Menschen angibt, die man bei optimaler Verteilung der zur Verfügung stehenden Nahrungsmittel zu Zeit $ t $ ernähren könnte, eine lineare ist:
\begin{align*}
	n(t) = a + b \cdot t.
\end{align*}
Die Weltbevölkerung $ w(t) $ betrug im Jahre 1800 ($ t= 0 $) rund eine Milliarde, d.h. $ w(0) = 10^9 $.
Wir wollen annehmen, dass die Weltbevölkerung rund $ 1\% $ pro Jahr wächst.
Außerdem wollen wir mit
\begin{align*}
	a = 2 \cdot 10^9 \ \textrm{und} \ b = 0.2 \cdot 10^9
\end{align*}
rechnen.
\begin{description}
	\item[Beweisen Sie:] 
	Es gibt genau einen Zeitpunkt $ t^\star > 400 $, an dem die Weltbevölkerung gerade noch ernährt werden kann, d.h., für $ w(t^\star)  = n(t^\star )$ ist.
\end{description}  
\
\\
\textbf{Lösung:}
\begin{mdframed}
\underline{\textbf{Vorgehensweise:}}
\renewcommand{\labelenumi}{\theenumi.}
\begin{enumerate}
\item Stelle den mathematischen Rahmen her.
\item Zeige die Aussage.

\end{enumerate}
\end{mdframed}
\underline{1. Stelle den mathematischen Rahmen her}\\
Wir sind an der Existenz und Eindeutigkeit eines Zeitpunktes $ t^\star $ interessiert, an welchem die vorhandenen Lebensmittel gerade noch ausreichen, um die Weltbevölkerung zu ernähren. Sei $ n(t)  $ die Anzahl an Personen, welche zu dem Zeitpunkt $ t $ ernährt werden können und $ w(t) $ die Weltbevölkerung zum Zeitpunkt $ t $.
Wenn 
\begin{align*}
	w(t) = n(t) \ \Leftrightarrow \ w(t) - n(t) = 0
\end{align*}
gilt, kann die Weltbevölkerung gerade noch ernährt werden. Aus diesem Grund definieren wir die Funktion
\begin{align*}
	f(t) := w(t) - n(t).
\end{align*}
Unser Ziel ist: Zeige das ein eindeutiger Zeitpunkt $ t^\star \in (400, \infty) $ mit $ f(t^\star) = 0 $ existiert.
Hier kommt uns der Nullstellensatz von Bolzano zur Hilfe. Dieser besagt: \\
Sei $ g  $ stetig auf dem Intervall $  [a,b] $, $ g(a) $ und $ g(b) $ haben verschiedene Vorzeichen. Dann hat $ g $ mindestens eine Nullstelle in $ [a,b] $. Ist g auch noch streng monoton, dann existiert nur eine Nullstelle sprich der Zeitpunkt ist eindeutig. \\
Nun ist es an der Zeit $ f $ konkret aufzustellen. Hierfür betrachten wir zuerst die Funktion für die exponentiell wachsende Weltbevölkerung:
\begin{align*}
	w(t) = w(0) \cdot (1 + 1 \% )^t= 10^9 \cdot 1.01^t.
\end{align*}
Diese ist als Zusammensetzung stetiger Funktionen stetig auf $ \R $. Das lineare Wachstum der ernährbaren Menschen ergibt sich aus der Aufgabenstellung:
\begin{align*}
	n(t) = a + b \cdot t
	= 2 \cdot 10^9 + 0.2 \cdot 10^9 \cdot t.
\end{align*}
Auch diese Funktion ist als Zusammensetzung stetiger Funktionen stetig auf $ \R $.
Daraus ergibt sich die auf $ \R $ stetige Funktion 
\begin{align*}
	f(t) = w(t) - n(t)
	=
	10^9 \cdot 1.01^t - (2 \cdot 10^9 + 0.2 \cdot 10^9 \cdot t).
\end{align*}

\underline{2. Zeige die Aussage}\\
Da $ f $ auf $ \R $ stetig ist, ist diese auch auf $ [400, b] $ mit $ b > 400 $ stetig.
Wir bestimmen:
\begin{align*}
	f(400) &= 10^9 \cdot 1.01^{ 400} - (2 \cdot 10^9 + 0.2 \cdot 10^9 \cdot 400 )
	= 10^9 \cdot (1.01^{ 400} - 2 - 0.2 \cdot 400)\\
	&= 10^9 \cdot (1.01^{ 400} - 2 - 80) 
	=  10^9 \cdot (1.01^{ 400} -82) 
	\approx 10^9 \cdot (- 28.48) < 0\\
	f(500) &= 10^9 \cdot (1.01^{ 500} - 2 - 0.2 \cdot 500)
	 = 10^9 \cdot (1.01^{ 500} - 2 - 0.2 \cdot 500)
	= 10^9 \cdot (1.01^{ 500} - 102)\\
	&\approx 10^9 \cdot 42.77 > 0.
\end{align*}
Nach dem Nullstellensatz von Bolzano hat $ f $ in dem Intervall $ [400,500] $  mindestens eine Nullstelle. Wir müssen zeigen, dass $ f  $ auf diesem Intervall streng monoton wachsend ist. Dann ist diese Nullstelle eindeutig.
Hierfür bestimmen wir die Ableitungen von $ w $ und $ n $:
\begin{align*}
	w(t) = 10^9 1.01^t = 10^9 e^{\ln(1.01^t)}
	=
	10^9 e^{t \ln(1.01)} \ \Rightarrow \
	w^\prime(t) &= 10^9 \ln(1.01) e^{t \ln(1.01)} = 10^9 \ln(1.01) 1.01^t\\
	n^\prime(t) &= 0.2\cdot 10^9.
\end{align*}
Damit ergibt sich
\begin{align*}
	f^\prime(t) = w^\prime(t) - n^\prime(t)
	= 
	10^9 \cdot  \ln(1.01) \cdot 1.01^t - 0.2\cdot 10^9 > 0
\end{align*}
für $ t > 400 $. 
Dies erkennt man an $ f^\prime(400) >0  $ und der strengen Monotonie von $ 1.01^t $.
Also ist $ f $ streng monoton wachsend auf $ [400,500] $ und die Nullstelle ist eindeutig.
 
\newpage

\subsection*{\aufgabe{a2}{6}}
Thomas Robert Malthus (1766-1834), ein britischer Ökonom und Pfarrer, veröffentlichte
1798 seinen vielbeachteten \textit{Essay on the Principle of Population}.
Er formulierte das Axiom, dass die Weltbevölkerung exponentiell wachse, die Nahrung dagegen nur linear. 
Das bedeutet, dass die Funktion $ n(t) $, welche die Anzahl Menschen angibt, die man bei optimaler Verteilung der zur Verfügung stehenden Nahrungsmittel zu Zeit $ t $ ernähren könnte, eine lineare ist:
\begin{align*}
	n(t) = a + b \cdot t.
\end{align*}
Die Weltbevölkerung $ w(t) $ betrug im Jahre 1800 ($ t= 0 $) rund eine Milliarde, d.h. $ w(0) = 10^9 $.
Wir wollen annehmen, dass die Weltbevölkerung rund $ 1\% $ pro Jahr wächst.
Außerdem wollen wir mit
\begin{align*}
	a = 2 \cdot 10^9 \ \textrm{und} \ b = 0.2 \cdot 10^9
\end{align*}
rechnen.\\
\\
Es gibt genau einen Zeitpunkt $ t^\star > 400 $, an dem die Weltbevölkerung gerade noch ernährt werden kann, d.h., für $ w(t^\star)  = n(t^\star )$ ist.\\
\\
Verwenden Sie eine Taylor-Approximation zweiter Ordnung im Punkt $ t_0 = 400 $, um eine Näherung für $ t^\star $ zu finden.
\\ \\
\textbf{Lösung:}
\begin{mdframed}
\underline{\textbf{Vorgehensweise:}}
\renewcommand{\labelenumi}{\theenumi.}
\begin{enumerate}
\item Bestimme das Taylorpolynom. 
\item Approximiere den Zeitpunkt.
\end{enumerate}
\end{mdframed}

\underline{1. Bestimme das Taylorpolynom}\\
In dieser Aufgabe wollen wir $ t^\star $ aus der letzten Aufgabe mit Hilfe des Taylorpolynoms zweiter Ordnung $ P_2 $ von $ f $ bei $ t_0 = 400 $ approximieren.
Das bedeutet: Wir lösen 
\begin{align*}
	P_2(t) = 0
\end{align*}
als Näherung der ursprünglichen Gleichung $ f(t) = 0 $. Zunächst müssen wir das Taylorpolynom zweiter Ordnung $ P_2 $ von $ f $ bei $ t_0 = 400 $ bestimmen.
Formal ausgeschrieben ist dies:
\begin{align*}
	P_2(t)
	=
	f(400) + f^\prime(400) \cdot (t -400) + \frac{f^{\prime \prime }(400)}{2}(t -400)^2. 
\end{align*}
Wir benötigen noch die erste und zweite Ableitung von $ f $:
\begin{align*}
	f^\prime(t) &= 10^9 \cdot  \ln(1.01) \cdot 1.01^t - 0.2\cdot 10^9\\
	f^{\prime \prime}(t) &= 10^9 \cdot \ln(1.01)^2 \cdot 1.01^t.
\end{align*}
Hierbei haben wir den selben Trick verwendet, wie bei der Ableitung von $ w $ in der letzten Aufgabe. Nun müssen wir $ f $ und dessen Ableitungen noch auswerten:
\begin{align*}
	f(400) &= 10^9 \cdot 1.01^{400} - (2 \cdot 10^9 + 0.2 \cdot 10^9 \cdot 400)
	= 10^9 \cdot (1.01^{400} - 82)\\
	f^\prime(400)
	&=
	10^9 \cdot (\ln(1.01) \cdot  1.01^{400}- 0.2)\\
	f^{\prime \prime}(400)
	&=
	10^9 \cdot  \ln(1.01)^2 \cdot 1.01^{400}.
\end{align*}
Damit erhalten wir das Taylorpolynom zweiter Ordnung $ P_2 $ von $ f $ in $ t_0 = 400 $:
\begin{align*}
	P_2(t)
	&=
	10^9 \cdot (1.01^{400} - 82) + 10^9 \cdot (\ln(1.01) \cdot  1.01^{400}- 0.2) \cdot (t -400) + \frac{10^9 \cdot  \ln(1.01)^2 \cdot 1.01^{400}}{2}(t -400)^2\\
	&=
	10^9 \cdot 
	\left(
	1.01^{400} - 82 + (\ln(1.01) \cdot  1.01^{400}- 0.2) \cdot (t -400)
	+
	\frac{1}{2} \cdot  \ln(1.01)^2 \cdot 1.01^{400}  \cdot (t -400)^2
	\right).
\end{align*}
Für die Gleichung $ P_2(t) = 0  $ ist der Faktor $ 10^9  $ nicht relevant. Mit der Substitution $  s = t - 400 $ erhalten wir die Gleichung
\begin{align*}
	1.01^{400} - 82 + (\ln(1.01) \cdot  1.01^{400}- 0.2) \cdot s
	+
	\frac{1}{2} \cdot  \ln(1.01)^2 \cdot 1.01^{400}  \cdot s^2.
\end{align*}
Hierauf können wir die Mitternachtsformel anwenden und erhalten die Lösungen:
\begin{align*}
	s_{\nicefrac{1}{2}}
	=
	\frac{-\left(\ln(1.01) \cdot  1.01^{400}- 0.2\right) \pm 
	\sqrt{\left(\ln(1.01) \cdot  1.01^{400}- 0.2\right)^2 -
		4 \cdot \frac{1}{2} \cdot  \ln(1.01)^2 \cdot 1.01^{400} \cdot (1.01^{400} - 82 )}}
	{2 \cdot \frac{1}{2} \cdot  \ln(1.01)^2 \cdot 1.01^{400 }}.
\end{align*}
Nachdem man dies in den Taschenrechner eingegeben hat, erhält man die Lösungen\\
$ s_1 = -189.94 $ und $ s_2 = 58.43 $.
Mit der Rücksubstitution $ t = s + 400 $ folgt:
\begin{align*}
	t_1 &= 216.06\\
	t_2 &= 458.43.
\end{align*}
Die Lösung $ t_1 $ erfüllt die Bedingung $ t^\star > 400  $ nicht. Damit erhalten wir
\begin{align*}
	t^\star \approx 458.43
\end{align*}  
als Approximation von $ t^\star $.
\newpage
\subsection*{\aufgabe{b}{10}}
Max hat über seine Verhältnisse gelebt. 
Deshalb hat er nun Schulden von mehreren hunderttausend Franken.
Die Schuldenberatung vermittelt ihm am Anfang des Jahres einen Privatkredit mit $ i = 5 \% $ in der genannten Höhe von $ S = 355'000 \ [\textrm{CHF}] $, den er in $ 12 $ gleich grossen Raten $ C $ jeweils per Jahresende abbezahlen muss. 
\begin{enumerate}
	\item[(b1)] Fügen Sie die Ereignisse und Mittelflüsse dem Zeitstrahl hinzu.
	\item[(b2)] Berechnen Sie die Höhe der Ratenzahlung $ C $.
\end{enumerate}
Am Ende des vierten Jahres gewinnt Max im Lotto CHF $ 160'000 $.
Er beschließt in diesem Jahr statt der normalen Rate $ C $ den ganzen Lottogewinn zur Abzahlung des Kredits zu benützen.
\begin{enumerate}
	\item[(b3)] Ergänzen Sie die Information am Zeitstrahl und berechnen Sie die Restschuld $ \overline{S} $ nach der Einzahlung am Ende des vierten Jahres.
\end{enumerate}
Max zahlt nun weiter am Ende jeden Jahres den Betrag $ C $ zu Tilgung seiner Restschuld $ \overline{S} $.
\begin{enumerate}
	\item[(b4)] Wie viele Zahlungen muss er machen, bis er die Restschuld vollkommen getilgt hat?
	\item[(b5)] Mit der letzten Rate muss nicht mehr der komplette Betrag $ C $ geleistet werden. Wie hoch ist die Zahlung genau?
\end{enumerate}
\begin{center}
	\includegraphics[scale=0.3]{pictures/zeitstrahl_1_b}
\end{center}
\ \\
\textbf{Lösung:}
\begin{mdframed}
\underline{\textbf{Vorgehensweise:}}
\begin{enumerate}
\item[(b1)] Füge den Kredit und die Raten zum Zeitstrahl hinzu.
\item[(b2)] Bestimme die Höhe der Ratenzahlung $ C $.
\item[(b3)] Ergänze den Lottogewinn und berechne die Restschuld.
\item[(b4)] Bestimme die restliche Laufzeit mit Hilfe der Restschuld.
\item[(b5)] Bestimme die letzte Rate.
\end{enumerate}
\end{mdframed}

\newpage
\underline{(b1) Füge den Kredit und die Raten zum Zeitstrahl hinzu}\\
\begin{center}
	\includegraphics[scale=0.55]{pictures/zeitstrahl_1_b_filled_1}
\end{center}

\underline{(b2) Bestimme die Höhe der Ratenzahlung $ C $}\\
Die konstanten Zahlungen $ C $ über $ 12 $ Jahre am Ende des Jahres entspricht einer nachschüssigen Rentenzahlung.
Damit die Schuld vollständig beglichen ist, muss der Endwert der nachschüssigen Rente der über $ 12 $ Jahre aufgezinsten Schuld entsprechen.
Das heißt:
\begin{align*}
	\underbrace{355'000 \cdot (1+ 0.05)^{12}}_{\textrm{Schuldenwert nach $ 12 $ Jahren}}
	=
	\underbrace{C \frac{(1+0.05)^{12} -1}{0.05}}_{\textrm{Endwert einer zwölfjährigen nachschüssigen Rente}}.
\end{align*}
Dies ist äquivalent zu:
\begin{align*}
	C = 355'000 \cdot (1+ 0.05)^{12} \cdot \frac{0.05}{(1+0.05)^{12} -1}
	=
	\frac{355'000\cdot (1+ 0.05)^{12} \cdot 0.05 }{(1+0.05)^{12} -1}
	\approx
	40'053.02 \ \mathrm{(CHF)}.
\end{align*}

\underline{(b3) Ergänze den Lottogewinn und berechne die Restschuld}\\
\begin{center}
	\includegraphics[scale=0.55]{pictures/zeitstrahl_1_b_filled_2}
\end{center}
Der offene Betrag am Ende  des vierten Jahren entspricht der aufgezinsten Schuld nach $ 4 $ Jahren abzüglich der bereits gezahlten Renten.
Hiervon müssen wir noch den Lottogewinn von $ 160'000 $ abziehen:
\begin{align*}
	\overline{S}
	=
	\underbrace{355'000 \cdot (1.05)^{4}}_{\textrm{aufgezinster Wert nach vier Jahren} }
	&-
	\underbrace{C\frac{1.05^4 - 1}{0.05}}_{\textrm{Endwert der Rente am Schluss des vierten Jahres}}\\
	&-\underbrace{(160'000 -40'053.02 ) }_{\textrm{Lottogewinn abzügl. der am Ende des $ 4. $ Jahres fälligen Rate}}\\
	&\approx 
	138'924.30 \ \mathrm{(CHF)}.
\end{align*}

\underline{(b4) Bestimme die restliche Laufzeit mit Hilfe der Restschuld}\\
Wir gehen davon aus, dass Max mit der Zahlung $ C $ fortfährt.
Also ist die Schuld getilgt, wenn die aufgezinste Restschuld $ \overline{S} $ nach dem vierten Jahr dem Endwert der Rente $ C $ entspricht. Das bedeutet:
\begin{align*}
	\underbrace{1.05^n \cdot \overline{S}}_{\textrm{aufgezinster Schuldenwert}}
	&=
	\underbrace{C\frac{1.05^n -1}{0.05}}_{\textrm{Endwert der nachschüssigen Rente}}
	=
	\frac{C \cdot 1.05^n}{0.05} - \frac{C}{0.05}\\
	\ \Leftrightarrow \
	1.05^n \left(\overline{S} - \frac{C}{0.05}\right) &= - \frac{C}{0.05}
	\ \Leftrightarrow \
	1.05^n \cdot \frac{0.05 \cdot \overline{S} -C}{0.05}  = -\frac{C}{0.05}\\
	\ \Leftrightarrow \
	1.05^n &= -\frac{C}{0.05} \cdot \frac{0.05}{0.05 \cdot \overline{S} -C}
	 \  \Leftrightarrow \
	1.05^n =  \frac{C}{C - 0.05 \cdot \overline{S}}\\
	\ \Leftrightarrow \
	\ln(1.05^n) &= \ln\left(\frac{C}{C - 0.05 \cdot \overline{S}}\right)\\
	\ \Leftrightarrow \
	n &= \ln\left(\frac{C}{C - 0.05 \cdot \overline{S}}\right)\cdot \frac{1}{\ln(1.05)}
	\approx 3.9 \ \textrm{(Jahre)}.
\end{align*}
Damit sind insgesamt noch $ 4 $ Zahlungen notwendig.\\
\\
\underline{(b5) Bestimme die letzte Rate}\\
Der Endwert der $ 4 $ Zahlungen aus (b4) ist
\begin{align*}
	C \cdot \frac{1.05^4 -1 }{0.05}
	\approx 172'633.52 \ \mathrm{(CHF)}.
\end{align*}
Der Endwert der ausstehenden Schulden beträgt hingegen
\begin{align*}
	1.05^4 \overline{S}
	\approx 
	168'863.35 \ \mathrm{(CHF)}.
\end{align*}
Demnach muss die letzte Zahlung kleiner als $ C = 40'053.02 \ \mathrm{(CHF)} $ sein.
Diese erhalten wir mit:
\begin{align*}
	40'053.02 - (172'633.52 - 168'863.35)
	=36'282.85 \ \mathrm{(CHF)}.
\end{align*}

\newpage
\subsection*{\aufgabe{c}{6}}
Nach der Relativitätstheorie gilt für die Masse $ m $ eines Körpers, der sich mit der Geschwindigkeit $ v $ bewegt
\begin{align*}
	m(v)
	=
	\frac{m_0}{\sqrt{1 - \frac{v^2}{c^2}}}.
\end{align*}
Dabei ist $ c $ die Lichtgeschwindigkeit $ (299'792'458 \nicefrac{m}{s}) $ und $ m_0 $ die Masse in Ruhe.
\begin{enumerate}
	\item[(c1)] Berechnen Sie die Elastizität $ \varepsilon_m $ von $ m(v) $.
	\item[(c2)] Um wie viel Prozent ändert sich näherungsweise die Masse, wenn die Geschwindigkeit von $ v_0 = 0.5c $ um $ 5 \% $ erhöht wird?
\end{enumerate}
\ \\
\textbf{Lösung:}
\begin{mdframed}
\underline{\textbf{Vorgehensweise:}}
\begin{enumerate}
\item[(c1)] Bestimme die Elastizität.
\item[(c1)] Alternative: Bestimme die Elastizität mit Hilfe des Logarithmus.
\item[(c2)] Approximiere die Massenänderung mit der Elastizität.
\end{enumerate}
\end{mdframed}

\underline{(c1) Bestimme die Elastizität}\\
Die Elastizität ist definiert durch
\begin{align*}
	\varepsilon_m(v) = v \frac{m^\prime(v)}{m(v)}.
\end{align*}
Um die Ableitung von $ m $ zu bestimmen, stellen wir $ m $ zunächst um:
\begin{align*}
	m(v)
	=
	\frac{m_0}{\sqrt{1 - \frac{v^2}{c^2}}}
	= 
	m_0 \cdot \left(1 - \frac{v^2}{c^2}\right)^{-\frac{1}{2}}.
\end{align*}
Damit folgt:
\begin{align*}
	m^\prime(v)
	=
	m_0 \left(-\frac{1}{2}\right) \left(1 - \frac{v^2}{c^2}\right)^{-\frac{3}{2}} \cdot  \left(-\frac{2v}{c^2}\right)
	=
	m_0 \frac{v}{c^2} \left(1 - \frac{v^2}{c^2}\right)^{-\frac{3}{2}}.
\end{align*}
Somit erhalten wir für die Elastizität:
\begin{align*}
	\varepsilon_m(v)
	=
	v \frac{m^\prime(v)}{m(v)}
	=
	v
	\frac{m_0 \frac{v}{c^2} \left(1 - \frac{v^2}{c^2}\right)^{-\frac{3}{2}}}{m_0  \left(1 - \frac{v^2}{c^2}\right)^{-\frac{1}{2}}}
	=
	v
	\frac{\frac{v}{c^2} \left(1 - \frac{v^2}{c^2}\right)^{\frac{1}{2}}}{ \left(1 - \frac{v^2}{c^2}\right)^{\frac{3}{2}}}
	=
	v
	\frac{\frac{v}{c^2} }{ \left(1 - \frac{v^2}{c^2}\right)}
	=
	\frac{\frac{v^2}{c^2} }{ \left(1 - \frac{v^2}{c^2}\right)}.
\end{align*}
\ \\
\underline{(c1) Alternative: Bestimme die Elastizität mit Hilfe des Logarithmus}\\
Mit der Kettenregel erhalten wir:
\begin{align*}
	\frac{\mathrm{d}}{\mathrm{dv}} \ln(m(v))
	=
	\frac{m^\prime(v)}{m(v)}.
\end{align*}
Deswegen lässt sich die Definition der Elastizität auch durch 
\begin{align*}
		\varepsilon_m(v) = v \frac{\mathrm{d}}{\mathrm{dv}}\ln(m(v))
\end{align*}
beschreiben. Hiermit folgt:
\begin{align*}
	\frac{\mathrm{d}}{\mathrm{dv}}\ln(m(v))
	&=
	\frac{\mathrm{d}}{\mathrm{dv}}
	\ln\left(	m_0 \cdot \left(1 - \frac{v^2}{c^2}\right)^{-\frac{1}{2}}\right)
	=
	\frac{\mathrm{d}}{\mathrm{dv}}
	\ln\left(	m_0 \right)+ \ln\left( \left(1 - \frac{v^2}{c^2}\right)^{-\frac{1}{2}}\right)\\
	&=
	\frac{\mathrm{d}}{\mathrm{dv}}\left(
	\ln\left(	m_0 \right)- \frac{1}{2} \ln\left( 1 - \frac{v^2}{c^2}\right)
	\right)
	=
	- \frac{1}{2}\frac{\mathrm{d}}{\mathrm{dv}}\ln\left( 1 - \frac{v^2}{c^2}\right)
	=
	\left(- \frac{1}{2}\right) \cdot  \frac{1}{1 - \frac{v^2}{c^2}} \cdot \left(-\frac{2v}{c^2}\right)\\
	&=
	\frac{\frac{v}{c^2}}{1 - \frac{v^2}{c^2}}.
\end{align*}
Damit ergibt sich die Elastizität 
\begin{align*}
	\varepsilon_m(v) = v \frac{\mathrm{d}}{\mathrm{dv}}\ln(m(v))=
	v \frac{\frac{v}{c^2}}{1 - \frac{v^2}{c^2}}
	=\frac{\frac{v^2}{c^2}}{1 - \frac{v^2}{c^2}}.
\end{align*}
\ \\
\underline{(c2) Approximiere die Massenänderung mit der Elastizität}\\
Die Massenänderung lässt sich durch 
\begin{align*}
	\frac{\Delta m(v_0)}{m(v_0)}
	\approx 
	\varepsilon_m(v_0) \frac{\Delta v}{v_0}
\end{align*}
approximieren. Hierbei ist $  \Delta m(v_0) = m(v_0 + \Delta v) - m(v_0)$.
Mit $ v_0 = 0.5 c $ und $ \frac{\Delta v}{v_0}  = 5 \% = 0.05$ erhalten wir für die Massenänderung:
\begin{align*}
	\frac{\Delta m(v_0)}{m(v_0)}
	=
	\frac{\frac{v^2}{c^2}}{1 - \frac{v^2}{c^2}} 5 \%
	=
	\frac{\frac{(0.5 c)^2}{c^2}}{1 - \frac{(0.5 c)^2}{c^2}} \cdot 0.05
	=
	\frac{\left(\frac{1}{2}\right)^2}{1 - \left(\frac{1}{2}\right)^2}\cdot 0.05
	=
	\frac{\frac{1}{4}}{1- \frac{1}{4}} \cdot 0.05
	=
	\frac{1}{3} \cdot 0.05 \approx 1.67 \%.
\end{align*}
Damit ändert sich die Masse um $ 1.67 \% $, wenn sich die Geschwindigkeit $ v_0 =0.5 c $ um $ 5 \% $ erhöht.


\newpage
\subsection*{\aufgabe{d}{8}}
Bei einer Absatzmenge $ x \geq 20  $ kann ein Monopolist mit dem Ertrag $ E(x) $ und den Kosten $ K(x) $ rechnen, die wie folgt gegeben sind:
\begin{align*}
	E(x) = -0.5 x^{1.5} +100 x \ \textrm{und} \ K(x) = -200\sqrt{x} + 50 x +120.
\end{align*}
Von seinem Bruttogewinn muss der Hersteller $ i \% $  $ (0 < i < 100) $ an Steuern bezahlen.
\begin{enumerate}
	\item[(d1)]
	Drücken Sie den Gewinn $ G(x) $ nach Steuern in Abhängigkeit der verkauften Menge $ x $ aus.
	\item[(d2)] 
	Für welche Menge $ x^\star $ maximiert der Hersteller seinen Gewinn nach Steuern?
\end{enumerate}
\ \\
\textbf{Lösung:}
\begin{mdframed}
	\underline{\textbf{Vorgehensweise:}}
	\begin{enumerate}
		\item[(d1)] Stelle eine geeignete Funktion auf.
		\item[(d2)] Bestimme das Maximum der Funktion.
 	\end{enumerate}
\end{mdframed}


\underline{(d1) Stelle eine geeignete Funktion auf}\\
Da wir am Gewinn interessiert sind, ist es naheliegend die Kosten $ K(x) $ vom Ertrag $ E(x) $ abzuziehen. Dies ist der Gewinn vor Steuern.
Damit erhalten wir die Funktion
\begin{align*}
	P(x) := E(x) - K(x)
	=
	-0.5x^{1.5} + 100 x- (-200\sqrt{x} + 50 x +120)
	=
	-0.5x^{1.5} + 50 x + 200 \sqrt{x} -120.
\end{align*}
Der Gewinn nach Steuern ergibt sich dann aus
\begin{align*}
	G(x) = P(x) - \frac{i}{100} P(x)
	=
	\left(1- \frac{i}{100}\right)P(x)
	=
	\left(1- \frac{i}{100}\right) \cdot (-0.5x^{1.5} + 50 x + 200 \sqrt{x} -120),
\end{align*}
wobei $ i \% = \frac{i}{100}$ der Steuersatz mit $ 0 < i < 100 $ ist.\\
\\
\underline{(d2) Bestimme das Maximum der Funktion}\\
Für das Maximum von $ G $ bestimmen wir zunächst die Ableitung von $ G $:
\begin{align*}
	G^\prime(x)
	= \left(1 - \frac{i}{100}\right)
	\cdot \left(-0.5 \cdot 1.5 x^{0.5} + 50 + 200 \cdot \frac{1}{2} x^{- \frac{1}{2}} \right)
	=
	\left(1 - \frac{i}{100}\right)
	\cdot \left(-0.75 x^{0.5} + 50 + 100 x^{- \frac{1}{2}} \right).
\end{align*}
Für ein Maximum muss die notwendige Bedingung $ G^\prime(x) = 0 $ erfüllt sein.
Dies ist äquivalent zu:
\begin{align*}
	\left(1 - \frac{i}{100}\right)
	\cdot \left(-0.75 x^{0.5} + 50 + 100 x^{- \frac{1}{2}} \right) = 0
	\ &\Leftrightarrow \
	-0.75 x^{0.5} + 50 + 100 x^{- \frac{1}{2}}  = 0\\
	\ &\overset{\cdot x^{\frac{1}{2}}}{\Leftrightarrow}\
	-0.75 x + 50 x^{\frac{1}{2}} + 100   = 0.
\end{align*}
Mit der Substitution $ y = x^{\frac{1}{2}} $ erhalten wir die quadratische Gleichung
\begin{align*}
	-0.75 x + 50 x^{\frac{1}{2}} + 100 
	=
	-0.75 y^2 + 50 y + 100 = 0.
\end{align*}
Deren Lösungen sind gegeben durch:
\begin{align*}
	y_{\nicefrac{1}{2}}
	=
	\frac{-50 \pm \sqrt{50^2 - 4 \cdot (-0.75) \cdot 100}}{2 \cdot (-0.75)}
	= 
	\frac{-50 \pm \sqrt{2800}}{-1.5}.
\end{align*}
Wegen $ y_1 = \frac{-50 + \sqrt{2800}}{-1.5}  < 0 $ und $ y = x^{\frac{1}{2}} =  \sqrt{x} $ ist diese Lösung für unsere ursprüngliche Gleichung nicht relevant.
Damit bleibt der Kandidat 
\begin{align*}
	x^\star = y_2^2 = 
	\left(\frac{-50 - \sqrt{2800}}{-1.5}\right)^2
	\approx
	4707.33
\end{align*}
für eine Extremstelle von $ G $. Wir müssen überprüfen, ob diese Stelle ein Maximum ist.
Hierfür bestimmen wir die zweite Ableitung von $ G $:
\begin{align*}
	G^{\prime \prime}(x)
	\left(1 - \frac{i}{100}\right)\cdot 
	\left(
	-0.75 \cdot 0.5\cdot x^{- \frac{1}{2}} - \frac{1}{2} 100 x^{- \frac{3}{2}}
	\right).
\end{align*}
Hieraus ist ersichtlich, dass $ G^{\prime \prime}(x) < 0 $ ist.
Demnach muss $ x^\star $ ein Maximum sein.



