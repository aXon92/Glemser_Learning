\fancyhead[C]{\normalsize\textbf{$\qquad$ Teil II: Multiple-Choice}}
\section*{Aufgabe 2 (34 Punkte)}
\vspace{0.4cm}
\subsection*{\frage{1}{3}}
$ A $ und $ B $ seien zwei Aussagen. Welche der folgenden zwei zusammengesetzten Aussagen sind äquivalent? 
\renewcommand{\labelenumi}{(\alph{enumi})}
\begin{enumerate}
	\item $ A \Rightarrow  B $ und $A \vee B$.
	\item $ A \vee B $ und $\neg (A \vee B)$.
	\item $ (A \Rightarrow \neg B) $ und $\neg ( A \vee B )$
	\item Keine der obigen Antworten ist richtig.
\end{enumerate}\ \\
\textbf{Lösung:}
\begin{mdframed}
\underline{\textbf{Vorgehensweise:}}
\renewcommand{\labelenumi}{\theenumi.}
\begin{enumerate}
\item Löse die Aufgabe mithilfe einer Wahrheitstabelle.
\end{enumerate}
\end{mdframed}

\underline{1. Löse die Aufgabe mithilfe einer Wahrheitstabelle}\\
Eine Aussage kann nur die Wahrheitswerte wahr ($ W $) oder falsch ($ F $) annehmen.
Dementsprechend gibt es bei zwei Aussagen $ A $, $ B $ genau vier Kombinationsmöglichkeiten der Wahrheitswerte.
Aus diesen Kombinationen ergeben sich dann die Wahrheitswerte der verknüpften Aussagen.
\begin{center}
	\begin{tabular}{cllll}
		\hline
		\multicolumn{1}{c|}{$A$} & \multicolumn{4}{l}{$W$ $W$ $F$ $F$} \\
		\multicolumn{1}{c|}{$B$} & \multicolumn{4}{l}{$W$ $F$ $W$ $F$} \\ 
		\multicolumn{1}{c|}{$\neg B$} & \multicolumn{4}{l}{$F$ $W$ $F$ $W$} \\ 
		\hline
		\multicolumn{1}{c|}{$ A \vee B$} & \multicolumn{4}{l}{$W$ $W$ $W$ $F$} \\ 
		\multicolumn{1}{c|}{$ \neg(A \vee B)$} & \multicolumn{4}{l}{$F$ $F$ $F$ $W$} \\ 
		\hline
		\multicolumn{1}{c|}{$ A \Rightarrow B$} & \multicolumn{4}{l}{$W$ $F$ $W$ $W$} \\ 
		\multicolumn{1}{c|}{$ A \Rightarrow  \neg B$} & \multicolumn{4}{l}{$F$ $W$ $W$ $W$} \\ 
		\hline
	\end{tabular}
\end{center}
Zwei Aussagen sind genau dann äquivalent, wenn deren Wahrheitswerte übereinstimmen. Wir sehen an der Wahrheitstafel, dass dies für keine der Antwortmöglichkeiten (a) - (c) erfüllt ist.\\
\\
Damit ist die Antwort (d) korrekt.


\newpage

\subsection*{\frage{2}{4}}
Welche der folgenden zusammengesetzten Aussagen ist eine Tautologie? 
\renewcommand{\labelenumi}{(\alph{enumi})}
\begin{enumerate}
	\item $(A \Rightarrow B ) \wedge (B \Rightarrow A)$.
	\item $ ((A \wedge B) \Rightarrow B) \vee A$.
	\item $ ((A \vee B) \Rightarrow B) \vee B$
	\item $(A \Rightarrow B ) \vee \neg A \vee B$.
\end{enumerate}
\ \\
\textbf{Lösung:}
\begin{mdframed}
	\underline{\textbf{Vorgehensweise:}}
	\renewcommand{\labelenumi}{\theenumi.}
	\begin{enumerate}
		\item Löse die Aufgabe mithilfe geeigneter Gegenbeispiele.
		\item Löse die Aufgabe mithilfe einer Wahrheitstafel.
	\end{enumerate}
\end{mdframed}
\underline{1. Löse die Aufgabe mithilfe geeigneter Gegenbeispiele.}\\
Eine Aussage heißt Tautologie, wenn sie jede Wahrheitswertekombination wahr ist.
Für (a) haben wir mit der Kombination $A = W, \ B = F$ ein Gegenbeispiel. Für diese Belegung ist $A \Rightarrow B$ falsch und $  B \Rightarrow A$ wahr. Da diese Aussagen mit UND verknüpft sind, ist Aussage insgesamt falsch.
Für die Antwort (c) und (d) ist $A = W, \ B = F$ ein Gegenbeispiel:
\begin{align*}
	((A \vee B) \Rightarrow B) \vee B
	\ &\Leftrightarrow \
	\neg(A \vee B ) \vee B \vee B
	\ \Leftrightarrow \
	(\neg A \wedge \neg B) \vee B.
	\ \Leftrightarrow \
	\neg A \vee B \wedge \neg B \vee B\\
	\ &\Leftrightarrow \
	\neg A \vee B 
	\ \Leftrightarrow \
	A \Rightarrow B
	\ \Leftrightarrow \
	(A \Rightarrow B) \vee \neg A \vee B.
\end{align*}
Nebenbei haben wir hier gezeigt, dass die Antwortmöglichkeiten (c) und (d) äquivalent sind. Übrig bleibt die Antwortmöglichkeit (b). Durch die Umformungen
\begin{align*}
	((A \wedge B) \Rightarrow B) \vee A
	\ &\Leftrightarrow \
	(\neg (A \wedge B) \vee B) \vee A
	\ \Leftrightarrow \
	(\neg A \vee \neg B \vee B) \vee A\\
	\ &\Leftrightarrow \
	\neg A \vee A \vee \neg B \vee B
	\ \Leftrightarrow \
	W \vee W \ \Leftrightarrow W
\end{align*}
erkennen wir das bei Möglichkeit (b) eine Tautologie vorliegt.\\
\\
\underline{2. Löse die Aufgabe mithilfe einer Wahrheitstafel}\\
\begin{center}
	\begin{tabular}{cllll}
		\hline
		\multicolumn{1}{c|}{$A$} & \multicolumn{4}{l}{$W$ $W$ $F$ $F$} \\
		\multicolumn{1}{c|}{$B$} & \multicolumn{4}{l}{$W$ $F$ $W$ $F$} \\ 
		\multicolumn{1}{c|}{$\neg A$} & \multicolumn{4}{l}{$F$ $F$ $W$ $W$} \\ 
		\hline
		\multicolumn{1}{c|}{$A \wedge B$} & \multicolumn{4}{l}{$W$ $F$ $F$ $F$} \\
		\multicolumn{1}{c|}{$A \vee B$} & \multicolumn{4}{l}{$W$ $W$ $W$ $F$} \\
		\multicolumn{1}{c|}{$ \neg A \vee B$} & \multicolumn{4}{l}{$W$ $F$ $W$ $W$} \\
		\hline
		\multicolumn{1}{c|}{$ A \Rightarrow B$} & \multicolumn{4}{l}{$W$ $F$ $W$ $W$} \\ 
		\multicolumn{1}{c|}{$ B \Rightarrow A$} & \multicolumn{4}{l}{$W$ $W$ $F$ $W$} \\ 
		\multicolumn{1}{c|}{$ (A \wedge B) \Rightarrow B$} & \multicolumn{4}{l}{$W$ $W$ $W$ $W$} \\ 
		\multicolumn{1}{c|}{$ (A \vee B) \Rightarrow B$} & \multicolumn{4}{l}{$W$ $F$ $W$ $W$} \\ 
		\hline		
		\multicolumn{1}{c|}{(a) $ (A \Rightarrow B) \wedge (B \Rightarrow A)$} & \multicolumn{4}{l}{$W$ $F$ $F$ $W$} \\
		\multicolumn{1}{c|}{(b) $ ((A \wedge B) \Rightarrow B) \vee A$} & \multicolumn{4}{l}{$W$ $W$ $W$ $W$} \\
		\multicolumn{1}{c|}{(c) $((A \vee B) \Rightarrow B) \vee B$} & \multicolumn{4}{l}{$W$ $F$ $W$ $W$} \\
		\multicolumn{1}{c|}{(d) $(A \Rightarrow B ) \vee \neg A \vee B$} & \multicolumn{4}{l}{$W$ $F$ $W$ $W$} \\
		\hline
	\end{tabular}
\end{center}
Die Antwort (b) ist immer wahr und somit korrekt.
 

\newpage
\subsection*{\frage{3}{3}}
Seien $\{a_k\}_{k \in \mathbb{N}}$ und $\{b_k\}_{k \in \mathbb{N}}$ zwei geometrische Folgen mit $a_1 = 1$, $a_2= 1.01$ und $b_1 = 3$, $b_2 = -2$.\\
\\
Welche der folgenden Aussagen ist korrekt? 
\renewcommand{\labelenumi}{(\alph{enumi})}
\begin{enumerate}
	\item 
	$\{a_k\}_{k \in \mathbb{N}}$ und $\{b_k\}_{k \in \mathbb{N}}$ konvergieren beide.
	\item 
	Nur $\{a_k\}_{k \in \mathbb{N}}$ konvergiert.
	\item 
	Nur $\{b_k\}_{k \in \mathbb{N}}$ konvergiert.
	\item
	Weder $\{a_k\}_{k \in \mathbb{N}}$ noch $\{b_k\}_{k \in \mathbb{N}}$ konvergieren.
\end{enumerate}
\ \\
\textbf{Lösung:}
\begin{mdframed}
\underline{\textbf{Vorgehensweise:}}
\renewcommand{\labelenumi}{\theenumi.}
\begin{enumerate}
\item Betrachte den Quotienten der benachbarten Folgenglieder.
\end{enumerate}
\end{mdframed}

\underline{1. Betrachte den Quotienten der benachbarten Folgenglieder}\\
Bei einer geometrischen Folge $a_k$ ist der Quotient zweier benachbarten Folgenglieder konstant, d.h. es gilt
\begin{align*}
	\frac{a_{k+1}}{a_k} = q \neq 0
\end{align*}
für alle $k \in \N$. Deswegen genügt die Angabe der ersten beiden Folgenglieder.
Für die Quotienten der Aufgabenstellung gilt:
\begin{align*}
	q^a &= \frac{a_2}{a_1} = \frac{1.01}{1} = 1.01\\
	q^b &= \frac{b_2}{b_1} = \frac{-2}{3} = - \frac{2}{3}.
\end{align*}
Wegen $\left|q^a\right| > 1$ ist die Folge $\{a_k\}_{k \in \N}$ divergent. Weiter ist $\{b_k\}_{k \in \N}$ wegen $\left|q^b\right| < 1$ konvergent.\\
\\
Damit ist (c) korrekt.


\newpage

\subsection*{\frage{4}{4}}
Seien $\{a_k\}_{k \in \mathbb{N}}$ und $\{b_k\}_{k \in \mathbb{N}}$ zwei geometrische Folgen mit $a_1 = 1$, $a_2= 1.1$ und $b_1 = 3$, $b_2 = 3.1$.
Seien $\{s_n^a \}_{n \in \mathbb{N}}$ und $\{s_n^b \}_{n \in \mathbb{N}}$ die entsprechenden Reihen.\\
\\
Welche der folgenden Aussagen ist korrekt? 
\renewcommand{\labelenumi}{(\alph{enumi})}
\begin{enumerate}
	\item 
	$\lim_{n \to \infty} \frac{s_n^a}{s_n^b} = 3$.
	\item
	$\lim_{n \to \infty} \frac{s_n^a}{s_n^b} = \frac{1}{3}$.
	\item
	$\lim_{n \to \infty} \frac{s_n^a}{s_n^b} = \infty$.
	\item
	$\lim_{n \to \infty} \frac{s_n^a}{s_n^b} = 0$.
\end{enumerate}
\ \\
\textbf{Lösung:}
\begin{mdframed}
\underline{\textbf{Vorgehensweise:}}
\renewcommand{\labelenumi}{\theenumi.}
\begin{enumerate}
\item Bestimme die Partialsummenfolgen.
\item Bestimme den Quotienten und dessen Grenzwert.
\end{enumerate}
\end{mdframed}

\underline{1. Bestimme die Partialsummenfolgen}\\
Die Quotienten der Folgen sind gegeben durch:
\begin{align*}
	q^a &= \frac{a_2}{a_1} = \frac{1.1}{1} = 1.1\\
	q^b &= \frac{b_2}{b_1} = \frac{3.1}{3} = 1.0\overline{3}.
\end{align*}
Die Partialsummenfolgen sind gegeben durch:
\begin{align*}
	s_n^a
	&=
	\sum \limits_{k=1}^n 1 \cdot 1.1^{k-1}
	=
	1 \cdot 
	\frac{1 -1.1^n}{1 - 1.1}
	= 
	\frac{1 - 1.1^n}{-0.1}
	=
	10 \cdot( 1.1^n - 1)\\
	s_n^b
	&=
	\sum \limits_{k=1}^n 3 \cdot 1.0\overline{3}^{k-1}
	=
	3 \cdot 
	\frac{1 -1.0\overline{3}^n}{1 - 1.0\overline{3}}
	= 
		3 \cdot 
	\frac{1 - 1.0\overline{3}^n}{1 - \frac{31}{30}}
	=
	3 \cdot 
	\frac{1 - 1.0\overline{3}^n}{- \frac{1}{30}}
	=
	90 (1.0\overline{3}^n -1).
\end{align*}
\ \\
\underline{2. Bestimme den Quotienten und dessen Grenzwert}\\
Für den Quotienten gilt:
\begin{align*}
	\frac{s_n^a}{s_n^b}
	=
	\frac{10 \cdot( 1.1^n - 1)}{90 (1.0\overline{3}^n -1)}
	=
	\frac{1}{9}
		\frac{
			1.0\overline{3}^n 
			\left(
			\left(
				\frac{1.1}{1.0\overline{3}}
				\right)^n 
				- \frac{1}{1.0\overline{3}^n}
			\right)
		}
		{
			1.0\overline{3}^n 
				\left(
					1 - 	\frac{1}{1.0\overline{3}^n}
				\right)		
		}
	=
	\frac{1}{9}
	\frac{	
		\left(
		\frac{1.1}{1.0\overline{3}}
		\right)^n 
		- \frac{1}{1.0\overline{3}^n}
	}
	{
		1 - 	\frac{1}{1.0\overline{3}^n}	
	}.
\end{align*}
Mit den Grenzwerten
\begin{align*}
	&\lim 
	\limits_{n \to \infty}
	\left(\frac{1.1}{1.0\overline{3}}\right)^n = \infty\\
	&\lim 
	\limits_{n \to \infty}
	\frac{1}{1.0\overline{3}^n} = 0
\end{align*}
erhalten wir:
\begin{align*}
	\lim 
	\limits_{n \to \infty}
	\frac{s_n^a}{s_n^b} = \infty.
\end{align*}
Damit ist die Antwort (c) korrekt.

\newpage
\subsection*{\frage{5}{3}}
Ein Projekt erfordert eine Anfangsinvestition von $10'000$ Schweizer Franken und generiert Erträge in Höhe von $500$ Schweizer Franken am Ende jedes Jahres für $10$ Jahre, sowie zusätzliche $12'000$ Schweizer Franken am Ende des zehnten Jahres.\\
\\
Der interne Zinssatz des Projekts ist:
\renewcommand{\labelenumi}{(\alph{enumi})}
\begin{enumerate}
	\item 
	Echt grösser als $5 \%$.
	\item 
	Gleich $5 \%$.
	\item
	Echt kleiner als $5 \%$.
	\item
	Die Frage kann nicht beantwortet werden, ohne zu wissen, welcher Zinssatz $i$ gilt.
\end{enumerate}
\ \\
\textbf{Lösung:}
\begin{mdframed}
\underline{\textbf{Vorgehensweise:}}
\renewcommand{\labelenumi}{\theenumi.}
\begin{enumerate}
\item 
\end{enumerate}
\end{mdframed}

\underline{1. }\\



 \newpage

\subsection*{\frage{6}{3}}
Seien $h$, $f$ und $g$ differenzierbare Funktionen einer reellen Variable. Welche der folgenden Formeln ist korrekt?
\renewcommand{\labelenumi}{(\alph{enumi})}
\begin{enumerate}
	\item 
	$ \frac{d}{dx}h(f(x) + g(x)) = h^\prime(x) ( f^\prime(x) + g^\prime(x))$.
	\item 
	$ \frac{d}{dx}h(f(x) + g(x)) = h^\prime(x) ( f(x) + g(x))^\prime$.
	\item
	$ \frac{d}{dx}h(f(x) + g(x)) = h^\prime(f(x))  f^\prime(x) + h^\prime(g(x)) g^\prime(x)$.
	\item
	$ \frac{d}{dx}h(f(x) + g(x)) = h^\prime(f(x) + g(x))  (f^\prime(x) +  g^\prime(x) )$.
	\item
	$ \frac{d}{dx}h(f(x) + g(x)) = h^\prime(f(x) + g(x))  f^\prime(x)  g^\prime(x)$.
\end{enumerate}
\ \\
\textbf{Lösung:}
\begin{mdframed}
\underline{\textbf{Vorgehensweise:}}
\renewcommand{\labelenumi}{\theenumi.}
\begin{enumerate}
\item Verwende die Summen und Kettenregel.
\end{enumerate}
\end{mdframed}

\underline{1. Verwende die Summen und Kettenregel}\\
Wir definieren uns die Hilfsfunktion $k$ durch
\begin{align*}
	k(x) = f(x) + g(x).
\end{align*}
Für diese gilt mit der Summenregel (Linearität der Ableitung):
\begin{align*}
	k^\prime(x) = f^\prime(x) + g^\prime(x).
\end{align*}
Damit erhalten wir mit der Kettenregel:
\begin{align*}
	\frac{d}{dx}h(f(x) + g(x))
	=
	\frac{d}{dx}h(k(x))
	=
	h^\prime(k(x)) k^\prime(x)
	=
	h^\prime(f(x) + g(x)) (f^\prime(x) + g^\prime(x)).
\end{align*}
Also ist die Antwort (d) korrekt.


\newpage
\subsection*{\frage{7}{4}}
Welche der folgenden Aussagen ist korrekt für eine differenzierbare Funktion $f$ auf einem Intervall $[a,b]$ und $x_0 \in (a,b)$?
\renewcommand{\labelenumi}{(\alph{enumi})}
\begin{enumerate}
	\item 
	$x_0$ ist genau dann ein stationärer Punkt, wenn er ein Extrempunkt ist.
	\item
	Wenn $x_0$ ein Wendepunkt ist, dann ist er auch ein stationärer Punkt.
	\item
	Wenn $x_0$ der einzige lokale Extrempunkt in $(a,b)$ ist, dann ist er auch ein globaler Extrempunkt in $[a,b]$.
	\item
	Wenn $f^\prime(x_0) = 0$, dann ist $x_0$ ein Extrempunkt.
	\item
	Keine der obigen Antworten ist richtig.
\end{enumerate}
\ \\
\textbf{Lösung:}
\begin{mdframed}
\underline{\textbf{Vorgehensweise:}}
\renewcommand{\labelenumi}{\theenumi.}
\begin{enumerate}
\item 
\end{enumerate}
\end{mdframed}

\underline{1. }\\




\newpage

\subsection*{\frage{8}{3}}
Seien $f$ und $g$ konkave, zweimal differenzierbare Funktionen. Welche der folgenden Aussagen ist korrekt?
\renewcommand{\labelenumi}{(\alph{enumi})}
\begin{enumerate}
	\item 
	$f \ g$ ist konkav.
	\item
	$f \ g$ ist konvex.
	\item
	$f \circ g $ ist konkav.
	\item
	$f +g $ ist konvex.
	\item 
	$f+g$ ist konkav.
\end{enumerate}
\ \\
\textbf{Lösung:}
\begin{mdframed}
\underline{\textbf{Vorgehensweise:}}
\renewcommand{\labelenumi}{\theenumi.}
\begin{enumerate}
\item Verwende Konkavitätskriterium der zweiten Ableitung.
\end{enumerate}
\end{mdframed}

\underline{1. Verwende Konkavitätskriterium der zweiten Ableitung}\\
Eine Funktion $f$ heißt genau dann konkav, falls 
\begin{align*}
	f^{\prime \prime} (x) \leq 0 
\end{align*}
für alle $x \in D_f$ gilt. \\
\\
Für die korrekte Antwort sind die Möglichkeiten mit der Summe die vielversprechendsten Kandidaten. Für die anderen Antwortmöglichkeiten muss man zweimal die Produktregel oder Kettenregel anwenden.\\
\\
Unter den Voraussetzungen der Aufgabe gilt mit der Summenregel
\begin{align*}
	(f +g)^{\prime \prime}(x)  = 
	\underbrace{f^{\prime \prime}(x)}_{\leq 0} + \underbrace{g^{\prime \prime}(x)}_{\leq 0}
	\leq 0
\end{align*}
für $x \in D_f \cap D_g$.
Damit ist die Funktion $f +g$ wieder konkav und die Antwort (e) korrekt.
 



\newpage
\subsection*{\frage{9}{3}}
Sei $f$ eine differenzierbare Funktion. Welcher der folgenden Ausdrücke approximiert 
$x_0 \frac{f(x_0 + \Delta x) - f(x_0)}{f(x_0) \Delta x}$, wenn $\Delta x$ klein ist?
\renewcommand{\labelenumi}{(\alph{enumi})}
\begin{enumerate}
	\item 
	Die Ableitung $f^\prime(x_0)$.
	\item
	Die Änderungsrate $\rho_f(x_0)$.	
	\item
	Das Differential $df$.
	\item
	Die Elastizität $\varepsilon_f(x_0)$.
\end{enumerate}
\ \\
\textbf{Lösung:}
\begin{mdframed}
	\underline{\textbf{Vorgehensweise:}}
	\renewcommand{\labelenumi}{\theenumi.}
	\begin{enumerate}
		\item Verwende den Differenzenquotienten.
	\end{enumerate}
\end{mdframed}

\underline{1. Verwende den Differenzenquotienten}\\
Der Differenzenquotient approximiert für $\Delta x $ nahe $0$ die erste Ableitung in $x_0$:
\begin{align*}
	f^\prime(x_0) \approx \frac{f(x_0 + \Delta x) - f(x_0)}{\Delta x}.
\end{align*}
Daher gilt:
\begin{align*}
	x_0 \frac{f(x_0 + \Delta x) - f(x_0)}{f(x_0) \Delta x}
	= 
	x_0 \frac{\frac{f(x_0 + \Delta x) - f(x_0)}{\Delta x}}{f(x_0) }
	\approx 
	x_0 \frac{f^\prime(x_0)}{f(x_0)}
	=: \varepsilon_f(x_0).
\end{align*}
Damit wird durch diesen Ausdruck die Elastizität von $f$ in $x_0$ approximiert.\\
\\
Die Antwort (d) ist korrekt.


\newpage
\subsection*{\frage{10}{4}}
Sei $f$ eine Funktion und $x_0 \in D_f$ so, dass $f(x_0) = 1$ und $f^{(k)}(x_0) = 1$ für $k = 1,2,3$ gilt. Seien $P_2$ und $P_3$ jeweils das Taylor-Polynom zweiter und dritter Ordnung von $f$ in $x_0$. Es folgt, dass:
\renewcommand{\labelenumi}{(\alph{enumi})}
\begin{enumerate}
	\item 
	$P_3(x) - P_2(x) = 0 $ für alle $x$.
	\item
	$P_3(x) - P_2(x) = (x - x_0)^2 $ für alle $x$.
	\item
	$P_3(x) - P_2(x) = \frac{1}{6} (x - x_0)^3 $ für alle $x$.
	\item
	Es ist nicht möglich die Frage zu beantworten, ohne die Funktionsvorschrift von $f$ zu kennen.
\end{enumerate}
\ \\
\textbf{Lösung:}
\begin{mdframed}
	\underline{\textbf{Vorgehensweise:}}
	\renewcommand{\labelenumi}{\theenumi.}
	\begin{enumerate}
		\item Bestimme das Taylorpolynom zweiter und dritter Ordnung.
	\end{enumerate}
\end{mdframed}

\underline{1. Bestimme das Taylorpolynom zweiter und dritter Ordnung}\\
Das Taylorpolynom $n$-ter Ordnung in $x_0 \in D_f$ ist gegeben durch:
\begin{align*}
	P_n(x)
	=
	\sum \limits_{k=0}^n \frac{f^{(k)}(x_0)}{k!} (x-x_0)^k.
\end{align*}
Mit den bekannten Werten an $x_0$ erhalten wir dann:
\begin{align*}
	P_2(x)
	&=
	f(x_0) + \frac{f^\prime(x_0)}{1!} (x- x_0) + \frac{f^{\prime \prime}(x_0)}{2!} (x- x_0)^2\\
	&=
	f(x_0) +  (x- x_0) + \frac{1}{2} (x- x_0)^2\\
	P_3(x)
	&=
	f(x_0) + \frac{f^\prime(x_0)}{1!} (x- x_0) + \frac{f^{\prime \prime}(x_0)}{2!} (x- x_0)^2
	+ \frac{f^{(3)}(x_0)}{3!} (x - x_0)^3\\
	&=
	f(x_0) +  (x- x_0) + \frac{1}{2} (x- x_0)^2 + \frac{1}{6} (x - x_0 )^3.
\end{align*}
Wir sehen das sich die Taylorpolynome zweiter und dritter Ordnung um einen Summanden unterscheiden. Somit bleibt $\frac{1}{6} (x - x_0)^3$ bei $P_3(x) - P_2(x)$.\\
\\
Die Antwort (c) ist korrekt.\\
\\
Die Aussage gilt auch unabhängig davon welche Werte die Funktion $f$ annimmt:
\begin{align*}
	P_{n+1}(x) - P_n(x) = 
	\frac{f^{(n)}(x_0)}{n!}(x - x_0)^n
	\ \Leftrightarrow \
	P_{n+1}(x) =  P_n(x) + \frac{f^{(n)}(x_0)}{n!}(x - x_0)^n. 
\end{align*} 

