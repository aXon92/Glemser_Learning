\section*{Aufgabe 3 (28 Punkte)}
\vspace{0.4cm}
\subsection*{\frage{1}{3}}
Welche der folgenden Folgen konvergiert gegen $1$?
\renewcommand{\labelenumi}{(\alph{enumi})}
\begin{enumerate}
	\item 
	$ \{a_n\}_{n \in \mathbb{N}} $ mit $a_n = \frac{\sin(n) - \cos(n)}{n}$.
	\item
	$ \{b_n\}_{n \in \mathbb{N}} $ mit $b_n = \frac{n^2}{2n^2 - 3n +1 }$.
	\item
	$ \{c_n\}_{n \in \mathbb{N}} $ mit $c_n = \sqrt{n^2 + 1 }  - n$.
	\item
	$ \{d_n\}_{n \in \mathbb{N}} $ mit $d_n = \frac{n^2}{e^n}$.
	\item
	$ \{e_n\}_{n \in \mathbb{N}} $ mit $e_n = \frac{(n+2)(n^2 + 3n -1 )}{n^3 + 4 }$.
	\item
	$ \{f_n\}_{n \in \mathbb{N}} $ mit $f_n = \frac{\ln(n)}{n}$.
\end{enumerate}
\ \\
\textbf{Lösung:}
\begin{mdframed}
\underline{\textbf{Vorgehensweise:}}
\renewcommand{\labelenumi}{\theenumi.}
\begin{enumerate}
\item Bestimme die Grenzwerte.
\end{enumerate}
\end{mdframed}

\underline{1. Bestimme die Grenzwerte}\\
Für die Folge in (a) gilt:
\begin{align*}
	|a_n| =
	\left|
	\frac{\sin(n) - \cos(n)}{n}
	\right|
	\leq
	\left|
	\frac{\sin(n)}{n}
	\right|
	+
	\left|
	\frac{\cos(n)}{n}
	\right|
	\leq 
	\frac{1}{n} + \frac{1}{n} \to 0 \ \textrm{für } n \to \infty.
\end{align*}
Hierbei haben wir die Dreiecksungleichung angewendet.
Für (b) erhalten wir:
\begin{align*}
	b_n = \frac{n^2}{2n^2 - 3n +1 }
	= \frac{n^2}{n^2 \cdot \left( 2+ -\frac{3}{n} + \frac{1}{n^2}\right)}
	=
	\frac{1}{   2+ -\frac{3}{n} + \frac{1}{n^2}}
	\to \frac{1}{2} \ \textrm{für } n \to \infty.
\end{align*}
Mit der dritten Binomischen Formel gilt für die (c)
\begin{align*}
	c_n = \sqrt{n^2 +1} - n
	= \frac{\sqrt{n^2 +1}^2  - n^2}{\sqrt{n^2 +1 } + n}
	= \frac{n^2 +1  - n^2}{\sqrt{n^2 +1 } + n}
	=
	\frac{1}{\sqrt{n^2 +1 } + n} \to 0 \textit{ für } n \to \infty.
\end{align*}
Für (d) und (f) gilt:
\begin{align*}
	&d_n = \frac{n^2}{ e^n} \to 0  \ \textrm{für } n \to \infty\\
	&f_n = \frac{\ln(n)}{n} \to 0 \ \textrm{für } n \to \infty.
\end{align*}
Übrig bleibt (e), wofür gilt:
\begin{align*}
	e_n
	&=
	\frac{(n+2) (n^2 + 3n -1)}{n^3 +4 }
	= \frac{n^3 +3 n^2 - n + 2 n^2 +6n -2}{n^3 + 4}
	= \frac{n^3 +6 n^2 +5 n -2}{n^3 + 4}\\
	&= \frac{n^3 \left( 1 +\frac{6}{n} +\frac{5 }{n^2} -\frac{2}{n^3} \right)}{n^3 \left( 1 + \frac{4}{n^3} \right)}
	=
	\frac{ 1 +\frac{6}{n} +\frac{5 }{n^2} -\frac{2}{n^3}}{  1 + \frac{4}{n^3}}
	\to 1 \ \textrm{für } n \to \infty.
\end{align*}
Damit ist die Antwort (e) korrekt.
\newpage

\subsection*{\frage{2}{4}}
Welche der folgenden Konditionen ist am besten?
\renewcommand{\labelenumi}{(\alph{enumi})}
\begin{enumerate}
	\item 
	Jährlicher Zinssatz von $5 \%$ mit jährlicher Verzinsung.
	\item
	Jährlicher Zinssatz von $4.95 \%$ mit halbjährlicher Verzinsung.
	\item
	Jährlicher Zinssatz von $4.9 \%$ mit vierteljährlicher Verzinsung.
	\item
	Jährlicher Zinssatz von $4.85 \%$ mit monatlicher Verzinsung.
	\item
	Jährlicher Zinssatz von $4.8 \%$ mit stetiger Verzinsung.	
\end{enumerate}
\ \\
\textbf{Lösung:}
\begin{mdframed}
\underline{\textbf{Vorgehensweise:}}
\renewcommand{\labelenumi}{\theenumi.}
\begin{enumerate}
\item Bestimme jeweils den effektiven Jahreszins.
\end{enumerate}
\end{mdframed}

\underline{1. Bestimme jeweils den effektiven Jahreszins}\\
Es gilt für einen beliebigen Betrag $A  > 0$:
\begin{align*}
	A \cdot 
	\left(
	1 + 5 \%
	\right)
	\ \Rightarrow \ &i_{\mathrm{eff,a}} = 5  \% \\
	A \cdot 
	\left(
	1 + \frac{4.95 \%}{2}
	\right)^2
	\approx 1.0501 \cdot A
	\ \Rightarrow \ &i_{\mathrm{eff,b}} = 5.01  \% \\
	A \cdot 
	\left(
	1 + \frac{4.9 \% }{4} 
	\right)^4
	\approx 1.0499 \cdot A
	\ \Rightarrow \ &i_{\mathrm{eff,c}} = 4.99  \% \\
	A \cdot 
	\left(
	1 + \frac{4.85 \%}{12} 
	\right)^12
	\approx 1.0496 \cdot A
	\ \Rightarrow \ &i_{\mathrm{eff,d}} = 4.96  \% \\
	A \cdot 
	e^{4.8 \% }
	\approx 1.0492 \cdot A
	\ \Rightarrow \ &i_{\mathrm{eff,e}} = 4.92  \% 
\end{align*}
Die Konditionen der Antwortmöglichkeit (b) hat den höchsten effektiven Jahreszins.\\
\\
Damit ist Antwort (b) korrekt.


\newpage
\subsection*{\frage{3}{4}}
Gegeben ist die Funktion
\begin{align*}
	f(x) = \sin(\ln(x^2 + x + 1)).
\end{align*}
Die Ableitung von $f$ in $x_0 = 0 $ ist:
\renewcommand{\labelenumi}{(\alph{enumi})}
\begin{enumerate}
	\item 
	$ 0$.
	\item
	$ 1 $.
	\item
	$ 2 $.
	\item
	$ \pi $.
	\item
	$f$ ist nicht differenzierbar in $x_0 = 0$.
\end{enumerate}
\ \\
\textbf{Lösung:}
\begin{mdframed}
\underline{\textbf{Vorgehensweise:}}
\renewcommand{\labelenumi}{\theenumi.}
\begin{enumerate}
\item Verwende die Kettenregel.
\end{enumerate}
\end{mdframed}
%\allowdisplaybreaks
\underline{1. Verwende die Kettenregel.}\\
Die Ableitung von $f$ ist mit der Kettenregel gegeben durch:
\begin{align*}
	f^\prime(x) = (2x +1) \cdot \frac{1}{x^2 +x +1 } \cdot  \cos (\ln(x^2 + x +1)).
\end{align*}
Damit folgt:
\begin{align*}
	f^\prime(0)  =(2\cdot 0 +1 ) \cdot \frac{1}{0^2 + 0 +1} \cdot \cos(\ln(1))
	= 1 \cdot 1 \cdot \cos(0) = 1.
\end{align*}

\newpage
\subsection*{\frage{4}{4}}
Sei $P_3$ das Taylor-Polynom dritter Ordnung von $f(x) = \ln(1+ x) $ in $x_0 = 0$.\\
\\
Es gilt, dass:
\renewcommand{\labelenumi}{(\alph{enumi})}
\begin{enumerate}
	\item 
	$P_3(1) = 1 $.
	\item 
	$P_3(1) = \frac{1 }{2}$.
	\item
	$P_3(1) =  0$.
	\item
	$P_3(1) = \frac{1 }{6}$.
	\item
	$P_3(1) = \frac{5 }{6}$.
\end{enumerate}
\ \\
\textbf{Lösung:}
\begin{mdframed}
\underline{\textbf{Vorgehensweise:}}
\renewcommand{\labelenumi}{\theenumi.}
\begin{enumerate}
\item Gebe das Taylorpolynom dritter Ordnung allgemein an.
\item Bestimme die entsprechenden Ableitungen und werte diese aus.
\end{enumerate}
\end{mdframed}

\underline{1. Gebe das Taylorpolynom dritter Ordnung allgemein an}\\
Das Taylorpolynom dritter Ordnung an der Entwicklungstelle $x_0 = 0$ ist gegeben durch:
\begin{align*}
	P_3(x)
	=
	f(x_0)
	+
	f^{\prime}(x_0)x
	+
	\frac{f^{\prime \prime}(x_0)}{2} x^2
	+
	\frac{f^{\prime \prime \prime } (x_0) }{6} x^3.	
\end{align*}
\ \\
\underline{2.Bestimme die entsprechenden Ableitungen und werte diese aus}\\
Die ersten drei Ableitungen sind mithilfe der Kettenregel gegeben durch:
\begin{align*}
	f^\prime(x) 
	&= 
	\frac{1}{1+x}\\
	f^{\prime \prime }(x)
	&= 
	-\frac{1}{(1+x)^2}
	\\
	f^{\prime \prime \prime} (x)
	&=
	-(-2)\frac{1}{(1+x)^3}
	=
	\frac{2}{(1+x)^3}
	.
\end{align*}
Damit folgt:
\begin{align*}
	f(0) = \ln(1) = 0, \quad
	f^\prime(0) = 1, \quad
	f^{\prime \prime}(0) = -1, \quad
	f^{\prime \prime \prime} ( 0) = 2.
\end{align*}
Das Taylorpolynom ist gegeben durch
\begin{align*}
	P_3(x)
	=
	x - 
	\frac{1}{2} x^2
	+
	\frac{2}{6} x^3
	=
	x -\frac{1}{2} x^2 + \frac{1}{3} x^3.
\end{align*}
Damit erhalten wir:
\begin{align*}
	P_3(1) = 1 - \frac{1}{2} + \frac{1}{3}
	= 1 - \frac{3 }{6} + \frac{2}{6}
	= 1 - \frac{1}{6} = \frac{5}{6}
\end{align*}
Somit ist die Antwort (e) korrekt.


\newpage

\subsection*{\frage{5}{3}}
Sei $f$ die Funktion zweier reeller Variablen definiert durch $f(x,y) = \cos(x^2 +y^2 + 2xy).$\\
\\
Die partiellen Ableitungen von $f$ in $(x_0, y_0) = \left(\sqrt{\frac{\pi}{2}}, \sqrt{\frac{\pi}{2}}\right)$ sind:
\renewcommand{\labelenumi}{(\alph{enumi})}
\begin{enumerate}
	\item 
	$f_x(x_0,y_0) = 0$ und $f_y(x_0,y_0) = 0$.
	\item
	$f_x(x_0,y_0) = 4 \sqrt{\frac{\pi}{2}} $ und $f_y(x_0,y_0) = 0$.
	\item
	$f_x(x_0,y_0) = 0 $ und $f_y(x_0,y_0) = 4 \sqrt{\frac{\pi}{2}}$.
	\item
	$f_x(x_0,y_0) = 4 \sqrt{\frac{\pi}{2}} $ und $f_y(x_0,y_0) = 4 \sqrt{\frac{\pi}{2}}$.
\end{enumerate}
\ \\
\textbf{Lösung:}
\begin{mdframed}
\underline{\textbf{Vorgehensweise:}}
\renewcommand{\labelenumi}{\theenumi.}
\begin{enumerate}
\item Bestimme die partiellen Ableitungen.
\end{enumerate}
\end{mdframed}

\underline{1. Bestimme die partiellen Ableitungen}\\
Die partiellen Ableitungen sind gegeben durch:
\begin{align*}
	f_x(x,y)
	&=
	-(2x + 2y) \sin(x^2 +y^2 +2xy) 
	\\
	f_y(x,y)
	&=
	-(2y + 2x) \sin(x^2 +y^2 +2xy).
\end{align*}
Damit erhalten wir für $(x_0, y_0) = \left(\sqrt{\frac{\pi}{2}}, \sqrt{\frac{\pi}{2}}\right)$ zunächst
\begin{align*}
	\sin(x_0^2 + y_0^2 + 2 x_0 y_0)
	=
	\sin\left(\frac{\pi}{2} + \frac{\pi}{2} +2 \sqrt{\frac{\pi}{2}} \sqrt{\frac{\pi}{2}} \right)
	=
	\sin\left(\pi +2 \frac{\pi}{2} \right)
	= 
	\sin(2 \pi ) = 0
\end{align*}
und es folgt
\begin{align*}
	f_x(x_0,y_0) = f_y(x_0,y_0) = 0.
\end{align*}
Somit ist die Antwort (a) korrekt.


\newpage

\subsection*{\frage{6}{4}}
Wir betrachten die Gleichung
\begin{align*}
	\ln(x) + x y^2 + \frac{2x}{x + 2y} - 2 = 0,
\end{align*}
die für $(x_0,y_0) = (1,0)$ erfüllt ist.\\
\\
Die marginale Änderung in $y$, wenn $x$ ausgehend von $x_0$ marginal geändert wird und die Gleichung weiterhin erfüllt ist, ist:
\renewcommand{\labelenumi}{(\alph{enumi})}
\begin{enumerate}
	\item 
	$ 0 $.
	\item
	$ \frac{1}{4}$.
	\item
	$ -\frac{1}{4}$.
	\item
	$ 1$.
	\item
	Keine der obigen Antworten ist korrekt.
\end{enumerate}
\ \\
\textbf{Lösung:}
\begin{mdframed}
\underline{\textbf{Vorgehensweise:}}
\renewcommand{\labelenumi}{\theenumi.}
\begin{enumerate}
\item Verwende den Satz der impliziten Funktion.
\end{enumerate}
\end{mdframed}

\underline{1. Verwende den Satz der impliziten Funktion}\\
Sei 
\begin{align*}
	\varphi(x,y) 
	=
	\ln(x) + xy^2
	+
	\frac{2}{x+2y}
	-
	2
\end{align*}
die rechte Seite der Gleichung.
Mit dem Satz der implizten Funktion ergibt sich die marginale Änderung in $y$ bei einer marignalen Änderung in $x$ folgendermaßen:
\begin{align*}
	\frac{dy}{dx} \bigg|_{(x_0,y_0) = (1,0)}
	&=
	- 
	\frac{\varphi_x(x_0,y_0)}{\varphi_y(x_0,y_0)}
	=
	-
	\frac{\frac{1}{x_0} + y_0^2 + \frac{2(x_0 + 2 y_0) - 2x_0}{(x_0 + 2y_0)^2}}{2 x_0 y_0 + \frac{-2 x_0 \cdot 2}{(x_0 + 2y_0)^2}}\\
	&=
	- 
	\frac{1 + 0 + 0}{0-4}
	= \frac{1}{4}.
\end{align*}
Somit ist die Antwort (b) korrekt.


\newpage

\subsection*{\frage{7}{2}}
Gegeben ist die Funktion
\begin{align*}
	f(x,y) 
	=
	2 x^3 e^{\frac{ 2x^2+y^2}{x^2 - 3y^2 }}
	-
	xy^2 e^{\frac{x+y}{x-y}}
	+
	x y^3 \ln \left( \frac{x+y }{2y} \right)
	\quad \textrm{für } x>0,y>0.
\end{align*}
Welche der folgenden Aussagen ist korrekt?
\renewcommand{\labelenumi}{(\alph{enumi})}
\begin{enumerate}
	\item
	$ f  $ ist homogen vom Grad $ 0 $.
	\item
	$ f  $ ist homogen vom Grad $ 2 $.
	\item
	$ f  $ ist homogen vom Grad $ 2 $..	
	\item 
	$ f  $ ist homogen vom Grad $ 4 $.
	\item
	$ f $ ist nicht homogen.
\end{enumerate}
\ \\
\textbf{Lösung:}
\begin{mdframed}
\underline{\textbf{Vorgehensweise:}}
\renewcommand{\labelenumi}{\theenumi.}
\begin{enumerate}
\item Überlege, ob $ f $ homogen sein kann.
\end{enumerate}
\end{mdframed}

\underline{1. Überlege, ob $ f $ homogen sein kann}\\
Wir betrachten die einzelnen Summanden von $ f $ und untersuchen diese auf Homogenität:
\begin{align*}
	f_1(x,y)
	&=
	2x^3 e^{\frac{2x^2+y^2}{x^2 - 3y^2}}\\
	f_2(x,y)
	&=
	-
	x y^2  e^{\frac{x+y}{x-y}}\\
	f_3(x,y)
	&=
	x y^3\ln \left( \frac{x+y}{2y} \right).
\end{align*} 
Wir untersuchen nun jeweils die Homogenität.
Sei $ \lambda \in \R $ beliebig. Dann gilt:
\begin{align*}
	f_1(\lambda x, \lambda y)
	&=
	2(\lambda x)^3 e^{\frac{2(\lambda x)^2 (\lambda y)^2} { (\lambda x)^2 - 3 (\lambda y)^2}}
	=
	2 \lambda^3 x^3 e^{\frac{\lambda^2( 2x^2+  y^2) } { \lambda^2(x^2 - 3y^2) }}
	=
	\lambda^3 2  x^3 e^{\frac{ 2x^2+  y^2}{ x^2 - 3y^2}}
	= \lambda^3 f_1(x,y)
	\\
	f_2(\lambda x, \lambda y)
	&=
	-
	\lambda x (\lambda y)^2 e^{\frac{\lambda x+ \lambda y}{ \lambda x- \lambda y}}
	=
	-
	\lambda^3 xy^2 e^{\frac{\lambda (x+  y)}{ \lambda (x-  y)}}
	=
	-
	\lambda^3 xy^2 e^{\frac{x+  y}{  x-  y}}
	=
	\lambda^3 f_2(x,y)\\
	f_3(\lambda x, \lambda y)
	&=
	\lambda x (\lambda y)^3 \ln \left( \frac{\lambda x + \lambda y}{2 \lambda y} \right)
	=
	\lambda^4 x y^3 \ln \left(\frac{\lambda}{\lambda} \frac{ x + y}{ 2y} \right)
	=
	\lambda^4 x y^3 \ln \left( \frac{ x + y}{ 2y} \right)
	=
	\lambda^4 f_3(x,y).
\end{align*}
Damit sind die ersten beiden Summanden homogen vom Grad $ 3 $.
Der dritte Summand ist homogen vom Grad $ 4 $. Aufgrund des Distributivgesetz müssten alle Summanden von $ f $ dieselbe Homogenität besitzen, damit $ f $ homogen ist. Also ist $ f $ nicht homogen.\\
\\
Damit ist Antwort (e) korrekt.


\newpage

\subsection*{\frage{8}{4}}
Gegeben ist die Funktion
\begin{align*}
	f(x,y)
	=
	\frac{x^{3a} y^b}{x^3 +y^3} e^{\frac{x^a y}{y^{a+1}}}
	- 
	\frac{1}{x^3 y^{3b} + x y^{2 + 3b}},
\end{align*}
wobei $ x > 0, y > 0 $ und $ a,b \in \R $.\\
\\
Für welche Werte von $ a $ und $ b $ gilt
\begin{align*}
	\varepsilon_x(x,y) + \varepsilon_y(x,y) = 2 \ \textrm{für alle } x>0, y>0\textrm{?}
\end{align*}
\renewcommand{\labelenumi}{(\alph{enumi})}
\begin{enumerate}
	\item 
	$a = \frac{10}{9}$ und $ b=-\frac{5}{3} $.
	\item
	$a = \frac{17}{9}$ und $ b=-\frac{2}{3} $.
	\item
	$a = \frac{20}{9}$ und $ b=-\frac{5}{3} $.
	\item
	$a = \frac{20}{9}$ und $ b=-\frac{2}{3} $.
	\item
	$a = \frac{19}{9}$ und $ b=-\frac{5}{3} $.
	\item
	Es gibt keine Kombination $ a $ und $ b $, die die Bedingung erfüllt.
\end{enumerate}
\ \\
\textbf{Lösung:}
\begin{mdframed}
\underline{\textbf{Vorgehensweise:}}
\renewcommand{\labelenumi}{\theenumi.}
\begin{enumerate}
\item Verwende die Eulersche Relation.
\end{enumerate}
\end{mdframed}

\underline{1. Verwende die Eulersche Relation}\\
Die folgende Eulersche Relation 
\begin{align*}
	\varepsilon_{f,x}(x,y) + \varepsilon_{f,y}(x,y) 
	= 2.
\end{align*}
ist erfüllt, falls $f$ homogen vom Grad $2$ ist. Hierfür betrachten wir:
\begin{align*}
	f(\lambda x , \lambda y)
	&=
	\frac{\lambda^{3a + b} x^{3a} y^b}{\lambda^3 (x^3 + y^3)}
	e^{\frac{\lambda^a x^a \lambda y}{\lambda^{a+1} y^{a+1}}}
	-
	\frac{1}{
		\lambda^{3 + 3b}
		(
		x^3 y^{3b} + x y^{2+ 3b}
		)
	}\\
	&= 
	\frac{\lambda^{3a + b}}{\lambda^3}
	\frac{x^{3a} y^b}{x^3 + y^3} 	e^{\frac{\lambda^{a+1} x^a  y}{\lambda^{a+1} y^{a+1}}}
	-
	\frac{1}{\lambda^{3 + 3b}}
	\frac{1}{
		x^3 y^{3b} + x y^{2+ 3b}
	}\\
	&= 
	\lambda^{3a + b -3 }
	\frac{x^{3a} y^b}{x^3 + y^3} 
	e^{\frac{ x^a  y}{ y^{a+1}}}
	-
	\lambda^{-3 - 3b}
	\frac{1}{
		x^3 y^{3b} + x y^{2+ 3b}
	}.
\end{align*}
Für die Homogenität von $1$ müssen die Bedingungen
\begin{align*}
	3a + b - 3 &= 2\\
	-3 - 3b &= 2
\end{align*}
erfüllt sein. Für die zweite Gleichung gilt:
\begin{align*}
	-3 -3 b = 2  
	\ \Leftrightarrow \
	-3 b = 5
	\ \Leftrightarrow \
	b = - \frac{5}{3}.
\end{align*}
Eingesetzt in die erste Gleichung liefert dies:
\begin{align*}
	3a - \frac{5}{3}  -3 = 3
	\ \Leftrightarrow \
	3a - \frac{14}{3}  = 2 
	\ \Leftrightarrow \
	3a = \frac{20}{3}
	\ \Leftrightarrow \
	a  = \frac{20}{9}.
\end{align*}
Die eulersche Relation ist somit für $a  = \frac{20}{9}$ und $b = - \frac{5}{3}$ erfüllt.\\
\\
Damit ist die Antwort (c) korrekt.


