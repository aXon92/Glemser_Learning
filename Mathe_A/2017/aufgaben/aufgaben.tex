%\newcommand{\ein}[2]{(#1) (#2 Punkte)}


\begin{Large}
\textbf{Teil I: Offene Aufgaben (50 Punkte)}
\end{Large}
\\
\\
\\
\textbf{Allgemeine Anweisungen für offene Fragen:}
\\
\renewcommand{\labelenumi}{(\roman{enumi})}
\begin{enumerate}
\item
Ihre Antworten müssen alle Rechenschritte enthalten,
diese müssen klar ersichtlich sein.
Verwendung korrekter mathematischer Notation wird erwartet
und fliesst in die Bewertung ein.

\item
Ihre Antworten zu den jeweiligen Teilaufgaben müssen in den dafür vorgesehenen Platz geschrie-
ben werden. Sollte dieser Platz nicht ausreichen, setzen Sie Ihre Antwort auf der Rückseite oder
dem separat zur Verfügung gestellten Papier fort. Verweisen Sie in solchen Fällen ausdrücklich
auf Ihre Fortsetzung. Bitte schreiben Sie zudem Ihren Vor- und Nachnamen auf jeden separaten
Lösungsbogen.

\item
Es werden nur Antworten im dafür vorgesehenen Platz bewertet. Antworten auf der Rückseite
oder separatem Papier werden nur bei einem vorhandenen und klaren Verweis darauf bewertet.

\item
Die Teilaufgaben werden mit den jeweils oben auf der Seite angegebenen Punkten bewertet.

\item
Ihre endgültige Lösung jeder Teilaufgabe darf nur eine einzige Version enthalten.

\item
Zwischenrechnungen und Notizen müssen auf einem getrennten Blatt gemacht werden. Diese
Blätter müssen, deutlich als Entwurf gekennzeichnet, ebenfalls abgegeben werden.
\end{enumerate}

\newpage
\section*{\hfil Aufgaben \hfil}
\vspace{1cm}
\section*{Aufgabe 1 (26 Punkte)}
\vspace{0.4cm}
\subsection*{\aufgabe{a1}{4}}
Gegeben ist die Funktion 
\begin{align*}
f  \ : \ D_f \to \mathbb{R},
\
x \mapsto \ln(\sqrt{x-2} - 4 ) + \ln(\sqrt{x-2} + 4).
\end{align*}
Ermitteln Sie den Definitionsbereich $ D_f $ und den Wertebereich $ W_f $ von $ f$.\\
\\
\textit{Hinweis:} Vereinfachen Sie zunächst die Logarithmusterme.
\\
\\
\subsection*{\aufgabe{a2}{3}}
Gegeben ist die Funktion 
\begin{align*}
f  \ : \ D_f \to \mathbb{R},
\
x \mapsto \ln(\sqrt{x-2} - 4 ) + \ln(\sqrt{x-2} + 4).
\end{align*}
Ist die Funktion $ f  $ auf ihrem Definitionsgebiet streng konkav (Beweis)?
\\
\\<
\subsection*{\aufgabe{a3}{3}}
Gegeben ist die Funktion 
\begin{align*}
f  \ : \ D_f \to \mathbb{R},
\
x \mapsto \ln(\sqrt{x-2} - 4 ) + \ln(\sqrt{x-2} + 4).
\end{align*}
Ermitteln Sie die Umkehrfunktion $ f^{-1} $ von $ f $.
\\
\\
\subsection*{\aufgabe{b}{6}}
Um seine Geschäftsidee zu finanzieren, nimmt ein Start-up einen Kredit in Höhe von
$ 1' 0 00 '0 0 0 $ CHF auf. Die Bank stimmt einem niedrigeren jährlichen Zinssatz von $ 0.5\% $
während der ersten 5 Jahre zu, in denen das Start-up am Ende jeden Jahres $ 10'000 $ CHF
zurückzahlen muss. Danach steigt der Zinssatz auf $ 2 \% $ p.a. und es werden konstante
Zahlungen in Höhe von $ C^I $ CHF vereinbart, die wieder am Ende jeden Jahres fällig sind.
Der Plan sieht vor, dass der Kredit in 15 Jahren zurückgezahlt ist.
\\
\\
Wie hoch müssen die jährlichen Zahlungen $ C^I $ sein, sodass der Plan des Start-ups umsetzbar ist?
\\
\\
\subsection*{\aufgabe{c}{4}}
Berechnen Sie 
\begin{align*}
\lim \limits_{x \to 0} \frac{2}{x^4} e^{- \frac{1}{x^2}}.
\end{align*}
\\
\\
\subsection*{\aufgabe{d}{6}}
Eine professionelle Langstreckenläuferin läuft in der ersten Stunde 20 Kilometer. Danach nimmt ihre Leistung in jeder weiteren Stunde des Laufens um einen Faktor $ a \in (0,1] $ ab, dass heisst beispielsweise in der zweiten Stunde läuft sie noch $ 20 \cdot (1-a) $ Kilometer.
Für welche Werte $ a \in (0,1] $ und $ b \geq 20  $ wird die Läuferin einen Wettkampf der Länge von $ b  $ Kilometern bewältigen können, gegeben dass sie beliebig lange laufen kann?
\\
\\
Stellen Sie die Lösungsmenge graphisch (in einem (a,b)-System) dar.
\\
\\
\newpage
\section*{Aufgabe 2 (24 Punkte)}
\vspace{0.4cm}
\subsection*{\aufgabe{a1}{5}}
Sei $ a_k = \ln \left( 1 + \left( \frac{1}{2}\right)^k \right) $ für $ k = 1,2,... $\\
\\
Verwenden Sie das Taylorpolynom $ P_2 $ zweiter Ordnung der Funktion
\begin{align*}
f \ : \ D_f \to \mathbb{R}, \ x \mapsto y = f(x) = \ln(1+x)
\end{align*}
im Punkt $ x_0 = 0 $, um einen Näherungswert für 
$ \sum_{k=1}^\infty a_k $ zu bestimmen.
\\
\\
\subsection*{\aufgabe{a2}{4}}
Gegeben ist die Funktion
\begin{align*}
f \ : \ D_f \to \mathbb{R}, \ x \mapsto y = f(x) = \ln(1+x).
\end{align*}
$ R_2 $ bezeichne das Restglied zweiter Ordnung von $ f  $ in $ x_0 = 0 $.\\
\\
Zeigen Sie
\begin{align*}
	\sum \limits_{k=1}^\infty R_2 \left( \left( \frac{1}{2}\right)^k \right) \leq \frac{1}{21}.
\end{align*}
\\
\\
\subsection*{\aufgabe{b}{4}}
Gegeben ist die Funktion 
\begin{align*}
f(x,y)
=
\frac{\ln(9 - 9 x^2 - y^2)}{(x-y) \sqrt{4 - x^2 - y^2}}.
\end{align*}

Ermitteln Sie den Definitionsbereich $ D_f $ von $ f $ und stellen Sie diesen graphisch dar.
\\
\\
\subsection*{\aufgabe{c}{5}}
Gegeben sei die Nutzenfunktion
\begin{align*}
u(c_1,c_2) = c_1^\alpha c_2^{1-\alpha}
\end{align*}
für $ \alpha \in (0,1) $, wobei $ c_1,c_2 $ die konsumierten Mengen der Güter 1 und 2 sind, und die Budgetrestriktion
\begin{align*}
C \ : \ p_1 c_1 + p_2 c_2 = 10
\end{align*}
für Preise $ p_1 > 0  $ und $ p_2 > 0 $.\\
\\
Für welche Werte der Parameter $ \alpha, p_1, p_2 $ berührt die Niveaulinie (Indifferenzkurve) $ u(c_1,c_2) = \sqrt{2} $ die Budgetlinie $ C $ im Konsumgüterbündel $ (c_1^\star,c_2^\star)  = (1,2)$?
\\ 
\\
\subsection*{\aufgabe{d}{6}}
Die Funktionen $ f $ und $ g $ sind auf $ \mathbb{R}^2_{++} $ definiert und haben den Wertebereich $ \mathbb{R}_{++} $.
Außerdem ist die Funktion $ f $ homogen vom Grad $ r $ und die Funktion $ g $ homogen vom Grad $ r -2 $.\\
Für die Funktion $ h $ gilt:
\begin{align*}
h(x,y) &= \frac{f(x,y)}{g(x,y)}\\
h_y(x,y) &= x  - \frac{3}{2} x^{0.5} y^{0.5}
\end{align*}
und
\begin{align*}
\varepsilon_{h,x}(x,y)
= 
\frac{xy - \frac{1}{2}x^{0.5} y^{1.5}}{xy - x^{0.5} y^{1.5}}
\end{align*}
Ermitteln Sie $ h(x,y) $ und vereinfachen Sie den Funktionsterm.
\newpage


\fancyhead[C]{\normalsize\textbf{$\qquad$ Teil II: Multiple-Choice}}
\begin{Large}
\textbf{Teil II: Multiple-Choice-Fragen (50 Punkte)}
\end{Large}
\\
\\
\\
\textbf{Allgemeine Anweisungen für Multiple-Choice-Fragen:}
\\
\renewcommand{\labelenumi}{(\roman{enumi})}
\begin{enumerate}
\item
Die Antworten auf die Multiple-Choice-Fragen müssen im dafür vorgesehenen Antwortbogen ein-
getragen werden. Es werden ausschliesslich Antworten auf diesem Antwortbogen bewertet. Der
Platz unter den Fragen ist nur für Notizen vorgesehen und wird nicht korrigiert.

\item
Jede Frage hat nur eine richtige Antwort. Es muss also auch jeweils nur eine Antwort angekreuzt
werden.

\item
Falls mehrere Antworten angekreuzt sind, wird die Antwort mit 0 Punkten bewertet, auch wenn
die korrekte Antwort unter den angekreuzten ist.

\item
Bitte lesen Sie die Fragen sorgfältig.

\end{enumerate}
\newpage
\section*{Aufgabe 3 (24 Punkte)}
\vspace{0.4cm}
\subsection*{\frage{1}{4}}
Seien $ A $ und $ B $ zwei Aussagen. Die zusammengesetzte Aussage $ A \vee (\neg A \Rightarrow B) $ ist äquivalent zu
\renewcommand{\labelenumi}{(\alph{enumi})}
\begin{enumerate}
\item $ A $.
\item $ B $.
\item $ A \vee B $.
\item $ A \wedge B $.
\end{enumerate}
\ \\
\subsection*{\frage{2}{3}}
Sei $ f $ eine stetige Funktion. 
Sei $ \lbrace a_n \rbrace_{n \in \mathbb{N}} $ eine monotone und konvergente Folge mit $ a_n \in D_f $ für alle $ n \in \mathbb{N} $.
Die Folge $ \lbrace b_n \rbrace_{n \in \mathbb{N}} $ definiert durch $ b_n = f(a_n) $ für alle $ n \in \mathbb{N} $ ist
\renewcommand{\labelenumi}{(\alph{enumi})}
\begin{enumerate}
\item konvergent.
\item divergent.
\item
monoton.
\item
Keine der obigen Antworten ist richtig.
\end{enumerate}
\ \\
\subsection*{\frage{3}{2}}
Welche der folgenden Aussagen über eine Funktion $ f $ und einen Punkt $ x_0 \in D_f $ ist wahr ? 
\renewcommand{\labelenumi}{(\alph{enumi})}
\begin{enumerate}
\item 
Wenn $ f $ in $ x_0 $ stetig ist, dann ist $ f $ in $ x_0 $ differenzierbar.
\item 
Wenn $ f $ in $ x_0 $ differenzierbar ist, dann ist $ f $ in $ x_0 $ stetig.
\item 
$ f $ ist in $ x_0 $ stetig genau dann, wenn $ f $ in $ x_0 $ differenzierbar ist.
\item
Wenn $ f $ in $ x_0 $ differenzierbar ist, dann ist $ f $ in $ x_0 $ unstetig.
\end{enumerate}
\ \\
\subsection*{\frage{4}{3}}
Ein Investor hat die Wahl zwischen zwei Projekten:\\
Projekt I erfordert eine Anfangsinvestition von CHF $ 100'000  $ und zahlt CHF $ 50'000 $ in $ 6 $ Monaten sowie CHF $ 60'000 $ in $ 1 $ Jahr aus. \\
Projekt II erfordert eine Anfangsinvestition von CHF $ 100'000 $ und zahlt in $ 1 $ Jahr CHF $ 110'000 $ aus.
\renewcommand{\labelenumi}{(\alph{enumi})}
\begin{enumerate}
\item 
Projekt I ist Projekt II vorzuziehen, gegeben, dass der Zinssatz strikt positiv ist.
\item
Projekt II ist Projekt I vorzuziehen, gegeben, dass der Zinssatz strikt positiv ist.
\item
Projekt I und Projekt II haben denselben Nettobarwert.
\item
Ob Projekt I dem Projekt II vorzuziehen ist, oder Projekt II dem Projekt I, hängt von der Höhe des strikt positiven Zinssatzes ab.
\end{enumerate}
\ \\
\subsection*{\frage{5}{3}}
Ein Finanzberater schlägt seinem Kunden zwei Optionen für die Rückzahlung eines Hypothekenkredits vor:
Option 1 sieht die Rückzahlung des Kredits mit konstanten Zahlungen $ C^D  $ vor, welche über $ n^D $ Jahre am Jahresanfang erfolgen. Bei Option 2 dagegen wird derselbe Kredit mit konstanten Zahlungen $ C^l $ am Jahresende über $ n^I $ Jahre zurückgezahlt.\\
\\
Unter der Voraussetzung, dass der Zinssatz strikt positiv ist, folgt
\renewcommand{\labelenumi}{(\alph{enumi})}
\begin{enumerate}
\item 
$ C^I = C^D$, wenn $ n^I = n^D $.
\item
$ C^I > C^D$, wenn $ n^I = n^D $.
\item
$ C^I < C^D$, wenn $ n^I = n^D $.
\item
$ C^I > C^D $ genau dann, wenn $ n^I > n^D $.
\end{enumerate}
\ \\
\subsection*{\frage{6}{3}}
Wir betrachten die Funktion
\begin{align*}
f \ : \
\mathbb{R} \to \mathbb{R}, \quad 
x \mapsto
y = 
\begin{cases}
\frac{\sin(x)}{a(x- \pi)} &\qquad \text{für} \ x \neq \pi\\
a &\qquad \text{für} \ x = \pi
\end{cases}
\end{align*}

Für welches $ a \in \mathbb{R} $ ist $ f $ überall stetig?
\renewcommand{\labelenumi}{(\alph{enumi})}
\begin{enumerate}
\item 
$ a = 1 $.
\item
$ a = -1 $.
\item
$ a \in \{ -1, 1\} $.
\item
$ f $ ist für kein $ a \in \mathbb{R} $ überall stetig.
\end{enumerate}
\ \\
\subsection*{\frage{7}{3}}
Sei $ f(x) = 1 + 3  x- 4  x^4 $ und $ P_4 $ das Taylorpolynom vierter Ordnung von $ f  $ in $ x_0 = 1 $.
Welche der folgenden Aussagen über das Restglied vierter Ordnung $ R_4 $ in $ x_0 = 1 $ ist wahr?
\renewcommand{\labelenumi}{(\alph{enumi})}
\begin{enumerate}
\item 
$ R_4(x) > 0  $ für alle $ x $.
\item
$ R_4(x) <0  $ für alle $ x $.
\item
$ R_4(x) =0  $ für alle $ x $.
\item
Jeder der Fälle $ R_4(x ) > 0 $, $ R_4(x) < 0 $ und $ R_4(x) = 0 $ ist für entsprechende $ x \in \mathbb{R} $ möglich.
\end{enumerate}
\ \\
\subsection*{\frage{8}{3}}
Für eine Funktion $ f $ ist die Elastizität $ \varepsilon_f(x) $ gegeben durch:
\begin{align*}
\varepsilon_f(x) =x \ \ln (x) + e^{3 \ x}.
\end{align*}
Sei $ g $ die Funktion definiert durch $ g(x) = f( a \ x) $ für $ a > 0 $.\\
\\
Dann gilt:
\renewcommand{\labelenumi}{(\alph{enumi})}
\begin{enumerate}
\item 
$ \varepsilon_g(x) = x \ \ln(x)  + e^{3 \ x}$.
\item
$ \varepsilon_g(x) =a \  x \ \ln(x)  +a \   e^{3 \ x} $.
\item
$ \varepsilon_g(x) =\frac{x}{a} \ \ln(x)  +\frac{x}{a}   e^{3 \ x} $.
\item
Keine der obigen Antworten ist richtig.
\end{enumerate}


\newpage
\section*{Aufgabe 4 (26 Punkte)}
\vspace{0.4cm}

\subsection*{\frage{1}{3}}
Geben ist die Funktion
\begin{align*}
f \ : \ \mathbb{R}^2_{++} \to \mathbb{R}^2_{++},
\ \
(x,y) \mapsto f(x,y) = (x^2 + 2\ x \ y + y^2) e^{x+y}. 
\end{align*}
Ihre partiellen Elastizitäten $ \varepsilon_{f,x}(x,y) $ und
$ \varepsilon_{f,y}(x,y) $ genügen der Ungleichung
\renewcommand{\labelenumi}{(\alph{enumi})}
\begin{enumerate}
\item 
$ \varepsilon_{f,x} > \varepsilon_{f,y}  $ für alle $ (x,y) \in \mathbb{R}^2_{++} $.
\item
$ \varepsilon_{f,x} < \varepsilon_{f,y}  $ für alle $ (x,y) \in \mathbb{R}^2_{++} $.
\item
$ \varepsilon_{f,x} < \varepsilon_{f,y}  $ für $ (x,y) \in \mathbb{R}^2_{++} $ mit $ x > y $.
\item
$ \varepsilon_{f,x} < \varepsilon_{f,y}  $ für $ (x,y) \in \mathbb{R}^2_{++} $ mit $ x < y $.
\end{enumerate}
\ \\
\subsection*{\frage{2}{4}}

\renewcommand{\labelenumi}{(\alph{enumi})}
Gegeben ist die Funktion
\begin{align*}
f \ : \ \mathbb{R} \to \mathbb{R}_{++}, \ \ 
x \mapsto f(x) = x^2 \ e^{x^2} + 1
\end{align*}
\begin{enumerate}
\item 
$ f $ hat ein lokales Maximum in $ x_0 = 0 $.
\item
$ f $ hat ein lokales Minimum in $ x_0 = 0 $.
\item
$ f $ hat einen Wendepunkt in $ x_0 = 0 $.
\item
$ f $ hat keine stationären Punkte.
\end{enumerate}
\ \\
\subsection*{\frage{3}{4}}
Gegeben ist die Funktion
\begin{align*}
f \ : \ D_f  \to \mathbb{R}, x \mapsto f(x) = \frac{1}{1+x}.
\end{align*}
$ P_3 $ und $ P_4 $ seien die Taylorpolynome dritter und vierter Ordnung von $ f $ in $ x_0 = 0 $.\\
\\
Dann gilt:
\renewcommand{\labelenumi}{(\alph{enumi})}
\begin{enumerate}
\item 
$ P_3(x) > P_4(x) $ für alle $ x \in D_f \setminus \{ x_0\} $.
\item
$ P_3(x) < P_4(x) $ für alle $ x \in D_f \setminus \{ x_0\} $.
\item
$ P_3(x) = P_4(x) $ für alle $ x \in D_f \setminus \{ x_0\} $.
\item
Jeder der Fälle $ P_3(x) > P_4(x) $, $ P_3(x) < P_4(x) $ oder $ P_3(x) = P_4(x) $ ist für entsprechende $ x \in D_f \setminus \{ x_0\} $ möglich.
\end{enumerate}
\ \\
\subsection*{\frage{4}{4}}
Gegeben sind die Funktionen
\begin{align*}
f(x,y) = \sqrt{1 - 4  x^2 - y^2}
\end{align*}
und
\begin{align*}
g(x,y)
= 
\ln(2  x - x^2 - y^2 + 8) 
\end{align*}
mit den entsprechenden Definitionsgebieten $ D_f $ und $ D_g $.\\
\\
Dann gilt:
\renewcommand{\labelenumi}{(\alph{enumi})}
\begin{enumerate}
\item 
$ D_f \subseteq D_g $.
\item
$ D_g \subseteq D_f $.
\item
$ D_f = D_g $.
\item
$ D_f \cap D_g = \emptyset $.
\end{enumerate}

\newpage
\subsection*{\frage{5}{3}}
Sei $ f(x) = \sin(x) $ und $ P_3  $ das Taylorpolynom  dritter Ordnung von $ f $ in $ x_0 = 0 $.\\
\\
Welche der folgenden Aussagen bezüglich des Restglieds dritter Ordnung $ R_3 $ von $ f  $ in $ x_0 = 0 $ ist wahr?
\renewcommand{\labelenumi}{(\alph{enumi})}
\begin{enumerate}
\item 
$ |R_3(x)| \leq \frac{|x|^4}{128} $ für alle $ x \in \mathbb{R} $.
\item
$ |R_3(x)| \leq \frac{|x|^4}{64} $ für alle $ x \in \mathbb{R} $.
\item
$ |R_3(x)| \leq \frac{|x|^4}{32} $ für alle $ x \in \mathbb{R} $.
\item
$ |R_3(x)| \leq \frac{|x|^4}{16} $ für alle $ x \in \mathbb{R} $.
\end{enumerate}
\ \\
\subsection*{\frage{6}{2}}
Gegeben ist die Funktion
\begin{align*}
f(x,y) = 8 \ \left( \frac{1}{x} + \frac{1}{5  y} \right)^{-0.5}
+ \sqrt{3  x} + \sqrt{y} \ \ \
(x > 0 , y > 0).
\end{align*}
\renewcommand{\labelenumi}{(\alph{enumi})}
\begin{enumerate}
\item 
$ f $ ist linear homogen.
\item
$ f $ ist homogen vom Grad $ -0.5 $.
\item
$ f $ ist homogen vom Grad $ 0.5 $.
\item
$ f $ ist nicht homogen.
\end{enumerate}
\ \\
\subsection*{\frage{7}{3}}
Gegeben sind die Funktionen
\begin{align*}
f(x,y) =  \frac{x^2}{y} + 1 + \sqrt{x^2 + 5 \ y^2} \ \ \
(x > 0 , y > 0)
\end{align*}
und
\begin{align*}
g(x,y) = f( a  x, a  y),
\end{align*}
wobei $ a >0 $.
\renewcommand{\labelenumi}{(\alph{enumi})}
\begin{enumerate}
\item 
$ g $ ist linear homogen.
\item
$ g $ ist homogen vom Grad $ a $.
\item
$ g $ ist homogen vom Grad $ 2 \  a $.
\item
$ g $ ist nicht homogen.
\end{enumerate}
\ \\
\subsection*{\frage{8}{3}}
Gegeben sei die Funktion
\begin{align*}
f(x,y) = 
x^{a+1} \sqrt{y^{4a + 4 }}+ ( xy)^{\frac{3a + 3}{2}} \ \ \ (x > 0, y > 0),
\end{align*}
wobei $ a \in \mathbb{R} $, mit partiellen Elastizitäten $ \varepsilon_{f,x} $ und $ \varepsilon_{f,y} $.\\
\\
Für welchen Wert von $ a $ gilt
\begin{align*}
\varepsilon_{f,x} + \varepsilon_{f,y} = 3?
\end{align*}
\renewcommand{\labelenumi}{(\alph{enumi})}
\begin{enumerate}
\item 
$a = 0$.
\item
$a = 1$.
\item
$a = 2$.
\item
$a = 3$.
\end{enumerate}