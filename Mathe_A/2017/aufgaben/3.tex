\fancyhead[C]{\normalsize\textbf{$\qquad$ Teil II: Multiple-Choice}}
\section*{Aufgabe 3 (24 Punkte)}
\vspace{0.4cm}
\subsection*{\frage{1}{4}}
Seien $ A $ und $ B $ zwei Aussagen. Die zusammengesetzte Aussage $ A \vee (\neg A \Rightarrow B) $ ist äquivalent zu
\renewcommand{\labelenumi}{(\alph{enumi})}
\begin{enumerate}
	\item $ A $.
	\item $ B $.
	\item $ A \vee B $.
	\item $ A \wedge B $.
\end{enumerate}
\ \\
\textbf{Lösung:}
\begin{mdframed}
\underline{\textbf{Vorgehensweise:}}
\renewcommand{\labelenumi}{\theenumi.}
\begin{enumerate}
\item Löse die Aufgabe mithilfe einer Wahrheitstafel.
\item Alternative Lösung I.
\item Alternative Lösung II.
\end{enumerate}
\end{mdframed}

\underline{1. Löse die Aufgabe mithilfe einer Wahrheitstafel}\\
Eine Aussage kann nur die Wahrheitswerte wahr ($ W $) oder falsch ($ F $) annehmen.
Dementsprechend gibt es bei zwei Aussagen $ A $, $ B $ genau vier Kombinationsmöglichkeiten der Wahrheitswerte.
Aus diesen Kombinationen ergeben sich dann die Wahrheitswerte der verknüpften Aussagen.
Zwei Verknüpfungen heißen äquivalent, falls deren Wahrheitswerte bei jeder möglichen Kombination übereinstimmen.
\begin{center}
	\begin{tabular}{cllll}
		\hline
		\multicolumn{1}{c|}{$A$} & \multicolumn{4}{l}{$W$ $W$ $F$ $F$} \\
		\multicolumn{1}{c|}{$B$} & \multicolumn{4}{l}{$W$ $F$ $W$ $F$} \\ \hline
		\multicolumn{1}{c|}{$\neg A$} & \multicolumn{4}{l}{$F$ $F$ $W$ $W$} \\ 
		\multicolumn{1}{c|}{$A \vee B$} & \multicolumn{4}{l}{$W$ $W$ $W$ $F$}  \\ 
		\multicolumn{1}{c|}{$A \wedge B$} & \multicolumn{4}{l}{$W$ $F$ $F$ $F$}  \\ 
		\multicolumn{1}{c|}{$A \Rightarrow B$} & \multicolumn{4}{l}{$W$ $F$ $W$ $W$}  \\
		\hline
		
		\multicolumn{1}{c|}{$\neg A \Rightarrow B$} & \multicolumn{4}{l}{$W$ $W$ $W$ $F$}  \\ 
	
		\hline
	\end{tabular}
\end{center}
In dem ersten Abschnitt der Tabelle sind alle möglichen Kombinationen der Aussagen $ A $ und $ B $ aufgelistet.
In dem zweiten Abschnitt die grundlegenden Verknüpfungen und 
im letzten Abschnitt die gewünschte Aussage.\\
Die Aussagen $ \neg A \Rightarrow B  $ und $ A \vee B  $ sind äquivalent, da diese bei jeder Kombination von $ A $ und $ B $ denselben Wahrheitswert besitzen.
\\
\\
Damit ist Antwort (c) korrekt.
\\
\\
\newpage
\underline{2. Alternative Lösung I}\\
Die Implikation ist durch
\begin{align*}
A \Rightarrow B \ : \Leftrightarrow \ \neg A \vee B
\end{align*}
definiert.
Mit 
\begin{align*}
A \vee (\neg A \Rightarrow B) 
\ \Leftrightarrow \
A \vee ( \neg (\neg A)) \vee B )
\ \Leftrightarrow \
A \vee (A \vee B)
\ \Leftrightarrow \
A \vee B
\end{align*}
ist Antwort (c) korrekt.
\\
\\
\underline{3. Alternative Lösung II}\\
Mit 
\begin{center}
	\begin{tabular}{cllll}
		\hline
		\multicolumn{1}{c|}{$A$} & \multicolumn{4}{l}{$W$ $W$ $F$ $F$} \\
		\multicolumn{1}{c|}{$B$} & \multicolumn{4}{l}{$W$ $F$ $W$ $F$} \\ \hline
		\multicolumn{1}{c|}{$\neg A \Rightarrow B$} & \multicolumn{4}{l}{$W$ $W$ $W$ $F$}  \\ 
		
		\multicolumn{1}{c|}{$A \vee B$} & \multicolumn{4}{l}{$W$ $W$ $W$ $F$}  \\ \hline
	\end{tabular}
\end{center}
erhalten wir die Äquivalenz von $ \neg A \Rightarrow B $ und $ A \vee B $.
Demnach gilt:
\begin{align*}
A \vee (\neg A \Rightarrow B) 
\ \Leftrightarrow \
A \vee (A \vee B)
\ \Leftrightarrow \
A \vee B 
\end{align*}
\\
\\
Somit ist Antwort (c) korrekt.

\newpage

\subsection*{\frage{2}{3}}
Sei $ f $ eine stetige Funktion. 
Sei $ \lbrace a_n \rbrace_{n \in \mathbb{N}} $ eine monotone und konvergente Folge mit $ a_n \in D_f $ für alle $ n \in \mathbb{N} $.
Die Folge $ \lbrace b_n \rbrace_{n \in \mathbb{N}} $ definiert durch $ b_n = f(a_n) $ für alle $ n \in \mathbb{N} $ ist
\renewcommand{\labelenumi}{(\alph{enumi})}
\begin{enumerate}
	\item konvergent.
	\item divergent.
	\item
	monoton.
	\item
	Keine der obigen Antworten ist richtig.
\end{enumerate}
\ \\
\textbf{Lösung:}
\begin{mdframed}
	\underline{\textbf{Vorgehensweise:}}
	\renewcommand{\labelenumi}{\theenumi.}
	\begin{enumerate}
		\item 
		Überlege dir zu den Aussagen (a) - (c) passende Gegenbeispiele.
	\end{enumerate}
\end{mdframed}

\underline{1. Überlege dir zu den Aussagen (a) - (c) passende Gegenbeispiele}\\
Wir betrachten die Funktion 
\begin{align*}
f \ : \ ] 0 , \infty [ \to \mathbb{R}, \ x \to \frac{1}{x}.
\end{align*}
und eine Folge $ a_n \in D_f $ für alle $ n \in \mathbb{N} $ mit $ a_n \to 0  $. Wegen 
\begin{align*}
f(a_n ) = \frac{1}{a_n } \to \infty
\end{align*}
ist Aussage (a) falsch. Ein konkretes Beispiel ist hier durch
\begin{align*}
a_n = \frac{1}{n} \ 
\Rightarrow 
\ 
f(a_n ) = \frac{1}{\frac{1}{n}} = n \to \infty
\end{align*}
gegeben.\\
\\
Für die zweite Aussage betrachten wir eine beliebige stetige Funktion mit $ D_f = \mathbb{R} $. Dann gilt 
\begin{align*}
a_n \to a_0 \ \Rightarrow \ f(a_n) \to f(a_0)
\end{align*}
für alle $ a_0 \in \mathbb{R} $. Konkret könnte man dies am einfachsten mit einer konstanten Funktion begründen.
Wir betrachten
\begin{align*}
f \ : \ \mathbb{R} \to \mathbb{R}, \ x \mapsto 0,
\end{align*}
womit sogar für jede Folge $ f(a_n) = 0  $ für alle $ n \in \mathbb{N} $ gilt.\\
\\
Für die nächste Aussage werden wir ein konkretes Gegenbeispiel angeben.
Wir definieren die Funktion
\begin{align*}
f(x) = \sin \left(\pi \frac{1}{x} \right)
\end{align*}
und die monotone Folge $ a_n = \frac{1}{n} $.
Dann erhalten wir durch
\begin{align*}
f(a_n) = \sin \left( \pi \frac{1}{\frac{1}{n}}\right)
= \sin (\pi n ) = (-1)^n
\end{align*}
eine nicht monotone Folge.
\\
\\ 
Somit ist Antwort (d) korrekt.

\newpage

\subsection*{\frage{3}{2}}
Welche der folgenden Aussagen über eine Funktion $ f $ und einen Punkt $ x_0 \in D_f $ ist wahr ? 
\renewcommand{\labelenumi}{(\alph{enumi})}
\begin{enumerate}
	\item 
	Wenn $ f $ in $ x_0 $ stetig ist, dann ist $ f $ in $ x_0 $ differenzierbar.
	\item 
	Wenn $ f $ in $ x_0 $ differenzierbar ist, dann ist $ f $ in $ x_0 $ stetig.
	\item 
	$ f $ ist in $ x_0 $ stetig genau dann, wenn $ f $ in $ x_0 $ differenzierbar ist.
	\item
	Wenn $ f $ in $ x_0 $ differenzierbar ist, dann ist $ f $ in $ x_0 $ unstetig.
\end{enumerate}
\ \\
\textbf{Lösung:}
\begin{mdframed}
\underline{\textbf{Vorgehensweise:}}
\renewcommand{\labelenumi}{\theenumi.}
\begin{enumerate}
\item Kenne die die Definition von Stetigkeit \& Differenzierbarkeit.
\item Zusatz: Weise die korrekte Antwort nach.
\end{enumerate}
\end{mdframed}

\underline{1. Kenne die die Definition von Stetigkeit \& Differenzierbarkeit}\\
Ist die Funktion $ f $ differenzierbar in $ x_0 \in D_f $, dann ist sie auch stetig in $ x_0 \in D_f $.
Somit ist Antwort (b) richtig und (d) falsch.
Durch die Betragsfunktion
\begin{align*}
f \ : \ \mathbb{R} \to \mathbb{R}, \ x \mapsto |x|
\end{align*}
erhalten wir ein direktes Gegenbeispiel zur Antwort (a).
Die Ableitung repräsentiert die Steigung in einem Punkt. Links von $ x_0 = 0 $ ist die Steigung $ -1 $ und rechts davon $ 1 $.
Um dies genauer zu untersuchen, betrachten wir die Definition der Differenzierbarkeit. Eine Funktion $ f $ heißt differenzierbar in $ x_0 $, falls
\begin{align*}
\lim \limits_{x \to x_0} \frac{f(x) - f(x_0)}{x- x_0} = f^\prime(x_0)
\end{align*}
existiert. Für die Betragsfunktion gelten nun
\begin{align*}
\lim \limits_{x \to 0, x <0 } \frac{f(x) - f(0)}{x- 0}
&= \lim \limits_{x \to 0, x <0 } \frac{-x }{x} = -1\\
\lim \limits_{x \to 0, x >0 } \frac{f(x) - f(0)}{x- 0}
&= \lim \limits_{x \to 0, x >0 } \frac{x }{x} = 1,
\end{align*}
wodurch der Grenzwert nicht existiert. Damit ist Antwort (a) falsch.
Da (c) eine Kombination von (a) und (b) ist, ist (c) auch falsch.\\
\\
Somit ist Antwort (b) korrekt.
\\
\\
\underline{2. Zusatz: Weise die korrekte Antwort nach}\\
Sei $ f $ in $ x_0 $ differenzierbar. 
Wir wollen zeigen, dass $ f(x) \to f(x_0) $ für $ x \to x_0 $ folgt.
Dann ist $ f $ stetig.
Wir erhalten
\begin{align*}
\lim \limits_{x \to x_0} f(x) - f(x_0)
= 
\lim \limits_{x \to x_0} \frac{f(x) - f(x_0)}{x- x_0}(x-x_0)
=
f^\prime(x_0) \cdot 0 = 0
\
\Rightarrow 
\
f(x) \to f(x_0)
\end{align*}
für $ x \to x_0 $. Damit ist $ f $ stetig und Antwort (d) falsch.\\
\\
Somit ist Antwort (b) korrekt.
 \newpage

\subsection*{\frage{4}{3}}
Ein Investor hat die Wahl zwischen zwei Projekten:\\
Projekt I erfordert eine Anfangsinvestition von CHF $ 100'000  $ und zahlt CHF $ 50'000 $ in $ 6 $ Monaten sowie CHF $ 60'000 $ in $ 1 $ Jahr aus. \\
Projekt II erfordert eine Anfangsinvestition von CHF $ 100'000 $ und zahlt in $ 1 $ Jahr CHF $ 110'000 $ aus.
\renewcommand{\labelenumi}{(\alph{enumi})}
\begin{enumerate}
	\item 
	Projekt I ist Projekt II vorzuziehen, gegeben, dass der Zinssatz strikt positiv ist.
	\item
	Projekt II ist Projekt I vorzuziehen, gegeben, dass der Zinssatz strikt positiv ist.
	\item
	Projekt I und Projekt II haben denselben Nettobarwert.
	\item
	Ob Projekt I dem Projekt II vorzuziehen ist, oder Projekt II dem Projekt I, hängt von der Höhe des strikt positiven Zinssatzes ab.
\end{enumerate}
\ \\
\ \\
\textbf{Lösung:}
\begin{mdframed}
\underline{\textbf{Vorgehensweise:}}
\renewcommand{\labelenumi}{\theenumi.}
\begin{enumerate}
\item Gebe den Barwerte beider Projekte an und finde eine Bedingung für die richtige Wahl.
\item Finde die korrekte Antwort.
\item Alternativer Lösungsweg.
\end{enumerate}
\end{mdframed}

\underline{1. Gebe den Barwerte beider Projekte an und finde eine Bedingung für die richtige Wahl}\\
Der Barwert von Projekt I ist durch
\begin{align*}
-100'000 + \frac{50'000}{(1+i)} + \frac{60'000}{(1+i)^2}
\end{align*}
gegeben. Den Barwert von Projekt II erhalten wir durch
\begin{align*}
-100'000 + \frac{110'000}{(1+i)^2}.
\end{align*}
Falls
\begin{align*}
-100'000 + \frac{50'000}{(1+i)} + \frac{60'000}{(1+i)^2}
>
-100'000 + \frac{110'000}{(1+i)^2}
\end{align*}
gilt, ist Projekt I dem Projekt II vorzuziehen.
\\
\\
\underline{2. Finde die korrekte Antwort}\\
Wir formen die gefundene Bedingung um:
\begin{align*}
-100'000 + \frac{50'000}{(1+i)} + \frac{60'000}{(1+i)^2}
&>
-100'000 + \frac{110'000}{(1+i)^2}
\\
\Leftrightarrow
-100'000 + \frac{50'000}{(1+i)}
&> -100'000 + \frac{50'000}{(1+i)^2}\\
 \Leftrightarrow 
\frac{50'000}{(1+i)} &> 
\frac{50'000}{(1+i)^2}
\ \Leftrightarrow \ 
\frac{1}{(1+i)} > \frac{1}{(1+i)^2}
 \ \Leftrightarrow \
(1+i)^2 > (1+i)
\end{align*}
Dies ist für jeden positives Zinssatz $ i > 0  $ erfüllt.\\
\\
Somit ist Antwort (a) korrekt.
\\
\\
\underline{3. Alternativer Lösungsweg}\\
Wir können die Aufgabe auch lösen, ohne zu rechnen.
Die beiden Auszahlungen sind in Summe identisch, d.h.
$ 50'000 + 60'000 $ oder $ 110'000 $.
Bei Projekt I werden die $ 50'000 $ nach $ 6 $ Monaten zu einem Zinssatz von $ (1 + i) $ angelegt.
Folglich ist wegen
\begin{align*}
(50'000) \cdot \underbrace{(1+i)^{\frac{1}{2}}}_{> 1} + 60'000 
> 110'000,
\end{align*}
das Projekt I vorzuziehen.\\
\\
Somit ist Antwort (a) korrekt.
\newpage

\subsection*{\frage{5}{3}}
Ein Finanzberater schlägt seinem Kunden zwei Optionen für die Rückzahlung eines Hypothekenkredits vor:
Option 1 sieht die Rückzahlung des Kredits mit konstanten Zahlungen $ C^D  $ vor, welche über $ n^D $ Jahre am Jahresanfang erfolgen. Bei Option 2 dagegen wird derselbe Kredit mit konstanten Zahlungen $ C^l $ am Jahresende über $ n^I $ Jahre zurückgezahlt.\\
\\
Unter der Voraussetzung, dass der Zinssatz strikt positiv ist, folgt
\renewcommand{\labelenumi}{(\alph{enumi})}
\begin{enumerate}
	\item 
	$ C^I = C^D$, wenn $ n^I = n^D $.
	\item
	$ C^I > C^D$, wenn $ n^I = n^D $.
	\item
	$ C^I < C^D$, wenn $ n^I = n^D $.
	\item
	$ C^I > C^D $ genau dann, wenn $ n^I > n^D $.
\end{enumerate}
\ \\
\textbf{Lösung:}
\begin{mdframed}
\underline{\textbf{Vorgehensweise:}}
\renewcommand{\labelenumi}{\theenumi.}
\begin{enumerate}
\item Überlege dir, welche Antworten du ausschließen kannst.
\item Finde die korrekte Antwort.
\end{enumerate}
\end{mdframed}

\underline{1. Überlege dir, welche Antworten du ausschließen kannst}\\
Wenn die Anzahl der Jahre $ n^I $ und $ n^D $ verschieden sind, können wir im Allgemeinen nicht entscheiden ob 
$ C^I > C^D $, $ C^I = C^D $ oder $ C^I < C^D $ erfüllt sind.
Also können wir Antwort (d) ausschließen.\\
\\
\underline{2. Finde die korrekte Antwort}\\
%Damit wissen wir, dass $ n^I = n^D $ gelten muss.
%Die Zinserträge bei Zahlungen am Anfang des Jahres sind echt größer als die Zinserträge bei Zahlungen am Ende des Jahres.
%Für den Kunden ist also Rückzahlung am Anfang des Jahres besser, da er weniger Zinsen zahlen muss. Dementsprechend gilt $ C^I > C^D $ bei gleicher Jahreszahl.
Wenn wir davon ausgehen, dass die Anzahl der Jahre identisch sind, dann muss $ C^I > C^D $ gelten ,da die Bank die  jeweilige Zahlung ein Jahr später erhält und so nicht mehr anlegen kann.
\\
\\
Damit ist Antwort (b) korrekt.

\newpage

\subsection*{\frage{6}{3}}
Wir betrachten die Funktion
\begin{align*}
f \ : \
\mathbb{R} \to \mathbb{R}, \quad 
x \mapsto
y = 
\begin{cases}
\frac{\sin(x)}{a(x- \pi)} &\qquad \text{für} \ x \neq \pi\\
a &\qquad \text{für} \ x = \pi
\end{cases}
\end{align*}

Für welches $ a \in \mathbb{R} $ ist $ f $ überall stetig?
\renewcommand{\labelenumi}{(\alph{enumi})}
\begin{enumerate}
	\item 
	$ a = 1 $.
	\item
	$ a = -1 $.
	\item
	$ a \in \{ -1, 1\} $.
	\item
	$ f $ ist für kein $ a \in \mathbb{R} $ überall stetig.
\end{enumerate}
\ \\
\textbf{Lösung:}
\begin{mdframed}
\underline{\textbf{Vorgehensweise:}}
\renewcommand{\labelenumi}{\theenumi.}
\begin{enumerate}
\item Überlege dir, was für die Stetigkeit von $ f $ gelten muss.
\item Bestimme die richtige Antwort.
\end{enumerate}
\end{mdframed}

\underline{1. Überlege dir, was für die Stetigkeit von $ f $ gelten muss}\\
Eine Funktion $ f $ ist stetig in $ x_0 $, falls
\begin{align*}
x \to x_0 \ 
\Rightarrow 
\ 
f(x) \to f(x_0)
\end{align*}
erfüllt ist. Wir müssen also
\begin{align*}
\lim \limits_{x \to \pi } \frac{\sin(x)}{a (x- \pi)}
\end{align*}
berechnen, um die Aufgabe zu lösen. 
\\
\\
\underline{2. Bestimme die richtige Antwort}\\
Unsere Funktion ist für $ x_0 \neq \pi  $ stetig.
Deswegen müssen wir nur $ x_0 = \pi $ untersuchen.
Wegen 
\begin{align*}
&\lim \limits_{x \to \pi} \sin(x) = 0\\
&\lim \limits_{x \to \pi} a(x - \pi) =0
\end{align*}
können wir die Regel von de l'H\^{o}pital anwenden. Es gilt
\begin{align*}
\lim \limits_{x \to \pi } \frac{\sin(x)}{a (x- \pi)}
=
\lim \limits_{x \to \pi } \frac{\sin(x)}{a x- a\pi)}
=
\lim \limits_{x \to \pi } \frac{\cos(x)}{a }
=
-\frac{1}{a},
\end{align*}
wodurch
\begin{align*}
- \frac{1}{a} = a
\
\Leftrightarrow
\
a^2 = -1
\end{align*}
erfüllt sein muss. Diese Gleichung besitzt jedoch keine reelle Lösung, wodurch $ f $ für kein $ a \in \mathbb{R} $ überall stetig ist.\\
\\
Damit ist Antwort (d) korrekt.
\newpage
\subsection*{\frage{7}{3}}
Sei $ f(x) = 1 + 3  x- 4  x^4 $ und $ P_4 $ das Taylorpolynom vierter Ordnung von $ f  $ in $ x_0 = 1 $.
Welche der folgenden Aussagen über das Restglied vierter Ordnung $ R_4 $ in $ x_0 = 1 $ ist wahr?
\renewcommand{\labelenumi}{(\alph{enumi})}
\begin{enumerate}
	\item 
	$ R_4(x) > 0  $ für alle $ x $.
	\item
	$ R_4(x) <0  $ für alle $ x $.
	\item
	$ R_4(x) =0  $ für alle $ x $.
	\item
	Jeder der Fälle $ R_4(x ) > 0 $, $ R_4(x) < 0 $ und $ R_4(x) = 0 $ ist für entsprechende $ x \in \mathbb{R} $ möglich.
\end{enumerate}
\ \\
\textbf{Lösung:}
\begin{mdframed}
\underline{\textbf{Vorgehensweise:}}
\renewcommand{\labelenumi}{\theenumi.}
\begin{enumerate}
\item Überlege dir, was du über das Restglied von $ f $ sagen kannst und bestimme die Antwort.
\end{enumerate}
\end{mdframed}

\underline{1. Überlege dir, was du über das Restglied von $ f $ sagen kannst und bestimme die Antwort}\\
Das Restglied $ n $-ter Ordnung für $ f $ im Punkt $ x_0 $ ist durch
\begin{align*}
R_n(x) = \frac{f^{n+1}(\xi)}{(n+1)!}(x - x_0)^{(n+1)} 
\end{align*}
für $ \xi \in [x_0,x] $ gegeben. $ R_4(x) $ enthält also die fünfte Ableitung $ f^{(5)} $. Nun gilt
\begin{align*}
f^\prime(x) 
= 16 x^3 + 3
\ \Rightarrow \
f^{\prime \prime}(x) = 48 x^2 
\ \Rightarrow \ 
f^{3}(x) = 96 x
\ \Rightarrow \
f^{4}(x) = 96
\ \Rightarrow \ 
f^{(5)}(x) = 0,
\end{align*}  
wodurch $ R_4(x) = 0  $ für alle $ x $ gilt.



\newpage

\subsection*{\frage{8}{3}}
Für eine Funktion $ f $ ist die Elastizität $ \varepsilon_f(x) $ gegeben durch:
\begin{align*}
\varepsilon_f(x) =x \ \ln (x) + e^{3 \ x}.
\end{align*}
Sei $ g $ die Funktion definiert durch $ g(x) = f( a \ x) $ für $ a > 0 $.\\
\\
Dann gilt:
\renewcommand{\labelenumi}{(\alph{enumi})}
\begin{enumerate}
	\item 
	$ \varepsilon_g(x) = x \ \ln(x)  + e^{3 \ x}$.
	\item
	$ \varepsilon_g(x) =a \  x \ \ln(x)  +a \   e^{3 \ x} $.
	\item
	$ \varepsilon_g(x) =\frac{x}{a} \ \ln(x)  +\frac{x}{a}   e^{3 \ x} $.
	\item
	Keine der obigen Antworten ist richtig.
\end{enumerate}
\ \\
\textbf{Lösung:}
\begin{mdframed}
\underline{\textbf{Vorgehensweise:}}
\renewcommand{\labelenumi}{\theenumi.}
\begin{enumerate}
\item Berechne mithilfe der Definition der Elastizität die korrekte Antwort.
\end{enumerate}
\end{mdframed}

\underline{1. Finde mithilfe der Definition der Elastizität die korrekte Antwort}\\
Mit der Kettenregel erhalten wir:
\begin{align*}
\varepsilon_g (x)
&= 
 \frac{g^\prime(x)}{g(x)} x\\
&=
 \frac{ f^\prime (ax)}{f(ax)} ax \\
&=
ax \frac{ f^\prime (ax)}{f(ax)}  \\
&=
(a \cdot x) \frac{ f^\prime (ax)}{f(ax)}\\
&=  \varepsilon_f(ax)
=ax \ln(ax)) + e^{3ax}
\end{align*}
Damit ist die Antwort (d) korrekt.

\newpage