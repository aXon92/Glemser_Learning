\section*{Aufgabe 3 (25 Punkte)}
\vspace{0.4cm}
\subsection*{\frage{1}{2}}
Die Aussage
\begin{align*}
(A \ \wedge \ B) \ \vee \ (B \ \vee \ A)
\end{align*}
hat die Wahrheitstabelle
\renewcommand{\labelenumi}{(\alph{enumi})}
\begin{enumerate}
\item 
\ 
\ 
\
\
\
\begin{tabular}{cllll}
\hline
\multicolumn{1}{c|}{$A$} & \multicolumn{4}{l}{$W$ $W$ $F$ $F$} \\
\multicolumn{1}{c|}{$B$} & \multicolumn{4}{l}{$W$ $F$ $W$ $F$} \\ \hline
\multicolumn{1}{c|}{$(A \ \wedge \ B) \ \vee \ (B \ \vee \ A)$} & \multicolumn{4}{l}{$W$ $W$ $W$ $W$}  \\ \hline
\end{tabular}


\item
\ 
\ 
\
\
\
\begin{tabular}{cllll}
\hline
\multicolumn{1}{c|}{$A$} & \multicolumn{4}{l}{$W$ $W$ $F$ $F$} \\
\multicolumn{1}{c|}{$B$} & \multicolumn{4}{l}{$W$ $F$ $W$ $F$} \\ \hline
\multicolumn{1}{c|}{$(A \ \wedge \ B) \ \vee \ (B \ \vee \ A)$} & \multicolumn{4}{l}{$W$ $W$ $W$ $F$}  \\ \hline
\end{tabular} 
\item
\ 
\ 
\
\
\
\begin{tabular}{cllll}
\hline
\multicolumn{1}{c|}{$A$} & \multicolumn{4}{l}{$W$ $W$ $F$ $F$} \\
\multicolumn{1}{c|}{$B$} & \multicolumn{4}{l}{$W$ $F$ $W$ $F$} \\ \hline
\multicolumn{1}{c|}{$(A \ \wedge \ B) \ \vee \ (B \ \vee \ A)$} & \multicolumn{4}{l}{$F$ $F$ $F$ $W$}  \\ \hline
\end{tabular} 
\item 
\ 
\ 
\
\
\
\begin{tabular}{cllll}
\hline
\multicolumn{1}{c|}{$A$} & \multicolumn{4}{l}{$W$ $W$ $F$ $F$} \\
\multicolumn{1}{c|}{$B$} & \multicolumn{4}{l}{$W$ $F$ $W$ $F$} \\ \hline
\multicolumn{1}{c|}{$(A \ \wedge \ B) \ \vee \ (B \ \vee \ A)$} & \multicolumn{4}{l}{$W$ $F$ $F$ $F$}  \\ \hline
\end{tabular}
\end{enumerate}
\textbf{Lösung:}
\begin{mdframed}
\underline{\textbf{Vorgehensweise:}}
\renewcommand{\labelenumi}{\theenumi.}
\begin{enumerate}
\item Stelle die Wahrheitstafel auf oder überlege dir eine äquivalente Aussage.
\end{enumerate}
\end{mdframed}

\underline{1. Stelle die Wahrheitstafel auf oder überlege dir eine äquivalente Aussage}\\
Durch
\begin{center}
\begin{tabular}{cllll}
\hline
\multicolumn{1}{c|}{$A$} & \multicolumn{4}{l}{$W$ $W$ $F$ $F$} \\
\multicolumn{1}{c|}{$B$} & \multicolumn{4}{l}{$W$ $F$ $W$ $F$} \\ \hline
\multicolumn{1}{c|}{$A \ \wedge \ B $} & \multicolumn{4}{l}{$W$ $F \ $  $F$ $F$}  \\ \hline
\multicolumn{1}{c|}{$ B \ \vee \ A$} & \multicolumn{4}{l}{$W$ $W$ $W$ $F$}  \\ \hline
\multicolumn{1}{c|}{$(A \ \wedge \ B) \ \vee \ (B \ \vee \ A)$} & \multicolumn{4}{l}{$W$ $W$ $W$ $F$}  \\ \hline
\end{tabular} 
\end{center}
erhalten wir (b) als korrekte Antwort.
Hierbei ist $\wedge$ die UND-Verknüpfung.
Diese wird nur wahr ($W$), wenn $A$ und $B$ wahr sind.
Für die ODER-Verknüpfung verwenden wir das Symbol $\vee$.
Diese wird wahr, wenn entweder $A$ oder $B$ oder beide wahr sind.
\\
Wir erhalten 
\begin{align*}
&(A  \wedge  B)  \vee  (B  \vee  A)\\
\Leftrightarrow
&(A \vee (B \vee A) ) \wedge (B \vee ( B \vee A))\\
\Leftrightarrow
&(A \vee B)  \wedge (A \vee B)\\
\Leftrightarrow
&A \vee B
\end{align*}
durch das Distributivgesetz für Aussagenlogik eine äquivalente Aussage.
Wir können aber auch direkt sehen, dass $A \wedge B$ keinen Einfluss auf unsere Aussage hat.\\
Somit ist Antwort (b) korrekt.

\newpage



\subsection*{\frage{2}{3}}
Gegeben ist die Folge $ \lbrace a_n \rbrace_{n \in \mathbb{N}}$
definiert durch
\begin{align*}
a_n = \frac{e  2^n + n}{n^e + 2n}.
\end{align*}
\renewcommand{\labelenumi}{(\alph{enumi})}
\begin{enumerate}
\item $ \lbrace a_n \rbrace_{n \in \mathbb{N}}$ hat den Grenzwert $0$.
\item $ \lbrace a_n \rbrace_{n \in \mathbb{N}}$ hat den Grenzwert $1$.
\item $ \lbrace a_n \rbrace_{n \in \mathbb{N}}$ hat den Grenzwert $e$.
\item $ \lbrace a_n \rbrace_{n \in \mathbb{N}}$ ist divergent.
\end{enumerate}
\ \\
\textbf{Lösung:}
\begin{mdframed}
\underline{\textbf{Vorgehensweise:}}
\renewcommand{\labelenumi}{\theenumi.}
\begin{enumerate}
\item Überlege dir, ob diese Folge konvergent sein kann.
\end{enumerate}
\end{mdframed}

\underline{1. Überlege dir, ob diese Folge konvergent sein kann}\\
Wir könnten direkt sagen, dass jede Expontialfolge der Form
$a^n$ mit $a > 1$ schneller wächst als jede Potenzfolge.
Demzufolge kann diese Folge nur divergent sein.
Divergent bedeutet, dass diese Folge keinen Grenzwert besitzt.
Durch
\begin{align*}
\lim \limits_{n \rightarrow \infty}
\frac{e  2^n + n}{n^e + 2n}
=
\lim \limits_{n \rightarrow \infty}
\frac{n^e(\frac{e  2^n}{n^e}  + \frac{n}{n^e})}{n^e (1 + 2\frac{n}{n^e})}
=
\lim \limits_{n \rightarrow \infty}
\frac{\frac{e  2^n}{n^e}  + \frac{n}{n^e}}{1 + 2\frac{n}{n^e}}
=
\infty
\end{align*}
können wir dies noch untermauern.\\
Somit Antwort (d) korrekt.
\newpage

\subsection*{\frage{3}{2}}
Der Grenzwert 
\begin{align*}
\lim \limits_{x \rightarrow \frac{\pi}{3}}
\frac{\sin(3x)}{1 - 2 \ cos(x)}
\end{align*}
ist
\renewcommand{\labelenumi}{(\alph{enumi})}
\begin{enumerate}
\item $0$.
\item $- \sqrt{3}$
\item $\frac{1}{3}$
\item $3 \pi$.
\end{enumerate}
\ \\
\textbf{Lösung:}
\begin{mdframed}
\underline{\textbf{Vorgehensweise:}}
\renewcommand{\labelenumi}{\theenumi.}
\begin{enumerate}
\item Bestimme den Grenzwert durch die Regel von de l'H\^{o}pital .
\end{enumerate}
\end{mdframed}

\underline{1. Bestimme den Grenzwert durch die Regel von de l'H\^{o}pital }\\
Wegen
\begin{align*}
\sin \left(3 \cdot \frac{\pi}{3} \right) &= \sin(\pi) = 0\\
1 - 2 \ \cos \left( \frac{\pi}{3} \right)
&= 1 - 2 \cdot \frac{1}{2}
= 1- 1 = 0 
\end{align*}
können wir die Regel von l'H\^{o}pital anwenden.
Wir erhalten durch
\begin{align*}
\lim \limits_{x \rightarrow \frac{\pi}{3}}
\frac{\sin(3x)}{1 - 2 \ \cos(x)}
= \lim \limits_{x \rightarrow \frac{\pi}{3}}
\frac{3 \ \cos(3x)}{ 2 \ \sin(x)}
= 
\frac{-3 }{ 2 \ \frac{1}{2} \sqrt{3}}
= \frac{-3}{\sqrt{3}}
= - \sqrt{3}
\end{align*}
den Grenzwert.
\\
Somit ist Antwort (b) korrekt.
\newpage

\subsection*{\frage{4}{3}}
Ein Projekt benötigt ein anfängliches Investment in Höhe von $2'000'000$ CHF und zahlt in $5$ Jahren CHF $2'500'000$ aus.
Das Projekt hat den höchsten Nettobarwert für einen jährlichen
Zinssatz $i$ von
\renewcommand{\labelenumi}{(\alph{enumi})}
\begin{enumerate}
\item $i = 2.25 \%$.
\item $i = 4.56 \%$
\item $i = 6.65 \%$
\item Der Nettobarwert ist unabhängig vom jährlichen Zinssatz $i$.
\end{enumerate}
\textbf{Lösung:}
\begin{mdframed}
\underline{\textbf{Vorgehensweise:}}
\renewcommand{\labelenumi}{\theenumi.}
\begin{enumerate}
\item Rufe dir die Formel des Nettobarwerts in Erinnerung.
\item Untersuche die Auswirkung des Zinssatzes. 
\end{enumerate}
\end{mdframed}

\underline{1. Rufe dir die Formel des Nettobarwerts in Erinnerung}\\
Allgemein lässt sich der Nettobarwert nach $T$ Jahren durch
\begin{align*}
NPV =
-C_0 + \sum \limits_{t=0}^T \frac{C_t}{(1+i)^t}
\end{align*}
berechnen.
\\
\\

\underline{2. Untersuche die Auswirkung des Zinsatzes}\\
Eingesetzt erhalten wir
\begin{align*}
NPV
= -2'000'000 + 
 \frac{2'500'000}{(1+i)^5}
\end{align*}
und erkennen, dass der Wert größer wird, wenn der Nenner kleiner wird.\\
\\
Somit ist Antwort (a) korrekt.
\newpage

\subsection*{\frage{5}{4}}
Der Term
\begin{align*}
\ln \left( \frac{4a}{b} \right)
-
\ln \left( \frac{b}{2a} \right)
\quad
(a > 0 , b > 0)
\end{align*}
ist äquivalent zu
\renewcommand{\labelenumi}{(\alph{enumi})}
\begin{enumerate}
\item $0$.
\item $- \ln(2)$
\item $2  \ln \left( \frac{2 \sqrt{2}  a}{b} \right)$
\item $\ln \left( \frac{4  a}{b} - \frac{b}{2  a } \right)$.
\end{enumerate}
\ \\
\textbf{Lösung:}
\begin{mdframed}
\underline{\textbf{Vorgehensweise:}}
\renewcommand{\labelenumi}{\theenumi.}
\begin{enumerate}
\item Forme den Term mit Hilfe der Logarithmusregeln um.
\end{enumerate}
\end{mdframed}

\underline{1. Forme den Term mit Hilfe der Logarithmusregeln um}\\
Durch Anwenden der Logarithmusregeln erhalten wir:
\begin{align*}
\ln \left( \frac{4a}{b} \right)
-
\ln \left( \frac{b}{2a} \right)
&= 
\ln \left(\frac{\frac{4a}{b}}{\frac{b}{2a}} \right)\\
&= \ln \left( \frac{4a}{b} \cdot \frac{2a}{b} \right)\\
&= \ln \left( \frac{8  a^2}{b^2} \right)\\
&= \ln \left( \frac{( 2 \sqrt{2}  a)^2}{b^2} \right)\\
&= \ln \left( \frac{ 2 \sqrt{2}  a}{b} \right)^2\\
&= 2 \ \ln \left( \frac{ 2 \sqrt{2}  a}{b} \right)
\end{align*}
Somit ist Antwort (c) korrekt.

\newpage

\subsection*{\frage{6}{5}}
Gegeben ist die Funktion 
\begin{align*}
f \ : \ D_f \to \mathbb{R}, \ \
x \mapsto y=
\begin{cases}
\frac{x^2-1}{x^2 - 2x -3} , &\quad \text{für} \ x\neq -1, \ x \neq 3 \\
\quad \ \ \frac{1}{2}, &\quad \text{für} \ x = -1 \\
\quad -2, &\quad \text{für} \ x = 3
\end{cases}
\end{align*}
\renewcommand{\labelenumi}{(\alph{enumi})}
\begin{enumerate}
\item $f$ ist überall stetig.
\item $f$ ist nur in $x = -1 $ unstetig.
\item $f$ ist nur in $x = 3 $ unstetig.
\item $f$ ist nur in $x = -1 $ und $x = 3 $ unstetig.
\end{enumerate}
\ \\
\textbf{Lösung:}
\begin{mdframed}
\underline{\textbf{Vorgehensweise:}}
\renewcommand{\labelenumi}{\theenumi.}
\begin{enumerate}
\item Rufe dir in Erinnerung, was Stetigkeit bedeutet.
\item Untersuche die Stetigkeit in $-1$.
\item Untersuche die Stetigkeit in $3$.
\end{enumerate}
\end{mdframed}
\underline{1. Rufe dir in Erinnerung, was Stetigkeit bedeutet}\\
Eine Funktion $f \ : \ \mathbb{R} \to \mathbb{R}$ heißt stetig in $x_0$,
falls 
\begin{align*}
x \rightarrow x_0 \ \Rightarrow \ f(x) \rightarrow f(x_0)
\end{align*}
gilt.
Das heißt: Wenn $x$ gegen $x_0$ geht, muss $f(x)$ gegen $f(x_0)$ gehen.
Wir müsssen also
\begin{align*}
\lim \limits_{x \rightarrow x_0} f(x)
\end{align*}
bestimmen um die Stetigkeit in $x_0 = -1$ oder $x_0=3$ zu überprüfen.
In allen anderen Punkten ist $f$ bereits stetig.\\
\\

\underline{2. Untersuche die Stetigkeit in $-1$}\\
Wir beobachten zunächst, dass
\begin{align*}
x^2 -2x -3 = 0
\Leftrightarrow
x_{\nicefrac{1}{2}} &= \frac{-(-2) \pm \sqrt{(-2)^2 - 4\cdot 1 \cdot (-3)}}{2 \cdot 1}\\
&=
\frac{2 \pm \sqrt{4 + 12}}{2}\\
&=
\frac{2 \pm \sqrt{16}}{2}\\
&=
\frac{2 \pm 4}{2} \\
&=
1 \pm 2\\
\Leftrightarrow
x_1 &= -1, \ \ x_2 = 3
\end{align*}
mit der $abc$-Formel folgt.
Damit erhalten wir
\begin{align*}
f(x) = \frac{x^2 -1}{x^2 -2x -3}
= \frac{(x-1) (x+1)}{(x+1)(x-3)}
= \frac{x-1}{x-3}
\end{align*}
für $x \neq -1$ und $x \neq 3$.
Also folgt durch
\begin{align*}
\lim \limits_{x \rightarrow -1} f(x)
=
\lim \limits_{x \rightarrow -1} \frac{x-1}{x-3}
= 
\frac{-2}{-4}
= 
\frac{1}{2}
\end{align*}
die Stetigkeit in $-1$.
\\
\\
\underline{3. Untersuche die Stetigkeit in $3$}\\
Wegen 
\begin{align*}
\lim \limits_{x \rightarrow 3} f(x)
= 
\lim \limits_{x \rightarrow 3} \frac{x-1}{x-3}
=
\infty
\end{align*}
ist $f$ in $3$ unstetig.\\
\\
Somit ist Antwort (c) korrekt.

\newpage

\subsection*{\frage{7}{4}}
Gegeben ist die Funktion
\begin{align*}
g(t) = x^t  \cos(2x).
\end{align*}
Welche der folgenden Ableitungen beschreibt die erste Ableitung von $g(t)$?
\renewcommand{\labelenumi}{(\alph{enumi})}
\begin{enumerate}
\item $g^\prime(t) = -2  t  x^{t-1}  \sin(2x)$.
\item $g^\prime(t) = \ln(x)   x^t  \cos(2x) $.
\item $g^\prime(t) = t  x^{t-1}  \cos(2x) - 2 x^t  \sin(2x)$.
\item $g^\prime(t) = \ln(t)  x^t  \cos(2x)$.
\end{enumerate}

\textbf{Lösung:}
\begin{mdframed}
\underline{\textbf{Vorgehensweise:}}
\renewcommand{\labelenumi}{\theenumi.}
\begin{enumerate}
\item Berechne die Ableitung von $g$.
\end{enumerate}
\end{mdframed}
\ \\
\underline{1. Berechne die Ableitung von $g$}\\
Wir müssen beachten, dass die Funktion von $t$ und nicht von $x$ abhängt.
Wir erhalten
\begin{align*}
\frac{\td{ \ }}{\td{t}} x^t = \frac{\td{ \ }}{\td{t}} e^{\ln(x^t)}
= \frac{\td{ \ }}{\td{t}} e^{t \ \ln(x)}
= \ln(x) \ e^{t \ \ln(x)}
= \ln(x) \ x^t
\end{align*}
mit Hilfe der Kettenregel.
Da $\cos(2x)$ ein konstanter Faktor ist, erhalten wir durch
\begin{align*}
g^\prime(t)
= \ln(x) \ x^t \cos(2x)
\end{align*}
die Ableitung nach $t$.
\\
Somit ist Antwort (b) korrekt.

\newpage

\subsection*{\frage{8}{2}}
Welche der folgenden Bedingungen ist hinreichend dafür, dass eine auf $I=(a,b)$ differenzierbare Funktion in
$(x_0,f(x_0))$ (mit $x_0 \in I)$) ein lokales Maximum hat?
\renewcommand{\labelenumi}{(\alph{enumi})}
\begin{enumerate}
\item $f^\prime(x_0) = f^{\prime \prime}(x_0) = 0$ und $f^{\prime \prime \prime}(x_0) < 0$.
\item $f^\prime(x_0) = f^{\prime \prime}(x_0) = 0 = f^{\prime \prime \prime}(x_0) $ und $f^{(4)}(x_0)<0$.
\item $f^\prime(x_0)=0$ und $f^{\prime \prime}(x_0) >0$.
\item $f^{\prime \prime}(x_0) \neq 0$ und $f^{\prime \prime \prime}(x_0)=0$.
\end{enumerate}
\ \\
\textbf{Lösung:}
\begin{mdframed}
\underline{\textbf{Vorgehensweise:}}
\renewcommand{\labelenumi}{\theenumi.}
\begin{enumerate}
\item Wende den passenden Satz der Vorlesung an.
\end{enumerate}
\end{mdframed}

\underline{1. Wende den passenden Satz aus der Vorlesung an}\\
Sei eine Funktion $f \ : \ I \to \mathbb{R}$ mit der Bedingung
\begin{align*}
f^\prime(x_0)= \dots = f^{(n-1)}(x_0) =0, \quad f^{(n)}(x_0) \neq 0
\end{align*}
gegeben.
Dann gilt:
\begin{itemize}
\item Falls $n$ gerade ist, besitzt $f$ in $x_0$ eine Extremstelle.
Für $f^{(n)}(x_0) < 0$ ist diese ein lokales Maximum.
Andernfalls ein lokales Minimum.

\item Falls $n$ ungerade ist, besitzt $f$ in $x_0$ einen Sattelpunkt.
\end{itemize}
Durch diesen Satz können wir Antwort (a) direkt ausschließen.
Aber wir erkennen auch sofort, dass Antwort (b) korrekt ist.