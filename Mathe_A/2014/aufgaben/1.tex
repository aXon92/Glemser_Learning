\vspace{1cm}
\fancyhead[C]{\normalsize\textbf{$\qquad$ Teil I: Offene Aufgaben}}
\renewcommand{\labelenumi}{\theenumi.}

\section*{Aufgabe 1 (25 Punkte)}
\subsection*{\nummer{a}{6}}
Für welche Werte von $a$ ist die Reihe 
\begin{align*}
\sum \limits_{k=0}^\infty \left( \frac{2 a -1}{a+1} \right)^k
\end{align*}
konvergent?\\
Ermitteln sie den zugehörigen Grenzwert.
\\
\\
\textbf{Lösung:}
\begin{mdframed}
\underline{\textbf{Vorgehensweise:}}
\begin{enumerate}
\item Erkenne die Art der Reihe.
\item Erinnere dich, wann diese Art von Reihe konvergent ist.
\item Bestimme $a$.
\item Bestimme den Grenzwert.
\end{enumerate}
\end{mdframed}

\underline{1. Erkenne die Art der Reihe}\\
Wir erkennen, dass Summanden der Form
\begin{align*}
q_a^k = \left( \frac{2 a -1}{a+1} \right)^k
\end{align*}
vorliegen. Damit haben wir eine geometrische Reihe gegeben.\\
\\

\underline{2. Erinnere dich, wann diese Art von Reihe konvergent ist}\\
Die geometrische Reihe
\begin{align*}
\sum \limits_{k=0}^\infty q^k
\end{align*}
ist genau dann konvergent, wenn $|q| < 1$ ist.
Wir müssen die Ungleichung 
\begin{align*}
|q_a| = \left| \frac{2 a -1}{a+1} \right| < 1
\end{align*}
lösen, um $a$ zu bestimmen.\\
\\
\underline{3. Bestimme $a$}\\
Die oben genannte Bedingung können wir nun durch
\begin{align*}
\left| \frac{2 a -1}{a+1} \right| < 1
\Leftrightarrow
\left( \frac{2 a -1}{a+1} \right)^2 < 1^2 = 1
&\Leftrightarrow
\frac{(2a-1)^2}{(a+1)^2} < 1
\Leftrightarrow
\frac{4a^2 -4a +1}{a^2 +2a +1} < 1\\
\Leftrightarrow
4a^2 -4a +1 < a^2 +2a +1
&\Leftrightarrow
3a^2-6a < 0
\Leftrightarrow
3a (a -2)< 0\\
&\Leftrightarrow
a (a -2) < 0
\end{align*}
umformen.
Nun ist $a(a-2)$ eine nach oben geöffnete Parabel, die negativ zwischen den Nullstellen ist.
Damit gilt $a(a-2) < 0$ für $a \in (0,2)$.
Insgesamt haben wir 
\begin{align*}
|q_a| < 1 \Leftrightarrow a \in (0,2)
\end{align*}
gezeigt.
Unsere Reihe konvergiert also für $a \in (0,2)$.\\
\\

\underline{4. Bestimme den Grenzwert}\\
Der allgemeine Grenzwert der geometrischen Reihe ist durch
\begin{align*}
\sum \limits_{k=0}^\infty q^k = \frac{1}{1-q}
\end{align*}
für $|q| < 1 $ gegeben. Damit erhalten wir den Grenzwert
\begin{align*}
\sum \limits_{k=0}^\infty q_a^k 
= \frac{1}{1-q_a}
= \frac{1}{1- \frac{2a-1}{a+1}}
= \frac{1}{\frac{a+1}{a+1} - \frac{2a-1}{a+1}}
= \frac{1}{\frac{-a + 2}{a+1}}
=\frac{a+1}{-a+2}
\end{align*}
für $a \in (0,2)$.

\newpage
\subsection*{\nummer{b}{6}}
Ein Unternehmer wird auf das Ende des Jahres $2021$ in den Ruhestand treten.
Um sich nach seinem Rücktritt eine Rente zu sichern, sieht er vor,
ab einschliesslich $2001$ (bis $2021$) zu jedem Jahresende einen
konstanten Betrag $C$ auf ein Sparkonto (Zinssatz $i = 2\%)$ einzuzahlen.
Als Rentner will er ab $2022$ $15$ Jahre lang zu jedem Jahresende
CHF $40'000$ abheben.\\
Berechnen Sie $C$.
\\
\\
\textbf{Lösung:}
\begin{mdframed}
\underline{\textbf{Vorgehensweise:}}
\begin{enumerate}
\item Überlege dir, ob eine vor- oder nachschüssige Rente gegeben ist.
\item Gebe den Endwert nach $21$-jähriger Einzahlung an.
\item Gebe den Barwert nach $15$-jähriger Auszahlung an.
\item Stelle eine passende Gleichung auf und finde die Lösung.
\end{enumerate}
\end{mdframed}

\underline{1. Überlege dir, ob eine vor- oder nachschüssige Rente gegeben ist}\\
Da die Einzahlungen am Ende des Jahres stattfinden, haben wir eine nachschüssige Rente.
Damit ist der Endwert nach $T$ Jahren durch
\begin{align*}
A_T = \sum \limits_{k=0}^{T-1} C \cdot( 1+ i)^k = C \frac{(1+i)^{ T}- 1}{i}
\end{align*} 
gegeben.\\
Der Barwert nach $T$ Jahren ist durch
\begin{align*}
PV = \sum \limits_{k=1}^T \frac{C}{(1+i)^k}
= C \cdot \frac{1-(1+i)^{-T}}{i}
\end{align*}
gegeben.
\\
\\
\underline{2. Gebe den Endwert nach $21$-jähriger Einzahlung an}\\
Der Endwert nach $21$ Jahren ist durch
\begin{align*}
A_{21} = C \cdot \frac{(1.02)^{21}-1}{0.02}
\end{align*}
gegeben.
Das $C$ ist die Unbekannte nach welcher wir später umformen.
\\
\\
\underline{3. Gebe den Barwert nach $15$-jähriger Auszahlung an}\\
Für den Barwert gilt nun 
\begin{align*}
PV
&=
\sum \limits_{k=1}^{15} \frac{40000}{(1 +0.02)^k}\\ 
PV &= 40'000 \cdot \frac{1-(1.02)^{-15}}{0.02}
\approx 513'970,55,
\end{align*}
wodruch wir nun im nächsten Schritt eine Gleichung aufstellen kann.\\
\\
\underline{4. Stelle eine passende Gleichung auf und finde die Lösung}\\
Wir erhalten also
\begin{align*}
A_{21} &=
C \frac{1.02^{21}-1}{0.02} = 40'000 \cdot \frac{1 - 1.02^{-15}}{0.02}
= PV \approx 513'970,55\\
\Leftrightarrow
C &= \frac{40'000}{1.02^{15}}\frac{1.02^{15}-1}{1.02^{21}}
\approx 19'934,25
%\Leftrightarrow
%PV = C \cdot \frac{(1.02)^21-1}{0.02}
%\Leftrightarrow
%C = \frac{0.02}{(1.02)^21-1} \cdot PV \approx 19'934,25.
\end{align*}
als Lösung.

\newpage
\subsection*{\nummer{c}{3}}
Berechnen Sie
\begin{align*}
\lim \limits_{x \rightarrow 0}
\frac{x^3}{2e^x-x^2-2x-2}.
\end{align*}
\\
\textbf{Lösung:}
\begin{mdframed}
\underline{\textbf{Vorgehensweise:}}
\begin{enumerate}
\item Überlege dir, welcher Satz hier angewendet werden kann.
\item Berechne den Grenzwert.
\end{enumerate}
\end{mdframed}

\underline{1. Überlege dir, welcher Satz hier angewendet werden kann}\\
Wir erhalten 
\begin{align*}
\frac{0}{0},
\end{align*}
wenn wir $0$ in den Zähler und Nenner einsetzen.
Dies ist nicht definiert. 
Aus diesem Grund können wir den Grenzwert nicht direkt ziehen und müssen den Satz von de l'H\^{o}pital anwenden.
Die Frage ist nun, wie oft wir de l'H\^{o}pital nutzen müssen.
Wir setzten nun $f(x) := x^3$ und 
$g(x):= 2e^x -x^2 -2x -2$.
Dann erhalten wir
\begin{align*}
f^\prime (x) =3 x^2 , \ f^{\prime \prime}(x) = 6x ,\
f^{\prime \prime \prime}(x) = 6
\end{align*}
und
\begin{align*}
g^\prime (x) = 2e^x -2x -2, \ g^{\prime \prime}(x) = 2e^x -2 , \
g^{\prime \prime \prime}(x) = 2 e^x.
\end{align*}  
Wir sehen, dass $f$ und $g$ erst ab der dritten Ableitung ungleich null für $x = 0 $ sind.
Also wenden wir de l'H\^{o}pital dreimal an.
\\
\\
\underline{2. Berechne den Grenzwert}\\
Den Grenzwert erhalten wir nun durch dreimaliges Anwenden von de l'H\^{o}pital.
Mit
\begin{align*}
\lim \limits_{x \rightarrow 0} \frac{x^3}{2e^x-x^2-2x-2}
= \lim \limits_{x \rightarrow 0} \frac{3x^2}{2e^x-2x -2}
=\lim \limits_{x \rightarrow 0} \frac{6x}{2e^x-2}
= \lim \limits_{x \rightarrow 0} \frac{6}{2e^x}
=\frac{6}{2	} = 3
\end{align*}
ist $3$ der gesuchte Grenzwert.

\newpage

\subsection*{\nummer{d1}{4}}
Gegeben ist die Funktion
\begin{align*}
f \ : \ D_f \to \mathbb{R}, \ \ \
x \mapsto y = 2^{-2 + \sqrt{\ln(x+2)}}. 
\end{align*}
Ermitteln Sie den Definitionsbereich $D_f$ und den Wertebereich $W_f$ von $f$.\\
\\
\textbf{Lösung:}
\begin{mdframed}
\underline{\textbf{Vorgehensweise:}}
\begin{enumerate}
\item Rufe dir den Definitionsbereiche der Wurzelfunktion und des Logarithmus in Erinnerung.
\item Leite den Definitionsbereich $D_f$ her.
\item Nutze eine Eigenschaft von $2^x$ um den Wertebereich $W_f$ anzugeben.
\item Gebe einen anderen Lösungsweg für den Wertebereich $W_f$ an.
\end{enumerate}
\end{mdframed}

\underline{1. Rufe dir den Definitionsbereiche der Wurzelfunktion und des Logarithmus in Erinnerung }\\
Die Wurzelfunktion besitzt den Definitionsbereich $[0, \infty)$.
Der Logarithmus hat den Definitionsbereich $(0, \infty)$ und Wertebereich $(-\infty, \infty) $.
Darüber hinaus gilt
\begin{align*}
\ln(x) \geq 0
\end{align*}
für alle $x \in [1,\infty)$.  
\\
\\
\underline{2. Leite den Definitionsbereich $D_f$ her}\\
Aus unserer Vorüberlegung wissen wir, dass 
\begin{align*}
x+2 \geq 1
\Leftrightarrow
x \geq -1
\end{align*}
erfüllt sein muss.
Damit ist der vollständige Definitionsbereich durch
\begin{align*}
D_f = [-1, \infty)
\end{align*}
gegeben.
\\
\\
\underline{3. Nutze eine Eigenschaft von $2^x$ um den Wertebereich $W_f$ anzugeben}\\
Wir wissen, dass 
\begin{align*}
x < y \ \Rightarrow \ 2^x < 2^y
\end{align*}
für alle $x,y \in \mathbb{R}$ gilt.
Die Exponentialfunktion zur Basis $2$ ist also streng monoton wachsend.
Nun ist die Wurzel-und Logaritmusfunktion streng monoton wachsend.
Durch
\begin{align*}
f(-1) = 2^{-2+\sqrt{\ln(1)}} &= 2^{-2} = \frac{1}{4}\\
\lim \limits_{x \rightarrow \infty} f(x) &= \infty
\end{align*}
erhalten wir 
\begin{align*}
W_f = \left[\frac{1}{4}, \infty \right)
\end{align*}
als Wertebereich.
Dieser Lösungsweg nutzt leider in gewisser Weise das Wissen aus Aufgaben 1 d2.
Aus diesem Grund werden wir noch einen anderen Weg angeben.\\
\\
\underline{4. Gebe einen anderen Lösungsweg für den Wertebereich $W_f$ an}\\
Wir können den Wertebereich auch durch Äquivalenzumformungen bestimmen.
Durch 
\begin{align*}
1 \leq x+2 < \infty
&\Leftrightarrow
0 \leq \ln(x+2) < \infty\\
&\Leftrightarrow
0 \leq \sqrt{\ln(x+2)} < \infty\\
&\Leftrightarrow
-2 \leq -2 + \sqrt{\ln(x+2)} < \infty\\
&\Leftrightarrow
\frac{1}{4} =2^{-2} \leq 2^{-2 + \sqrt{\ln(x+2)}} < \infty
\end{align*}
erhalten wir wieder
\begin{align*}
W_f = \left[ \frac{1}{4}, \infty \right)
\end{align*}
als Wertebereich.

\newpage

\subsection*{\nummer{d2}{4}}
Gegeben ist die Funktion
\begin{align*}
f \ : \ D_f \to \mathbb{R}, \ \ \
x \mapsto y = 2^{-2 + \sqrt{\ln(x+2)}}. 
\end{align*}
Ist $f$ monoton (Beweis)?
\\
\\
\textbf{Lösung:}
\begin{mdframed}
\underline{\textbf{Vorgehensweise:}}
\begin{enumerate}
\item Mache dir klar, was Monotonie bedeutet.
\item Leite die Funktion ab.
\item Schaue, ob $f^\prime(x)$ immer $\geq 0$ oder $\leq 0$ ist.
\end{enumerate}
\end{mdframed}

\underline{1. Mache dir klar was Monotonie bedeutet}\\
Wir erinnern uns, dass eine Funktion monoton wachsend ist, falls
\begin{align*}
x < y \Rightarrow f(x) \leq f(y)
\end{align*}
gilt. Für fallende Monotonie folgt stattdessen $f(x) \geq f(y)$.
Wir nennen eine Funktion monoton, falls diese monoton wachsen oder fallend ist.
Wichtig ist, dass obige Definition auf dem ganzen Definitionsbereich gelten muss.\\
\\

%\underline{2. Überlege dir, wie sich Verkettung von monotonen Funktionen auf die Monotonie auswirkt}\\
%Wenn wir zwei monotone Funktionen $g$ und $h$ betrachten, so ist auch $g \circ h$ monoton. Dabei ist $g\circ h(x) = g(h(x))$. Wir zeigen dies exemplarisch für zwei monoton wachsende Funktionen. Seien $g$, $h$ monoton wachsende Funktionen. Dann ist mit
%\begin{align*}
%x < y \Rightarrow h(x) \leq h(y) \Rightarrow g(h(x)) \leq g(h(y))
%\end{align*} 
%auch $g\circ h$ monoton wachsend.
%Alle anderen Kombinationen funktionieren sehr ähnlich.
%Wir wissen nun, dass
%\begin{align*}
%-2 + \sqrt{\ln(x+2)}
%\end{align*}
%streng monoton wachsend ist.
%Dieser Ausdruck ist eine Verkettung von monotonen Funktionen plus eine Konstante.
%\\
%\\


\underline{2. Leite die Funktion ab}\\
Zunächst gilt
\begin{align*}
\frac{\td{ \ }}{\td{x}} 2^x
&= 
\frac{\td{ \ }}{\td{x}}e^{x \ln(2)}
=
\ln(2) e^{x\ln(2)} = \ln(2) 2^x\\
\frac{\td{ \ }}{\td{x}} \sqrt{\ln(x+2)}
&= \frac{1}{2 \ln(x+2)} \cdot \frac{1}{x+2}
\end{align*}
mit Hilfe der Kettenregel.
Mit der mehrfachen Anwendung der Kettenregel erhalten wir durch
\begin{align*}
f^{\prime}(x) =
\ln(2) \cdot 2^{-2 + \sqrt{\ln(x+2)}} 
\cdot \frac{1}{2 \sqrt{\ln(x+2)}} \cdot \frac{1}{x+2}
=
\frac{\ln(2) \cdot 2^{-2 + \sqrt{\ln(x+2)}}}{2 \sqrt{\ln(x+2)}} \cdot \frac{1}{x+2}
\end{align*}
die Ableitung.
\\
\\

\underline{3. Schaue, ob $f^\prime(x)$ immer $\geq 0$ oder $\leq 0$ ist}\\
Der Ausdruck
\begin{align*}
\frac{1}{x+2}
\end{align*}
ist positiv für $x \in D_f$.
Die anderen in der Ableitung vorkommenden Terme sind alle positiv.
Damit gilt
\begin{align*}
f^\prime(x) > 0
\end{align*}
für alle $x \in D_f$.
Also ist $f$ streng monoton wachsend.

\newpage

\subsection*{\nummer{d3}{3}}
Gegeben ist die Funktion
\begin{align*}
f \ : \ D_f \to W_f \subset \mathbb{R}, \ \ \
x \mapsto y = 2^{-2 + \sqrt{\ln(x+2)}}. 
\end{align*}
Ermitteln Sie die Umkehrfunktion $f^{-1}$ von $f$.
\\
\\
\textbf{Lösung:}
\begin{mdframed}
\underline{\textbf{Vorgehensweise:}}
\begin{enumerate}
\item Überlege dir, wie der Definitions-und Wertebereich von $f^{-1}$ aussieht.
\item Forme $y = f(x)$ nach $x$ um.
\end{enumerate}
\end{mdframed}

\underline{1. Überlege dir, wie der Definitions-und Wertebereich von $f^{-1}$ aussieht}\\
Unser Ziel ist es, die Funktion $f$ umzukehren.
Dafür müssen wir zunächst den Bildbereich von $f$ auf den Wertebereich einschränken.
Der Bildbereich ist die Menge der von der Funktion erreichten Werte.
Wir betrachten also nur noch die \glqq getroffenen\grqq~Elemente.
Dies ist notwendig. Die Exponentialfunktion hat beispielsweise nur positve Funktionswerte. Damit kann man auch nur positive Werte umkehren.  
Damit ist unsere Funktion durch
\begin{align*}
f \ : \ D_f \to W_f,\ \ \
x \mapsto y = 2^{-2 + \sqrt{\ln(x+2)}}
\end{align*}
gegeben.
Wir können nun jedem Element in $W_f$ eindeutig ein Element aus $D_f$ zuordnen.
%Das rechtfertigt auch den Begriff Umkehrfunktion.
Der Definitionsbereich von $f^{-1}$ ist $W_f$ und der Wertebereich ist $D_f$.\\
\\
\underline{2. Forme $y = f(x)$ nach $x$ um}\\
Durch 
\begin{align*}
y = f(x) 
&\Leftrightarrow
y = 2^{-2 + \sqrt{\ln(x+2)}}\\
&\Leftrightarrow
\ln(y) = \ln \left(2^{-2 + \sqrt{\ln(x+2)}} \right)
= \ln(2) \cdot (-2 + \sqrt{\ln(x+2)})\\
&\Leftrightarrow
\frac{\ln(y)}{\ln(2)} = -2 + \sqrt{\ln(x+2)}\\
&\Leftrightarrow
\frac{\ln(y)}{\ln(2)} +2 =  \sqrt{\ln(x+2)}\\
&\Leftrightarrow
\left(\frac{\ln(y)}{\ln(2)} +2 \right)^2 = \ln(x+2)\\
&\Leftrightarrow
e^{\left(\frac{\ln(y)}{\ln(2)} +2 \right)^2} = x+2 \\
&\Leftrightarrow
x = e^{\left(\frac{\ln(y)}{\ln(2)} +2 \right)^2} -2
\end{align*}
erhalten wir die Umkehrfunktion.
Mit
\begin{align*}
f^{-1} \ : \ W_f \to D_f, \ \ \
x \mapsto y = e^{\left(\frac{\ln(x)}{\ln(2)} +2 \right)^2} -2
\end{align*}
sind wir fertig.

