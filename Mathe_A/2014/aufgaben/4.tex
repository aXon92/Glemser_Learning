\newpage
\section*{Aufgabe 4 (25 Punkte)}
\vspace{0.4cm}
\subsection*{\frage{1}{3}}
Gegeben ist die Funktion 
\begin{align*}
f \ : \ \mathbb{R}_+ \to \mathbb{R},
t \mapsto f(t) = t^4.
\end{align*}
\renewcommand{\labelenumi}{(\alph{enumi})}
\begin{enumerate}
\item Die Wachstumrate $\rho_f(t)$ von $f$ ist streng monoton fallend in $t$.
\item Die Wachstumrate $\rho_f(t)$ von $f$ ist konstant in $t$.
\item Die Wachstumrate $\rho_f(t)$ von $f$ ist streng monoton wachsend in $t$.
\item Die Wachstumrate $\rho_f(t)$ von $f$ ist nicht monoton in $t$.
\end{enumerate}
\ \\
\textbf{Lösung:}
\begin{mdframed}
\underline{\textbf{Vorgehensweise:}}
\renewcommand{\labelenumi}{\theenumi.}
\begin{enumerate}
\item Bestimme die Wachstumsrate $\rho_f$.
\end{enumerate}
\end{mdframed}

\underline{1. Bestimme die Wachstumsrate $\rho_f$}\\
Die Wachstumsrate $\rho_f$ von $f$ ist durch
\begin{align*}
\rho_f(t) = \frac{f^\prime(t)}{f(t)}
\end{align*} 
gegeben.
Damit erhalten wir mit
\begin{align*}
\rho_f(t) = \frac{f^\prime(t)}{f(t)}
= \frac{4  t^3}{t^4}
= \frac{4}{t^3}
\end{align*} 
die Wachstumsrate.
Diese ist auf $\mathbb{R}_+$ streng monoton fallenend.
Also ist Antwort (a) korrekt.

\newpage

\subsection*{\frage{2}{5}}
Gegeben ist die Funktion 
\begin{align*}
f \ : \ \mathbb{R}_+ \to \mathbb{R}_+, \
x \mapsto f(x) = \frac{1}{x} \ x^x.
\end{align*}
\renewcommand{\labelenumi}{(\alph{enumi})}
\begin{enumerate}
\item $f$ hat ein lokales Minimum in $x_0 = 1$.
\item $f$ hat ein lokales Minimum in $x_0 = \ln(2)$.
\item $f$ hat ein lokales Minimum in $x_0 = e$.
\item $f$ hat ein lokales Minimum in $x_0 = e^{-1}$.


\end{enumerate}
\ \\
\textbf{Lösung:}
\begin{mdframed}
\underline{\textbf{Vorgehensweise:}}
\renewcommand{\labelenumi}{\theenumi.}
\begin{enumerate}
\item Bestimme die Ableitung von $f$.
\item Bestimme die Extrempunkte.
\end{enumerate}
\end{mdframed}

\underline{1. Bestimme die Ableitung von $f$}\\
Zunächst bemerken wir, dass
\begin{align*}
x^x = e^{\ln(x^x)} = e^{x  \ln(x)}
\end{align*}
gilt. Damit gilt auch
\begin{align*}
\frac{\td{ \ }}{\td{x}} x^x
&= \frac{\td{ \ }}{\td{x}} e^{x \ln(x)}\\
&=\left(\ln(x) + x \cdot \frac{1}{x}\right)\cdot e^{x\ln(x)}\\
&= ( \ln(x) + 1) \cdot e^{x  \ln(x)}\\
&= ( \ln(x) + 1) \cdot x^x
\end{align*}
durch die Produkt-und Kettenregel.
Mit der Produktregel erhalten wir durch
\begin{align*}
f^\prime(x) &= \frac{1}{x} \ ( \ln(x) + 1) \cdot x^x
- \frac{1}{x^2} \ x^x\\
&= \left(  ( \ln(x) + 1) - \frac{1}{x} \right) \cdot  \frac{1}{x}\cdot x^x\\
\end{align*}
die Ableitung von $f$.\\
\\
\underline{2. Bestimme die Extrempunkte}\\
Wir überprüfen nun die Nullstellen der Ableitung.
Hierfür ist nur
\begin{align*}
\ln(x) + 1 - \frac{1}{x} = 0
\end{align*}
interessant.
Diese Gleichung ist nicht explizit lösbar.
Jedoch können streng monotone Funktionen nur einen Schnittpunkt haben.
Wir wollen also wissen, für welches $x$
\begin{align*}
\ln(x) + 1 = \frac{1}{x}
\end{align*}
gilt.
Durch Überlegen (oder Testen von (a) bis (d)) erhalten wir $x_0 =1$.
Links von $1$ ist die Steigung von $f$ negativ und rechts positiv.
Also haben wir ein Minimum an $x_0 = 1$.\\
\\
Somit ist Antwort (a) korrekt.

\newpage

\subsection*{\frage{3}{3}}
Gegeben ist das Taylorpolynom $4.$ Ordnung der Funktion $f$ im Punkt $x_0 = 0$:
\begin{align*}
P_4(x) = \frac{1}{8}x^4 + \frac{1}{2} x^3 + \frac{3}{2} x^2 + 3x +1
\end{align*}
Welches ist die zugehörige Funktion?
\renewcommand{\labelenumi}{(\alph{enumi})}
\begin{enumerate}
\item $f(x) = e^x$.
\item $f(x) = 2  e^x - 1$.
\item $f(x) = 3  e^x - 2$.
\item $f(x) = 5  e^x - 4$.
\end{enumerate}

\textbf{Lösung:}
\begin{mdframed}
\underline{\textbf{Vorgehensweise:}}
\renewcommand{\labelenumi}{\theenumi.}
\begin{enumerate}
\item Gebe einen allgemeinen Zusammenhang für das $4.$ Taylorpolynom an.
\item Überprüfe deine Lösung.
\item Alternativer Lösungsweg.
\end{enumerate}
\end{mdframed}

\underline{1. Gebe einen allgemeinen Zusammenhang für das $4.$ Taylorpolynom an }\\
Im Allgemeinen gilt:
\begin{align*}
f(0) &= P_4(0)\\
f^\prime(0) &= P_4^\prime(0) 
\end{align*}
Durch
\begin{align*}
f(0) &= P_4(0) = 0\\
f^\prime(0) &= P_4^\prime(0) =3 
\end{align*}
erhalten wir (c) als korrekte Antwort.
Für keine andere Möglichkeit gilt $f^\prime(0) = 3$.
\\
\\
\underline{2. Überprüfe deine Lösung}\\
Für $f(x) = 3e^x -2$ gilt
\begin{align*}
f^{(i)}(0) = 3
\end{align*}
für alle $i \in \mathbb{N}$.
Also ist das vierte Taylorpolynom durch
\begin{align*}
P_4(x) = 1  + 3x + \frac{3}{2!} x^2 + \frac{3}{3!} x^3 + \frac{3}{4!} x^4
= 1 + 3x + \frac{3}{2} x^2 + \frac{1}{2} x^3 + \frac{1}{8} x^4
\end{align*}
gegeben.
\\
\\
\newpage

\underline{3. Alternativer Lösungsweg}\\
Das Taylorpolynom der Ordnung $4.$ von $e^x$ ist durch
\begin{align*}
\frac{1}{4!} x^4 + \frac{1}{3!} x^3 + \frac{1}{2!} x^2  + \frac{1}{1} x +1
= 
\frac{1}{24} x^4 + \frac{1}{6} x^3 + \frac{1}{2} x^2 +x +1
\end{align*}
gegeben. Somit können wir Antwort (a) auschließen.\\
Wir erhalten 
\begin{align*}
\frac{1}{8} x^4  + \frac{1}{2} x^3 + \frac{3}{2} x^2 + 3 x + 3
\end{align*}
durch Multiplizieren von $3$ das Taylorpolynom von $3 \ e^x$.
Wir sehen, dass nur der konstante Term noch nicht passt.
Durch Subtraktion von $2$ erhalten wir mit
\begin{align*}
\frac{1}{8} x^4  + \frac{1}{2} x^3 + \frac{3}{2} x^2 + 3 x + 1
\end{align*}
das Taylorpolynom von $3  e^x - 2$.
Damit ist Antwort (c) korrekt.



\newpage

\subsection*{\frage{4}{3}}
Die Niveaulinien $f(x,y) = c$ der Funktion
\begin{align*}
f(x,y) = \sqrt{x^2 - y^2 -1}
\end{align*}
für $c > 1$
\renewcommand{\labelenumi}{(\alph{enumi})}
\begin{enumerate}
\item sind Parabeln
\item sind Hyperbeln.
\item sind Ellipsen.
\item sind keine Kurven der unter (a) - (c) genannten Art.
\end{enumerate}


\textbf{Lösung:}
\begin{mdframed}
\underline{\textbf{Vorgehensweise:}}
\renewcommand{\labelenumi}{\theenumi.}
\begin{enumerate}
\item Stelle eine Gleichung auf und Forme sie zu einer bekannten Gleichung um.
\end{enumerate}
\end{mdframed}

\underline{1. Stelle eine Gleichung auf und Forme sie zu einer bekannten Gleichung um}\\
Wir wollen die Gleichung 
\begin{align*}
\sqrt{x^2 - y^2 -1} = c
\end{align*}
für $c > 1 $ auf ihre geometrische Eigenschaft untersuchen.
Wir erhalten durch
\begin{align*}
\sqrt{x^2 - y^2 -1} = c
\Leftrightarrow
x^2 - y^2 - 1 = c^2
\Leftrightarrow
x^2 - y^2 = c^2 +1
\Leftrightarrow
\frac{x^2}{c^2 + 1} - \frac{y^2}{c^2 + 1} = 1
\end{align*}
eine Hyperbelgleichung.

\newpage

\subsection*{\frage{5}{3}}
Gegeben ist die Funktion 
\begin{align*}
f(x,y) = \left( x^a y^{1-a} + x^{-3} y^4 \right)^{0.5}, \ \ \
(x > 0, y>0,a>0)
\end{align*}
\renewcommand{\labelenumi}{(\alph{enumi})}
\begin{enumerate}
\item $f$ ist homogen vom Grad $0.5$.
\item $f$ ist linear homogen.
\item $f$ ist homogen vom Grad $a$.
\item $f$ ist nicht homogen.
\end{enumerate}
\ \\
\textbf{Lösung:}
\begin{mdframed}
\underline{\textbf{Vorgehensweise:}}
\renewcommand{\labelenumi}{\theenumi.}
\begin{enumerate}
\item Überlege dir, wann eine Funktion homogen ist.
\item Zeige allgemein, ob die Funktion homogen oder nicht homogen ist.
\end{enumerate}
\end{mdframed}

\underline{1. Überlege dir, wann eine Funktion homogen ist} \\
Eine Funktion $f \ : \ D_f  \to \mathbb{R}$ heißt homogen vom Grad $n$, falls
\begin{align*}
f(\lambda x, \lambda y) = \lambda^n \cdot f(x,y)
\end{align*}
für alle $x \in D_f$ und $\lambda > 0$ gilt.
\\
\\

\underline{2. Zeige allgemein, ob die Funktion homogen oder nicht homogen ist}\\
Wir wählen ein beliebiges $\lambda >0 $. 
Für $\lambda = 0 $ erhalten wir die Gleichung $0 = 0$, wodurch wir diesen Fall direkt auschließen können.
Wir erhalten nun durch
\begin{align*}
f( \lambda x , \lambda y )
&=
\left((\lambda x)^a (\lambda y)^{1-a}
+ (\lambda x)^{-3} (\lambda y)^4\right)^{0.5}
= 
\left(\lambda^a x^a \lambda^{1-a} y^{1-a}
+ \lambda^{-3} x^{-3} \lambda^4 y^4)\right)^{0.5}\\
&=
\left(\lambda^1 x^a  y^{1-a}
+ \lambda^1 x^{-3}  y^4)\right)^{0.5}
=
\lambda^{0.5} \left(x^a  y^{1-a}
+  x^{-3}  y^4)\right)^{0.5}
= \lambda^{0.5} f(x,y)
\end{align*}
die Homogenität vom Grad $0.5$.

\newpage

\subsection*{\frage{6}{3}}
Gegeben ist die Funktion 
\begin{align*}
f(x,y) = \frac{1}{3} \ln(x) + \frac{2}{3} \ln(y), \ \ \
(x>0 , y>0 )
\end{align*}
\renewcommand{\labelenumi}{(\alph{enumi})}
\begin{enumerate}
\item $f$ ist homogen vom Grad $0$.
\item $f$ ist linear homogen.
\item $f$ ist homogen vom Grad $\frac{1}{3}$.
\item $f$ ist nicht homogen.
\end{enumerate}
\ \\
\textbf{Lösung:}
\begin{mdframed}
\underline{\textbf{Vorgehensweise:}}
\renewcommand{\labelenumi}{\theenumi.}
\begin{enumerate}
\item Überlege dir, ob diese Funktion homogen sein kann.
\item Untermauere deine Überlegung.
\end{enumerate}
\end{mdframed}

\underline{1. Überlege dir, ob diese Funktion homogen sein kann}\\
Wir rufen uns die Logarithmusregel
\begin{align*}
\ln( \lambda  x ) =\ln(\lambda) + \ln(x)
\end{align*}
für $\lambda, x > 0$.
Unsere Funktion $f$ setzt sich aus zwei Logarithmustermen zusammen.
Aus der Rechenregel sehen wir, dass die Multiplikation zur Summe wird. Damit kann $f$ nicht homogen sein.\\
\\
\underline{2. Untermauere deine Überlegung}\\
Wir rechnen
\begin{align*}
f(\lambda x, \lambda y) 
&= 
\frac{1}{3} \ln(\lambda x) + \frac{2}{3} \ln(\lambda y)
=
\frac{1}{3}( \ln(\lambda) + \ln(x)) + \frac{2}{3} ( \ln(\lambda) + \ln(y))\\
&=
\ln(\lambda) + \frac{1}{3} \ln(x) + \frac{2}{3} \ln(y)
= 
\ln(\lambda) + f(x,y)
\end{align*}
für $\lambda >0 $ nach.
Damit ist $f$ ist nicht homogen\\
\\
Somit ist Antwort (d) korrekt.

\newpage

\subsection*{\frage{7}{3}}
Gegeben ist die Funktion
\begin{align*}
f(x,y) = 
\sqrt{4  x^{0.2} y^{0.4}} + \sqrt[4]{x^{1.6} y^{0.8}}
\ \ \
(x>0, y>0)
\end{align*}
\renewcommand{\labelenumi}{(\alph{enumi})}
\begin{enumerate}
\item $f$ ist homogen vom Grad $0.6$.
\item $f$ ist linear homogen.
\item $f$ ist homogen vom Grad $2.4$.
\item $f$ ist nicht homogen.
\end{enumerate}

\textbf{Lösung:}
\begin{mdframed}
\underline{\textbf{Vorgehensweise:}}
\renewcommand{\labelenumi}{\theenumi.}
\begin{enumerate}
\item Weise nach, ob die Funktion homogen ist oder nicht.
\end{enumerate}
\end{mdframed}

\underline{1. Weise nach, ob die Funktion homogen ist oder nicht}\\
Wir betrachten die einzelnen Summanden der Funktion für $\lambda > 0$. Wir erhalten
\begin{align*}
\sqrt{4 (\lambda x)^{0.2} (\lambda y)^{0.4}}
= \sqrt{4 \ \lambda^{0.2} x^{0.2} \lambda^{0.4} y^{0.4}}
= \sqrt{4 \ \lambda^{0.6} x^{0.2}  y^{0.4}}
= \lambda^{0.3} \ \sqrt{4 \  x^{0.2}  y^{0.4}}
\end{align*}
für den ersten Summand.
Jedoch gilt
\begin{align*}
\sqrt[4]{(\lambda x)^{1.6} (\lambda y)^{0.8}}
=
\sqrt[4]{\lambda^{1.6} x^{1.6} \lambda^{0.8} y^{0.8}}
=
\sqrt[4]{\lambda^{2.4} x^{1.6} y^{0.8}}
=
\lambda^{0.6} \ \sqrt[4]{ x^{1.6} y^{0.8}}
\end{align*}	
für den zweiten Summanden.
Die Potenzen sind unterschiedlich, womit $f$ nicht homogen ist.
Also ist Antwort (d) korrekt.

\newpage

\subsection*{\frage{8}{2}}
Eine differenzierbare Funktion $f(x,y)$ sei homogen vom Grad $k = 0.7$ und es gelte $f(x,y) > 0$ für alle $x,y$.
Dann gilt:
\renewcommand{\labelenumi}{(\alph{enumi})}
\begin{enumerate}
\item $\varepsilon_{f,x}(x_0,y_0) + \varepsilon_{f,y}(x_0,y_0) = 0$ in jedem Punkt $(x_0,y_0)$ des Definitionsbereichs.
\item $\varepsilon_{f,x}(x_0,y_0) + \varepsilon_{f,y}(x_0,y_0) = 0.7$ in jedem Punkt $(x_0,y_0)$ des Definitionsbereichs.
\item $\varepsilon_{f,x}(x_0,y_0) + \varepsilon_{f,y}(x_0,y_0) = 1.4$ in jedem Punkt $(x_0,y_0)$ des Definitionsbereichs.
\item Es lässt sich keine allgemeine Aussage über $\varepsilon_{f,x}(x_0,y_0) + \varepsilon_{f,y}(x_0,y_0)$ machen,
d.h. $\varepsilon_{f,x}(x_0,y_0) + \varepsilon_{f,y}(x_0,y_0)$ ist abhängig von $(x_0,y_0)$.
\end{enumerate}
\textbf{Lösung:}
\begin{mdframed}
\underline{\textbf{Vorgehensweise:}}
\renewcommand{\labelenumi}{\theenumi.}
\begin{enumerate}
\item Überlege dir, was laut Euler für homogene Funktionen vom Grad $k$ gilt.

\end{enumerate}
\end{mdframed}

\underline{1. Überlege dir, was laut Euler für homogene Funktionen vom Grad $k$ gilt }\\
Wir wissen, dass der Zusammenhang
\begin{align*}
k \ f(x_0,y_0) = x_0 f_x(x_0,y_0) + y_0 f_y(x_0,y_0)
\end{align*}
für eine homogene Funktion $f$ vom Grad $k$ gilt.
Da $f(x_0,x_0) >0 $ ist, erhalten wir durch
\begin{align*}
k \ f(x_0,y_0) = x_0 f_x(x_0,y_0) + y_0 f_y(x_0,y_0)
\Leftrightarrow
k  &= x_0 \frac{f_x(x_0,y_0)}{f(x_0,y_0)} + y_0 \frac{f_x(x_0,y_0)}{f(x_0,y_0)}\\
&= \varepsilon_{f,x}(x_0,y_0) + \varepsilon_{f,y}(x_0,y_0)
\end{align*}
(b) als korrekte Antwort.