\begin{Large}
\textbf{Teil I: Offene Aufgaben (50 Punkte)}
\end{Large}
\\
\\
\\
\textbf{Allgemeine Anweisungen für offene Fragen:}
\\
\renewcommand{\labelenumi}{(\roman{enumi})}
\begin{enumerate}
\item
Ihre Antworten müssen alle Rechenschritte enthalten,
diese müssen klar ersichtlich sein.
Verwendung korrekter mathematischer Notation wird erwartet
und fliesst in die Bewertung ein.

\item
Ihre Antworten zu den jeweiligen Teilaufgaben müssen in den dafür vorgesehenen Platz geschrie-
ben werden. Sollte dieser Platz nicht ausreichen, setzen Sie Ihre Antwort auf der Rückseite oder
dem separat zur Verfügung gestellten Papier fort. Verweisen Sie in solchen Fällen ausdrücklich
auf Ihre Fortsetzung. Bitte schreiben Sie zudem Ihren Vor- und Nachnamen auf jeden separaten
Lösungsbogen.

\item
Es werden nur Antworten im dafür vorgesehenen Platz bewertet. Antworten auf der Rückseite
oder separatem Papier werden nur bei einem vorhandenen und klaren Verweis darauf bewertet.

\item
Die Teilaufgaben werden mit den jeweils oben auf der Seite angegebenen Punkten bewertet.

\item
Ihre endgültige Lösung jeder Teilaufgabe darf nur eine einzige Version enthalten.

\item
Zwischenrechnungen und Notizen müssen auf einem getrennten Blatt gemacht werden. Diese
Blätter müssen, deutlich als Entwurf gekennzeichnet, ebenfalls abgegeben werden.
\end{enumerate}

\newpage
\section*{\hfil Aufgaben \hfil}
\vspace{1cm}
\section*{Aufgabe 1 (25 Punkte)}
\vspace{0.4cm}
\subsection*{\nummer{a}{6}}
Für welche Werte von $a$ ist die Reihe 
\begin{align*}
\sum \limits_{k=0}^\infty \left( \frac{2 a -1}{a+1} \right)^k
\end{align*}
konvergent?\\
Ermitteln sie den zugehörigen Grenzwert.
\\
\\
\subsection*{\nummer{b}{6}}
Ein Unternehmer wird auf das Ende des Jahres $2021$ in den Ruhestand treten.
Um sich nach seinem Rücktritt eine Rente zu sichern, sieht er vor,
ab einschliesslich $2001$ (bis $2021$) zu jedem Jahresende einen
konstanten Betrag $C$ auf ein Sparkonto (Zinssatz $i = 2\%)$ einzuzahlen.
Als Rentner will er ab $2022$ $15$ Jahre lang zu jedem Jahresende
CHF $40'000$ abheben.\\
Berechnen Sie $C$.
\\
\\
\subsection*{\nummer{c}{3}}
Berechnen Sie
\begin{align*}
\lim \limits_{x \rightarrow 0}
\frac{x^3}{2e^x-x^2-2x-2}.
\end{align*}
\ \\
\\
\subsection*{\nummer{d1}{4}}
Gegeben ist die Funktion
\begin{align*}
f \ : \ D_f \to \mathbb{R}, \ \ \
x \mapsto y = 2^{-2 + \sqrt{\ln(x+2)}}. 
\end{align*}
Ermitteln Sie den Definitionsbereich $D_f$ und den Wertebereich $W_f$ von $f$.
\ \\
\\
\subsection*{\nummer{d2}{3}}
Gegeben ist die Funktion
\begin{align*}
f \ : \ D_f \to \mathbb{R}, \ \ \
x \mapsto y = 2^{-2 + \sqrt{\ln(x+2)}}. 
\end{align*}
Ist $f$ monoton (Beweis)?
\ \\
\\
\subsection*{\nummer{d3}{3}}
Gegeben ist die Funktion
\begin{align*}
f \ : \ D_f \to W_f \subset \mathbb{R}, \ \ \
x \mapsto y = 2^{-2 + \sqrt{\ln(x+2)}}. 
\end{align*}
Ermitteln Sie die Umkehrfunktion $f^{-1}$ von $f$.
\newpage
\section*{Aufgabe 2 (25 Punkte)}
\subsection*{\nummer{a1}{4}}
Gegeben ist die Funktion
\begin{align*}
f \ : \ D_f \to \mathbb{R}, \ \ \
x \mapsto y = \sin(2x) 
\end{align*}
Bestimmen Sie das Taylorpolynom $P_3(x)$ dritter Ordnung von $f$ im Punkt $x_0 = \pi$.
\ \\
\\
\subsection*{\nummer{a2}{3}}
Gegeben ist die Funktion
\begin{align*}
f \ : \ D_f \to \mathbb{R}, \ \ \
x \mapsto y = \sin(2x) 
\end{align*}
$R_3(x)$ bezeichne das Restglied dritter Ordnung von $f$ in $x_0= \pi$.
Zeigen Sie, dass für $x \in [\pi -0.1, \pi+0.1]$ gilt:
\begin{align*}
|R_3(x)| \leq \frac{1}{10^4}.
\end{align*}
\ \\
\\
\subsection*{\nummer{b}{4}}
Gegeben ist die Funktion
\begin{align*}
f(x,y) = 
\ln(4 \ x^2 + y^2- 16) + \sqrt[3]{25 - x^2 - y^2}. 
\end{align*}
Ermitteln Sie den Definitionsbereich $D_f$ von $f$
und stellen Sie diesen graphisch dar.
\ \\
\\
\subsection*{\nummer{c}{4}}
Gegeben ist die Funktion
\begin{align*}
f(x,y) = \cos(x) \ \sin(y) + a \ (x+3)^2.
\end{align*}
Für welches $a \in \mathbb{R}$ nimmt die Tangentensteigung
an die Niveaulinie im Punkt $(x_0,y_0) = ( 0, \pi) $
den Wert $3$ an?
\ \\
\\
\subsection*{\nummer{d1}{3}}
Gegeben ist die Produktionsfunktion
\begin{align*}
P(K,A)= 
8  \left( \frac{\lambda}{K} +\frac{1}{2A} \right)^{-0.5}
\quad ( \lambda >0 ).
\end{align*}
Berechnen Sie die Grenzerträge $P_K$ und $P_A$
sowie die partiellen Elastizitäten 
$\varepsilon_{P,K}$ und $\varepsilon_{P,A}$.
\ \\
\\
\subsection*{\nummer{d2}{3}}
Gegeben ist die Produktionsfunktion
\begin{align*}
P(K,A)= 
8 \ \left( \frac{\lambda}{K} +\frac{1}{2A} \right)^{-0.5}
\end{align*}
mit $\lambda = 2$.\\
Berechnen Sie das totale Differential $\td{P}$ von $P$
in $(K_0,A_0)= \left( 1, \frac{1}{4} \right)$.\\
Berechnen Sie mit Hilfe des totalen Differentials einen Näherungswert für $P(0.98,0.29)$.
\ \\
\\
\subsection*{\nummer{d3}{4}}
Gegeben ist die Produktionsfunktion
\begin{align*}
P(K,A)= 
8 \ \left( \frac{\lambda}{K} +\frac{1}{2A} \right)^{-0.5}.
\end{align*}
Für welche Werte von $\lambda \in \mathbb{R}_+$ ist die technische Substitutionsrate im Punkt $(K_0,A_0) = \left( 1, \frac{1}{4} \right)$ gleich $-1$.
\newpage


\fancyhead[C]{\normalsize\textbf{$\qquad$ Teil II: Multiple-Choice}}
\begin{Large}
\textbf{Teil II: Multiple-Choice-Fragen (50 Punkte)}
\end{Large}
\\
\\
\\
\textbf{Allgemeine Anweisungen für Multiple-Choice-Fragen:}
\\
\renewcommand{\labelenumi}{(\roman{enumi})}
\begin{enumerate}
\item
Die Antworten auf die Multiple-Choice-Fragen müssen im dafür vorgesehenen Antwortbogen eingetragen werden. Es werden ausschliesslich Antworten auf diesem Antwortbogen bewertet. Der
Platz unter den Fragen ist nur für Notizen vorgesehen und wird nicht korrigiert.

\item
Jede Frage hat nur eine richtige Antwort. Es muss also auch jeweils nur eine Antwort angekreuzt
werden.

\item
Falls mehrere Antworten angekreuzt sind, wird die Antwort mit 0 Punkten bewertet, auch wenn
die korrekte Antwort unter den angekreuzten ist.

\item
Bitte lesen Sie die Fragen sorgfältig.

\end{enumerate}

\newpage



\section*{Aufgabe 3 (25 Punkte) }
\vspace{0.4cm}
\subsection*{\frage{1}{2}}
Die Aussage
\begin{align*}
(A \ \wedge \ B) \ \vee \ (B \ \vee \ A)
\end{align*}
hat die Wahrheitstabelle
\renewcommand{\labelenumi}{(\alph{enumi})}
\begin{enumerate}
\item 
\ 
\ 
\
\
\
\begin{tabular}{cllll}
\hline
\multicolumn{1}{c|}{$A$} & \multicolumn{4}{l}{$W$ $W$ $F$ $F$} \\
\multicolumn{1}{c|}{$B$} & \multicolumn{4}{l}{$W$ $F$ $W$ $F$} \\ \hline
\multicolumn{1}{c|}{$(A \ \wedge \ B) \ \vee \ (B \ \vee \ A)$} & \multicolumn{4}{l}{$W$ $W$ $W$ $W$}  \\ \hline
\end{tabular}


\item
\ 
\ 
\
\
\
\begin{tabular}{cllll}
\hline
\multicolumn{1}{c|}{$A$} & \multicolumn{4}{l}{$W$ $W$ $F$ $F$} \\
\multicolumn{1}{c|}{$B$} & \multicolumn{4}{l}{$W$ $F$ $W$ $F$} \\ \hline
\multicolumn{1}{c|}{$(A \ \wedge \ B) \ \vee \ (B \ \vee \ A)$} & \multicolumn{4}{l}{$W$ $W$ $W$ $F$}  \\ \hline
\end{tabular} 
\item
\ 
\ 
\
\
\
\begin{tabular}{cllll}
\hline
\multicolumn{1}{c|}{$A$} & \multicolumn{4}{l}{$W$ $W$ $F$ $F$} \\
\multicolumn{1}{c|}{$B$} & \multicolumn{4}{l}{$W$ $F$ $W$ $F$} \\ \hline
\multicolumn{1}{c|}{$(A \ \wedge \ B) \ \vee \ (B \ \vee \ A)$} & \multicolumn{4}{l}{$F$ $F$ $F$ $W$}  \\ \hline
\end{tabular} 
\item 
\ 
\ 
\
\
\
\begin{tabular}{cllll}
\hline
\multicolumn{1}{c|}{$A$} & \multicolumn{4}{l}{$W$ $W$ $F$ $F$} \\
\multicolumn{1}{c|}{$B$} & \multicolumn{4}{l}{$W$ $F$ $W$ $F$} \\ \hline
\multicolumn{1}{c|}{$(A \ \wedge \ B) \ \vee \ (B \ \vee \ A)$} & \multicolumn{4}{l}{$W$ $F$ $F$ $F$}  \\ \hline
\end{tabular}
\end{enumerate}
\ \\
\\
\subsection*{\frage{2}{3}}
Gegeben ist die Folge $ \lbrace a_n \rbrace_{n \in \mathbb{N}}$
definiert durch
\begin{align*}
a_n = \frac{e  2^n + n}{n^e + 2n}.
\end{align*}
\renewcommand{\labelenumi}{(\alph{enumi})}
\begin{enumerate}
\item $ \lbrace a_n \rbrace_{n \in \mathbb{N}}$ hat den Grenzwert $0$.
\item $ \lbrace a_n \rbrace_{n \in \mathbb{N}}$ hat den Grenzwert $1$.
\item $ \lbrace a_n \rbrace_{n \in \mathbb{N}}$ hat den Grenzwert $e$.
\item $ \lbrace a_n \rbrace_{n \in \mathbb{N}}$ ist divergent.
\end{enumerate}
\newpage
\subsection*{\frage{3}{2}}
Der Grenzwert 
\begin{align*}
\lim \limits_{x \rightarrow \frac{\pi}{3}}
\frac{\sin(3x)}{1 - 2 \ cos(x)}
\end{align*}
ist
\renewcommand{\labelenumi}{(\alph{enumi})}
\begin{enumerate}
\item $0$.
\item $- \sqrt{3}$
\item $\frac{1}{3}$
\item $3 \pi$.
\end{enumerate}
\ \\
\\
\subsection*{\frage{4}{3}}
Ein Projekt benötigt ein anfängliches Investment in Höhe von $2'000'000$ CHF und zahlt in $5$ Jahren CHF $2'500'000$ aus.
Das Projekt hat den höchsten Nettobarwert für einen jährlichen
Zinssatz $i$ von
\renewcommand{\labelenumi}{(\alph{enumi})}
\begin{enumerate}
\item $i = 2.25 \%$.
\item $i = 4.56 \%$
\item $i = 6.65 \%$
\item Der Nettobarwert ist unabhängig vom jährlichen Zinssatz $i$.
\end{enumerate}
\ \\
\\
\subsection*{\frage{5}{4}}
Der Term
\begin{align*}
\ln \left( \frac{4a}{b} \right)
-
\ln \left( \frac{b}{2a} \right)
\quad
(a > 0 , b > 0)
\end{align*}
ist äquivalent zu
\renewcommand{\labelenumi}{(\alph{enumi})}
\begin{enumerate}
\item $0$.
\item $- \ln(2)$
\item $2  \ln \left( \frac{2 \sqrt{2}  a}{b} \right)$
\item $\ln \left( \frac{4  a}{b} - \frac{b}{2  a } \right)$.
\end{enumerate}
\ \\
\\
\subsection*{\frage{6}{5}}
Gegeben ist die Funktion 
\begin{align*}
f \ : \ D_f \to \mathbb{R}, \ \
x \mapsto y=
\begin{cases}
\frac{x^2-1}{x^2 - 2x -3} , &\quad \text{für} \ x\neq -1, \ x \neq 3 \\
\quad \ \ \frac{1}{2}, &\quad \text{für} \ x = -1 \\
\quad -2, &\quad \text{für} \ x = 3
\end{cases}
\end{align*}
\renewcommand{\labelenumi}{(\alph{enumi})}
\begin{enumerate}
\item $f$ ist überall stetig.
\item $f$ ist nur in $x = -1 $ unstetig.
\item $f$ ist nur in $x = 3 $ unstetig.
\item $f$ ist nur in $x = -1 $ und $x = 3 $ unstetig.
\end{enumerate}
\ \\
\\
\subsection*{\frage{7}{4}}
Gegeben ist die Funktion
\begin{align*}
g(t) = x^t  \cos(2x).
\end{align*}
Welche der folgenden Ableitungen beschreibt die erste Ableitung von $g(t)$?
\renewcommand{\labelenumi}{(\alph{enumi})}
\begin{enumerate}
\item $g^\prime(t) = -2  t  x^{t-1}  \sin(2x)$.
\item $g^\prime(t) = \ln(x)   x^t  \cos(2x) $.
\item $g^\prime(t) = t  x^{t-1}  \cos(2x) - 2 x^t  \sin(2x)$.
\item $g^\prime(t) = \ln(t)  x^t  \cos(2x)$.
\end{enumerate}
\ \\
\\
\subsection*{\frage{8}{2}}
Welche der folgenden Bedingungen ist hinreichend dafür, dass eine auf $I=(a,b)$ differenzierbare Funktion in
$(x_0,f(x_0))$ (mit $x_0 \in I)$) ein lokales Maximum hat?
\renewcommand{\labelenumi}{(\alph{enumi})}
\begin{enumerate}
\item $f^\prime(x_0) = f^{\prime \prime}(x_0) = 0$ und $f^{\prime \prime \prime}(x_0) < 0$.
\item $f^\prime(x_0) = f^{\prime \prime}(x_0) = 0 = f^{\prime \prime \prime}(x_0) $ und $f^{(4)}(x_0)<0$.
\item $f^\prime(x_0)=0$ und $f^{\prime \prime}(x_0) >0$.
\item $f^{\prime \prime}(x_0) \neq 0$ und $f^{\prime \prime \prime}(x_0)=0$.
\end{enumerate}
\newpage

\section*{Aufgabe 4 (25 Punkte)}
\vspace{0.4cm}
\subsection*{\frage{1}{3}}
Gegeben ist die Funktion 
\begin{align*}
f \ : \ \mathbb{R}_+ \to \mathbb{R},
t \mapsto f(t) = t^4.
\end{align*}
\renewcommand{\labelenumi}{(\alph{enumi})}
\begin{enumerate}
\item Die Wachstumrate $\rho_f(t)$ von $f$ ist streng monoton fallend in $t$.
\item Die Wachstumrate $\rho_f(t)$ von $f$ ist konstant in $t$.
\item Die Wachstumrate $\rho_f(t)$ von $f$ ist streng monoton wachsend in $t$.
\item Die Wachstumrate $\rho_f(t)$ von $f$ ist nicht monoton in $t$.
\end{enumerate}
\ \\
\\
\subsection*{\frage{2}{5}}
Gegeben ist die Funktion 
\begin{align*}
f \ : \ \mathbb{R}_+ \to \mathbb{R}_+, \
x \mapsto f(x) = \frac{1}{x} \ x^x.
\end{align*}
\renewcommand{\labelenumi}{(\alph{enumi})}
\begin{enumerate}
\item $f$ hat ein lokales Minimum in $x_0 = 1$.
\item $f$ hat ein lokales Minimum in $x_0 = \ln(2)$.
\item $f$ hat ein lokales Minimum in $x_0 = e$.
\item $f$ hat ein lokales Minimum in $x_0 = e^{-1}$.


\end{enumerate}
\ \\
\\
\subsection*{\frage{3}{3}}
Gegeben ist das Taylorpolynom $4.$ Ordnung der Funktion $f$ im Punkt $x_0 = 0$:
\begin{align*}
P_4(x) = \frac{1}{8}x^4 + \frac{1}{2} x^3 + \frac{3}{2} x^2 + 3x +1
\end{align*}
Welches ist die zugehörige Funktion?
\renewcommand{\labelenumi}{(\alph{enumi})}
\begin{enumerate}
\item $f(x) = e^x$.
\item $f(x) = 2  e^x - 1$.
\item $f(x) = 3  e^x - 2$.
\item $f(x) = 5  e^x - 4$.
\end{enumerate}
\ \\
\\
\subsection*{\frage{4}{3}}
Die Niveaulinien $f(x,y) = c$ der Funktion
\begin{align*}
f(x,y) = \sqrt{x^2 - y^2 -1}
\end{align*}
für $c > 1$
\renewcommand{\labelenumi}{(\alph{enumi})}
\begin{enumerate}
\item sind Parabeln
\item sind Hyperbeln.
\item sind Ellipsen.
\item sind keine Kurven der unter (a) - (c) genannten Art.
\end{enumerate}
\ \\
\\
\subsection*{\frage{5}{3}}
Gegeben ist die Funktion 
\begin{align*}
f(x,y) = \left( x^a y^{1-a} + x^{-3} y^4 \right)^{0.5}, \ \ \
(x > 0, y>0,a>0)
\end{align*}
\renewcommand{\labelenumi}{(\alph{enumi})}
\begin{enumerate}
\item $f$ ist homogen vom Grad $0.5$.
\item $f$ ist linear homogen.
\item $f$ ist homogen vom Grad $a$.
\item $f$ ist nicht homogen.
\end{enumerate}
\ \\
\\
\subsection*{\frage{6}{3}}
Gegeben ist die Funktion 
\begin{align*}
f(x,y) = \frac{1}{3} \ln(x) + \frac{2}{3} \ln(y), \ \ \
(x>0 , y>0 )
\end{align*}
\renewcommand{\labelenumi}{(\alph{enumi})}
\begin{enumerate}
\item $f$ ist homogen vom Grad $0$.
\item $f$ ist linear homogen.
\item $f$ ist homogen vom Grad $\frac{1}{3}$.
\item $f$ ist nicht homogen.
\end{enumerate}
\ \\
\\
\subsection*{\frage{7}{3}}
Gegeben ist die Funktion
\begin{align*}
f(x,y) = 
\sqrt{4  x^{0.2} y^{0.4}} + \sqrt[4]{x^{1.6} y^{0.8}}
\ \ \
(x>0, y>0)
\end{align*}
\renewcommand{\labelenumi}{(\alph{enumi})}
\begin{enumerate}
\item $f$ ist homogen vom Grad $0.6$.
\item $f$ ist linear homogen.
\item $f$ ist homogen vom Grad $2.4$.
\item $f$ ist nicht homogen.
\end{enumerate}
\ \\
\\
\subsection*{\frage{8}{2}}
Eine differenzierbare Funktion $f(x,y)$ sei homogen vom Grad $k = 0.7$ und es gelte $f(x,y) > 0$ für alle $x,y$.
Dann gilt:
\renewcommand{\labelenumi}{(\alph{enumi})}
\begin{enumerate}
\item $\varepsilon_{f,x}(x_0,y_0) + \varepsilon_{f,y}(x_0,y_0) = 0$ in jedem Punkt $(x_0,y_0)$ des Definitionsbereichs.
\item $\varepsilon_{f,x}(x_0,y_0) + \varepsilon_{f,y}(x_0,y_0) = 0.7$ in jedem Punkt $(x_0,y_0)$ des Definitionsbereichs.
\item $\varepsilon_{f,x}(x_0,y_0) + \varepsilon_{f,y}(x_0,y_0) = 1.4$ in jedem Punkt $(x_0,y_0)$ des Definitionsbereichs.
\item Es lässt sich keine allgemeine Aussage über $\varepsilon_{f,x}(x_0,y_0) + \varepsilon_{f,y}(x_0,y_0)$ machen,
d.h. $\varepsilon_{f,x}(x_0,y_0) + \varepsilon_{f,y}(x_0,y_0)$ ist abhängig von $(x_0,y_0)$.
\end{enumerate}