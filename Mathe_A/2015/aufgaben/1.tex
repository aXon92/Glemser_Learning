\vspace{1cm}
\fancyhead[C]{\normalsize\textbf{$\qquad$ Teil I: Offene Aufgaben}}
\renewcommand{\labelenumi}{\theenumi.}
\section*{Aufgabe 1 (26 Punkte)}
\vspace{0.4cm}
\subsection*{(a) ( 5 Punkte )}
Sei $q(x) = \frac{5x+2}{3x-10}$ für $x \in \mathbb{R} \setminus \left\lbrace \frac{10}{3} \right\rbrace$.
\\
\\
Für welche Werte von $x$ konvergiert die Reihe $\lbrace s_n \rbrace_{n \in \mathbb{N}}$
mit $s_n = \sum_{k=0}^{n-1} 3 \cdot q(x)^k$?
\\
\\
\textbf{Lösung:}
\begin{mdframed}
\underline{\textbf{Vorgehensweise:}}
\begin{enumerate}
\item Bestimme die Art der Reihe (geometrische Reihe).
\item Ermittle die Konvergenzbedingung der geometrischen Reihe.
\item Bestimme die reellen $x$, für welche die Reihe konvergiert.
\begin{enumerate}
\item Quadrieren
\item Fallunterscheidung
\end{enumerate} 
\end{enumerate}
\end{mdframed}


\underline{1. Bestimme die Art der Reihe (geometrische Reihe)}\\
Da Konstanten für die Konvergenz unwichtig sind, erhalten wir mit
\begin{align*}
s_n = \sum_{k=0}^{n-1} 3 \cdot q(x)^k = 3 \cdot\sum_{k=0}^{n-1}  q(x)^k
\end{align*}
eine von $x$ abhängige geometrische Reihe.
\\
\\
\underline{2. Ermittle die Konvergenzbedingung der geometrischen Reihe}\\
Unsere geometrische Reihe konvergiert also, wenn
\begin{align}
|q(x)| =\left| \frac{5x+2}{3x-10} \right|   < 1  
\end{align}
erfüllt ist. Die dazu passenden $x$ lassen sich durch verschiedene Varianten bestimmen. Davon werden wir zwei betrachten.\\
\\
\underline{3.(a) Quadrieren}\\
Die Konvergenzbedingung ist äquivalent zu
\begin{align*}
|q(x)|^2 
= \left| \frac{5x+2}{3x-10} \right|^2
= \left( \frac{5x+2}{3x-10} \right)^2
=\frac{(5x+2)^2}{(3x-10)^2} 
< 1^2 =1,
\end{align*}
wodurch wir keine lästige Fallunterscheidung machen müssen.
Der Preis dafür ist jedoch, dass wir mit quadratischen Termen rechnen müssen.
Durch Anwenden der binomischen Formeln erhalten wir
\begin{align*}
\frac{(5x+2)^2}{(3x-10)^2} <1
&\Leftrightarrow
(5x+2)^2 < (3x-10)^2\\
&\Leftrightarrow
25x^2 + 20x +4 < 9x^2 -60x +100\\
&\Leftrightarrow
16x^2 + 80x -96 < 0\\
&\Leftrightarrow
x^2 +5x -6 <0
\end{align*}
und wir müssen nur noch die Nullstellen eines quadratischen Polynoms bestimmen. Wir haben eine nach oben geöffnete Parabel. Also gilt mit der Mitternachtsformel 
\begin{align*}
x^2 +5x -6 = 0 
&\Leftrightarrow
x_{\nicefrac{1}{2}}=\frac{-5 \pm \sqrt{25 -4\cdot 1\cdot(-6)}}{2}
= \frac{-5 \pm \sqrt{49}}{2}
= \frac{-5 \pm 7}{2}
\\
&\Leftrightarrow
x_1 = \frac{-5 + 7}{2} = 1, \quad x_2 = \frac{-5-7}{2} = -6,
\end{align*}
wodurch $x^2+5x-6 < 0$ für $x \in (-6,1)$ folgt.
Damit wissen wir durch
\begin{align*}
|q(x) | < 1 
\Leftrightarrow
\dots
\Leftrightarrow
x^2+5x-6 < 0,
\end{align*}
dass unsere geometrische Reihe $s_n$ für $x \in (-6,1)$ konvergiert.
\\
\\
\underline{3.(b) Fallunterscheidung}\\
Wir können in der Konvergenzbedingung auch die Beträge in den Bruch ziehen.
Das bedeutet
\begin{align*}
|q(x)| =\left| \frac{5x+2}{3x-10} \right| =\frac{|5x+2|}{|3x-10|} < 1,  
\end{align*}
wodurch wir für den Betrag im Zähler und Nenner jeweils zwei Fälle unterscheiden müssen.
Die Definition des Betrags gibt uns durch
\begin{align*}
|5x +2| 
&= 
\begin{cases}
 5x + 2, &\ \text{für} \ 5x +2 \geq 0\\
 -5x-2,  &\ \text{für} \ 5x +2 < 0
\end{cases}
=
\begin{cases}
 5x + 2, &\ \text{für} \ x \geq -\frac{2}{5}\\
 -5x-2,  &\ \text{für} \ x < -\frac{2}{5}
\end{cases}\\
|3x -10|
&=
\begin{cases}
 3x - 10, &\ \text{für} \ 3x - 10 \geq 0\\
 -3x+10,  &\ \text{für} \ 3x - 10 < 0
\end{cases}
=
\begin{cases}
 3x - 10, &\ \text{für} \ x \geq \frac{10}{3}\\
 -3x+10,  &\ \text{für} \ x< \frac{10}{3}
\end{cases}
\end{align*}
drei Fälle. Durch logische Kombination der Fälle der einzelnen Beträge erhalten wir mit 
\begin{align*}
\frac{|5x+2|}{|3x-10|} 
= 
\begin{cases}
\frac{5x+2}{3x-10}, &\ \text{für} \ x \geq \frac{10}{3}\\
\frac{5x+2}{-3x+10} &\ \text{für} \ -\frac{2}{5} \leq x < \frac{10}{3}\\
\frac{-5x-2}{-3x+10} &\ \text{für} \ x < -\frac{2}{5}
\end{cases}
\end{align*}
die drei zu überprüfenden Fälle.
Wenn $x \geq \nicefrac{10}{3}$ ist, gilt sicher $x \geq -\nicefrac{2}{5}$.
Damit haben wir den ersten Fall.
Falls $x \geq \nicefrac{10}{3}$ ist, müssen wir unterscheiden, ob 
$x \geq -\nicefrac{2}{5}$ oder $x < -\nicefrac{2}{5}$ ist.
Daraus erhalten wir den zweiten und dritten Fall.
\begin{enumerate}
\item Im ersten Fall ist $x \geq \nicefrac{10}{3}$ und durch
\begin{align*}
&\ \ \ \ \frac{5x+2}{3x-10} < 1\\ 
&\Leftrightarrow
5x+2 < 3x-10\\
&\Leftrightarrow
2x < -12\\
&\Leftrightarrow
x< -6
\end{align*}
erhalten wir einen Widerspruch. Also besitzt dieser Fall keine Lösung.
\item
Im zweiten Fall ist $-\nicefrac{2}{5} \leq x < \nicefrac{10}{3}$ und durch
\begin{align*}
&\ \ \ \  \frac{5x+2}{-3x+10} < 1\\
&\Leftrightarrow
5x + 2 < -3x +10\\
&\Leftrightarrow
8x < 8\\
&\Leftrightarrow
x < 1
\end{align*}
erhalten wir die Lösungsmenge $\mathbb{L}_2=[-\nicefrac{2}{3},1) $ für diesen Fall.
\item
Im letzten Fall ist $x < -\nicefrac{2}{3}$ und durch
\begin{align*}
&\ \ \ \ \ \frac{-5x-2}{-3x+10} < 1\\
&\Leftrightarrow
-5x -2 < -3x +10\\
&\Leftrightarrow
-12 < 2x \\
&\Leftrightarrow
-6 < x
\end{align*}
erhalten wir die Lösungsmenge $\mathbb{L}_3=(-6,-\nicefrac{2}{3})$ für diesen Fall.
\end{enumerate}
Die Gesamtlösungsmenge für $|q(x)|< 1$ erhalten wir durch $\mathbb{L} = \mathbb{L}_2 \cup \mathbb{L}_3 = (-6,1)$.

\newpage
\subsection*{(b) (7 Punkte)} 
Eine Kreditnehmerin leiht sich $P=2'500'000$ CHF, um ihr Haus zu finanzieren.
Sie kann jährliche Zahlungen in Höhe von $C=50'000$ CHF aufbringen.
Der jährliche Zinssatz liegt bei $i = 1 \%$. Wie lange benötigt die Kundin, um das Darlehen zurückzuzahlen, wenn die Zahlungen zum Ende des Jahres stattfinden? 
\\
\\
\textbf{Lösung:}
\begin{mdframed}
\underline{\textbf{Vorgehensweise:}}
\begin{enumerate}
\item Überlege, ob es sich um eine vor- oder nachschüssige Rente handelt, d.h.
\begin{align*}
PV = C \frac{1-(1+i)^{-T+1}}{i} 
\ \text{oder} \
PV = C \frac{1-(1+i)^{-T}}{i}
\end{align*}
\item Setze die Werte ein und forme nach $T$ um.   
\end{enumerate}
\end{mdframed}

\underline{1. Überlege, ob es sich um eine vor- oder nachschüssige Rente handelt}\\
Da die konstanten Zahlungen gegen Ende des Jahres geleistet werden, handelt es sich um eine nachschüssige Rente. Demnach ist der Barwert nach $T$ Jahren durch
\begin{align*}
PV =  C \frac{1-(1+i)^{-T}}{i}
\end{align*}
gegeben.\\
\\
\underline{2. Setze die Werte ein und forme nach $T$ um}\\
Um die Anzahl $T$ der Jahre zu ermitteln, lösen wir die Gleichung $P = PV$ nach $T$ auf. Zunächst gilt
\begin{align*}
PV = C \frac{1-(1+i)^{-T}}{i} 
= 50'000 \frac{1-1,01^{-T}}{0,01}
= 5'000'000 (1-1,01^{-T}) 
\end{align*}
und mit 
\begin{align*}
P = PV 
&\Leftrightarrow
5'000'000(1-1,01^{-T}) =2'500'000\\
&\Leftrightarrow
2(1-1.01^{-T})=1\\
&\Leftrightarrow
1-1,01^{-T} = \frac{1}{2}\\
&\Leftrightarrow
1,01^{-T} = \frac{1}{2}\\
&\Leftrightarrow
1,01^T = 2\\
&\Leftrightarrow
\ln(1,01^T) = T\ln(1,01) = ln(2)\\
&\Leftrightarrow
T = \frac{\ln(2)}{\ln(1,01)} \approx 69,66
\end{align*}
erhalten wir eine Rückzahlung von circa 70 Jahren.

\newpage

\subsection*{(c) (5 Punkte)}
Berechnen Sie 
\begin{align*}
\lim \limits_{x \rightarrow \infty} \left(\frac{x+3}{x+2}\right)^{3x+6}.
\end{align*}
\\
\\
\textbf{Lösung:}
\begin{mdframed}
\underline{\textbf{Vorgehensweise:}}
\begin{enumerate}
\item Überlege dir, zu welchen bekannten Grenzprozessen eine Ähnlichkeit besteht.
\item Berechne den Grenzwert   
\end{enumerate}
\end{mdframed}

\underline{1.Überlege dir, zu welchen bekannten Grenzprozessen eine Ähnlichkeit besteht}\\
Durch die Umformungen 
\begin{align*}
\frac{x+3}{x+2} = \frac{x+2+1}{x+2} = \frac{x+2}{x+2} + \frac{1}{x+2} = 1 + \frac{1}{x+2}
\end{align*}
und
\begin{align*}
3x+6 = 3(x+2)
\end{align*}
ergibt sich zusammengesetzt
\begin{align*}
\left(1 + \frac{1}{x+2}\right)^{3(x+2)}
=
\left( \left(1 + \frac{1}{x+2}\right)^{x+2} \right)^3. 
\end{align*} 
Daran erkennen wir, dass eine Ähnlichkeit zu dem Grenzprozess
\begin{align*}
e = \lim \limits_{x \rightarrow \infty} \left(1+\frac{1}{x}\right)^x
= \lim \limits_{x \rightarrow \infty} \left(1+\frac{1}{x+2}\right)^{x+2}
\end{align*}
besteht.\\
\\
\underline{2. Berechne den Grenzwert}\\
Unter Ausnutzung der Stetigkeit von $x^3$ erhalten wir mit
\begin{align*}
\lim \limits_{x \rightarrow \infty} \left( \left(1 + \frac{1}{x+2}\right)^{x+2} \right)^3
= \left( \lim \limits_{x \rightarrow \infty} \left(1 + \frac{1}{x+2}\right)^{x+2} \right)^3
= e^3 
\end{align*}
den gesuchten Grenzwert.

\newpage
\subsection*{(d1) (5 Punkte)}
Gegeben sei die Funktion
\begin{align*}
f \ : \ D_f \to \mathbb{R}, \ \ x \mapsto y = \sqrt{\ln(x^2+4x+4)}.
\end{align*}
Bestimmen Sie den Definitionsbereich $D_f$.
\\
\\
\textbf{Lösung:}
\begin{mdframed}
\underline{\textbf{Vorgehensweise:}}
\begin{enumerate}
\item Rufe dir die Defintionsbereiche des Logarithmus und der Wurzelfunktion in Erinnerung.
\item Bestimme den Definitionsbereich.
\end{enumerate}
\end{mdframed}

\underline{1.Rufe dir die Definitionsbereiche des Logaritmus und der Wurzelfunktion in Erinnerung}\\
Der natürliche Logarithmus ist die Umkehrfunktion der Exponentialfunktion.
Diese hat den Wertebereich $(0, \infty)$, womit der Logarithmus auf $(0,\infty)$ definiert ist.
Die Wurzelfunktion besitzt den Definitionsbereich $[0,\infty)$.\\
\\
Der Logarithmus ist auf dem Intervall $[1,\infty)$ größer gleich $0$. Wir erkennen das an
\begin{align*}
e^0 = 1, \ \lim \limits_{x \rightarrow \infty} e^x = \infty,
\ \lim \limits_{x \rightarrow -\infty} e^x = 0
\end{align*}
und der strengen Monotonie der Exponentialfunktion.\\
Wegen der Wurzelfunktion suchen wir die passenden $x \in \mathbb{R}$, sodass
\begin{align*}
\ln(x^2+4x+4) &\geq 0\\
\Leftrightarrow
e^{\ln(x^2+4x+4)} &\geq 1\\
\Leftrightarrow
x^2 +4x +4 &\geq 1\\
\Leftrightarrow
x^2 +4x +3 &\geq 0
\end{align*}
gilt. Da $x^2+4x+4$ eine nach oben oben geöffnete Normalparabel ist, könnten wir die Bedingung 
\begin{align*}
x^2 +4x +4 \geq 1
\end{align*}
auch direkt ablesen.
\\
\\
\underline{2. Bestimme den Definitionsbereich}\\
Mit
\begin{align*}
x^2 +4x+3 =0
\Leftrightarrow
x_{\nicefrac{1}{2}}
= \frac{-4 \pm \sqrt{4^2 - 4 \cdot 1 \cdot 3}}{2} 
= \frac{-4 \pm \sqrt{16 -12 }}{2}
= \frac{-4 \pm \sqrt{4}}{2}
= \frac{-4 \pm 2}{2}
= -2 \pm 1
\end{align*}
erhalten wir $x_1 = -1$ und $x_2 = -3$ als Nullstellen der Parabel. Da diese nach oben geöffnet ist, ergibt sich 
\begin{align*}
D_f = (-\infty,-3]\cup [-1,\infty)
= \mathbb{R} \setminus (-3,-1)
\end{align*}
als Definitionsbereich von $f$.

\newpage

\subsection*{(d2) (4 Punkte)}
Gegeben sei die Funktion
\begin{align*}
f \ : \ D_f \to \mathbb{R}, \ x \mapsto y = \sqrt{\ln(x^2 +4x +4)}. 
\end{align*}
Ist $f$ monoton (Beweis)?
\\
\\
\textbf{Lösung:}
\begin{mdframed}
\underline{\textbf{Vorgehensweise:}}
\begin{enumerate}
\item Bestimme die Monotoniebedingung.
\item Leite die Funktion ab.
\item Bestimme das Monotonieverhalten.
\item Alternativer Lösungsweg. 
\end{enumerate}
\end{mdframed}

\underline{1. Bestimme die Monotoniebedingung}\\
Eine Funktion $g$ heißt monoton wachsend bzw. fallend, falls
\begin{align*}
g^\prime(x) \geq 0 \ \text{bzw.} \ g^\prime(x) \leq 0  
\end{align*}
für alle $x$ aus dem Definitionsbereich gilt. Ändert sich das Vorzeichen der Ableitung ist eine Funtktion nicht monoton.
\\

\underline{2. Leite die Funktion ab}\\
Durch zweifaches Anwenden der Kettenregel erhalten wir mit
\begin{align*}
f^\prime(x) &= \frac{1}{2\sqrt{(\ln(x^2+4x+4)}} \cdot \frac{1}{x^2+4x+4}\cdot (2x+4)\\
&= \frac{2x+4}{2\sqrt{(\ln(x^2+4x+4)}\cdot(x^2+4x+4)}
\end{align*}
die Ableitung von $f$. Nun ist der Definitionsbereich so gewählt, dass $x^2 +4x+4 \geq 1$ für alle $x \in D_f$ (siehe Aufgabe 1 d1). Also ist nur $2x +4 $ für das Vorzeichen verantwortlich.
\\
\\
\underline{3. Bestimme das Monotonieverhalten}\\ 
Wir erkennen recht schnell, dass mit
\begin{align*}
f^\prime(-3) < 0 \ \text{und} \ f^\prime(-1)>0
\end{align*}
ein Widerspruch zu Monotonie vorliegt.
\\
\\
\underline{4. Alternativer Lösungsweg}\\
Wir erinnern uns, dass eine Funktion monoton wachsend ist, falls
\begin{align*}
x < y \Rightarrow f(x) \leq f(y)
\end{align*}
gilt. Für fallende Monotonie folgt stattdessen $f(x) \geq f(y)$.
Wir nennen eine Funktion monoton, falls diese monoton wachsen oder fallend ist.
Wichtig ist, dass obige Definition auf dem ganzen Definitionsbereich gelten muss.
\\
Wenn wir zwei monotone Funktionen $g$ und $h$ betrachten, so ist auch $g \circ h$ monoton. Dabei ist $g\circ h(x) = g(h(x))$. Wir zeigen dies exemplarisch für zwei monoton wachsende Funktionen. Seien $g$, $h$ monoton wachsende Funktionen. Dann ist mit
\begin{align*}
x < y \Rightarrow h(x) \leq h(y) \Rightarrow g(h(x)) \leq g(h(y))
\end{align*} 
auch $g\circ h$ monoton wachsend.
Alle anderen Kombinationen funktionieren sehr ähnlich.
Doch warum das Ganze?
Die Wurzel -und Logarithmusfunktion sind beide monoton wachsend. Deswegen genügt es zu untersuchen, ob $x^2+4x+4$ monoton ist.
\\
Wir haben gesehen, dass es ausreicht
\begin{align*}
h(x) := x^2 + 4x + 4
\end{align*}
zu untersuchen. Wenn man das Bild einer Parabel im Kopf hat, ist $h$ wahrscheinlich nicht monoton. Wir suchen nun einen Widerspruch zur Definition der Monotonie.
Es gilt $-3 < -2 $ und es gilt $h(-3) =1 > 0 = h(-2)$. Also kann $h$ schonmal nicht monoton wachsend sein. Mit $-2 < -1$ erhalten wir $h(-2) =0 < h(-1) = 1$. 
Damit sehen wir, dass die Bedingung für Monotonie verletzt ist.
Also ist $h$ nicht monoton und somit auch $f$ nicht.


\newpage
