\fancyhead[C]{\normalsize\textbf{$\qquad$ Teil II: Multiple-Choice}}
\section*{Aufgabe 3 (25 Punkte)}
\vspace{0.4cm}
\subsection*{\frage{1}{2}}
Gegeben seien die Aussagen
\begin{align*}
A(x) &= \text{\glqq $\frac{x}{4}$ ist eine positive ganze Zahl\grqq}\\
B(x) &= \text{\glqq $x$ ist eine gerade Zahl\grqq}.
\end{align*}
Welche der Aussagen ist wahr:
\renewcommand{\labelenumi}{(\alph{enumi})}
\begin{enumerate}
\item $A(x) \Rightarrow B(x)$.
\item $A(x) \Leftrightarrow B(x)$.
\item $\neg A(x) \Rightarrow B(x)$.
\item $A(x) \Rightarrow \neg B(x)$.
\end{enumerate}
\ \\
\textbf{Lösung:}
\begin{mdframed}
\underline{\textbf{Vorgehensweise:}}
\renewcommand{\labelenumi}{\theenumi.}
\begin{enumerate}
\item Setze Zahlen ein um die Aussagen zu überprüfen.
Es reicht ein Gegenbeispiel für eine Aussage zu finden, um sie zu widerlegen.
\item Zur Übung beweisen wir die richtige Aussage.
\end{enumerate}
\end{mdframed}

\underline{1. Setze Zahlen ein um die Aussagen zu überprüfen}
\renewcommand{\labelenumi}{(\alph{enumi})}
\begin{enumerate}
\item $A(x) \Rightarrow B(x)$ bedeutet:
Wenn $A(x)$ gilt, dann gilt $B(x)$ bzw. aus $A(x)$ folgt $B(x)$.\\
Wir setzen $x=4$ ein, sodass
\begin{equation}
\frac{x}{4}= \frac{4}{4} = 1
\end{equation}
eine positive ganze Zahl ist.
Nun ist $x=4$ gerade, womit die Aussage für dieses $x$ stimmt.
Es gilt
\begin{align*}
1 = \frac{4}{4},
\end{align*}
womit $x = 4 $ und gerade ist.
Vermutlich ist dies die korrekte Aussage.
\item $A(x) \Leftrightarrow B(x)$ bedeutet:
$A(x)$ gilt genau dann, wenn $B(x)$ gilt.\\
Wir erinnern uns, dass
\begin{align*}
A(x) \Leftrightarrow B(x)
\end{align*}
dasselbe wie 
\begin{align*}
A(x) \Rightarrow B(x) \ \text{und} \ A(x) \Leftarrow B(x)
\end{align*}
bedeutet. Also aus $A(x)$ folgt $B(x)$ und aus $B(x)$ folgt $A(x)$.
Wir nehmen die einfachste gerade Zahl $x = 2 $.
Dann gilt 
\begin{align*}
\frac{x}{4} = \frac{2}{4} = \frac{1}{2} = 0,5 \,
\end{align*}
wodurch wir ein Gegenbeispiel für $A(x) \Leftarrow B(x)$ gefunden haben.
\item $\neg A(x) \Rightarrow B(x)$ bedeutet:
Wenn $A(x)$ nicht gilt, dann gilt $B(x)$\\
Das Symbol $\neg$ ist die Negation oder Umkehrung einer Aussage.
Wir haben also 
\begin{align*}
\neg A(x) = \text{\glqq $\frac{x}{4}$ ist keine positve ganze Zahl\grqq}
\end{align*}
als Aussage.
Wir setzen $x= 3$.
Damit ist die Zahl $\nicefrac{x}{4} = \nicefrac{3}{4} = 0.75$ ist keine positive ganze Zahl, aber die $3$ ist nicht gerade.
Dementsprechend kann $B(x)$ nicht erfüllt sein.
Also ist diese Aussage falsch.
\item $A(x) \Rightarrow \neg B(x)$ bedeutet:
Wenn $A(x)$ gilt, so gilt $B(x)$ nicht.\\
Wenn $\nicefrac{x}{4}$ eine positive ganze Zahl ist, ist $x$ durch $4$ teilbar.
Somit ist $x$ auch durch $2$ teilbar.
Womit nicht ungerade sein kann.
Wir wählen wieder $x= 4$ und sehen direkt das $x$ gerade ist.
\end{enumerate}
Wir haben gesehen, dass (b),(c) und (d) falsch sind. 
Also muss (a) korrekt sein.\\
\\
\underline{2. Zur Übung beweisen wir die richtige Aussage}\\
Eine Aussage der Form 
\begin{align*}
A(x) \Rightarrow B(x)
\end{align*}
beweist man, indem man $A(x)$ als gegeben ansieht und $B(x)$.
Sei also $\nicefrac{x}{4}$ eine positive ganze Zahl.
Damit ist $x$ durch $4$ teilbar.
Wir wissen also 
\begin{align*}
x = 4 \cdot y = 2 \cdot 2 \cdot y
\end{align*}
mit $y \in \mathbb{N}$. Damit ist $x$ gerade, da durch $2$ teilbar.
\newpage
\subsection*{\frage{2}{3}}
Gegeben seien die Folgen $\lbrace a_n \rbrace_{ n\in \mathbb{N}}$ und $\lbrace b_n\rbrace_{ n\in \mathbb{N}}$.
Beide Folgen sind konvergent.
Sei $\lbrace c_n\rbrace_{ n\in \mathbb{N}}$ die Folge definiert durch $c_n = a_n - (-1)^n \ b_n$.\\
\\
Dann gilt: 
\renewcommand{\labelenumi}{(\alph{enumi})}
\begin{enumerate}
\item $\lbrace c_n\rbrace_{ n\in \mathbb{N}}$ ist konvergent.
\item $\lbrace c_n\rbrace_{ n\in \mathbb{N}}$ ist divergent.
\item $\lbrace c_n\rbrace_{ n\in \mathbb{N}}$ kann abhängig von $\lbrace a_n\rbrace_{ n\in \mathbb{N}}$ und $\lbrace b_n\rbrace_{ n\in \mathbb{N}}$ konvergent oder divergent sein.
\item $\lbrace c_n\rbrace_{ n\in \mathbb{N}}$ ist konvergent genau dann, wenn
$ \lim \limits_{n  \rightarrow \infty} a_n = \lim \limits_{n  \rightarrow \infty} b_n = 0  $.
\end{enumerate}
\ \\
\textbf{Lösung:}
\begin{mdframed}
\underline{\textbf{Vorgehensweise:}}
\renewcommand{\labelenumi}{\theenumi.}
\begin{enumerate}
\item Überlege  welche Eigenschaften $\lbrace c_n\rbrace_{ n\in \mathbb{N}}$ besitzt.
\item Überprüfe die Eigenschaften und finde gegebenenfalls Gegenbeispiele.
\end{enumerate}
\end{mdframed}

\underline{1. Überlege  welche Eigenschaften $\lbrace c_n\rbrace_{ n\in \mathbb{N}}$ besitzt}\\
Wir wissen, dass $a_n$ und $b_n$ konvergent sind. Demnach sind beide Folgen beschränkt.
Eine Folge heißt beschränkt, falls sie eine obere und untere Schranke besitzt.
Insbesondere bleibt eine beschränkte Folge beschränkt, wenn man sie mit dem Faktor $(-1)^n$ ergänzt. Die Summe zweier beschränkten Folgen ist wiederum beschränkt.
Damit wissen wir, dass $c_n$ beschränkt ist.
Wir betrachten $(-1)^n$ als Beispielfolge.
Diese Folge ist durch $-1$ nach unten und durch $1$ nach oben beschränkt.
Aber leider ist diese Folge divergent, da sie zwischen $-1$ und $1$ hin und herspringt.
\\
\\
\underline{2. Überprüfe die Eigenschaften und finde gegebenenfalls Gegenbeispiele}\\
\renewcommand{\labelenumi}{(\alph{enumi})}
\begin{enumerate}
\item Wähle $a_n = 0$, $b_n = 1$.
Dann ist durch
\begin{align*}
c_n = a_n - (-1)^n \ b_n = -(-1)^n = (-1)^{n+1}
\end{align*}
ein Gegenbespiel gefunden, da $c_n$ zwischen $-1$ und $1$ hin und her springt.
\item Wähle $a_n = 1$, $b_n = 0$.
Dann ist durch
\begin{align*}
c_n = a_n - (-1)^n \ b_n 
= 1 - (-1)^n \cdot 0 = 1
\end{align*}
ein Gegenbeispiel gefunden, da konstante Folgen konvergent sind.

\item
Diese Antwort ist richtig. Wir betrachten
\begin{align*}
c_n = a_n - (-1)^n \ b_n 
\end{align*}
genauer.
Der einzige Term an dem die Konvergenz scheitern kann ist $b_n$.
Da $b_n$ konvergent ist, können wir 
\begin{align*}
(-1)^n  \ b_n \ \text{konvergent} 
\Leftrightarrow
\lim \limits_{n \rightarrow \infty} b_n = 0
\end{align*}
als Hilfsmittel verwenden.
Wenn $b_n \rightarrow a \neq 0 $ gilt, dann springt $(-1)^n \ b_n $ zwischen $-a$ und $a$ hin und her.
Also sehen wir, dass die Konvergenz von der Wahl von $b_n$ abhängt.

\item
Vergleiche mit (b).
\end{enumerate}
Also muss Antwort (c) korrekt sein.
\newpage

\subsection*{\frage{3}{3}}
Die Folge $\lbrace a_n \rbrace_{n \in \mathbb{N}}$ ist konvergent und $a_n > 0$
für alle $n$.
Sei $\lbrace b_n \rbrace_{n \in \mathbb{N}}$ die Folge definiert durch
$b_n = \ln(a_n)$ für $n \in \mathbb{N}$.
Dann gilt:\\

\renewcommand{\labelenumi}{(\alph{enumi})}
\begin{enumerate}
\item $\lbrace b_n \rbrace_{n \in \mathbb{N}}$ ist beschränkt und monoton.
\item $\lbrace b_n \rbrace_{n \in \mathbb{N}}$ ist beschränkt oder monoton.
\item $\lbrace b_n \rbrace_{n \in \mathbb{N}}$ ist beschränkt und konvergent.
\item Keine der vorangegangenen Antworten ist richtig.
\end{enumerate}

\ \\
\textbf{Lösung:}
\begin{mdframed}
\underline{\textbf{Vorgehensweise:}}
\renewcommand{\labelenumi}{\theenumi.}
\begin{enumerate}
\item Überlege dir, wie eine Folge konstruiert werden kann, welche allen Eigenschaften widerspricht. 
\end{enumerate}
\end{mdframed}

\underline{1. Überlege dir, wie eine Folge konstruiert werden kann, welche allen Eigenschaften widerspricht}\\
Wir wählen 
\begin{align*}
a_n =
\begin{cases}
\frac{1}{3}, &\ \text{falls} \ n = 1 \\
\frac{1}{n}, &\ \text{falls} \ n \geq 2
\end{cases}
\end{align*}
als Folge. Diese Folge ist nicht monoton, weil
\begin{align*}
a_1 < a_2 \ \text{und} \ a_1 > a_4
\end{align*}
ist. Damit kann 
\begin{align*}
b_n = \ln(a_n)
\end{align*}
aufgrund der strengen Monotonie des Logarithmus auch nicht monoton sein.
Nun zeigen wir noch das $b_n$ unbeschränkt ist.
Mit 
\begin{align*}
b_n = \ln\left( \frac{1}{n} \right) = \ln(1) - \ln(n) = - \ln(n)
\stackrel{n \rightarrow \infty}{\rightarrow } -\infty
\end{align*}
erhalten wir die Unbeschränktheit.
Damit haben wir eine Folge gefunden, sodass keine der Eigenschaften erfüllt ist.\\
Also ist (d) die richtige Antwort.

\newpage

\subsection*{\frage{4}{2}}
Ein Projekt benötigt ein anfängliches Investment in Höhe von $2'000'000$ CHF
und zahlt $1'000'000$ CHF in $10$ Jahren, $1'500'000$ CHF in $20$ Jahren sowie $1'000'000$ CHF in $40$ Jahren aus.
Das Projekt besitzt den höchsten Nettobarwert für einen jährlichen Zinssatz $i$ von
\renewcommand{\labelenumi}{(\alph{enumi})}
\begin{enumerate}
\item $i = 2.35 \%$.
\item $i = 3.45 \%$.
\item $i = 4.65 \%$.
\item $i = 5.05 \%$.
\end{enumerate}

\ \\
\textbf{Lösung:}
\begin{mdframed}
\underline{\textbf{Vorgehensweise:}}
\renewcommand{\labelenumi}{\theenumi.}
\begin{enumerate}
\item Bestimme die Formel des Nettobarwerts in allgemeiner Form.
\item Untersuche die Auswirkung des Zinssatzes auf den Nettobarwert. 
\end{enumerate}
\end{mdframed}

\underline{1. Bestimme die Formel des Nettobarwerts in allgemeiner Form}\\
Allgemein lässt sich der Nettobarwert durch
\begin{align*}
NPV =
-C_0 + \sum \limits_{t=0}^T \frac{C_t}{(1+i)^t}
\end{align*}
berechnen, wobei bei uns nur $C_{10}$, $C_{20}$ und $C_{40}$ ungleich null sind.
\\
\\

\underline{2. Untersuche die Auswirkung des Zinsatzes}\\
Eingesetzt erhalten wir
\begin{align*}
NPV(i) 
= 
-2'000'000 + \frac{1'000'000}{(1+i)^{10}} + \frac{1'500'000}{(1+i)^{20}}
+ \frac{1'000'000}{(1+i)^{40}}
\end{align*}
und erkennen, dass der Wert größer wird, wenn die Nenner kleiner werden.
Aus diesem Grund wird der kleinste Abdiskuntierungszinssatz
den höchsten Nettobarwert ergeben.\\
Also ist Antwort (a) richtig.

\newpage

\subsection*{\frage{5}{4}}
Die Gleichung
\begin{align*}
\ln(x^3) - \ln \left( 1 - \frac{4}{5}x \right) + \ln(x) = \ln(5)
\end{align*}
besitzt die Lösungsmenge
\renewcommand{\labelenumi}{(\alph{enumi})}
\begin{enumerate}
\item $\lbrace -5,1 \rbrace$.
\item $\lbrace 1 \rbrace$.
\item $\mathbb{R} \setminus \lbrace -5, 1 \rbrace$.
\item $\lbrace -5 \rbrace$.
\end{enumerate}
\ \\
\textbf{Lösung:}
\begin{mdframed}
\underline{\textbf{Vorgehensweise:}}
\renewcommand{\labelenumi}{\theenumi.}
\begin{enumerate}
\item Forme die Gleichung um.
\item Löse die Gleichung. 
\end{enumerate}
\end{mdframed}


\underline{1. Forme die Gleichung um}\\
Durch Umformen erhalten wir
\begin{align*}
\ln(x^3) - \ln \left( 1 - \frac{4}{5}x \right) + \ln(x) &= \ln(5)\\
\Leftrightarrow
\ln(x^3 \cdot x) - \ln \left( \frac{5 - 4 x}{5} \right) &= \ln(5)\\
\Leftrightarrow
\ln \left(x^4 \cdot \frac{5}{5-4x}\right) &= \ln(5). 
\end{align*}
\\
\underline{2. Löse die Gleichung}\\
Durch 
\begin{align*}
&\ \ \  \ x^4 \frac{5}{5-4x} = 5\\
&\Leftrightarrow 
 \frac{x^4}{5-4x} = 1\\
&\Leftrightarrow  
x^4 = 5 - 4x\\
&\Leftrightarrow 
x^4 + 4x - 5 = 0\\
&\Leftrightarrow 
x =1
\end{align*}
kommen wir auf die korrekte Lösungsmenge.\\
Damit ist (b) die korrekte Antwort.

\newpage

\subsection*{\frage{6}{5}}
Gegeben sei die Funktion $f \ : \ \mathbb{R} \to \mathbb{R}$ definiert durch
\begin{align*}
f(x) 
= 
\begin{cases}
\frac{\sin(x)}{x}& , \ x \neq 0\\
a& ,		\  x = 0
\end{cases}.
\end{align*}
Für welche Werte von $a$ ist die Funktion stetig?
\begin{enumerate}
\item $a=0$.
\item $a=1$.
\item $a = \frac{\pi}{2}$.
\item $a = \pi$.
\end{enumerate}
\ \\
\textbf{Lösung:}
\begin{mdframed}
\underline{\textbf{Vorgehensweise:}}
\renewcommand{\labelenumi}{\theenumi.}
\begin{enumerate}
\item Rufe dir in Erinnerung, was Stetigkeit bedeutet.
\item Bestimme den Grenzwert
\begin{align*}
\lim \limits_{x \rightarrow 0} \frac{\sin(x)}{x}
\end{align*}
und leite  davon die richtige Antwort ab.
\end{enumerate}
\end{mdframed}

\underline{1. Rufe dir in Erinnerung, was Stetigkeit bedeutet}\\
Eine Funktion $f \ : \ \mathbb{R} \to \mathbb{R}$ heißt stetig in $x_0$,
falls 
\begin{align*}
x \rightarrow x_0 \ \Rightarrow \ f(x) \rightarrow f(x_0)
\end{align*}
gilt.
Das heißt: Wenn $x$ gegen $x_0$ geht, muss $f(x)$ gegen $f(x_0)$ gehen.
Wir müsssen also 
\begin{align*}
\lim \limits_{x \rightarrow x_0} f(x)
\end{align*}
bestimmen um das passende $a$ für die Stetigkeit in $x_0 = 0$ zu erhalten.
In allen anderen Punkten ist $f$ bereits stetig.\\
\\
\underline{2. Bestimme den Grenzwert}\\
Um den Grenzwert 
\begin{align*}
\lim \limits_{x \rightarrow 0} \frac{\sin(x)}{x}
\end{align*}
zu bestimmen, wenden wir die Regel von de l'H\^{o}pital an.
Die Regel von de l'H\^{o}pital wenden wir an, wenn wir einen Ausdruck der Form 
\begin{align*}
\lim \limits_{x \to x_0} \frac{f(x)}{g(x)}
\end{align*}
vorgegeben haben und mit
\begin{align*}
\frac{f(x_0)}{g(x_0)} &= \frac{0}{0} \\
\frac{f(x_0)}{g(x_0)} &= \frac{\infty}{\infty}\\
\frac{f(x_0)}{g(x_0)} &= \frac{-\infty}{\infty}
\end{align*}
einen undefinierten Term erhalten.
Dann wendet man durch
\begin{align*}
\lim \limits_{x \to x_0} \frac{f(x)}{g(x)}
=
\lim \limits_{x \to x_0} \frac{f^\prime(x)}{g^\prime(x)}
\end{align*}
die Regel von de l'H\^{o}pital an.\\
Da wir \glqq $\nicefrac{0}{0}$\grqq~gegeben haben, können wir mit
\begin{align*}
\lim \limits_{x \rightarrow 0} \frac{\sin(x)}{x}
= \lim \limits_{x \rightarrow 0} \frac{\cos(x)}{1} = 1
\end{align*}
direkt de l'H\^{o}pital anwenden. Es gilt also 
\begin{align*}
x \rightarrow 0 \ \Rightarrow \ f(x) \rightarrow  1 = f(0), 
\end{align*}
wenn $a = 1 $ ist.\\
Somit ist Antwort (b) richtig.

\newpage

\subsection*{\frage{7}{4}}
Welche der folgenden Funktionen ist in ihrem kompletten Definitionsbereich \textit{konvex}?
\renewcommand{\labelenumi}{(\alph{enumi})}
\begin{enumerate}
\item $f_1$ definiert durch $f_1(x) = \ln \left(\frac{1}{2x +1} \right)$.
\item $f_2$ definiert durch $f_2(x) = \ln(x^2 + 2x +1 )$.
\item $f_3$ definiert durch $x^3 + 3 \ x + 4 $.
\item Keine der obigen Funktionen ist in ihrem ganzen Definitionsbereich konvex.
\end{enumerate}
\ \\
\textbf{Lösung:}
\begin{mdframed}
\underline{\textbf{Vorgehensweise:}}
\renewcommand{\labelenumi}{\theenumi.}
\begin{enumerate}
\item Stelle die allgemeine Bedingung der Konvexität auf.
\item Bestimme die Ableitungen.
\item Ermittle die konvexe Funktion.
%\item Wie könntest du geometrisch direkt auf die richtige Antwort kommen?
\end{enumerate}
\end{mdframed}

\underline{1. Stelle die allgemeine Bedingung der Konvexität auf}\\
Eine Funktion $f\ : \ D_f \to \mathbb{R}$ heißt auf ihrem ganzen Definitionsbereich konvex, falls
\begin{align*}
f^{\prime \prime}(x) \geq 0
\end{align*}
für alle $x \in D_f $ gilt.\\
\\
\underline{2. Bestimme die Ableitungen}\\
Wir bestimmen die Ableitungen: 
\begin{align*}
f_1^\prime(x) &= \frac{1}{\frac{1}{2x+1}} \cdot 2
= 2 \cdot(2x+1)
= 4x + 2\\
f_1^{\prime \prime}(x) &= 4\\
f_2^\prime(x) &=\frac{1}{x^2+ 2x +1} \cdot( 2x+2)
= \frac{2(x+1)}{(x+1)^2} = \frac{2}{x+1}
= 2 \cdot (x+1)^{-1}\\
f_2^{\prime \prime}(x)
&=\frac{-2}{(x+1)^2}\\
f_3^\prime(x) &= 3x^2 +3\\
f_3^{\prime \prime}(x) &= 6x 
\end{align*}
\\
\\
\underline{3. Ermittle die konvexe Funktion}
\renewcommand{\labelenumi}{(\alph{enumi})}
\noindent
\begin{enumerate}[leftmargin=0.6cm]
\item Wir sehen, dass
\begin{align*}
f^{\prime \prime}(x) = 4 > 0 
\end{align*}
für alle $x \in D_f= (-\frac{1}{2}, \infty)$ gilt.
Die Funktion $f_1$ ist auf $D_f$ sogar strikt konvex.
\item
Wir sehen, dass
\begin{align*}
f_2^{\prime \prime}(x)=\frac{-2}{(x+1)^2},
\end{align*}
immer negativ ist für alle $x \in D_f = (-\frac{1}{2},\infty)$
Damit ist $f_2$ nicht konvex.

\item
Wir sehen wie, dass bei
\begin{align*}
f^{\prime \prime}(x) = 6x
\end{align*}
ein Vorzeichenwechsel im Definitionsbereich vorliegt.
\end{enumerate}
Somit ist Antwort (a) korrekt.



%\underline{4. Wie könntest du geometrisch direkt auf die richtige Antwort kommen }\\
%
%\begin{center}
%
%\begin{tikzpicture}[scale=1]
%\begin{axis}[samples=1000,domain=-0.5:15,mark=none, scale = 1.4,
%axis lines = middle,
%xlabel={$x$},
%ylabel={$f(x)$}]
%\addplot[mark = none,color= blue]
%expression{ ln(1/(2*x+1)};
%\addplot[mark=none,]
%\end{axis}
%\end{tikzpicture}
%\end{center}


\newpage

\subsection*{\frage{8}{2}}
Die Funktion $f$ definiert durch $f(x) = e^{x^2+3x+2}$
\renewcommand{\labelenumi}{(\alph{enumi})}
\begin{enumerate}
\item hat ein lokales Maximum bei $x_0 = -\frac{3}{2}$.
\item hat ein lokales Minimum bei $x_0 = -\frac{3}{2}$.
\item hat einen Sattelpunkt bei $x_0 = -\frac{3}{2}$.
\item Keine der vorangehenden Antworten ist richtig.
\end{enumerate}
\ \\
\textbf{Lösung:}
\begin{mdframed}
\underline{\textbf{Vorgehensweise:}}
\renewcommand{\labelenumi}{\theenumi.}
\begin{enumerate}
\item Überlege, welche Eigenschaften die Exponentialfunktion
besitzt.
\item Bestimme die $x$-Koordinate des Scheitelpunkts von $x^2 + 3x+ 2$.
\item Alternative: Löse diese Aufgabe noch mithilfe der Ableitung.
\end{enumerate}
\end{mdframed}

\underline{1. Überlege, welche Eigenschaften die Exponentialfunktion besitzt}\\
Die Exponentialfunktion 
\begin{align*}
\text{exp} \ : \ \mathbb{R} \to (0,\infty), \ x \mapsto e^x
\end{align*}
ist streng monoton wachsend und immer positiv.
Wir haben mit
\begin{align*}
x^2+3x+2
\end{align*}
eine nach oben geöffnete Parabel gegeben.
Deswegen erreicht $f$ an deren Scheitelpunkt ein Minimum.\\


\underline{2. Bestimme die $x$-Koordinate des Scheitelpunkts von $x^2 + 3x+ 2$}\\
Den Scheitelpunkt können wir mithilfe quadratischer Ergänzung bestimmen.
Durch
\begin{align*}
x^2 +3x + 2 = x^2  + 2 \cdot \frac{3}{2} \cdot x + 2 
= x^2 +2 \cdot \frac{3}{2} \cdot x + \left(\frac{3}{2} \right)^2 -
\left(\frac{3}{2} \right)^2 + 2
= \left( x + \frac{3}{2} \right)^2 - \left(\frac{3}{2} \right)^2 + 2
\end{align*}
erhalten wir die Scheitelpunktform.
Der Scheitelpunkt besitzt die $x$-Koordinate $x_0 = -\frac{3}{2}$.
Damit besitzt $f$ in $x_0$ ein Minimum. 
\\
\\

\underline{3. Alternative: Löse diese Aufgabe noch mithilfe der Ableitung}\\
Die Ableitung von $f$ ist durch
\begin{align*}
f^\prime(x)=(2x +3) \cdot e^{x^2+3x+2}
\end{align*}
gegeben. Für die Ableitung gilt
\begin{align*}
f^\prime(x) &= 0\\ 
\Leftrightarrow
2x  +3 &= 0\\
\Leftrightarrow
x &= -\frac{3}{2},
\end{align*}
womit $x_0= -\frac{3}{2}$ ein kritischer Punkt ist.
Die Ableitung $f^\prime$ hat in diesem Punkt ein Vorzeichenwechsel von negativ zu positiv.
Damit besitzt $f$ in $x_0$ ein Minimum.
\\
\\
Somit ist Antwort (b) korrekt.