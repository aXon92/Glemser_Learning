\section*{Aufgabe 2 (24 Punkte)}
\vspace{0.4cm}
\subsection*{(a1) (6 Punkte)}
Gegeben sei die Funktion
\begin{align*}
f: D_f \to \mathbb{R}, \ x \mapsto y= \frac{1}{1+x}.
\end{align*}
Bestimmen Sie das Taylor-Polynom $P_3(x)$ dritter Ordnung von $f$ in $x_0 = 0$.\\
\\
Verwenden Sie $P_3(x)$, um einen Näherungswert von $f(0,1)$ zu bestimmen.
\\
\\
\textbf{Lösung:}
\begin{mdframed}
\underline{\textbf{Vorgehensweise:}}
\begin{enumerate}
\item Leite mit der allgemeinen Formel für das $n$-te Taylorpolynom die Formel für $P_3(x)$ her.
\item Bestimme die Ableitungen sowie die entsprechenden Werte.
\item Berechne den Näherungswert. 
\end{enumerate}
\end{mdframed}
\underline{1. Leite mit der allgemeinen Formel für das $n$-te Taylorpolynom die Formel für $P_3(x)$ her. }\\
Die allgemeine Formel für das $n$-te Taylorpolynom ist durch
\begin{align*}
P_n(x) = \sum \limits_{k=0}^n \frac{f^{(k)}(x_0)}{k!} \cdot (x-x_0)^k
\end{align*}
gegeben. Also müssen wir
\begin{align*}
P_3(x) 
= f(x_0) + f^\prime(x_0)\cdot x + \frac{f^{\prime \prime}(x_0)}{2}\cdot x^2+
\frac{f^{\prime \prime \prime}(x_0)}{6} \cdot x^3
\end{align*}
berechnen.
\\ 
\\
\underline{2. Bestimme die Ableitungen sowie die entsprechenden Werte}\\
Zunächst berechnen wir mit der Kettenregel
\begin{align*}
f^\prime(x) &= \frac{-1}{(1+x)^2}\\
f^{\prime \prime}(x) &= \frac{2}{(1+x)^3}\\
f^{\prime \prime \prime}(x) &= \frac{-6}{(1+x)^4}
\end{align*}
die nötigen Ableitungen. Mit $x_0=0$ ergeben sich eingesetzt 
\begin{align*}
f(x_0) &= 1\\
f^\prime(x_0) &=-1\\
f^{\prime \prime}(x_0) &= 2 \\
f^{\prime \prime \prime}(x_0) &= -6.
\end{align*}
Durch Einsetzen erhalten wir
\begin{align*}
P_3(x) = 1 - x +\frac{2}{2} \cdot x^2 + \frac{-6}{6} \cdot	x^3
=1 -x +x^2 -x^3 
\end{align*}
das gesuchte Taylorpolynom.
\\
\\
\underline{3. Berechne den Näherungswert}\\
Wir nutzen nun aus, dass
\begin{align*}
f(0,01) \approx P_3(0,01) 
&= 1 - 0,1+0,1^2-0,1^3\\ 
&=1 -0,1 +0,01-0,001\\
&= 0,909
\end{align*}
gilt und erhalten so unseren Näherungswert.

\newpage 

\subsection*{(a2) (3 Punkte)}
Gegeben sei die Funktion
\begin{align*}
f: D_f \to \mathbb{R}, \ x \mapsto y= \frac{1}{1+x}.
\end{align*}
$R_3(x)$ bezeichne das Restglied dritter Ordnung von $f$ in $x_0=0$.\\
Zeigen Sie, dass für $x \geq 0 $ gilt:
\begin{align*}
|R_3(x)| \leq x^4.
\end{align*}
\textbf{Lösung:}
\begin{mdframed}
\underline{\textbf{Vorgehensweise:}}
\begin{enumerate}
\item Bestimme, was der Satz von Taylor über das Restglied des $n$-ten Taylorpolynoms aussagt. 
\item Berechne die vierte Ableitung von $f$ und suche eine passende Abschätzung dafür.
\item Finde eine geschickte Abschätzung für das Restglied. 
\end{enumerate}
\end{mdframed}

\underline{1. Bestimme, was der Satz von Taylor über das Restglied des $n$-ten Taylorpolynoms aussagt.}\\
Das Restglied des $n$-ten Taylorpolynoms ist durch
\begin{align*}
R_n(x) = \frac{f^{(n+1)}(\xi)}{(n+1)!}\cdot (x -x_0)^{n+1}
\end{align*}
für $\xi \in [x_0, x]$ gegeben. Daraus ergibt sich für uns
\begin{align*}
R_3(x) = \frac{f^{(4)}(\xi)}{4!}\cdot x^4  
= \frac{f^{(4)}(\xi)}{24} \cdot x^4
\end{align*}
die gesuchte Restgliedformel.
\\
\\
\underline{2. Berechne die vierte Ableitung von $f$ und suche eine passende Abschätzung dafür }\\
Die vierte Ableitung ist durch
\begin{align*}
f^{(4)}(x) = \left(f^{(3)}(x)\right)^\prime = \frac{24}{(1+x)^5}
\end{align*}
gegeben. Nun wissen wir, dass
\begin{align*}
0 \leq \frac{1}{(1+x)^5} \leq 1
\end{align*}
für $x \geq 0 $ gilt. Damit erhalten wir für $x \geq 0$ mit
\begin{align*}
f^{(4)}(x) = \frac{24}{(1+x)^5} \leq 24 
\end{align*}
eine passende Abschätzung.
\\ 
\\
\underline{3. Finde eine geschickte Abschätzung für das Restglied}\\
Durch unsere Vorarbeit erhalten wir direkt mit
\begin{align*}
|R_3(x)| = \frac{|f^{(4)}(\xi)|}{24} \cdot x^4
\leq \frac{24}{24} \cdot x^4 = x^4
\end{align*}
die passende Abschätzung für $\xi \in [0,x]$.

\newpage
\subsection*{(b) (3 Punkte)}
Gegeben sei die Funktion
\begin{align*}
f\: \ D_f \to \mathbb{R}, \ (x,y) \mapsto 3 e^{2ax+by+5} 
\end{align*}
wobei $a,b \in \mathbb{R}$
\\ 
\\
Bestimmen Sie die Werte von $a$ und $b$ so, dass die partiellen Elastizitäten $\varepsilon_{f,x}(c,c)$ und $\varepsilon_{f,y}(c,c)$ identisch sind für alle $c > 0 $.
\\
\\
\textbf{Lösung:}
\begin{mdframed}
\underline{\textbf{Vorgehensweise:}}
\begin{enumerate}
\item Bestimme, wie die partiellen Elastizitäten $\varepsilon_{f,x}(x,y)$ und $\varepsilon_{f,y}(x,y)$ definiert sind. 
\item Ermittle die partiellen Ableitungen.
\item Stelle eine Gleichung auf, welche du nach $c$ auflöst.
\end{enumerate}
\end{mdframed}

\underline{1. Bestimme, wie die partiellen Elastizitäten $\varepsilon_{f,x}(x,y)$ und $\varepsilon_{f,y}(x,y)$ definiert sind }\\
Wir erinneren uns, dass für die partiellen Elastizitäten
\begin{align*}
\varepsilon_{f,x}(x,y) =  \frac{f_x(x,y)}{f(x,y)}x
\ \text{und} \
\varepsilon_{f,y}(x,y) =  \frac{f_y(x,y)}{f(x,y)}y
\end{align*}
gilt. Wir müssen also zunächst die partiellen Ableitungen von $f$ bestimmen.\\ 
\\
\underline{2. Ermittle die partiellen Ableitungen}\\
Wir fangen mit der partiellen Ableitung von $f$ nach $x$ an.
Diese ist mit der Kettenregel durch
\begin{align*}
f_x(x,y) = 3 \cdot 2a \cdot e^{2ax + by +5}
= 6a e^{2ax + by +5}
\end{align*}
gegeben. Die Ableitung nach $y$ erhalten wir mit
\begin{align*}
f_y(x,y) = 3 \cdot b \cdot e^{2ax + by +5}
= 3b e^{2ax + by +5}
\end{align*}
ganz analog.\\
\\
\underline{3. Stelle eine Gleichung auf, welche du nach $c$ auflöst}\\
Zunächst setzen wir alles in die Definition der partiellen Elastizität ein, wodurch wir
\begin{align*}
\varepsilon_{f,x}(x,y) 
&=  \frac{f_x(x,y)}{f(x,y)}x
=  \frac{6a e^{2ax+by+5}}{3e^{2ax+by+5}}x
=  \frac{2a}{1}x = 2ax\\
\varepsilon_{f,y}(x,y)
&=  \frac{f_y(x,y)}{f(x,y)}y
= \frac{3b e^{2ax+by+5}}{3 e^{2ax+by+5}}y
=  \frac{b}{1}y = by
\end{align*}
erhalten.
Nun suchen wir $a,b \in \mathbb{R}$, sodass 
\begin{align*}
\varepsilon_{f,x}(c,c) = \varepsilon_{f,y}(c,c) 
\end{align*}
für alle $c >0 $ gilt.
Durch
\begin{align*}
\varepsilon_{f,x}(c,c) &= \varepsilon_{f,y}(c,c)
\Leftrightarrow\\
2ac &= bc
\Leftrightarrow\\
2a &= b
\Leftrightarrow\\
a &= \frac{b}{2}
\end{align*}
erhalten wir ein passendes Verhältnis von $a$ und $b$.

\newpage
\subsection*{(c) (8 Punkte)}
Die Kurve $C$ in der $xy$-Ebene sei gegeben durch die Gleichung
\begin{align*}
C  \ : \ x^2+5xy+12y-a= 0,
\end{align*}
wobei $a \in \mathbb{R}$ \\
\\
Bestimmen Sie $a \in \mathbb{R}$ und $c \in \mathbb{R}$ so, dass
der Punkt $P = (c,c)$ zur Kurve gehört und die Steigung der Tangente an $C$ in $P$ $-1$ beträgt.
\\
\\
\textbf{Lösung:}
\begin{mdframed}
\underline{\textbf{Vorgehensweise:}}
\begin{enumerate}
\item Überlege, wie man mithilfe des Satzes über implizite Funktionen die Steigung der Tangente berechnet.
\item Bestimme die Ableitungen nach $x$ bzw. $y$.
\item Bestimme ein passendes $a$ bzw. $c$.
%\item Können wir den Satz über implizite Funktionen umgehen?
\end{enumerate}
\end{mdframed}

\underline{1. Überlege, wie man mithilfe des Satzes über implizite Funktionen die Steigung der Tangente berechnet}\\
Wir haben die Kurve
\begin{align*}
\varphi(x,y) = x^2+5xy+12y-a=0
\end{align*}
gegeben. Die Steigung der Tangente in einem Punkt $(x_0,y_0)$ können wir mithilfe des Satzes über implizite Funktionen durch
\begin{align*}
\frac{\td{y}}{\td{x}}(x_0,y_0)  
= -\frac{\varphi_x(x_0,y_0)}{\varphi_y(x_0,y_0)}
\end{align*}
bestimmen.
\\
\\
\underline{2. Bestimme die Ableitungen nach $x$ bzw. $y$}\\
Die partiellen Ableitungen von $\varphi$ sind durch
\begin{align*}
\varphi_x(x,y) = 2x +5y \ 
\text{und} \
\varphi_y(x,y) = 5x + 12
\end{align*}
gegeben. Durch Einsetzen erhalten wir 
\begin{align*}
\frac{\td{y}}{\td{x}}(x_0,y_0)  
= -\frac{\varphi_x(x_0,y_0)}{\varphi_y(x_0,y_0)}
=-\frac{2x+5y}{5x+12}
\end{align*}
als Resultat.\\
\\
\underline{3. Bestimme ein passendes $a$ bzw. $c$}\\
Wir nehmen an, dass der Punkt $P = (c,c)$ auf der Kurve liegt.
Dann erhalten wir durch
\begin{align*}
&\ \ \ \  \ \frac{\td{y}}{\td{x}}(c,c)
= - \frac{2c+5c}{5c+12} = -1\\
&\Leftrightarrow
  \frac{7c}{5c+12} = 1\\
&\Leftrightarrow
7c = 5c +12\\
&\Leftrightarrow
2c = 12\\
&\Leftrightarrow
c= 6
\end{align*}
das passende $c$.
Nun fehlt noch das $a$, sodass $(6,6)$ auf der Kurve liegt.
Durch Einsetzen und Umformen erhalten wir mit
\begin{align*}
\varphi(6,6) = 6^2 +5\cdot 6^2  + 12 \cdot 6 -a = 0
\Leftrightarrow
a = 6^2 +5\cdot 6^2  + 12 \cdot 6
=6^3 + 12 \cdot 6 = 216 + 72 = 288
\end{align*}
das passende $a$.
Somit sind für $a = 288$ und $c = 6$ alle geforderten Bedingungen erfüllt.
\\
\\
%\underline{Können wir den Satz über implizite Funktionen umgehen?}
%Ja wir können die Analysis aus der Schule verwenden.
%Zunächst erhalten wir durch
%\begin{align*}
%5\Leftrightarrow
%x^2 + y (5x +12) -a =0 
%\Leftrightarrow
%y ( 5x +12) = -x^2 +a
%\Leftrightarrow
%y = \frac{-x^2+a}{5x+12}
%\end{align*}
%eine explizite Funktionsvorschrift. Die Kurve ist also der Funktionsgraph.
%Wir definieren uns
%\begin{align*}
%f(x)  := \frac{-x^2+a}{5x+12}
%\end{align*}
%und erhalten mit der Quotientenregel
%\begin{align*}
%f^\prime(x) = \frac{-2x \cdot (5x +12) +(x^2+a) \cdot 5}{(5x+12)^2}
%= \frac{ -10x^2 - 24 x + 5x^2 + 5a}{25x^2 + 120x + 144}
%= \frac{-5x^2-24x +5a}{25x^2 + 120x + 144}
%\end{align*}
%als Ableitung.

\newpage
\subsection*{\ein{d1}{2}}
Gegeben sei die Produktionsfunktion
\begin{align*}
P(K,A) = (a K^{0,25} + A^{0,75})^4,
\end{align*}
wobei $a > 0$.
\\
\\
Bestimmen Sie die Grenzerträge $P_K$ und $P_A$.
\\
\\
\textbf{Lösung:}
\begin{mdframed}
\underline{\textbf{Vorgehensweise:}}
\begin{enumerate}
\item Bestimme die Grenzerträge in allgemeiner Form.
\item Bestimme die Grenzerträge durch partielles Ableiten.
\end{enumerate}
\end{mdframed}

\underline{1. Bestimme die Grenzerträge in allgemeiner Form}\\
Die Grenzerträge entsprechen den partiellen Ableitungen nach $K$ bzw. $A$.
Wir müssen also 
\begin{align*}
\frac{\partial P(K,A)}{\partial \mathrm{K}} \
\text{und} \
\frac{\partial P(K,A)}{\partial \mathrm{A}}
\end{align*}
berechnen.\\
\\

\underline{2. Bestimme die Grenzerträge durch partielles Ableiten}\\
Durch partielles Ableiten nach $K$ bzw. $A$ erhalten wir mit der Kettenregel
\begin{align*}
P_K(K,A) &= 4 \left( a K^{0,25} + A^{0,75}\right)^3 \cdot 0,25 a \ K^{-0,75}
= \frac{a \left( a K^{0,25} + A^{0,75}\right)^3 }{K^{0,75}}\\
P_A(K,A) &= 4 \left( a K^{0,25} + A^{0,75}\right)^3 \cdot 0,75 \ A^{-0,25}
= \frac{3 \left( a K^{0,25} + A^{0,75}\right)^3}{A^{0,25}}
\end{align*}
die Grenzerträge.
\newpage
\subsection*{\ein{d2}{2}}
Gegeben sei die Produktionsfunktion
\begin{align*}
P(K,A) = (a K^{0,25} + A^{0,75})^4,
\end{align*}
wobei $a > 0$.\\
\\
Für welche Werte von $a \in \mathbb{R}$ ist die technische Substitutionsrate im Punkt $(1,16)$ gleich $-\frac{4}{3}$?
\\
\\
\textbf{Lösung:}
\begin{mdframed}
\underline{\textbf{Vorgehensweise:}}
\begin{enumerate}
\item Bestimme die technische Substitutionsrate in allgemein mit dem Satzes über implizite Funktionen.
\item Berechne die technische Substitutionsrate.
\end{enumerate}
\end{mdframed}

\underline{1. Bestimme die technische Substitutionsrate in allgemein mit dem Satzes über implizite Funktionen}\\
Die technische Substitutionsrate in einem Punkt $(K,A)$ entspricht
\begin{align*}
\frac{\td{A}}{\td{K}}(K,A).
\end{align*}
Durch den Satz über implizite Funktionen bekommen wir
\begin{align*}
\frac{\td{A}}{\td{K}}(K,A) 
= -\frac{P_K(K,A)}{P_A(K,A)}
\end{align*}
geliefert.\\
\\

\underline{2. Berechne die technische Substitutionsrate}\\
Durch Einsetzen erhalten wir
\begin{align*}
-\frac{P_K(K,A)}{P_A(K,A)} 
= -\frac{\frac{a \left( a K^{0,25} + A^{0,75}\right)^3 }{K^{0,75}}}{\frac{3 \left( a K^{0,25} + A^{0,75}\right)^3}{A^{0,25}}}
= -\frac{a \left( a K^{0,25} + A^{0,75}\right)^3}{3 \left( a K^{0,25} + A^{0,75}\right)^3}
\cdot \frac{A^{0,25}}{K^{0,75}}
=-\frac{a}{3} \cdot \frac{A^{0,25}}{K^{0,75}},
\end{align*}
wodurch
\begin{align*}
\frac{\td{A}}{\td{K}}(1,16) 
= -\frac{a}{3} \cdot \frac{16^{0,25}}{1} 
= -\frac{a}{3} \cdot \sqrt[4]{16} = -\frac{2a}{3}
\end{align*}
folgt. Zum Schluss erhalten wir mit 
\begin{align*}
\frac{\td{A}}{\td{K}}(1,16) = -\frac{2a}{3} = - \frac{4}{3}
\Leftrightarrow a = 2
\end{align*}
das gesuchte $a$.
