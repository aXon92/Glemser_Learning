\fancyhead[C]{\normalsize\textbf{$\qquad$ Teil II: Multiple-Choice}}
\section*{Aufgabe 2 (34 Punkte)}
\vspace{0.4cm}
\subsection*{\frage{1}{2}}
Die Aussage \glqq Wenn Albert kommt, dann kommt auch Beatrice\grqq~sei richtig.
Dann ist die folgende Aussage sicher auch richtig:
\renewcommand{\labelenumi}{(\alph{enumi})}
\begin{enumerate}
	\item \glqq Wenn Beatrice kommt, dann kommt auch Albert\grqq
	\item \glqq Wenn Albert nicht kommt, dann kommt auch Beatrice nicht\grqq
	\item
	\glqq Wenn Beatrice nicht kommt, dann kommt auch Albert nicht\grqq
	
	\item
	Keine der Aussagen (a)-(c) ist sicher richtig.
\end{enumerate}
\ \\
\textbf{Lösung:}
\begin{mdframed}
\underline{\textbf{Vorgehensweise:}}
\renewcommand{\labelenumi}{\theenumi.}
\begin{enumerate}
\item Löse die Aufgabe mithilfe einer Wahrheitstabelle.

\end{enumerate}
\end{mdframed}

\underline{1. Löse die Aufgabe mithilfe einer Wahrheitstabelle}\\
Eine Aussage kann nur die Wahrheitswerte wahr ($ W $) oder falsch ($ F $) annehmen.
Dementsprechend gibt es bei zwei Aussagen $ A $, $ B $ genau vier Kombinationsmöglichkeiten der Wahrheitswerte.
Aus diesen Kombinationen ergeben sich dann die Wahrheitswerte der verknüpften Aussagen.
Zwei Verknüpfungen heißen äquivalent, falls deren Wahrheitswerte bei jeder möglichen Kombination übereinstimmen.
In unserem Fall steht $ A $ für Albert und $ B $ für Beatrice.

\begin{center}
	\begin{tabular}{cllll}
		\hline
		\multicolumn{1}{c|}{$A$} & \multicolumn{4}{l}{$W$ $W$ $F$ $F$} \\
		\multicolumn{1}{c|}{$B$} & \multicolumn{4}{l}{$W$ $F$ $W$ $F$} \\ \hline
		\multicolumn{1}{c|}{$ \neg A$} & \multicolumn{4}{l}{$F$ $F$ $W$ $W$} \\
		\multicolumn{1}{c|}{$ \neg B$} & \multicolumn{4}{l}{$F$ $W$ $F$ $W$} \\ \hline
		\multicolumn{1}{c|}{$ A \Rightarrow B$} & \multicolumn{4}{l}{$W$ $F$ $W$ $W$} \\ 
		\multicolumn{1}{c|}{(a) \ \ \ \ $ B \Rightarrow A$} & \multicolumn{4}{l}{$W$ $W$ $F$ $W$} \\
		\multicolumn{1}{c|}{(b) $  \neg A \Rightarrow \neg B$} & \multicolumn{4}{l}{$W$ $W$ $F$ $W$}  \\ 
		\multicolumn{1}{c|}{(c) $\neg B \Rightarrow \neg A$} & \multicolumn{4}{l}{$W$ $F$ $W$ $W$}  \\ 
		\hline
	\end{tabular}
\end{center}
Wir erkennen an der Wahrheitstabelle, dass Antwort (c) korrekt ist.

\newpage

\subsection*{\frage{2}{2}}
Die Folge $ \lbrace a_n \rbrace_{n \in \mathbb{N}} $ ist monoton wachsend und beschränkt. Die Folge $ \lbrace b_n \rbrace_{n \in \mathbb{N} }  $ ist definiert durch $ b_n = 2 a_n -1 $.\\
Dann ist $ \lbrace b_n \rbrace_{n \in \mathbb{N}} $
\renewcommand{\labelenumi}{(\alph{enumi})}
\begin{enumerate}
	\item konvergent.
	\item divergent.
	\item
	monoton fallend.
	\item
	unbeschränkt.
\end{enumerate}
\ \\
\textbf{Lösung:}
\begin{mdframed}
	\underline{\textbf{Vorgehensweise:}}
	\renewcommand{\labelenumi}{\theenumi.}
	\begin{enumerate}
		\item 
		Überlege dir, was für monotone und beschränkte Folgen gilt und beantworte die Frage.	
	\end{enumerate}
\end{mdframed}

\underline{1. Überlege dir, was für monotone und beschränkte Folgen gilt und beantworte die Frage}\\
Sei $ (a_n) $ eine monoton wachsende und beschränkte Folge.
Da diese monoton wachsend ist, ist $ s := a_0  $ die untere Schranke.
Sei $ S $ die kleinste obere Schranke. Insgesamt gilt
\begin{align*}
s \leq a_n \leq S
\end{align*}
für alle $ n \in \mathbb{N} $.
Da die Folge monoton wachsend ist, gilt dann auch
\begin{align*}
\lim \limits_{n \to \infty} a_n = S.
\end{align*}
Der intuitive Grund ist: Die Folge ist nach beschränkt und wächst immer weiter. Dementsprechend muss sie sich $ S $ immer weiter annähern.
Also ist $ (a_n) $ konvergent. Da Multiplikation und Addition von Konstanten die Konvergenzeigenschaft beibehalten ist $ b_n $ konvergent.\\
\\ 
Somit ist Antwort (a) korrekt.

\newpage

\subsection*{\frage{3}{3}}
Ein Kapital von $ P = 1'000'000 $ CHF wird angelegt bei einem Jahreszinssatz von $ i = 2\% $ mit monatlicher Verzinsung.\\
Der effektive Zinssatz $ i_{\textrm{eff}} $ beträgt näherungsweise 
\renewcommand{\labelenumi}{(\alph{enumi})}
\begin{enumerate}
	\item 
	$ 1.982 \% $.
	\item 
	$ 2 \% $.
	\item 
	$ 2.018 \% $.
	\item
	$ 2.020 \% $.
\end{enumerate}
\ \\
\textbf{Lösung:}
\begin{mdframed}
\underline{\textbf{Vorgehensweise:}}
\renewcommand{\labelenumi}{\theenumi.}
\begin{enumerate}
\item Leite eine Gleichung für den effektiven Zins her und beantworte die Frage.
\end{enumerate}
\end{mdframed}

\underline{1. Leite eine Gleichung für den effektiven Zins her und beantworte die Frage}\\
Der effektive Zins $ i_{\textrm{eff}} $ entspricht dem Zins der notwendig ist, um den selben Betrag durch jährliche Verzinsung zu erreichen, wie mit monatlicher Verzinsung zur jährlichen Rate $ i = 2 \% $.
Als Formel erhalten wir also:
\begin{align*}
1'000'000 \left( 1 + \frac{2 \%}{12}\right)^{12}
&=
1'000'000 (1 + i_{\mathrm{eff}})\\
\ \Leftrightarrow \
 \left( 1 + \frac{2 \%}{12}\right)^{12}
&=
 (1 + i_{\mathrm{eff}})\\
\ \Leftrightarrow \
i_{\mathrm{eff}}
&=
\left( 1 + \frac{2 \%}{12}\right)^{12} - 1
\approx 2.018 \%
\end{align*}
Damit Antwort (c) korrekt.

\newpage

\subsection*{\frage{4}{3}}
Die Funktion $ f $ ist gegeben durch
\begin{align*}
f : D_f \to \mathbb{R}, \ 
x \mapsto y = f(x) = - \frac{3}{x^5} +2
\end{align*}
Dann ist die inverse Funktion $ f^{-1} $ definiert auf 
\renewcommand{\labelenumi}{(\alph{enumi})}
\begin{enumerate}
	\item 
	$ \mathbb{R} $.
	\item
	$ \mathbb{R} \setminus \lbrace 0 \rbrace $.
	\item
	$ \mathbb{R} \setminus \lbrace 2 \rbrace $.
	\item
	Die Funktion $ f $ ist nicht injektiv. Deshalb ist es unmöglich, eine Umkehrfunktion $ f^{-1} $ anzugeben.
\end{enumerate}
\ \\
\ \\
\textbf{Lösung:}
\begin{mdframed}
\underline{\textbf{Vorgehensweise:}}
\renewcommand{\labelenumi}{\theenumi.}
\begin{enumerate}
\item Bestimme die Umkehrfunktion und schließe damit auf die richtige Antwort.
\end{enumerate}
\end{mdframed}

\underline{1. Bestimme die Umkehrfunktion und schließe damit auf die richtige Antwort}\\
Wir bestimmen die Umkehrfunktion:
\begin{align*}
y = - \frac{3}{x^5} +2
\ \Leftrightarrow \
\frac{3}{x^5} = 2 - y
\ \Leftrightarrow \
x^5 = \frac{3}{2 -y}
\ \Leftrightarrow \
x = f^{-1}(y) = \sqrt[5]{\frac{3}{2 -y}}
\end{align*}
Wir sehen, dass $ f^{-1}  $ für $ y \neq 2 $ definiert ist.
Damit gilt $ D_{f^{-1}} = \mathbb{R} \setminus \{ 2 \} $.\\
\\
Also ist Antwort (c) korrekt.\\
\\
%\underline{2. Alternativer Lösungsweg}\\
%Wir wissen, dass der Wertebereich von $ f $ dem Definitionsbereich von $ f^{-1} $ entspricht, d.h.
%\begin{align*}
%R_f = D_{f^{-1}}.
%\end{align*}
%Wegen 
%\begin{align*}
%\frac{3}{x^5} \neq 0
%\end{align*}
%für alle $ x \neq 0 $ wird die $ 2 $ von $ f $ nicht getroffen, d.h. $ R_f = \mathbb{R} \setminus \{ 2\} $.

\newpage

\subsection*{\frage{5}{4}}
Seien $ m,n,k > 0 $, $ m \neq 1 $, $ n \neq 1 $, $ k \neq 1 $.\\
Welche der folgenden Identitäten ist allgemein gültig?
\renewcommand{\labelenumi}{(\alph{enumi})}
\begin{enumerate}
	\item 
	$ n^{\log_m(k)} = k^{\log_n(m)} $.
	\item
	$ n^{\log_m(k)} = k^{\log_m(n)}$.
	\item
	$ n^{\log_m(k)} = m^{\log_n(k)}$.
	\item
	Keine der Identitäten (a) - (c) gilt allgemein.
\end{enumerate}
\ \\
\textbf{Lösung:}
\begin{mdframed}
\underline{\textbf{Vorgehensweise:}}
\renewcommand{\labelenumi}{\theenumi.}
\begin{enumerate}
\item Löse die Aufgabe mithilfe des natürlichen Logarithmus.
\end{enumerate}
\end{mdframed}

\underline{1. Löse die Aufgabe mithilfe des natürlichen Logarithmus}\\
Für den Logarithmus zur Basis $m  $ gilt folgender Zusammenhang:
\begin{align*}
\log_m(k) = \frac{\ln (k)}{\ln(m)}.
\end{align*}
Damit erhalten wir 
\begin{align*}
n^{\log_m(k)} 
=
\left(e^{\ln(n)}\right)^{\log_m(k)}
=
\left(e^{\ln(n)}\right)^{\frac{\ln (k)}{\ln(m)}}
=
\left(e^{1}\right)^{\frac{\ln(n) \ln (k)}{\ln(m)}}
=
\left(e^{\ln (k)}\right)^{\frac{\ln(n) }{\ln(m)}}
=k^{\log_m(n)}.
\end{align*}
\ \\
Also ist Antwort (b) korrekt.
\newpage

\subsection*{\frage{6}{3}}
Wir betrachten die Funktion
\begin{align*}
f(x) = \frac{1}{(\sin(x))^2}- 2 \pi, -2 \pi < x < 2 \pi.
\end{align*}

Welche der folgenden Behauptungen ist wahr?
\renewcommand{\labelenumi}{(\alph{enumi})}
\begin{enumerate}
	\item 
	$ f $ ist stetig für alle $ x \in (-2 \pi, 2 \pi) \setminus \lbrace 0 \rbrace $.
	\item
	$ f $ hat mehr als eine Unstetigkeitsstelle in $ (- 2 \pi , 2 \pi ) $.
	
	\item
	$ f $ ist überall stetig.
	\item
	$ f $ hat keine Polstelle.
\end{enumerate}
\ \\
\textbf{Lösung:}
\begin{mdframed}
\underline{\textbf{Vorgehensweise:}}
\renewcommand{\labelenumi}{\theenumi.}
\begin{enumerate}
\item Beantworte die Frage durch das Ausschließen von Antworten.
\end{enumerate}
\end{mdframed}

\underline{1. Beantworte die Frage durch das Ausschließen von Antworten}\\
Für die Antwort auf unsere Frage sind nur die Nullstellen von $ g(x) := (\sin(x))^2 $ relevant. Es gilt 
\begin{align*}
g(x) = 0 \ \Leftrightarrow \ 
\sin(x) = 0 
\ \Leftrightarrow \
x_1 = - \pi , x_2 = 0 , x_3 = \pi
\end{align*}
für $ -2 \pi < x < 2 \pi $.
Jede Nullstelle von $ g $ entspricht einer Unstetigkeitsstelle in $ f $. Damit ist (a) und (c) falsch.
Desweiteren gilt
\begin{align*}
\lim \limits_{x \to 0 }  f(x) = \infty,
\end{align*}
wodurch auch (d) falsch ist.\\
\\
Also ist Antwort (b) korrekt.


\newpage


\subsection*{\frage{7}{3}}
Gegeben ist die Funktion $ f $ durch
\begin{align*}
f : D_f \to \mathbb{R}, \ x \mapsto y = f(x) = \frac{x}{3} - \frac{e}{x^3}.
\end{align*}
Der Wertebereich von $ f^\prime$ ist
\renewcommand{\labelenumi}{(\alph{enumi})}
\begin{enumerate}
	\item 
	$ (- \infty,0) \cup (0,\infty) $.
	\item
	$ (-\infty,\infty) $.
	\item
	$ (0, \infty) $
	\item
	$ ( \frac{1}{3}, \infty) $
\end{enumerate}
\ \\
\textbf{Lösung:}
\begin{mdframed}
\underline{\textbf{Vorgehensweise:}}
\renewcommand{\labelenumi}{\theenumi.}
\begin{enumerate}
\item Bestimme die 1. Ableitung und den entsprechenden Wertebereich.
\item Alternativer Lösungsweg.
 
\end{enumerate}
\end{mdframed}


\underline{1. Bestimme die 1. Ableitung und den entsprechenden Wertebereich}\\
Durch Ableiten erhalten wir
\begin{align*}
f^\prime (x)
=
\frac{1}{3} - (-3) \frac{e}{x^4}
= \frac{1	}{3} + 3 e \frac{1}{x^4}.
\end{align*}
Nun gilt wegen $ \frac{1}{x^4} > 0 $ für $ x \neq 0  $ auch 
$ f^\prime(x) > \frac{1}{3} $ für $ x \neq 0 $.
Aufgrund von 
\begin{align*}
\lim \limits_{ x \to 0 } f^\prime(x)  = \infty
\end{align*}
gilt $ W_{f^\prime} = \left( \frac{1}{3} , \infty \right) $.\\
\\
Also ist Antwort (d) korrekt.\\
\\
\underline{2. Alternativer Lösungsweg}\\
Die Ableitung des ersten Summanden ist konstant.
Die Ableitung des zweiten Summanden hat wegen des Minus die Form
\begin{align*}
C \frac{1}{x^4}, \ \ C > 0.
\end{align*}
Der Wertebereich dieser Funktion ist $ (0,\infty) $. Sei $ K > 0  $ die Konstante der Ableitung des ersten Summanden.
Dann folgt 
\begin{align*}
W_{f^\prime} = K + (0,\infty) = ( K ,\infty).
\end{align*}
Also kann nur Antwort (d) korrekt sein.


\newpage

\subsection*{\frage{8}{3}}
Gegeben ist eine differenzierbare Funktion $ f $ mit Ableitung $ f^\prime $.\\
Welche der folgenden Aussagen ist falsch?
\renewcommand{\labelenumi}{(\alph{enumi})}
\begin{enumerate}
	\item 
	$ f  $ ist stetig.
	\item
	Wenn $ f $ gerade ist, dann ist auch $ f^\prime $ gerade.
	\item
	Wenn $ f $ gerade ist, dann ist $ f^\prime $ ungerade.
	\item
	Wenn $ f^\prime > 0 $ ist, dann ist $ f $ streng monoton wachsend.
\end{enumerate}
\ \\
\textbf{Lösung:}
\begin{mdframed}
\underline{\textbf{Vorgehensweise:}}
\renewcommand{\labelenumi}{\theenumi.}
\begin{enumerate}
\item Finde ein Gegenbeispiel für eine der Antworten.
\end{enumerate}
\end{mdframed}

\underline{1. Finde ein Gegenbeispiel für eine der Antworten}\\
Eine Funktion $ f $ heißt gerade, falls
\begin{align*}
f(x) = f(-x)
\end{align*}
für alle $ x \in D_f $ gilt. Sie heißt ungerade, falls
\begin{align*}
f(x) = - f(-x)
\end{align*}
gilt.
Wir betrachten $ f(x) = x^2  $. Dann gilt
\begin{align*}
f^\prime(x) = 2 x = - 2(-x) = -f^\prime(-x).
\end{align*}
Demnach ist $ f^\prime $ ungerade und wir haben ein Gegenbeispiel für (b) gefunden.\\
Wir können dies auch formal begründen. Sei $ f  $ gerade.
Dann folgt mit der Kettenregel:
\begin{align*}
f(x) = f(-x) \ \Rightarrow \ f^\prime(x) = - f^\prime(-x).
\end{align*}
\ \\
Die Antwort (b) ist also korrekt.


\newpage

\subsection*{\frage{9}{3}}
Für eine differenzierbare Funktion $ h:D_h \to \mathbb{R}, x \mapsto y = h(x) $ seid $ dh(x) $ ihr Differential an der Stelle $ x $.
Wir betrachten nun zwei differenzierbare Funktionen $ f :D_f \to \mathbb{R} $ und $ g: D_g \to \mathbb{R} $ und Konstanten $ a,b \in \mathbb{R} $.\\
Welche der folgenden Gleichungen ist im Allgemeinen falsch?
\renewcommand{\labelenumi}{(\alph{enumi})}
\begin{enumerate}
	\item 
	$ d(af +bg)(x) = a \ df(x) + b \ d g(x) $.
	\item
	$ d(fg)(x) = g(x) \ df(x) + f(x) \ dg(x) $.
	\item
	$ d(f\circ g) (x) = f^\prime(g(x)) \ dg(x) $.
	\item
	$ d \left( \frac{f}{g} \right)(x) = g(x) \ df(x) - f(x) \ dg(x) $.
\end{enumerate}

\ \\
\textbf{Lösung:}
\begin{mdframed}
	\underline{\textbf{Vorgehensweise:}}
	\renewcommand{\labelenumi}{\theenumi.}
	\begin{enumerate}
		\item Überlege dir, wie das Differential definiert ist.
		\item Betrachte die Quotientenregel, um die falsche Antwort zu erhalten.
	\end{enumerate}
\end{mdframed}

\underline{1. Überlege dir, wie das Differential definiert ist}\\
Das Differential einer differenzierbaren Funktion $ f $ ist definiert über
\begin{align*}
df(x) := f^\prime(x) \ dx,
\end{align*}
wobei $ dx \in \mathbb{R} $ eine beliebige reelle Zahl ist. $ d f(x)  $ beschreibt also eine Gerade durch den Ursprung mit der Steigung $ f^\prime(x) $.\\
\\
\underline{2. Betrachte die Quotientenregel, um die falsche Antwort zu erhalten}\\
Die Quotientenregel ist durch 
\begin{align*}
\left(\frac{f}{g}\right)^\prime
= \frac{f^\prime g - f g^\prime}{g^2}
\end{align*}
gegeben. Demzufolge müsste in (d) in irgendeiner Form ein $ g $ im Nenner auftauchen.
Das passiert aber nicht, womit (d) wahrscheinlich eine falsche Gleichung ist.
Dies rechnen wir nun nach:
\begin{align*}
d \left(\frac{f}{g}\right)(x)
&= 
\left(\frac{f}{g}\right)^\prime (x) \ dx
=
\left(\frac{f^\prime g - f g^\prime}{g^2}\right) (x)  \ dx
=
\frac{f^\prime(x) g(x ) - f(x) g^\prime(x)}{(g(x))^2} \ dx\\
&=
\frac{1}{g(x)} f^\prime(x) \ dx - \frac{f(x)}{g^2(x)} g^\prime(x) \ dx
= \frac{1}{g(x)} d f(x) - \frac{f(x)}{g^2(x)} d g(x).
\end{align*}
Somit war unsere Vermutung richtig und (d) ist eine falsche Gleichung.\\
\\
Die Antwort (d) ist korrekt.



\newpage

\subsection*{\frage{10}{3}}
Wir betrachten die Funktion
\begin{align*}
f : D_f \to \mathbb{R}, \ x \mapsto y = f(x) = | x^2 - 2x -24 |.
\end{align*}
Welche der folgenden Behauptungen ist wahr?
\renewcommand{\labelenumi}{(\alph{enumi})}
\begin{enumerate}
	\item 
	$ f $ hat ein lokales Maximum in $ x_0 = 1 $.
	\item
	$ f $ hat ein globales Maximum in $ x_0 = 1 $.
	\item
	$ f  $ hat ein eindeutiges globales Maximum.
	\item
	$ f $ hat ein eindeutiges lokales Minimum.
\end{enumerate}
\ \\
\textbf{Lösung:}
\begin{mdframed}
	\underline{\textbf{Vorgehensweise:}}
	\renewcommand{\labelenumi}{\theenumi.}
	\begin{enumerate}
		\item Überlege dir anhand der Struktur einer Parabel die korrekte Antwort.
	\end{enumerate}
\end{mdframed}

\underline{1. Überlege dir anhand der Struktur einer Parabel die korrekte Antwort}\\
Wir wissen, dass 
\begin{align*}
g(x) = x^2 -2x -24
\end{align*}
eine nach oben geöffnete Parabel ist.
Demnach liegt deren globales Minimum an dem Scheitelpunkt.
Es gilt 
\begin{align*}
g^\prime(x) &= 2x -2 = 2 (x-1)\\
g^{\prime \prime}(x) &= 2  >0
\end{align*}
Damit hat $ g $ das globale Minimum bei $ x_0 = 1 $. 
Also hat
\begin{align*}
f(x) = |g(x) | 
\end{align*}
ein lokales Maximum an $ x_0 = 1 $. Das Maximum ist lokal, da
\begin{align*}
\lim \limits_{x \to \pm \infty} f(x) = \infty
\end{align*}
gilt.\\
\\
Damit ist Antwort (a) korrekt.\\
\\
\textit{Hinweis:} Falls dir dieses Argument nicht klar ist, zeichne $ x^2 -1 $ und wende den Betrag hierauf an.
\newpage

\subsection*{\frage{11}{3}}
Sei 
\begin{align*}
P(x) = 1 + 2x + 5x^2 -x^3
\end{align*}
des Taylor-Polynoms 3. Ordnung an der Stelle $ x_0 = 0 $ einer unendliche oft differenzierbaren Funktion $ f $.\\
Welche der folgenden Aussagen sind korrekt?
\renewcommand{\labelenumi}{(\alph{enumi})}
\begin{enumerate}
	\item 
	$ f(0) = 2 $.
	\item
	$ f^\prime(0) = 1 $.
	\item
	$ f^{\prime \prime}(0) = 10 $.
	\item
	$ f^{(3)}(0) = -5 $.
\end{enumerate}
\ \\
\textbf{Lösung:}
\begin{mdframed}
	\underline{\textbf{Vorgehensweise:}}
	\renewcommand{\labelenumi}{\theenumi.}
	\begin{enumerate}
		\item Gebe die allgemeine Formel für das Taylorpolynom der 3. Ordnung an.
		\item Leite hieraus die korrekte Antwort her.
		\item Alternativer Lösungsweg.	
\end{enumerate}
\end{mdframed}

\underline{1. Gebe die allgemeine Formel für das Taylorpolynom der 3. Ordnung an}\\
Die allgemeine Formel für das Taylorpolynom 3. Ordnung in $ x_0 =0 $ ist:
\begin{align*}
P_3 (x) = \sum \limits_{k=0}^3 \frac{f^{(k)}(0)}{k!} x^k
= f(0) + \frac{f^\prime(0)}{1} x
+ \frac{f^{\prime \prime }(0)}{2}x^2 +
+ \frac{f^{\prime \prime \prime}(0)}{3!} x^3
\end{align*}
\ \\
\underline{2. Leite hieraus die korrekte Antwort her}\\
Wir führen einen Koeffizientenvergleich durch. Es gilt
\begin{align*}
 f(0) &= 1 \neq 2\\
 f^\prime(0) &= 2 \neq 1\\
 \frac{f^{\prime \prime}(0)}{2} x^2  &= \frac{10}{2} x^2= 5 x^2 .
\end{align*}
\ \\
Also ist die Antwort (c) korrekt.\\
\ \\
\underline{3. Alternativer Lösungsweg}\\
Für das dritte Taylorpolynom in $ x_0 = 0 $ gilt
\begin{align*}
P_3^{(k)}(0) = f^{(k)}(0).
\end{align*}
Wegen 
\begin{align*}
P_3^\prime(x) &= 2 + 10 x - 3 x^2\\
P_3^{\prime \prime}(x) &= 10 - 6 x
\ \Rightarrow \ 
P_3^{\prime \prime}(0) = 10
\end{align*}
ist die Antwort (c) korrekt.

\newpage

\subsection*{\frage{12}{2}}
Eine homogene Funktion $ f $ hat den Grad $ 1.8 $.
Ausserdem kennt man die partielle Elastizität
$ \varepsilon_{f,y}(x,y) = x +y -1.2 $.\\
Es folgt dann 
\renewcommand{\labelenumi}{(\alph{enumi})}
\begin{enumerate}
	\item 
	$ \varepsilon_{f,x}(x,y) = x-y -1.2 $.
	\item
	$ \varepsilon_{f,x}(x,y) = -x-y +1.2 $.
	\item
	$ \varepsilon_{f,x}(x,y) = -x-y +3 $.
	\item
	$ \varepsilon_{f,x}(x,y) $ ist durch die Angaben nicht eindeutig bestimmt.
\end{enumerate}
\ \\
\textbf{Lösung:}
\begin{mdframed}
	\underline{\textbf{Vorgehensweise:}}
	\renewcommand{\labelenumi}{\theenumi.}
	\begin{enumerate}
		\item Leite die Antwort mithilfe der Eulerschen Relation her.
	\end{enumerate}
\end{mdframed}

\underline{1. Leite die Antwort mithilfe der Eulerschen Relation her}\\
Die Eulersche Relation ist gegeben durch
\begin{align*}
\varepsilon_{f,x}(x,y) + \varepsilon_{f,y}(x,y) = k,
\end{align*}
wobei $ k $ der Grad der Homogenität von $ f $ ist.
Damit erhalten wir
\begin{align*}
\varepsilon_{f,x}(x,y) = k - \varepsilon_{f,y}(x,y)
= 1.8 - (x+y-1.2)
= 1.8 -x-y +1.2 =-x-y+3.
\end{align*}
\ \\
Damit ist die Antwort (c) korrekt.

\newpage