\fancyhead[C]{\normalsize\textbf{$\qquad$ Teil II: Multiple-Choice}}
\section*{Aufgabe 2 (32 Punkte)}
\vspace{0.4cm}
\subsection*{\frage{1}{4}}
Welche der folgenden Aussagen ist wahr, genau dann wenn $ A $ und $ B $ in ihrem Wahrheitsgehalt verschieden sind (eine Aussage ist wahr, die andere falsch)?
\renewcommand{\labelenumi}{(\alph{enumi})}
\begin{enumerate}
	\item $ A \vee B $.
	\item $ A \wedge B $.
	\item $ \neg (A \Rightarrow B) $.
	\item $ (A \vee B) \wedge (\neg(A \wedge B)) $.
\end{enumerate}\ \\
\textbf{Lösung:}
\begin{mdframed}
\underline{\textbf{Vorgehensweise:}}
\renewcommand{\labelenumi}{\theenumi.}
\begin{enumerate}
\item Löse die Aufgabe mithilfe einer Wahrheitstabelle.
\end{enumerate}
\end{mdframed}

\underline{1. Löse die Aufgabe mithilfe einer Wahrheitstabelle}\\
Eine Aussage kann nur die Wahrheitswerte wahr ($ W $) oder falsch ($ F $) annehmen.
Dementsprechend gibt es bei zwei Aussagen $ A $, $ B $ genau vier Kombinationsmöglichkeiten der Wahrheitswerte.
Aus diesen Kombinationen ergeben sich dann die Wahrheitswerte der verknüpften Aussagen.
\begin{center}
	\begin{tabular}{cllll}
		\hline
		\multicolumn{1}{c|}{$A$} & \multicolumn{4}{l}{$W$ $W$ $F$ $F$} \\
		\multicolumn{1}{c|}{$B$} & \multicolumn{4}{l}{$W$ $F$ $W$ $F$} \\ \hline
		\multicolumn{1}{c|}{(a) \ \ $ A \vee B$} & \multicolumn{4}{l}{$W$ $W$ $W$ $F$} \\ 
		\multicolumn{1}{c|}{(b) \ \ $ A \wedge B$} & \multicolumn{4}{l}{$W$ $F$ $F$ $F$} \\ 
		\multicolumn{1}{c|}{$ A \Rightarrow B$} & \multicolumn{4}{l}{$W$ $F$ $W$ $W$} \\ 
		\hline
		\multicolumn{1}{c|}{$\neg ( A \wedge B)$} & \multicolumn{4}{l}{$F$ $W$ $W$ $W$} \\ 
		\multicolumn{1}{c|}{(c) \ \ $\neg (A \Rightarrow B)$} & \multicolumn{4}{l}{$F$ $W$ $F$ $F$} \\ 
		\hline
		\multicolumn{1}{c|}{(d) \ \ $ (A \vee B) \wedge (\neg ( A \wedge B))$} & \multicolumn{4}{l}{$F$ $W$ $W$ $F$} \\
		\hline
	\end{tabular}
\end{center}
Damit sind die Antworten (a)--(c) falsch und die richtige Antwort bleibt übrig.\\
\\
Also ist die Antwort (d) korrekt.\\
\\
Alternativ lässt sich die Antwort auch anders herleiten. Durch Umformungen erhalten wir:
\begin{align*}
	(A\vee B) \wedge (\neg (A\wedge B))
	&=
	(A\vee B) \wedge (\neg A \vee  \neg B)
	=
	((A\vee B) \wedge (\neg A)) \vee ( (A\vee B) \wedge  (\neg B))\\
	&=(A \wedge \neg A) \vee (\neg A \wedge B) \vee (A \wedge \neg B) \vee (B \wedge \neg B)\\
	&=
	(\neg A \wedge B) \vee (A \wedge \neg B).
\end{align*}
Hierbei geht ein, dass $ (A \wedge \neg A)  $ unabhängig von dem Wahrheitsgehalt der Aussage falsch ist und für logische Operationen das Distributivgesetz gilt.\\
\\
Man nennt diese logische Verknüpfung auch exklusives Oder und entspricht der sprachlichen Verwendung von \grqq oder \grqq.

\newpage

\subsection*{\frage{2}{3}}
Sei $ A = \textrm{\grqq Jeanette und Hans kommen an die Party \grqq} $
und $ B = \textrm{\grqq Mark kommt an die Party \grqq}  $.
Wir wissen, dass Hans die Party nicht besuchte.
Über Jeanette und Mark ist jedoch nichts bekannt.\\
Welche Aussage ist demnach richtig?
\renewcommand{\labelenumi}{(\alph{enumi})}
\begin{enumerate}
	\item $ A \wedge B $.
	\item $ A \vee B $.
	\item $ A \Rightarrow B $.
	\item $ A \Leftrightarrow B $.
\end{enumerate}
\ \\
\textbf{Lösung:}
\begin{mdframed}
	\underline{\textbf{Vorgehensweise:}}
	\renewcommand{\labelenumi}{\theenumi.}
	\begin{enumerate}
		\item Übersetze die Aufgabenstellung in Wahrheitswerte und leite die korrekte Antwort her.
	\end{enumerate}
\end{mdframed}
\underline{1. Übersetze die Aufgabenstellung in Wahrheitswerte und leite die korrekte Antwort her}\\
Uns ist bekannt, dass Hans die Party nicht besuchte. 
Demnach gilt $ A = F $.
Wir wissen nichts über Mark. Also kann entweder  $ B= W $ oder $ B = F $ gelten.
Unser Ziel ist die Aussage zu finden, welche unabhängig von $ B  $ wahr ist.
Die Antwort (a) können wir direkt ausschließen, da diese wegen $ A $ falsch ist.
Die Aussage in (b) kann nur wahr sein, falls $ B = W $ und für (d) gilt dies umgekehrt.
Also bleibt die Antwort (c) übrig.\\
Die Wahrheitstafel in der letzten Aufgabe zeigt, dass die Aussage von (c) wirklich unabhängig von dem Wahrheitswert von $ B $ ist.
Man sagt auch: \grqq Aus etwas Falschem kann alles folgen\grqq.\\
\\
Damit ist die Antwort (d) korrekt.\\
\\
Alternativ lässt sich diese Aufgabe auch über eine Wahrheitstabelle lösen.
Hierbei ist zu bedenken, dass
\begin{align*}
	(A \Leftrightarrow B)
	=
	(A\Rightarrow B) \wedge (B \Rightarrow A)
\end{align*}
gilt. Damit lässt sich aus der Wahrheitstabelle der letzten Aufgabe die passende Wahrheitstabelle aufstellen.
 

\newpage
\subsection*{\frage{3}{3}}
Die Folge $ \{ a_n \}_{n \in \N} $ ist beschränkt, monoton und konvergent.
Zudem gilt $ a_n \neq 0  $ für alle $ n $.
Sei $ \{b_n\}_{n \in \N} $ die Folge definiert durch $ b_n = \frac{1}{a_n} $.
Dann folgt:
\renewcommand{\labelenumi}{(\alph{enumi})}
\begin{enumerate}
	\item 
	$ \{b_n \}_{n \in \N} $ ist beschränkt.
	\item 
	$ \{b_n\}_{n \in \N} $ ist monoton.
	\item 
	$ \{b_n\}_{n \in \N} $ ist konvergent.
	\item
	Keine der obigen Eigenschaften folgt.
\end{enumerate}
\ \\
\textbf{Lösung:}
\begin{mdframed}
\underline{\textbf{Vorgehensweise:}}
\renewcommand{\labelenumi}{\theenumi.}
\begin{enumerate}
\item Finde passende Gegenbeispiele.
\end{enumerate}
\end{mdframed}

\underline{1. Finde passende Gegenbeispiele}\\
Für die Beschränktheit und Konvergenz finden wir mit $ a_n = \frac{1}{n} $ ein  Gegenbeispiel. Dies erkannt man an:
\begin{align*}
	b_n = \frac{1}{a_n}
	=
	\frac{1}{\frac{1}{n}} = n.
\end{align*}
Die Folge $ \{ b_n\}_{n \in \N } $ ist unbeschränkt und somit auch divergent.
Also sind die Antworten (a) und (c) falsch.\\
Ein Gegenbeispiel für die Monotonie ist schwieriger zu finden.
Sollte sich das Vorzeichen der $ a_n $ nicht ändern bleibt die Monotonie für $ \{ b_n\}_{n \in \N } $ erhalten.
Wir benötigen also eine monotone, konvergente Folge $ \{a_n\}_n \in \N $, welche das das Vorzeichen ändert.
Hierfür setzen wir
\begin{align*}
	a_n = \frac{1}{n} - 0.9.
\end{align*} 
Wichtig ist hier, dass $ a_n \neq 0 $ eingehalten wird.
Dann gilt $ a_1 > 0 $ und $ a_n < 0 $ für alle $ n \geq 2 $.
Die Monotonie liefert
\begin{align*}
	a_n \geq a_{n+1}
	\ \Leftrightarrow \
	\frac{a_n}{a_{n+1}}  \leq 1
	\ \Leftrightarrow \ 
	\frac{1}{a_{n+1}} \geq  \frac{1}{a_n}
	\ \Leftrightarrow \ 
	b_{n+1} \geq  b_n
	\ \Leftrightarrow \ 
	b_{n} \leq  b_{n+1}
\end{align*} 
für $ n \geq 2 $. Hierbei ist wichtig, das die Multiplikation mit einer negativen Zahl die Richtung der Ungleichung verändert. Mit 
\begin{align*}
	a_1 \geq a_2
	\ \Leftrightarrow \
	1 \geq \frac{a_2}{a_1}
	\ \Leftrightarrow \
	\frac{1}{a_2} \leq \frac{1}{a_1}
	\ \Leftrightarrow \
	b_2 \leq b_1
	\ \Leftrightarrow \
	b_1 \geq b_2
\end{align*}
erhalten wir das Argument für das Gegenbeispiel. Da 
\begin{align*}
	b_n \leq b_{n+1}
\end{align*}
nicht für alle $ n \geq 1 $ erfüllt (für $ n = 1 $), ist $ \{b_n\}_{n \in \N} $ nicht monoton.\\
\\
Damit ist die Antwort (d) korrekt. 




\newpage

\subsection*{\frage{4}{4}}
Sei $ \{ a_n \}_{n \in \N} $ eine geometrische Folge mit 
$ \frac{a_{n+1}}{a_n} = q \in (0,1) $.
Sei $ \{ b_n \}_{n \in \N} $ eine andere geometrische Folge mit
$ \frac{b_{n+1}}{b_n} = \frac{q}{2} $.\\
Zudem gilt: $ \sum_{k = 1}^\infty a_k = \sum_{k = 1}^\infty b_k $.
Dann folgt
\renewcommand{\labelenumi}{(\alph{enumi})}
\begin{enumerate}
	\item 
	$ b_1 = 2 a_1 $, falls $ q = \frac{1}{3} $.
	\item
	$ b_1 = 2 a_1 $, falls $ q = \frac{1}{2} $.
	\item
	$ b_1 = 2 a_1 $, falls $ q = \frac{2}{3} $.
	\item
	Die Bedingung $ b_1 = 2 a_1 $ ist nie erfüllt.
\end{enumerate}
\ \\
\textbf{Lösung:}
\begin{mdframed}
\underline{\textbf{Vorgehensweise:}}
\renewcommand{\labelenumi}{\theenumi.}
\begin{enumerate}
\item Gebe die Reihen über die geometrischen Folgen an.
\item Verwende die Voraussetzung aus der Aufgabe um die korrekte Antwort zu finden.
\end{enumerate}
\end{mdframed}

\underline{1. Gebe die Reihen über die geometrischen Folgen an}\\
Wir betrachten die geometrische Folge $ \{ a_n \}_{n \in \N} $ mit $ \frac{a_{n+1}}{a_n} = q \in (0,1) $.
Wegen 
\begin{align*}
	a_{n+1} = a_n q
\end{align*}
für alle $ n \in \N $ erhalten wir
\begin{align*}
	a_{n} = a_1 q^{n-1}.
\end{align*}
Mit der geometrischen Reihe erhalten wir wegen $ q \in (0,1) $
\begin{align*}
	\sum \limits_{k = 1}^\infty a_k 
	=
	\sum \limits_{k = 1}^\infty a_1 q^{k-1}
	=
	a_1 \cdot  \sum \limits_{k = 0}^\infty q^k
	= 
	\frac{a_1}{1-q}.
\end{align*}
Mit $ q \in (0,1)  $ folgt auch $ \frac{q}{2} \in (0,1)  $.
Damit gilt durch analoge Argumentation:
\begin{align*}
	\sum \limits_{k = 1}^\infty b_k = ...=
	\frac{b_1}{1- \frac{q}{2}}
	=
	\frac{2b_1}{2  -q}.
\end{align*}
\ \\
\underline{2. Verwende die Voraussetzung aus der Aufgabe um die korrekte Antwort zu finden}\\
Wegen $ \sum_{k = 1}^\infty a_k = \sum_{k = 1}^\infty b_k $ muss 
\begin{align*}
\textrm{(I)}	\frac{a_1}{1-q} = \frac{2b_1}{2  -q}
\end{align*}
erfüllt sein.
Nach den Regeln der Bruchrechnung gilt dies, falls $ a_1 = 2 b_1 $ und $ 1- q = 2 - q $ erfüllt ist.
Wir setzen die geforderte Bedingung $ b_1 = 2 a_1 $ in (I) ein und erhalten
\begin{align*}
	\frac{a_1}{1- q} =\frac{4 a_1}{2- q}.
\end{align*}
Diese Gleichung ist sicher für $ a_1 = 0 $ und alle $ q \in (0,1) $ erfüllt.
Nun betrachten wir noch was für $ a_1 \neq 0 $ geschieht.
Dann gilt
\begin{align*}
	\frac{a_1}{1- q} =\frac{4 a_1}{2- q}
	\ \Leftrightarrow \
	\frac{1}{1- q} =\frac{4}{2- q}
	\ \Leftrightarrow \
	2 - q = 4 (1-q)
	\ \Leftrightarrow \
	2 - q = 4 - 4 q)
	\ \Leftrightarrow \
	q  =\frac{2}{3}.
\end{align*}
Damit ist $ b_1 = 2 a_1 $, falls $ q = \frac{2}{3} $ gilt.\\
\\
Also ist die Antwort (c) korrekt.

\newpage
\subsection*{\frage{5}{3}}
Am Tag, an dem Mark einen Kredit über $ P_1= 1'000'000 $ CHF mit Zinssatz $ i_1 = 1 \% $ aufnimmt, nimmt Lucie einen Kredit über $ P_2= 800'000 $ CHF mit Zinssatz $ i_2 = 1.2 \% $ auf.
Mark zahlt den Kredit in konstanten Raten von $ 10'500 $ CHF jeweils am Ende des Jahres zurück, während Lucie $ 9'500 $ CHF auch jeweils am Ende des Jahres zurückbezahlt.\\
Welche der folgenden Antworten ist richtig?
\renewcommand{\labelenumi}{(\alph{enumi})}
\begin{enumerate}
	\item 
	Mark bezahlt den Kredit vor Lucie zurück.
	\item 
	Lucie bezahlt den Kredit vor Mark zurück.
	\item
	Lucie und Mark bezahlen den Kredit zum gleichen Zeitpunkt zurück.
	\item
	Keine der obigen Antworten ist richtig..
\end{enumerate}
\ \\
\textbf{Lösung:}
\begin{mdframed}
\underline{\textbf{Vorgehensweise:}}
\renewcommand{\labelenumi}{\theenumi.}
\begin{enumerate}
\item Berechne die jährlichen Zinszahlungen von Mark und Lucie.
\end{enumerate}
\end{mdframed}

\underline{1. Vergleiche die jährlichen Zinszahlungen von Mark und Lucie}\\
Wir stellen uns die Frage ob Mark und Lucie ihre Zinsen überhaupt zurückzahlen können. Für Mark erhalten wir mit 
\begin{align*}
	1'000'000 \cdot 1 \% = 10'000,
\end{align*}
die Zinsen des ersten Jahres.
Damit kann Mark seine Schulden mit der Rate $ 10'500 $ in endlicher Zeit zurückzahlen.
Für Lucie erhalten wir
\begin{align*}
	800'000 \cdot 1.2\% = 9'600
\end{align*}
für die Zinszahlung des ersten Jahres. Die Zinszahlung übersteigt die Rate von Lucie. Also kann Lucie ihre Schulden mit dieser Rate nie zurückzahlen.\\
\\
Damit ist die Antwort (a) korrekt.

 \newpage

\subsection*{\frage{6}{3}}
Seien $ f $ und $ g $ Funktionen einer reellen Variable mit Definitionsbereich $ D_f $ beziehungsweise $ D_g $.\\
Sei $ h = f \circ g $.
Welche der folgenden Aussagen über den Definitionsbereich $ D_h $ ist richtig?
\renewcommand{\labelenumi}{(\alph{enumi})}
\begin{enumerate}
	\item 
	$ D_h = D_f \cup D_g$.
	\item 
	$ D_h \subseteq D_g $.
	\item
	$ D_h \subseteq D_f $.
	\item
	$ D_h = D_f \cap  D_g $.
\end{enumerate}
\ \\
\textbf{Lösung:}
\begin{mdframed}
\underline{\textbf{Vorgehensweise:}}
\renewcommand{\labelenumi}{\theenumi.}
\begin{enumerate}
\item Finde die korrekte Antwort mit Hilfe der Definition. 

\end{enumerate}
\end{mdframed}

\underline{1. Finde die korrekte Antwort mit Hilfe der Definition}\\
Gegeben seien die Funktionen $ f $ und $ g $ einer reellen Variable mit den Definitionsbereichen $ D_f $ und $ D_g $.
Dann ist die Verknüpfung definiert durch
\begin{align*}
	h(x) = (f \circ g)(x):= f(g(x))
\end{align*}
für $ x \in D_h $. Da zuerst $ g(x) $ ausgewertet wird, muss zwingend $ x \in D_g $ folgen. Wir haben also $ x \in D_h \Rightarrow x \in D_g $ gezeigt.
Wegen
\begin{align*}
	D_h \subseteq D_g
	\ :\Leftrightarrow \
	x \in D_h \Rightarrow x \in D_g
\end{align*}
ist die Antwort (b) korrekt.\\
\\
Für die anderen Antwortmöglichkeiten lassen sich Gegenbeispiele konstruieren.\\
Sei $ f(x)  =  x $ und $ g(x) = \ln(x) $. Dann gilt $ D_f = \R $ und $ D_g = (0, \infty)$.
Dies ist ein Gegenbeispiel für Antwort (a).\\
Sei $ f(x) = \ln(x) $ und $ g(x ) = e^x $. Dann gilt $ D_f = (0, \infty) $ und $ D_g = \R $. Dies ist ein Gegenbeispiel für die Antworten (c) und (d). 
\newpage
\subsection*{\frage{7}{3}}
Eine Funktion $ f $ mit Definitionsbereich $ D_f \subseteq \R_{++} $ habe eine Elastizität $ \varepsilon_f $, die konstant gleich $ 2 $ ist.\\
\\
Welche der folgenden Aussagen über die die Wachstumsrate $ \rho_f $ von $ f $ ist richtig?
\renewcommand{\labelenumi}{(\alph{enumi})}
\begin{enumerate}
	\item 
	$ \rho_f $ ist streng monoton fallend.
	\item
	$ \rho_f $ ist streng monoton wachsend.
	\item
	$ \rho_f $ ist konstant.
	\item
	$ \rho_f $ ist nicht monoton.
\end{enumerate}
\ \\
\textbf{Lösung:}
\begin{mdframed}
\underline{\textbf{Vorgehensweise:}}
\renewcommand{\labelenumi}{\theenumi.}
\begin{enumerate}
\item Leite mit der Definition der Elastizität die korrekte Antwort her.
\end{enumerate}
\end{mdframed}

\underline{1. Leite mit der Definition der Elastizität die korrekte Antwort her}\\
Die Wachstumsrate $ \rho_f(x) $ von $ f $ ist gegeben durch $ \rho_f(x) = \frac{f^\prime(x)}{f(x)} $
Die Elastizität von $ f $ ist definiert durch
\begin{align*}
	\varepsilon_f(x) := x \frac{f^\prime(x)}{f(x)} = x \rho_f(x).
\end{align*}
Da $ \varepsilon_f $ konstant ist, erhalten wir
\begin{align*}
	2 = x \rho_f(x)
	\ \Leftrightarrow \ 
	\rho_f(x) = \frac{2}{x}
\end{align*}
für $ x \in D_f \subseteq \R_{++} $.
Also ist $ \rho_f $ streng monoton fallend.\\
\\
Damit ist Antwort (a) korrekt.

\newpage

\subsection*{\frage{8}{4}}
Eine differenzierbare Funktion $ f : D_f \to \R  $ ist streng konkav und erfüllt $ f(x)  > 0 $ für alle $ x \in D_f $.
Zudem gibt es ein $ x_0 \in D_f, x_0 \neq 0 $, so dass für die Elastizität von $ f $ an der Stelle $ \varepsilon_f(x_0 ) = 0 $ gilt.\\
Welche der folgenden Aussagen ist richtig?
\renewcommand{\labelenumi}{(\alph{enumi})}
\begin{enumerate}
	\item 
	Bei $ x_0 $ ist $ f $ elastisch.
	\item
	Bei $ x_0 $ besitzt $ f $ ein lokales Minimum.
	
	
	\item
	Bei $ x_0 $ besitzt $ f $ ein lokales Maximum.
	\item
	Keine der obigen Antworten ist richtig.
\end{enumerate}
\ \\
\textbf{Lösung:}
\begin{mdframed}
\underline{\textbf{Vorgehensweise:}}
\renewcommand{\labelenumi}{\theenumi.}
\begin{enumerate}
\item Leite mit der Definition der Elastizität die korrekte Antwort her.
\end{enumerate}
\end{mdframed}

\underline{1. Leite mit der Definition der Elastizität die korrekte Antwort her}\\
Wir machen uns auch in dieser Aufgabe die Definition der Elastizität $ \varepsilon_f $ zunutze.
Es gilt 
\begin{align*}
	\varepsilon_f(x) := x \frac{f^\prime(x)}{f(x)}
\end{align*}
für $ x \in D_f $. 
Sei $ x_0 $ die in der Aufgabe beschriebene Stelle.
Wegen $ x_0 \neq 0 $ und $ f(x_0 ) > 0 $ gilt
\begin{align*}
	0 = \varepsilon_f(x_0) = x_0 \frac{f^\prime(x_0)}{f(x_0)}
	=
	\underbrace{\frac{x_0}{f(x_0)}}_{\neq 0} f^\prime(x_0)
	\ \Leftrightarrow \
	f^{\prime}(x_0) = 0
\end{align*} 
wegen des Satzes des Nullprodukts.
Da $ f $ auf $ D_f $ konkav ist, liegt ein globales Maximum vor.
Außerdem sind globale Maxima stets lokal.\\
\\
Damit ist Antwort (c) korrekt.
\newpage
\subsection*{\frage{9}{3}}
Sei $ f $ eine Funktion einer reellen Variable, die mindestens $ n $-mal differenzierbar ist.
Sei $ P_k $ das Taylor-Polynom $ k $-ter Ordnung von $ f $ an der Stelle $ x_0 $, für $ k = 1,...,n-1 $, und $ R_k $ das zugehörige Restglied $ k $-ter Ordnung.\\
Welche der folgenden Aussagen ist richtig? 
\renewcommand{\labelenumi}{(\alph{enumi})}
\begin{enumerate}
	\item 
	Für alle $ x \in D_f $ und $  k = 2,...,n-1 $ gilt,
	dass $ R_k(x) < R_{k-1}(x) $.
	\item
	Für alle $ x \in D_f $ und $  k = 2,...,n-1 $ gilt,
	dass $ R_k(x) > R_{k-1}(x) $.
	
	\item
	Für alle $ x \in D_f $ und $  k = 2,...,n-1 $ gilt,
	dass $ R_k(x) = R_{k-1}(x) $.
	\item
	Keine der obigen Aussagen ist richtig.
\end{enumerate}
\ \\
\textbf{Lösung:}
\begin{mdframed}
	\underline{\textbf{Vorgehensweise:}}
	\renewcommand{\labelenumi}{\theenumi.}
	\begin{enumerate}
		\item Verwende das Restglied nach Lagrange.
	\end{enumerate}
\end{mdframed}

\underline{1. Verwende das Restglied nach Lagrange}\\
Das Restglied nach Lagrange ist gegeben durch
\begin{align*}
	R_k(x) 
	= 
	\frac{f^{k+1}(\xi)}{(k+1)!} x^{k+1}
\end{align*}
für $ \xi \in (0,x)$ oder $ \xi  \in (x, 0) $ (abhängig vom Vorzeichen von $ x $).
Wir werden nun ein Gegenbeispiel für die Antworten (a)--(c) angeben.
Wir setzen $ f(x)  = \sin(x)$. Dann gilt
\begin{align*}
	R_1\left(\frac{\pi}{2} \right)
	&=
	\frac{f^{(2)}(\xi_1)}{2!} \left( \frac{\pi}{2}  \right)^{2}
	=
	\frac{-\sin(\xi_1)}{(2)!} \left( \frac{\pi}{2}  \right)^{2} < 0\\
	R_2\left(\frac{\pi}{2} \right)
	&=
	\frac{f^{(3)}(\xi_2)}{3!} \left( \frac{\pi}{2}  \right)^{3}
	=
	\frac{-\cos(\xi_2)}{3!} \left( \frac{\pi}{2}  \right)^{3} < 0\\
	R_3\left(\frac{\pi}{2} \right)
	&=
	\frac{f^{(4)}(\xi_3)}{4!} \left( \frac{\pi}{2}  \right)^{4}
	=
	\frac{\sin (\xi_3)}{4!} \left( \frac{\pi}{2}  \right)^{4} > 0\\
	R_4\left(\frac{\pi}{2} \right)
	&=
	\frac{f^{(5)}(\xi_4)}{5!} \left( \frac{\pi}{2}  \right)^{5}
	=
	\frac{\cos (\xi_4)}{5!} \left( \frac{\pi}{2}  \right)^{5} > 0\\
	R_5\left(\frac{\pi}{2} \right)
	&=
	\frac{f^{(6)}(\xi_5)}{6!} \left( \frac{\pi}{2}  \right)^{6}
	=
	\frac{-\sin (\xi_5)}{6!} \left( \frac{\pi}{2}  \right)^{6} < 0
\end{align*}
für $ \xi_1,..,\xi_5 \in \left(0 , \frac{\pi}{2}\right) $.
Damit haben wir einen Widerspruch zur strengen Monotonie und Gleichheit bezüglich der Ordnung des Restglieds.\\
\\
Also ist die Antwort (d) korrekt.

\newpage
\subsection*{\frage{10}{2}}
Eine Funktion zweier reellen Variablen $ f $ sei homogen von Grad $ 2 $ und ihre partielle Elastizität erfülle 
\begin{align*}
	\varepsilon_{f,y}(x,y) = e^{x+y} +1. 
\end{align*}
Dann folgt:
\renewcommand{\labelenumi}{(\alph{enumi})}
\begin{enumerate}
	\item 
	$ 	\varepsilon_{f,x}(x,y)  = 1 + e^{x+y}$.
	\item
	$ 	\varepsilon_{f,x}(x,y)  = 1 - e^{x+y}$.
	
	\item
	$ 	\varepsilon_{f,x}(x,y)  =  e^{x+y}$.
	\item
$ 	\varepsilon_{f,x}(x,y)  = - e^{x+y}$.
\end{enumerate}
\ \\
\textbf{Lösung:}
\begin{mdframed}
	\underline{\textbf{Vorgehensweise:}}
	\renewcommand{\labelenumi}{\theenumi.}
	\begin{enumerate}
		\item Verwende die eulersche Relation. 
	\end{enumerate}
\end{mdframed}

\underline{1. Verwende die eulersche Relation}\\
Da $ f $ homogen vom Grad $ 2 $ ist, liefert die eulersche Relation
\begin{align*}
	\varepsilon_{f,x}(x,y) + \varepsilon_{f,y}(x,y) = 2
	\ \Leftrightarrow \
	\varepsilon_{f,x}(x,y) = 2 -  \varepsilon_{f,y}(x,y).
\end{align*}
Mit der Tatsache $ \varepsilon_{f,y}(x,y) = e^{x+y} +1 $ erhalten wir 
\begin{align*}
	\varepsilon_{f,x}(x,y) = 2 -  \varepsilon_{f,y}(x,y)=
	2- (e^{x+y} +1) 
	=
	1 - e^{x+y}. 
\end{align*}
Damit ist die Antwort (b) korrekt.
