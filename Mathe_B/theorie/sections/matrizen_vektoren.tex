\section{Matrizen und Vektoren}

\subsection{Matrizen}
\begin{mybox}{Definition}\index{Matrix}
\index{Matrix}
Eine $m \times n$ \textit{Matrix} $A$ ist eine Anordnung von $m$ Zeilen mit jeweils $n$ Spalten.
Dies wird durch folgendes Schema dargestellt:
\begin{align*}
A = (a_{ij})_{1 \leq i \leq m,1 \leq j \leq n} 
\begin{pmatrix}
a_{11} & a_{12} & \dots & a_{1n}\\
a_{21} & a_{22} & \dots & a_{2n}\\
\vdots & \vdots & \vdots & \vdots\\
a_{m1} & a_{m2} & \hdots & a_{mn}
\end{pmatrix}
\end{align*}
Mit
\begin{align*}
a_{ij}, \ 1 \leq i \leq m, \ 1 \leq j \leq n
\end{align*}
bezeichen wir den $ij$-ten Eintrag der Matrix.\\
Wir nennen eine Matrix \textit{quadratisch}, wenn $m = n$ gilt.
\end{mybox}
Mit Matrizen können wir rechnen und diese auf Gleichheit überprüfen. 
Bis auf die Matrixmultiplikation sind alle Operationen relativ intuitiv.
Deswegen werden wir diese zuerst anschauen.\\
\\
\textbf{Elementare Matrixoperationen}
\index{Matrix!Elementare Operationen}
\renewcommand{\labelenumi}{\theenumi.}
\begin{enumerate}

\item \textbf{Gleichheit von Matrizen}
\vspace{0.2cm}\\
Seien die Matrizen $A $ und $B$ jeweils $m \times n $-Matrizen.
$A$ und $B$ sind gleich, wenn alle Einträge übereinstimmen.
Formal lässt sich das durch
\begin{align*}
a_{ij} = b_{ij}
\end{align*}
für $1 \leq i \leq m$ und $1 \leq j \leq n$ formulieren.\\ \\
Wir können dies durch
\begin{align*}
\begin{pmatrix}
a & b \\
c & d
\end{pmatrix}
=
\begin{pmatrix}
1 & 2 \\
3 & 4
\end{pmatrix}
\
\Leftrightarrow
\ 
a = 1, \ b = 2, \ c = 3, \ d = 4
\end{align*}
veranschaulichen.
\item \textbf{Addition von Matrizen}\vspace{0.2cm}\\
\index{Matrix!Addition}
Seien die Matrizen $A $ und $B$ jeweils $m \times n $-Matrizen.
Dann addieren wir die einzelnen Einträge:
\begin{align*}
A+B &:=(a_{ij} + b_{ij})_{1 \leq i \leq m,1 \leq j \leq n}\\
&=
\begin{pmatrix}
a_{11} & a_{12} & \dots & a_{1n}\\
a_{21} & a_{22} & \dots & a_{2n}\\
\vdots & \vdots & \vdots & \vdots\\
a_{m1} & a_{m2} & \hdots & a_{mn}
\end{pmatrix}
+
\begin{pmatrix}
b_{11} & b_{12} & \dots & b_{1n}\\
b_{21} & b_{22} & \dots & b_{2n}\\
\vdots & \vdots & \vdots & \vdots\\
b_{m1} & b_{m2} & \hdots & b_{mn}
\end{pmatrix}\\
&=
\begin{pmatrix}
a_{11} +b_{11} & a_{12} + b_{12} & \dots & a_{1n}+b_{1n}\\
a_{21} + b_{21} & a_{22} + b_{22} & \dots & a_{2n}+b_{2n}\\
\vdots & \vdots & \vdots & \vdots\\
a_{m1} + b_{m1} & a_{m2} + b_{m2} & \hdots & a_{mn} + b_{mn}
\end{pmatrix}
\end{align*}
\textbf{Wichtig:} Die Matrizen müssen die gleiche Anzahl an Zeilen und Spalten haben.

Dies sieht zunächst ziemlich abstrakt aus, weswegen wir ein Beispiel zur Verdeutlichung betrachten.
\begin{align*}
A = 
\begin{pmatrix}
1 & 2\\
1 & 1
\end{pmatrix}, \
B =
\begin{pmatrix}
3 & 1 \\
2 & -1
\end{pmatrix}
\end{align*} 
Die Addition beider Matrizen ergibt
\begin{align*}
A+B 
=
\begin{pmatrix}
1+3 & 2 +1\\
1 +2 & 1+(-1)
\end{pmatrix}
=
\begin{pmatrix}
4 & 3\\
3 & 0
\end{pmatrix}.
\end{align*}

\item \textbf{Skalare Multiplikation}\vspace{0.2cm}\\
Sei $A$ eine $m \times n$ Matrix.
Dann gilt
\begin{align*}
\lambda \cdot A
&=
\lambda
\begin{pmatrix}
a_{11} & a_{12} & \dots & a_{1n}\\
a_{21} & a_{22} & \dots & a_{2n}\\
\vdots & \vdots & \vdots & \vdots\\
a_{m1} & a_{m2} & \hdots & a_{mn}
\end{pmatrix}
=
\begin{pmatrix}
\lambda a_{11} & \lambda a_{12} & \dots &\lambda a_{1n}\\
\lambda a_{21} &\lambda a_{22} & \dots &\lambda a_{2n}\\
\vdots & \vdots & \vdots & \vdots\\
\lambda a_{m1} &\lambda a_{m2} & \hdots &\lambda a_{mn}
\end{pmatrix}\\
&=(\lambda a_{ij})_{1 \leq i \leq m,1 \leq j \leq n}
\end{align*}
für $\lambda \in \mathbb{R}$.
Wir multiplizieren also die einzelnen Einträge mit dem skalaren Faktor.

\item \textbf{Transponierte Matrix}\vspace{0.2cm}\\
Sei $A$ eine $m \times n$ Matrix von der Form
\begin{align*}
\begin{pmatrix}
a_{11} & a_{12} & \dots & a_{1n}\\
a_{21} & a_{22} & \dots & a_{2n}\\
\vdots & \vdots & \vdots & \vdots\\
a_{m1} & a_{m2} & \hdots & a_{mn}
\end{pmatrix}.
\end{align*}
\index{Matrix!transponiert}
Dann ist die \textit{transponierte} $m \times n$ Matrix durch
\begin{align*}
A^\top
:=
\begin{pmatrix}
a_{11} & a_{21} & \dots & a_{m1}\\
a_{12} & a_{22} & \dots & a_{m2} \\
\vdots & \vdots & \vdots & \vdots\\
a_{1n} & a_{2n} & \dots & a_{nm}
\end{pmatrix}
\end{align*}
definiert.
Wir tauschen einfach Zeilen mit Spalten.
Am besten verdeutlichen wir dies anhand von Beispielen.\\
\\
\textbf{Beispiele:}
\begin{align*}
A = 
\begin{pmatrix}
1 & 2 & 3
\end{pmatrix}
&\Rightarrow
A^\top
= \begin{pmatrix}
1 \\ 2 \\ 3
\end{pmatrix}\\
A = 
\begin{pmatrix}
2 & 4 & 3\\
1 & 6 & 5\\
\end{pmatrix}
&\Rightarrow
A^\top
= 
\begin{pmatrix}
2 & 1 \\
4 & 6  \\
3 & 5
\end{pmatrix}\\
I := 
\begin{pmatrix}
1 & 0 & 0\\
0 & 1 & 0\\
0 & 0 & 1
\end{pmatrix}
&\Rightarrow
I^\top = I
\end{align*}
\newpage
Für das Transponieren gelten folgende Rechenregeln:
\begin{itemize}
\item
$(A^\top)^\top =A$
\item
$(A + B)^\top = A^\top + B^\top$
\item
$(A^{-1})^\top = (A^\top)^{-1}$
\item
$(A \cdot B)^\top = B^\top \cdot A^\top$
\end{itemize}
Hierbei bezeichnet $A^{-1}$ die inverse Matrix.
Zu dieser kommen wir später.\\ \\
Wir nennen eine quadratische Matrix $A$ \textit{symmetrisch}, falls gilt:
\index{Matrix!symmetrisch}
\begin{align*}
A = A^\top
\end{align*}
\end{enumerate} 
\newpage
\subsection{Matrixmultiplikation}
\index{Matrixmultiplikation}
\begin{mybox}{Matrixmultiplikation}
Sei $A$ eine $ m \times n$ Matrix und $B$ eine $n \times k$ Matrix.
Dann ist $C := A \cdot B$ eine $m \times k$ Matrix und für deren Einträge gilt
\begin{align*}
c_{ij} := \sum \limits_{l = 1}^n a_{il} \cdot b_{lj}
= a_{i1} \cdot b_{1j} + a_{i2} \cdot b_{2j} + \cdots + a_{in} \cdot b_{nj}.
\end{align*}
Schematisch können wir dies durch
\begin{align*}
C = 
A \cdot B
&=
\underbrace{\begin{pmatrix}
\dots & \dots & \dots & \dots\\
\vdots & \vdots & \vdots & \vdots\\
a_{i1}& a_{i2} & \dots & a_{in}\\
\vdots & \vdots & \vdots & \vdots\\
\dots & \dots & \dots & \dots
\end{pmatrix}}_{n \ \text{Spalten}}
\cdot
\left. 
\begin{pmatrix}
\dots & \dots & b_{1j} & \dots \\
\vdots & \vdots & b_{2j} & \vdots\\
\vdots & \vdots & \vdots & \vdots\\
\dots & \dots & b_{nj} &  \dots
\end{pmatrix}
\right\rbrace \ n \ \text{\small Zeilen}
&=
\begin{pmatrix}
\dots &\dots &\dots &\dots\\
\vdots & \vdots & \vdots & \vdots\\
\ast & \ast & c_{ij} & \ast\\ 
\vdots & \vdots & \vdots & \vdots\\
\dots &\dots &\dots &\dots
\end{pmatrix}
\end{align*}
darstellen.

\end{mybox}
\begin{figure}[H]
\centering
\begin{tikzpicture}[scale = 1,>=latex]
% les matrices
\matrix (A) [matrix of math nodes,%
             nodes = {node style ge},%
             left delimiter  = (,%
             right delimiter = )] at (0,0)
{%
  a_{11} & a_{12} & \ldots & a_{1n}  \\
  \node[node style sp](A-2-1) {a_{21}};%
         & \node[node style sp] (A-2-2) {a_{22}};%
                  & \ldots%
                           & \node[node style sp] (A-2-4){a_{2n}}; \\
  \vdots & \vdots & \ddots & \vdots  \\
  a_{m1} & a_{m2} & \ldots & a_{mn}  \\
};
\node [draw,below=10pt] at (A.south) 
    { $A$ : \textcolor{red}{$m$ Zeilen} $n$ Spalten};

\matrix (B) [matrix of math nodes,%
             nodes = {node style ge},%
             left delimiter  = (,%
             right delimiter =)] at (6*\myunit,6*\myunit)
{%
  b_{11} & \node[node style sp](B-1-2) {b_{12}};%
                  & \ldots & b_{1k}  \\
  b_{21} & \node[node style sp](B-2-2) {b_{22}};%
                  & \ldots & b_{2k}  \\
  \vdots & \vdots & \ddots & \vdots  \\
  b_{n1} & \node[node style sp](B-4-2) {b_{n2}};%
                  & \ldots & b_{nk}  \\
};
\node [draw,above=10pt] at (B.north) 
    { $B$ : $n$ Zeilen \textcolor{red}{$k$ Spalten}};
% matrice résultat
\matrix (C) [matrix of math nodes,%
             nodes = {node style ge},%
             left delimiter  = (,%
             right delimiter = )] at (6*\myunit,0)
{%
  c_{11} & c_{12} & \ldots & c_{1k} \\
  c_{21} & \node[node style sp,red](C-2-2) {c_{22}};%
                  & \ldots & c_{2k} \\
  \vdots & \vdots & \ddots & \vdots \\
  c_{m1} & c_{m2} & \ldots & c_{mk} \\
};
% les fleches
\draw[blue] (A-2-1.north) -- (C-2-2.north);
\draw[blue] (A-2-1.south) -- (C-2-2.south);
\draw[blue] (B-1-2.west)  -- (C-2-2.west);
\draw[blue] (B-1-2.east)  -- (C-2-2.east);
\draw[<->,red](A-2-1) to[in=180,out=90]
	node[arrow style mul] (x) {$a_{21}\cdot b_{12}$} (B-1-2);
\draw[<->,red](A-2-2) to[in=180,out=90]
	node[arrow style mul] (y) {$a_{22}\cdot b_{22}$} (B-2-2);
\draw[<->,red](A-2-4) to[in=180,out=90]
	node[arrow style mul] (z) {$a_{2n}\cdot b_{n2}$} (B-4-2);
\draw[red,->] (x) to node[arrow style plus] {$+$} (y)%
                  to node[arrow style plus] {$+\raisebox{.5ex}{\ldots}+$} (z)%
                  to (C-2-2.north west);


\node [draw,below=10pt] at (C.south) 
    {$ C=A\cdot B$ : \textcolor{red}{$m$ Zeilen}  \textcolor{red}{$k$ Spalten}};

\end{tikzpicture}
\end{figure}
\newpage
Zur Verdeutlichung werden wir uns gemeinsam ein paar Beispiele anschauen.
\subsubsection*{Anwendungsbeispiel A}

Wir betrachten die $1 \times 3$ Matrix
\begin{align*}
A=
\begin{pmatrix}
1 &  2  & 1
\end{pmatrix}
\end{align*}
und die $3 \times 1$ Matrix
\begin{align*}
B=
\begin{pmatrix}
-1\\ 1\\  1
\end{pmatrix}.
\end{align*}
Berechnen Sie die Matrix $A \cdot B$.
Dann erhalten wir durch 
\begin{align*}
A \cdot B =
\begin{pmatrix}
1 & 2 &  1
\end{pmatrix}
\cdot
\begin{pmatrix}
-1\\ 1\\  1
\end{pmatrix}
=
\begin{pmatrix}
1 \cdot (-1) + 2 \cdot 1 + 1 \cdot 1
\end{pmatrix}
=
\begin{pmatrix}
2
\end{pmatrix}
= 2
\end{align*}
eine $1 \times 1$ Matrix.
Eine $1 \times 1$ Matrix ist ein Skalar, weswegen wir diese Form von Matrixmultiplikation auch 
\textit{Skalarprodukt} nennen.\index{Skalarprodukt}\\
\\
Was fällt auf?
$A$ besitzt eine Zeile mit drei Einträgen und $B$ besitzt eine Spalte mit drei Einträgen.
Wir verrechnen also Zeilen mit Spalten.
Umgangssprachlich sagt man auch \glqq Zeile mal Spalte\grqq.

\subsubsection*{Anwendungsbeispiel B}
Berechnen Sie die Matrix $A \cdot B$ mit
\begin{align*}
A 
=
\begin{pmatrix}
1 &  2  & 1\\
4 & 1 & 3
\end{pmatrix},
\quad
B=
\begin{pmatrix}
-1\\ 1\\  1
\end{pmatrix}.
\end{align*}
Wir erhalten:
\begin{align*}
A \cdot B
=
\begin{pmatrix}
1 &  2  & 1\\
4 & 1 & 3
\end{pmatrix}
\cdot
\begin{pmatrix}
-1\\ 1\\  1
\end{pmatrix}
= 
\begin{pmatrix}
1 \cdot (-1) + 2 \cdot 1 + 1 \cdot 1\\
4 \cdot (-1) + 1 \cdot 1 + 3 \cdot 1
\end{pmatrix}
=
\begin{pmatrix}
2\\
0
\end{pmatrix}
\end{align*}

\subsubsection*{Anwendungsbeispiel C}
Berechnen Sie die Matrix $A \cdot B$ mit 
\begin{align*}
A 
=
\begin{pmatrix}
1 &  2  & 1\\
4 & 1 & 3
\end{pmatrix},
\quad
B = 
\begin{pmatrix}
-1 & 2\\ 
 1 & -1\\
   1 & 2
\end{pmatrix}.
\end{align*}
Wir erhalten: 
\begin{align*}
A \cdot B
= 
\begin{pmatrix}
1 &  2  & 1\\
4 & 1 & 3
\end{pmatrix}
\cdot
\begin{pmatrix}
-1 & 2\\ 
 1 & -1\\
   1 & 2
\end{pmatrix}
= 
\begin{pmatrix}
1 \cdot (-1) + 2 \cdot 1  + 1 \cdot 1 & 1 \cdot 2 + 2 \cdot (-1) +1 \cdot 2\\
4 \cdot (-1) + 1 \cdot 1 + 3 \cdot 1 & 4 \cdot 2 + 1 \cdot (-1) + 3 \cdot 2
\end{pmatrix}
= 
\begin{pmatrix}
2 & 2\\
0 & 13
\end{pmatrix}
\end{align*}

\newpage
Wichtig ist noch, dass im Allgemeinen \textbf{nicht}
\begin{align*}
A \cdot B = B \cdot A
\end{align*}
gilt. 
Oft ist diese Operation nicht definiert.\\
%Für
%\begin{align*}
%A 
%=
%\begin{pmatrix}
%1 &  2  & 1\\
%4 & 1 & 3
%\end{pmatrix},
%\quad
%B=
%\begin{pmatrix}
%-1\\ 1\\  1
%\end{pmatrix}
%\end{align*}
%ist $B \cdot A$ nicht definiert.
%Wir rechnen \glqq Zeile mal Spalte\grqq.
%Nun haben die Zeilen von $B$ nur einen Eintrag und die Spalten von $A$ von zwei. 
%Wir betrachten
Wir schauen uns hierzu folgendes Beispiel an:
\begin{align*}
A =
\begin{pmatrix}
1 & 0 \\
0 & 0
\end{pmatrix},
\quad
B = 
\begin{pmatrix}
0 & 1 \\
0 & 0
\end{pmatrix}
\end{align*}
Wir rechnen \glqq Zeile mal Spalte\grqq und erhalten folgendes Resultat:
\begin{align*}
A \cdot B 
&= 
\begin{pmatrix}
1 & 0 \\
0 & 0
\end{pmatrix}
\cdot
\begin{pmatrix}
0 & 1 \\
0 & 0
\end{pmatrix}
=
\begin{pmatrix}
1\cdot 0 + 0 \cdot 0 & 1 \cdot 1 + 0 \cdot 0 \\
0\cdot 0+ 0 \cdot 0 & 0\cdot 1+ 0 \cdot 0 
\end{pmatrix}
=
\begin{pmatrix}
0 & 1 \\
0 & 0
\end{pmatrix}
\\
B \cdot A
&=
\begin{pmatrix}
0 & 1 \\
0 & 0
\end{pmatrix}
\cdot 
\begin{pmatrix}
1 & 0 \\
0 & 0
\end{pmatrix}
=
\begin{pmatrix}
0\cdot 1 + 1 \cdot 0 & 0 \cdot 0 + 1 \cdot 0 \\
0\cdot 1+ 0 \cdot 0 & 0\cdot 0+ 0 \cdot 0 
\end{pmatrix}
=
\begin{pmatrix}
0 & 0 \\
0 & 0
\end{pmatrix}
\end{align*}
Somit sehen wir, dass $A \cdot B \neq B \cdot A $ gilt.\\
\\
Eine Matrix, welche null in allen Einträgen ist, bezeichnen wir als \textit{Nullmatrix}.\index{Nullmatrix}
Diese kürzen wir mit $0$ ab, um uns Schreibarbeit zu ersparen.
An
\begin{align*}
B \cdot A
&= 
\begin{pmatrix}
0 & 0 \\
0 & 0
\end{pmatrix}
=0
\end{align*}
sehen wir, dass $B \cdot A = 0$ gelten kann, obwohl $A, B \neq 0$ sind.\\

\begin{mybox}{Einheitsmatrix}\index{Einheitsmatrix}
Wir bezeichnen mit
\begin{align*}
I := 
\begin{pmatrix}
1 & 0 & 0 & \dots & 0 \\
0 & 1 & 0 & \dots & 0 \\
0 & 0 & 1 & \dots & 0\\
\vdots &  &  & \ddots & \\
0 & 0 & 0 & \dots & 1 
\end{pmatrix}
\end{align*}
die $n \times n$ \textit{Einheitsmatrix}.
Sei $A$ eine $n \times n$ Matrix.
Dann gilt
\begin{align*}
I \cdot A = A \cdot I = A
\end{align*}
\end{mybox}\ \\
Die Einheitsmatrix hat also dieselbe Funktion wie die $1$ in den rellen Zahlen.
Hier gilt auch
\begin{align*}
a \cdot 1 = 1 \cdot a = a
\end{align*}
für alle $a \in \mathbb{R}$.

\newpage
\subsection{Vektoren}
In den letzten beiden Abschnitten wurden Matrizen und die dafür notwendigen Rechenoperationen eingeführt.
Diese lassen sich nun analog auf Vektoren anwenden.\\
\begin{mybox}{Definition}\index{Vektor}
Wir nennen eine $n \times 1$ Matrix
\begin{align*}
\textbf{v}
=
\begin{pmatrix}
x_1 \\
\vdots\\
x_n
\end{pmatrix}
\end{align*}
einen $n$-dimensionalen \textit{Spaltenvektor}.\\ \\
Analog nennen wir einen $1 \times n$ Matrix
\begin{align*}
\textbf{w}
=
\begin{pmatrix}
x_1 & \dots & x_n 
\end{pmatrix}
\end{align*}
einen $n$-dimensionalen \textit{Zeilenvektor}.
\end{mybox}\ \\
Uns fällt auf, dass
\begin{align*}
\textbf{v} = \textbf{w}^\top
\end{align*}
gilt.
Der transponierte Vektor von $\textbf{w}$ entspricht dem Vektor $\textbf{v}$.
Die Unterscheidung zwischen Zeilen -und Spaltenvektoren ist meist nicht notwendig.
Diese spielt eine Rolle bei der Multiplikation von einer Matrix mit einem Vektor.\\
Wenn $A$ eine $n \times m$ Matrix ist, muss $\textbf{v}$ für 
\begin{align*}
A \cdot \textbf{v}
\end{align*}
ein Spaltenvektor mit $m$ Einträgen sein.
Dies wird die häufigste Operation sein, bei welcher wir mit Matrizen und Vektoren arbeiten.
Für
\begin{align*}
\textbf{v} \cdot A
\end{align*}
ist ein Zeilenvektor $\textbf{v}$ mit $n$ Einträgen notwendig. 
Ansonsten sprechen wir meist einfach von \textit{Vektoren} mit der entsprechenden Dimension.\\

\begin{mybox}{Skalarprodukt}
Seien $\textbf{v}$ und $\textbf{w}$ zwei $n$-dimensionale Vektoren.
Dann ist durch
\begin{align*}
\textbf{v} \cdot \textbf{w}
:=
\textbf{v}^\top \cdot \textbf{w} = \textbf{v} \cdot \textbf{w}^\top
=
\begin{pmatrix}
v_1 & \dots & v_n
\end{pmatrix}
\cdot 
\begin{pmatrix}
w_1\\
\vdots\\
w_n
\end{pmatrix}
= 
v_1 \cdot w_1 + \dots + v_n \cdot w_n
\end{align*}\index{Skalarprodukt}
das \textit{Skalarprodukt} zwischen $\textbf{v}$ und $\textbf{w}$ definiert.
Wir nennen $\textbf{v}$ und $\textbf{w}$ zueinander \textit{orthogonal}, falls
\begin{align*}
\textbf{v} \cdot \textbf{w} = 0 
\end{align*}
gilt. \index{Vektor!orthogonal}
\end{mybox}
Wir sehen, dass sich hier jegliche Rechnung mit Vektoren auf die Multiplikation von Matrizen zurückführen lässt.
\subsubsection*{Anwendungsbeispiel A}
Wir betrachten:
\begin{align*}
\textbf{v}
= 
\begin{pmatrix}
1\\ 2 \\ 3
\end{pmatrix},
\
\textbf{w} 
=
\begin{pmatrix}
1\\ 1 \\ 1
\end{pmatrix}
\ 
&\Rightarrow
\textbf{v} \cdot \textbf{w} = 1 \cdot 1 + 2 \cdot 1 + 3 \cdot 1 = 6\\
\textbf{v}
= 
\begin{pmatrix}
1\\ 2 \\ 3
\end{pmatrix},
\
\textbf{w} 
=
\begin{pmatrix}
-2\\ -2 \\ 2
\end{pmatrix}
\ 
&\Rightarrow
\textbf{v} \cdot \textbf{w} = 1 \cdot (-2) + 2 \cdot (-2) + 3 \cdot 2 = -2 -4 + 6 = 0
\end{align*}

\newpage

\subsection{Gaußverfahren für Matrizen}
In diesem Abschnitt werden wir uns von technischer Seite dem Gaußverfahren nähern.
Dies machen wir, um Aussagen über Matrixeigenschaften zu treffen.
Unser Ziel ist es eine $n \times m$ Matrix $A$ durch elementare Matrixumformungen 
\begin{align*}
A \leadsto
\left(\mkern3mu
    % \hskip-\arraycolsep
    \begin{array}{*{13}c}
      \multicolumn{1}{|c}{a_{11}} & (*) & (*) & (*) & (*) & (*) & (*) & \ldots & \ldots & \ldots & \ldots & \ldots & (*) \\
      \cline{1-1}
      0 & \multicolumn{1}{|c}{a_{22}} & (*) & (*) & (*) & (*) & (*)  & \ldots & \ldots & \ldots & \ldots & \ldots & (*) \\
      \cline{2-3}
      0 & 0 & 0 & \multicolumn{1}{|c}{a_{34}} & (*) & (*) & (*) & \ldots & \ldots & \ldots & \ldots & \ldots & (*)\\
      \cline{4-6}
      0 & 0 & 0 & 0 & 0 & 0 & \multicolumn{1}{|c}{a_{47}} & (*) & \ldots & \ldots & \ldots & \ldots & (*) \\
      \cline{7-7}\cdashline{8-9}
      \vdots & & & & \vdots & & & & \vdots & \multicolumn{1}{:c}{} & & & \vdots \\
      0 & 0 & 0 & \ldots & \ldots & \ldots & \ldots & \ldots & 0 & \multicolumn{1}{|c}{a_{rl}} & (*) & \ldots & (*) \\
      \cline{10-13}
      0 & 0 & 0 & \ldots  & \ldots & \ldots & \ldots & \ldots & \ldots & \ldots & 0 & 0 & 0 \\
      0 & 0 & 0 & \ldots  & \ldots & \ldots & \ldots & \ldots & \ldots & \ldots & 0 & 0 & 0 \\
      \vdots & & & & \vdots & & & & \vdots & & & & \vdots \\
      0 & 0 & 0 & \ldots  & \ldots & \ldots & \ldots & \ldots & \ldots & \ldots & 0 & 0 & 0
    \end{array}
   \hskip-\arraycolsep\right)
\end{align*}
auf \textit{Zeilenstufenform} zu bringen.\index{Zeilenstufenform}\\ \\
\textbf{Anschauliche Bedeutung}: In jeder neuen Zeile stehen am Anfang der Zeile mehr Nullen als in der vorherigen Zeile.
Außer die Anzahl der Nullen entsprechen bereits der Spaltenanzahl.\\
Wir werden diese elementaren Matrixumformungen später zum Lösen von linearen Gleichungssystemen benötigen. Hierbei steht $\ast$ für eine beliebige Zahl.
\\
\index{Zeilenstufenform}

\begin{mybox}{Elementare Matrixumformungen}
Durch
\renewcommand\labelenumi{\arabic{enumi}.} 
\begin{enumerate}
\item Addition des $c$-fachen Werts einer Zeile zu einer anderen Zeile. 
\item Multiplikation einer Zeile mit Wert ungleich Null.
\item Vertauschen zweier Zeilen.
\end{enumerate}
lässt sich eine Matrix $A$ auf Zeilenstufenform bringen.
\end{mybox}
Wir schauen uns hierzu folgende Anwendungsbeispiele an:
\renewcommand\labelenumi{\arabic{enumi}.}
\begin{enumerate}
\item
\begin{align*}
A 
=
\begin{gmatrix}[p]
1 & 2 & -1 &3\\
-1 & 3 & 1 & 2\\
2 & 3 & 2 & 4
\rowops
\add[ \cdot 1]{0}{1}
\add[\cdot(-2)]{0}{2}	
\end{gmatrix}
\leadsto
\begin{gmatrix}[p]
1 & 2 & -1 &3\\
0 & 5 & 0 & 5\\
0 & -1 & 4 & -2
\rowops	
\end{gmatrix}
\end{align*}
Hierbei ist durch die Pfeile visualisiert:\\
Die Addition des $1$-fachen der ersten Zeile auf die zweite Zeile.\\
Die Addition des $(-2)$-fachen der ersten Zeile auf die dritte Zeile.
\item
\begin{align*}
A =
\begin{gmatrix}[p]
2 & 4 & 6\\
1 & 2 & 3
\rowops
\mult{0}{ \cdot \frac{1}{2}}
\end{gmatrix}
\leadsto
\begin{pmatrix}
1 & 2 & 3\\
1 & 2 & 3
\end{pmatrix}
\end{align*}
Hier multiplizieren wir die erste Zeile mit dem Faktor $\frac{1}{2}$.

\item
\begin{align*}
A
=
\begin{gmatrix}[p]
1 & 2\\
3 & 4
\rowops
\swap{0}{1}
\end{gmatrix}
\leadsto
\begin{pmatrix}
3 & 4 \\
1 & 2
\end{pmatrix}
\end{align*}
Hier vertauschen wir die erste mit der zweiten Zeile.
\end{enumerate}
Diese Operationen werden wir nun verwenden, um Matrizen auf Zeilenstufenform zu bringen.
Die folgenden Matrizen befinden sich bereits in Zeilenstufenform:
\begin{align*}
A &= \begin{pmatrix}
1 & \ast & \ast\\
0 & 0 & 0
\end{pmatrix}\\
B &= 
\begin{pmatrix}
1 & 2 & 3 & 2\\
0 & 1 & 0 & 1\\
0 & 0 & 1 & 2\\
0 & 0 & 0 & 0
\end{pmatrix}\\
C &=
\begin{pmatrix}
1 & \ast & \ast & \ast\\
0 & 1 & \ast & \ast\\
0 & 0 & 1 & \ast\\
0 & 0 & 0 & 1
\end{pmatrix}\\
D&=
\begin{pmatrix}
1 & \ast & \ast\\
0 & 0 & 0\\
0 & 0 & 0
\end{pmatrix}\\
I &= 
\begin{pmatrix}
1 & 0 & 0\\
0 & 1 & 0\\
0 & 0 & 1
\end{pmatrix}
\end{align*}
Zur Übung rechnen wir weitere Beispiele.
\subsubsection*{Anwendungsbeispiel A}
\begin{align*}
A 
=
\begin{gmatrix}[p]
1 & 2 & -1 &3\\
-1 & 3 & 1 & 2\\
2 & 3 & 2 & 4
\rowops
\add[ \cdot 1]{0}{1}
\add[\cdot(-2)]{0}{2}	
\end{gmatrix}
&\leadsto
\begin{gmatrix}[p]
1 & 2 & -1 &3\\
0 & 5 & 0 & 5\\
0 & -1 & 4 & -2
\rowops
\mult{1}{\cdot \frac{1}{5}}	
\end{gmatrix}
\leadsto
\begin{gmatrix}[p]
1 & 2 & -1 &3\\
0 & 1 & 0 & 1\\
0 & -1 & 4 & -2
\rowops
\add[ \cdot (-2)]{1}{0}
\add[\cdot 1] {1}{2}	
\end{gmatrix}\\
&\leadsto
\begin{gmatrix}[p]
1 & 0 & -1 &1\\
0 & 1 & 0 & 1\\
0 & 0 & 4 & -1
\rowops
\end{gmatrix}
\end{align*}
\subsubsection*{Anwendungsbeispiel B}
\begin{align*}
A =
\begin{gmatrix}[p]
2 & 4 & 6\\
1 & 2 & 3
\rowops
\mult{0}{ \cdot \frac{1}{2}}
\end{gmatrix}
\leadsto
\begin{gmatrix}[p]
1 & 2 & 3\\
1 & 2 & 3
\rowops
\add[\cdot (-1)]{0}{1}
\end{gmatrix}
\leadsto
\begin{gmatrix}[p]
1 & 2 & 3\\
0 & 0 & 0
\rowops
\add[\cdot (-1)]{0}{1}
\end{gmatrix}
\end{align*}
\subsubsection*{Anwendungsbeispiel C}
\begin{align*}
A = 
\begin{gmatrix}[p]
1 & 2  & 3 & 6 & \BAR & 5\\
1 & 3 & 4 & 8 &\BAR & 7 \\
2 & 1 & 3 & 4 &\BAR & -1
\rowops
\add[-1]{0}{1}
\add[-2]{0}{2}
\end{gmatrix}
&\leadsto
\begin{gmatrix}[p]
1 & 2  & 3 & 6 & \BAR & 5\\
0 & 1 & 1 & 2 &\BAR & 2 \\
0 & -3 & -3 & -8 &\BAR & -11
\rowops
\add[3]{1}{2}
\end{gmatrix}\\
&\leadsto
\begin{gmatrix}[p]
1 & 2  & 3 & 6 & \BAR & 5\\
0 & 1 & 1 & 2 &\BAR & 2 \\
0 & 0 & 0 & -2 &\BAR & -5
\rowops
\add[1]{2}{1}
\add[3]{2}{0}
\end{gmatrix}\\
&\leadsto
\begin{gmatrix}[p]
1 & 2  & 3 & 0 & \BAR & -10\\
0 & 1 & 1 & 0 &\BAR & -3 \\
0 & 0 & 0 & -2 &\BAR & -5
\rowops
\mult{2}{\cdot ( -1 )}
\end{gmatrix}\\
&\leadsto
\begin{gmatrix}[p]
1 & 2  & 3 & 0 & \BAR & -10\\
0 & 1 & 1 & 0 &\BAR & -3 \\
0 & 0 & 0 & 2 &\BAR & 5
\end{gmatrix}
\end{align*}
Hier hat man bereits nach dem zweiten Schritt eine Zeilenstufenform.
Hieran sehen wir auch, dass es keine eindeutige Zeilenstufenform geben muss.
Warum es sich oft trotzdem lohnt, weiter umzuformen, sehen wir bei linearen Gleichungssystemen.
\subsubsection*{Anwendungsbeispiel D}
\begin{align*}
A=
\begin{gmatrix}[p]
2 & 0 &0 & 0 \\
1 & 5 & 1  & -1\\
1 & 3 & -3 & 0 \\
0 & 1 & -1 & 0
\rowops
\add[-\frac{1}{2}]{0}{1}
\add[-\frac{1}{2}]{0}{2}
\end{gmatrix}
&\leadsto
\begin{gmatrix}[p]
2 & 0 &0 & 0 \\
0 & 5 & 1  & -1\\
0 & 3 & -3 & 0 \\
0 & 1 & -1 & 0
\rowops
\swap{1}{3}
\end{gmatrix}\\
&\leadsto
\begin{gmatrix}[p]
2 & 0 &0 & 0 \\
0 & 1& -1  & 0 \\
0 & 3 & -3 & 0 \\
0 & 5 & 1  & -1
\rowops
\add[-3]{1}{2}
\add[-5]{1}{3}
\end{gmatrix}\\
&\leadsto
\begin{gmatrix}[p]
2 & 0 &0 & 0 \\
0 & 1& -1  & 0 \\
0 & 0 & 0 & 0 \\
0 & 0 & 6  & -1
\rowops
\swap{2}{3}
\end{gmatrix}\\
&\leadsto
\begin{gmatrix}[p]
2 & 0 &0 & 0 \\
0 & 1& -1  & 0 \\
0 & 5 & 6  & -1 \\
0 & 0 & 0 & 0 
\rowops
\add[-5]{1}{2}
\end{gmatrix}
\\
&\leadsto
\begin{gmatrix}[p]
2 & 0 &0 & 0 \\
0 & 1& -1  &  0\\
0 & 0 & 11  & -1 \\
0 & 0 & 0 & 0 
\end{gmatrix}
\end{align*}

Wir haben in diesem Abschnitt nur elementare Matrixumformungen auf Zeilen betrachtet.
Theoretisch lassen sich diese analog auf Spalten anwenden.
\newpage
\subsection{Lineare Unabhängigkeit von Vektoren}

\begin{mybox}{Definition}
Das System $\textbf{v}_1, \dots, \textbf{v}_n$ von Vektoren heißt \textit{linear unabhängig}, falls
\begin{align*}
\lambda_1 \cdot  \textbf{v}_1 + \lambda_2 \cdot \textbf{v}_2 + \dots + \lambda_n \cdot \textbf{v}_n = \textbf{0}
\ \Rightarrow \
\lambda_1 = \dots = \lambda_n = 0
\end{align*}
gilt.
Ansonsten heißt das System \textit{linear abhängig}.
\index{Lineare!Abhängigkeit}\index{Lineare!Unabhängigkeit}
\end{mybox}
Diese Definition wirkt zunächst sehr abstrakt.
Deswegen betrachten wir das System
\begin{align*}
\textbf{e}_1 =
\begin{pmatrix}
1\\ 0 \\0
\end{pmatrix},
\textbf{e}_2 =
\begin{pmatrix}
0  \\ 1 \\ 0
\end{pmatrix},
\textbf{e}_3 = 
\begin{pmatrix}
0 \\ 0 \\ 1
\end{pmatrix}
\end{align*}
und überlegen, welche Möglichkeit wir haben, den Nullvektor $\textbf{0}$ darzustellen.
Für
\begin{align*}
\lambda_1 \textbf{e}_1 + \lambda_2 \textbf{e}_2 + \lambda_3 \textbf{e}_3 = \textbf{0}
\end{align*}
folgt direkt $\lambda_1 = \lambda_2 = \lambda_3 = 0$.
Mit diesen drei Vektoren lässt sich der dreidimensionale Raum darstellen, weswegen man das System
\begin{align*}
\textbf{e}_1, \textbf{e}_2, \textbf{e}_3
\end{align*}
auch \textit{Basis} von $\mathbb{R}^3$ nennt.\index{Basis}
Nun betrachten wir die zwei Vektoren
\begin{align*}
\begin{pmatrix}
1 \\2
\end{pmatrix},
\begin{pmatrix}
2 \\ 4
\end{pmatrix}.
\end{align*}
Für diese gilt
\begin{align*}
2 \cdot 
\begin{pmatrix}
1 \\2
\end{pmatrix}
= \begin{pmatrix}
2 \\4
\end{pmatrix}
\Leftrightarrow
2 \cdot 
\begin{pmatrix}
1 \\2
\end{pmatrix} 
+ (-1) \cdot
\begin{pmatrix}
2 \\4
\end{pmatrix}
= \textbf{0},
\end{align*}
wodurch \textbf{nicht} folgt, dass alle Vorfaktoren gleich Null sind.
Damit sind diese Vektoren linear abhängig.
\newpage
\begin{mybox}{Entscheidungsverfahren}
\index{Lineare!Entscheidungsverfahren}
Sei $\textbf{v}_1, \dots, \textbf{v}_n$ ein System von Vektoren und $A = \begin{pmatrix} \textbf{v}_1 & \dots & \textbf{v}_n \end{pmatrix} $ die Matrix mit den Vektoren $\textbf{v}_1, \dots, \textbf{v}_n$ als Spalten.
Nun wenden wir auf $A$ mit
\begin{align*}
A \leadsto \tilde{A}
\end{align*}
elementare Matrixumforungen an und erhalten eine Zeilenstufenform $\tilde{A}$.\\ \\
Nun gilt:
\begin{itemize}
\item
$\tilde{A}$ enthält keine Nullzeile:\\ 
Das System $\textbf{v}_1, \dots, \textbf{v}_n$ ist linear unabhängig.

\item
$\tilde{A}$ enthält mindestens eine Nullzeile:\\
Das System $\textbf{v}_1, \dots, \textbf{v}_n$ ist linear abhängig.
\end{itemize}
Wir können sogar etwas über die Anzahl der linear unabhängigen Vektoren aussagen.
Es gilt:
\begin{align*}
\text{Anzahl l. u. Vektoren} \ = \ \text{Anzahl Zeilen von } \tilde{A} - \text{Anzahl Nullzeilen von } \tilde{A}
\end{align*}
\end{mybox}
Wir testen dies für unser linear abhängiges System
\begin{align*}
\begin{pmatrix}
1 \\2
\end{pmatrix},
\begin{pmatrix}
2 \\ 4
\end{pmatrix}
\Rightarrow
A = \begin{pmatrix}
1 & 2 \\
2 & 4
\end{pmatrix}.
\end{align*}
Wir sehen an
\begin{align*}
\begin{gmatrix}[p]
1 & 2 \\
2 & 4
\rowops
\add[\cdot (-2)]{0}{1}
\end{gmatrix}
\leadsto
\begin{pmatrix}
1 & 2 \\
0 & 0
\end{pmatrix},
\end{align*}
dass das System linear abhängig ist, da wir eine Nullzeile haben und die Anzahl linear unabhängiger Vektoren ist eins.

\newpage
\subsection{Rang einer Matrix}
\begin{mybox}{Definition}
Sei $A = \begin{pmatrix}
\textbf{v}_1 & \dots & \textbf{v}_n
\end{pmatrix}$ eine Matrix, wobei $\textbf{v}_1, \dots, \textbf{v}_n$ die einzelnen Spalten(-vektoren) von $A$ sind.
Durch
\begin{align*}
\mathrm{rg}(A) := \ \text{Anzahl der linear unabhängigen Spalten von } A
\end{align*}
definieren wir den \textit{Rang} der Matrix $A$.\index{Matrix!Rang}
\end{mybox}
Wir greifen auf vorherige Beispiele zurück, um uns das Vorgehen zu veranschaulichen.\\

\textbf{Beispiele:}
\begin{align*}
A &= \begin{pmatrix}
1 & \ast & \ast\\
0 & 0 & 0
\end{pmatrix}
\Rightarrow
\mathrm{rg}(A) = 1\\
A &= 
\begin{pmatrix}
1 & 2 & 3 & 2\\
0 & 1 & 0 & 1\\
0 & 0 & 1 & 2\\
0 & 0 & 0 & 0
\end{pmatrix}
\Rightarrow
\mathrm{rg}(A) = 3\\
A&=
\begin{pmatrix}
1 & \ast & \ast & \ast\\
0 & 1 & \ast & \ast\\
0 & 0 & 1 & \ast\\
0 & 0 & 0 & 1
\end{pmatrix}
\Rightarrow
\mathrm{rg}(A) = 4\\
A &= \begin{pmatrix}
1 & \ast & \ast\\
0 & 0 & 0\\
0 & 0 & 0
\end{pmatrix}
\Rightarrow
\mathrm{rg}(A) = 1\\
I &= 
\begin{pmatrix}
1 & 0 & 0\\
0 & 1 & 0\\
0 & 0 & 1
\end{pmatrix}
\Rightarrow
\mathrm{rg}(A) = 3\\
A 
&=
\begin{gmatrix}[p]
1 & 2 & -1 &3\\
-1 & 3 & 1 & 2\\
2 & 3 & 2 & 4
\end{gmatrix}
\leadsto
\dots
\leadsto
\begin{gmatrix}[p]
1 & 0 & -1 &1\\
0 & 1 & 0 & 1\\
0 & 0 & 4 & -1
\rowops
\end{gmatrix}
\Rightarrow
\mathrm{rg}(A) = 3\\
A &=
\begin{gmatrix}[p]
2 & 4 & 6\\
1 & 2 & 3
\end{gmatrix}
\leadsto
\dots
\leadsto
\begin{gmatrix}[p]
1 & 2 & 3\\
0 & 0 & 0
\rowops
\end{gmatrix}
\Rightarrow 
\mathrm{rg}(A) = 1\\
A &= \begin{gmatrix}[p]
1 & 2  & 3 & 6 & \BAR & 5\\
1 & 3 & 4 & 8 &\BAR & 7 \\
2 & 1 & 3 & 4 &\BAR & -1
\end{gmatrix}
\leadsto
\dots
\leadsto
\begin{gmatrix}[p]
1 & 2  & 3 & 0 & \BAR & -10\\
0 & 1 & 1 & 0 &\BAR & -3 \\
0 & 0 & 0 & 2 &\BAR & 5
\end{gmatrix}
\Rightarrow
\mathrm{rg}(A) = 3\\
A &= 
\begin{gmatrix}[p]
2 & 0 &0 & 0 \\
1 & 5 & 1  & -1\\
1 & 3 & -3 & 0 \\
0 & 1 & -1 & 0
\rowops
\end{gmatrix}
\leadsto
\dots
\leadsto
\begin{gmatrix}[p]
2 & 0 &0 & 0 \\
0 & 1& -1  &  0\\
0 & 0 & 6  & -1 \\
0 & 0 & 0 & 0 
\rowops
\end{gmatrix}
\Rightarrow
\mathrm{rg}(A) = 3
\end{align*}

\newpage
\subsection{Inverse Matrix}\index{Matrix!invers}\index{Matrix!regulär}
\index{Matrix!invers}
\begin{mybox}{Definition:}
Sei $A$ eine quadratische $n \times n$ Matrix.
Falls eine quadratische $n \times n $ Matrix $B$ mit
\begin{align*}
A \cdot B = B \cdot A = I
\end{align*}
existiert, nennen wir diese die \textit{inverse} Matrix zu $A$.
Für gewöhnlich verwendet man $A^{-1} := B$ um dies zu kennzeichnen.\\
Wenn die inverse Matrix existiert, sagen wir auch $A$ ist \textit{regulär}.
Ansonsten sagen wir, $A$ ist \textit{singulär}.
\end{mybox}
Wir kennen dies schon von den reellen Zahlen.
Diese sind $1 \times 1$ Matrizen.
Für eine Zahl $a \in \mathbb{R} \setminus \lbrace 0 \rbrace$ gilt immer
\begin{align*}
a \cdot a^{-1} = a \cdot \frac{1}{a} = \frac{1}{a} \cdot a = 1.
\end{align*}
Falls $A$ invertierbar ist, gilt 
\begin{align*}
A \textbf{x} = \textbf{b}
\Leftrightarrow
\textbf{x} = A^{-1} \textbf{b}.
\end{align*}
Wir finden also ein eindeutiges $\textbf{x}$, welches diese Gleichung löst.
Dies wird uns später beim Lösen von linearen Gleichungssystemen helfen.\\

\textbf{Rechenregeln:}
\begin{itemize}
\item $(A^{-1})^{-1} = A$
\item $I^{-1} = I$
\item $(A\cdot B )^{-1} = B^{-1} \cdot A^{-1}$
\end{itemize}

Eine berechtigte Frage ist nun, wie wir inverse Matrizen berechnen können.
Hierfür benötigen wir die elementaren Matrixumformungen.
Wir werden dieses Verfahren nun für die Matrix
\begin{align*}
A = 
\begin{gmatrix}[p]
1 & 1\\
0 & 1
\end{gmatrix} 
\end{align*}
betrachten und es danach allgemein beschreiben.
Wir erhalten durch 
\begin{align*}
\begin{gmatrix}[p]
A & \BAR & I
\end{gmatrix}
=
\begin{gmatrix}[p]
1 & 1 & \BAR & 1 & 0 \\
0 & 1 & \BAR & 0 & 1 
\rowops
\add[\cdot (-1)]{1}{0}
\end{gmatrix}
\leadsto
\begin{gmatrix}[p]
1 & 0 & \BAR & 1 & -1 \\
0 & 1 & \BAR & 0 & 1 
\rowops
\end{gmatrix}
\end{align*}
die inverse Matrix
\begin{align*}
A^{-1} = 
\begin{pmatrix}
1 & -1\\
0 & 1
\end{pmatrix}
\end{align*}
zu $A$.
Durch Nachrechnen sieht man schnell, dass dies die korrekte inverse Matrix ist.
Wir sehen also, dass uns elementare Matrixumformungen helfen, die inverse Matrix zu bestimmen, falls diese existiert.\vspace{0.1cm}
\begin{mybox}{Verfahren zum Bestimmen der inversen Matrix}\index{Verfahren inv. Matrix}
Sei $A$ eine $n \times n$ Matrix. Falls die inverse Matrix $A^{-1}$ existiert, bestimmen wir diese durch elementare Matrixumformungen nach dem Schema
\begin{align*}
\begin{gmatrix}[p]
A & \BAR & I 
\end{gmatrix}
\leadsto
\dots 
\leadsto
\begin{gmatrix}[p]
I & \BAR & A^{-1}
\end{gmatrix}.
\end{align*}
Das bedeutet, dass wir auf der linken Seite die Matrix $A$ und auf der rechten Seite die Einheitsmatrix $I$ eintragen.
Dann bringen wir die linke Seite mit elementaren Matrixumformungen auf die Einheitsmatrix $I$.
\end{mybox}

Eine berechtigte Frage ist nun, wann die inverse Matrix existiert. 
Dies können wir bereits durch den Rang der Matrix beantworten, denn es gilt
\begin{align*}
A \ \text{invertierbar/regulär} \Leftrightarrow \mathrm{rg}(A) = n = \text{ Anzahl linear unabhängiger Zeilen/Spalten}.
\end{align*}
Aus dem Verfahren können wir herauslesen, dass invertierbare Matrizen durch elementare Zeilenumformungen auf die Einheitsmatrix gebracht werden können.
D.h. $A \leadsto \dots \leadsto I$.\\
Insbesondere gibt es dann den Zwischenschritt:
\begin{align*}
A \leadsto \dots \leadsto
\begin{pmatrix}	[cccc]
1 &  \ast  & \ldots & \ast \\
0  &  1 & \ldots & \ast \\
\vdots & \vdots & \ddots & \vdots \\
0  &   0       &\ldots & 1
\end{pmatrix}.
\end{align*}
Es gibt also keine Nullzeile. 
Damit sind die Spaltenvektoren linear unabhängig und es gilt $\mathrm{rg}(A) = n$.\\
Im Abschnitt über Determinanten werden wir noch das effektivere Kriterium
\begin{align*}
A \ \text{invertierbar/regulär} \Leftrightarrow \det(A) \neq 0
\end{align*}
kennenlernen.\\

\begin{mybox}{Kriterium für Invertierbatkeit/Regularität}\index{Krit. für Invert.}
Es gilt
\begin{align*}
A \ \text{invertierbar/regulär} 
\Leftrightarrow
\mathrm{rg}(A) = n
\Leftrightarrow
\det(A) \neq 0
\end{align*}
\end{mybox}

Bevor man die inverse Matrix berechnet, sollte man immer erst überprüfen, ob diese überhaupt existiert.

\newpage
\subsection{Formales Rechnen mit Matrizen}
\index{Matrix!formales Rechnen}
Wir haben in den letzten Abschnitten folgende Rechnengesetze für Matrizen kennengelernt:
\begin{itemize}
\item
$(A^\top)^\top =A$
\item
$(A + B)^\top = A^\top + B^\top$
\item
$(A^{-1})^\top = (A^\top)^{-1}$
\item
$(A \cdot B)^\top = B^\top \cdot A^\top$
\item $(A^{-1})^{-1} = A$
\item $I^{-1} = I$
\item $(A\cdot B )^{-1} = B^{-1} \cdot A^{-1}$
\end{itemize}
Dabei nennen wir eine Matrix $A$ symmetrisch, falls $A = A^\top$ gilt.
\ \\
\ \\
\subsubsection*{Anwendungsbeispiel A}
Die quadratischen Matrizen $A$, $B$ seien regulär. $B$ ist auch symmetrisch.
Zeigen Sie , dass gilt
\begin{align*}
B^{-1} ( A^\top B)^\top (BA)^\top B^{-1} = A A^\top
\end{align*}
Da $ B $ symmetrisch ist, gilt $ B = B^\top $.\\
Diese Aufgabe lösen wir durch Umformen der linken Seite:
\begin{align*}
B^{-1} ( A^\top B)^\top (BA)^\top B^{-1}
=
B^{-1}  B^\top A A^\top B^\top B^{-1}
=
B^{-1}  B A A^\top B B^{-1}
=
I \cdot A A^\top \cdot I
=
A A^\top
\end{align*}
\ \\
\subsubsection*{Anwendungsbeispiel B}
Die quadratischen Matrizen $A$, $B$ seien regulär. $A$ ist auch symmetrisch.
Zeigen Sie , dass gilt
\begin{align*}
(BB^\top)(AB^{-1})^\top(BA)^{-1}(BA^\top )= BA
\end{align*}
Diese Aufgabe lösen wir durch Umformen der linken Seite:
\begin{align*}
(BB^\top)(AB^{-1})^\top(BA)^{-1}(BA^\top )
=
(BB^\top)(B^{-1})^\top A^\top A^{-1} B^{-1} (BA^\top )
=
BB^\top(B^ \top)^{-1} A A^{-1} B^{-1} BA^\top 
=
B I I I A^\top =
B A^\top = B A
\end{align*}
\newpage
\subsection{Determinanten}
Mit Determinanten ist es möglich, quadratischen Matrizen eine Zahl zuzuordnen.
Damit erhalten wir die Möglichkeit, Aussagen über bestimmte Matrixeigenschaften zu treffen.
\begin{mybox}{Untermatrix}
\index{Untermatrix}
Sei $A$ eine $n \times n$ Matrix.\\
Dann bezeichnen wir mit $A_{ij}$ die $(n-1) \times (n-1)$ Untermatrix von $A$, bei 
der die $i$-te Zeile und $j$-te Spalte gestrichen werden.
\end{mybox}
Dies lässt sich am besten an einem Beispiel veranschaulichen.\\ \\
\textbf{Beispiel:}
\begin{align*}
A=
\begin{pmatrix}
1 & 2& 4 &7 \\
3 & 5 & 1 & 3\\
4 & 6 & 7 & 2\\
7 & 1 & 2 & 6
\end{pmatrix}
\Rightarrow
A_{11}
&=
\begin{pmatrix}
5 & 1 & 3\\
6 & 7 & 2 \\
1 & 2 & 6
\end{pmatrix}, \quad
A_{21}
=
\begin{pmatrix}
2 & 4 & 7 \\
6 & 7 & 2 \\
1 & 2 & 6
\end{pmatrix}\\
A_{34}
&=
\begin{pmatrix}
1 & 2 & 4 \\
3 & 5 & 1 \\
7 & 1 & 2
\end{pmatrix}, \quad
A_{44}
=
\begin{pmatrix}
1 & 2 & 4 \\
3 & 5 & 1 \\
4 & 6 & 7
\end{pmatrix}
\end{align*}
$A_{11}$ bedeutet, dass die erste Zeile und erste Spalte gestrichen wird.\\
$A_{21}$ bedeutet, dass die zweite Zeile und erste Spalte gestrichen wird.\\
$A_{34}$ bedeutet, dass die dritte Zeile und vierte Spalte gestrichen wird.\\
$A_{44}$ bedeutet, dass die vierte Zeile und vierte Spalte gestrichen wird.\\
\begin{mybox}{Elementare Formeln}
\index{Determinante!elementar}
\begin{itemize}
\item
Sei $A$ eine $2 \times 2$ Matrix.
Dann ist die Determinante von $A$ durch
\begin{align*}
\det(A)= \det 
\begin{pmatrix}
a_{11} & a_{12}\\
a_{21} & a_{22}
\end{pmatrix} 
= 
\begin{vmatrix}
a_{11} & a_{12}\\
a_{21} & a_{22}
\end{vmatrix}
=
a_{11} \cdot a_{22} - a_{21} \cdot a_{12}
\end{align*}
definiert.
\item
Sei $A$ eine $3 \times 3$ Matrix.\index{Regel von Sarrus}
Dann ist die \textit{Regel von Sarrus} durch
\begin{align*}
\det A
&=
\det
\begin{pmatrix}
a_{11} & a_{12} & a_{13}\\
a_{21} & a_{22} & a_{23}\\
a_{31} & a_{32} & a_{33}
\end{pmatrix}
= 
\begin{vmatrix}
a_{11} & a_{12} & a_{13}\\
a_{21} & a_{22} & a_{23}\\
a_{31} & a_{32} & a_{33}
\end{vmatrix}\\
&=
a_{11}a_{22}a_{33}
+
a_{12}a_{23}a_{31}
+
a_{13}a_{21}a_{32}
-
a_{31}a_{22}a_{13}
-
a_{32}a_{23}a_{11}
-
a_{33}a_{21}a_{12}
\end{align*}
gegeben.
Mit dem Schema
\begin{center}
\begin{tikzpicture}
     \matrix [%
       matrix of math nodes,
       column sep=1em,
       row sep=1em
     ] (sarrus) {%
       a_{11} & a_{12} & a_{13} & a_{11} & a_{12} \\
       a_{21} & a_{22} & a_{23} & a_{21} & a_{22} \\
       a_{31} & a_{32} & a_{33} & a_{31} & a_{32} \\
     }; 
 
     \path ($(sarrus-1-3.north east)+(0.5em,0)$) edge[dotted] ($(sarrus-3-3.south east)+(0.5em,0)$)
           (sarrus-1-1)                          edge         (sarrus-2-2)
           (sarrus-2-2)                          edge         (sarrus-3-3)
           (sarrus-1-2)                          edge         (sarrus-2-3)
           (sarrus-2-3)                          edge         (sarrus-3-4)
           (sarrus-1-3)                          edge         (sarrus-2-4)
           (sarrus-2-4)                          edge         (sarrus-3-5)
           (sarrus-3-1)                          edge[dashed] (sarrus-2-2)
           (sarrus-2-2)                          edge[dashed] (sarrus-1-3)
           (sarrus-3-2)                          edge[dashed] (sarrus-2-3)
           (sarrus-2-3)                          edge[dashed] (sarrus-1-4)
           (sarrus-3-3)                          edge[dashed] (sarrus-2-4)
           (sarrus-2-4)                          edge[dashed] (sarrus-1-5);
 
     \foreach \c in {1,2,3} {\node[anchor=south] at (sarrus-1-\c.north) {$+$};};
     \foreach \c in {1,2,3} {\node[anchor=north] at (sarrus-3-\c.south) {$-$};};
\end{tikzpicture}
\end{center}
müssen wir diese Regel nicht auswendig wissen.   
\end{itemize}
\end{mybox}

Da wir auch Determinaten für $4 \times 4$ Matrizen berechnen wollen, benötigen wir noch einen allgemeineren Zugang.
Wir führen diesen zurück auf die elementaren Formeln.\\

\begin{mybox}{Allgemeine Definition}\index{Determinante}
Sei $A$ eine quadratische $n \times n$ Matrix.
Dann ist die \textit{Determinante} durch
\begin{align*}
\det(A) = |A| 
&:= 
\sum \limits_{i=1}^n (-1)^{i+j} \cdot a_{ij} \cdot \det(A_{ij}) \ \text{(Entwicklung nach $j$-ter Spalte)}\\
&=
\sum \limits_{j=1}^n (-1)^{i+j} \cdot a_{ij} \cdot \det(A_{ij}) \ \text{(Entwicklung nach $i$-ter Spalte)}
\end{align*}
definiert. 
\end{mybox}
Wir wollen dies nun auf $4 \times 4$ Matrizen anwenden.
Das Schema 
\begin{align*}
\begin{pmatrix}
+ & - & + & -\\
- & + & - & +\\
+ & - & + & -\\
- & + & - & +
\end{pmatrix}
\end{align*}
hilft uns bei der Entwicklung nach Spalten oder Zeilen.
Gegeben sei
\begin{align*}
A = \begin{pmatrix}
a_{11} & a_{12} & a_{13} & a_{14}\\
a_{21} & a_{22} & a_{23} & a_{24}\\
a_{31} & a_{32} & a_{33} & a_{34}\\
a_{41} & a_{42} & a_{43} & a_{44}
\end{pmatrix}.
\end{align*}
Wir berechnen die Determinante von $A$, indem wir nach der ersten Spalte entwickeln.
Durch 
\begin{align*}
\det(A) = a_{11} \cdot \det (A_{11}) - a_{21} \cdot \det(A_{21}) + a_{31} \cdot \det(A_{31}) - a_{41} \cdot \det(A_{41})
\end{align*}
erhalten wir eine Formel für $4 \times 4$ Matrizen. Die Vorzeichenverteilung in unserer Formel stimmt mit dem obigen Vorzeichenschema überein.
Wir wiederholen den Vorgang, indem wir nach der zweiten Spalte entwickeln:
\begin{align*}
\det(A) =
-a_{12} \cdot \det(A_{12}) + a_{22} \cdot \det(A_{22}) - a_{32} \cdot \det(A_{32}) + a_{42} \cdot \det(A_{42})
\end{align*}
Die Untermatrizen sind alles $3 \times 3$ Matrizen.
Darauf können wir die Regel von Sarrus anwenden. Dies ist jedoch sehr aufwendig.
Deswegen werden wir nach einer Möglichkeit suchen, die Regel von Sarrus nur einmal anzuwenden.\\

\textbf{Rechenregeln für Determinanten:}\index{Determinante!Rechenregeln}
\begin{itemize}
\item $\det(A \cdot B) = \det(A) \cdot \det(B)$.
\item $\det(A) = \det(A^\top)$.\\
Aus diesem Grund ist irrelevant, ob wir nach Zeilen oder Spalten entwickeln.
\item $\det(A^{-1}) = \frac{1}{\det(A)}$.
\item Die Addition eines Vielfaches($\neq 0$) einer Zeile auf eine andere Zeile ändert den Wert der Determinante nicht.
Wir betrachten
\begin{align*}
A =
\begin{pmatrix}
1 & 1 & 3\\
1 & 2 & 4\\
1 & 3 & 2
\end{pmatrix}.
\end{align*}
Auf diese Matrix können wir die Regel von Sarrus anwenden 
oder das $(-1)$-fache der ersten Zeile auf die zweite und dritte Zeile addieren.
Dann gilt:
\begin{align*}
\det A
=
\begin{vmatrix}
1 & 1 & 3 \\
1 & 2 & 4\\
1 & 3 & 2
\end{vmatrix}
=
\begin{vmatrix}
1 & 1 & 3 \\
0 & 1 & 1\\
0 & 2 & -1
\end{vmatrix}
=
1 \cdot \det(A_{11}) - 0 \cdot \det(A_{21}) + 0 \cdot \det(A_{31})
=
\begin{vmatrix}
1 & 1 \\
2 & -1
\end{vmatrix}
= 1 \cdot (-1) - 2 = 3
\end{align*}
Wir haben das Problem auf das Berechnen einer Determinante einer $2 \times 2$ Matrix reduziert.
Nun betrachten wir
\begin{align*}
B =
\begin{pmatrix}
1 & \ast  & \ast & \ast\\
\ast &\ast &  \ast  & \ast\\
\ast &\ast &  \ast  & \ast\\
\ast &\ast &  \ast  & \ast
\end{pmatrix}.
\end{align*} 
Hierfür gilt
\begin{align*}
\det(B) 
=
\begin{vmatrix}
1 & \ast  & \ast & \ast\\
\ast &\ast &  \ast  & \ast\\
\ast &\ast &  \ast  & \ast\\
\ast &\ast &  \ast  & \ast
\end{vmatrix}
= \begin{vmatrix}
1 & \ast  & \ast & \ast\\
0&\ast &  \ast  & \ast\\
0 &\ast &  \ast  & \ast\\
0 &\ast &  \ast  & \ast
\end{vmatrix},
\end{align*}
wodurch wir die Regel von Sarrus nur einmal anwenden müssen.
Es kann auch sein, dass wir Glück haben und wir $\det(B_{11})$ ähnlich wie $\det(A)$ berechnen können.
\end{itemize}

\begin{mybox}{Zusammenhang Determinante-Invertierbarkeit-Rang}
Sei $A$ eine $n\times n$ Matrix.
Dann gilt:
\begin{align*}
\det(A) \neq 0 
&\Leftrightarrow
\mathrm{rg}(A) = n
\Leftrightarrow
\ \text{$A$ ist invertierbar bzw. regulär}\\
\det(A) = 0 
&\Leftrightarrow
\mathrm{rg}(A) < n
\Leftrightarrow
\ \text{$A$ ist nicht invertierbar bzw. singulär}
\end{align*}
\end{mybox}
\newpage
\subsubsection*{Anwendungsbeispiel A}

Das System von $3$-dimensionalen Vektoren
$\lbrace \textbf{u}_1, \textbf{u}_2, \textbf{u}_3 \rbrace$
ist linear abhängig.
Sei $A = [\textbf{u}_1, \textbf{u}_2, \textbf{u}_3]$ die Matrix mit Spaltenvektoren $\textbf{u}_1$, $\textbf{u}_2$ und $\textbf{u}_3$. 
\renewcommand{\labelenumi}{(\alph{enumi})}
\begin{enumerate}
\item 
$A^n$ ist regulär für alle $n \in \mathbb{N}$.
\item
$A^n$ ist singulär für alle $n \in \mathbb{N}$.
\item
$A^n$ ist regulär für $n$ ungerade und singulär für $n$ gerade.
\item
$A^n$ ist singulär für $n$ ungerade und regulär für $n$ gerade.
\end{enumerate}
\ \\
\textbf{Lösung:}
\begin{mdframed}
\underline{\textbf{Vorgehensweise:}}
\renewcommand{\labelenumi}{\theenumi.}
\begin{enumerate}
\item Überlege dir, ob $A$ regulär oder singulär ist und löse damit die Aufgabe.
\end{enumerate}
\end{mdframed}

\underline{1. Überlege dir, ob $A$ regulär oder singulär ist und löse damit die Aufgabe}\\
Eine quadratische Matrix ist genau dann regulär, wenn die Spalten linear unabhängig sind.
Da die Spalten von $A$ linear abhängig sind, ist $A$ singulär.
Damit gilt auch $\det(A) = 0$.
Aus dem Zusammenhang
\begin{align*}
\det\left(A^n\right) = \det(A)^n = 0^n = 0
\end{align*}
erhalten wir die Singularität von $A^n$ für alle $n \in \mathbb{N}$.
\\
\ \\
Damit ist Antwort (b) richtig.
\newpage
\subsubsection*{Anwendungsbeispiel B}
Gegeben sei die $3 \times 3 $ Matrix
\begin{align*}
A
= 
\begin{pmatrix}
1 & 1 & 0\\
1 & 0 & 1 \\
1 & 1 & 1
\end{pmatrix}.
\end{align*}
Dann gilt:
\renewcommand{\labelenumi}{(\alph{enumi})}
\begin{enumerate}
\item 
$A^{-1}
= 
\begin{pmatrix}
1 & 1 & -1 \\
0 & -1 & 1\\
-1 & 0 & 1
\end{pmatrix}$.
\item
$A^{-1}
= 
\begin{pmatrix}
1 & 1 & -1 \\
1 & -1 & 1\\
-1 & 0 & 1
\end{pmatrix}$.
\item
$A^{-1}
= 
\begin{pmatrix}
1 & 1 & -1 \\
0 & -1 & 1\\
-1 & 1 & 1
\end{pmatrix}$.
\item
$A$ ist singulär.
\end{enumerate}
\ \\
\textbf{Lösung:}
\begin{mdframed}
\underline{\textbf{Vorgehensweise:}}
\renewcommand{\labelenumi}{\theenumi.}
\begin{enumerate}
\item Überlege dir, welche Antworten falsch sind und gebe die richtige Antwort an.

\end{enumerate}
\end{mdframed}

\underline{1. Überlege dir, welche Antworten falsch sind und gebe die richtige Antwort an}\\
Zunächst gilt 
\begin{align*}
\det(A) 
&= 
\begin{gmatrix}[v]
1 & 1 & 0\\
1 & 0 & 1 \\
1 & 1 & 1
\rowops
\add[\cdot (-1)]{0}{1}
\add[\cdot (-1)]{0}{2}
\end{gmatrix}
=
\begin{gmatrix}[v]
1 & 1 & 0\\
0 & -1 & 1 \\
0 & 0 & 1
\end{gmatrix}\\
&=
1 \cdot
\begin{gmatrix}[v]
 -1 & 1 \\
 0 & 1
\end{gmatrix} 
- 0\cdot
\begin{gmatrix}[v]
 1 & 0 \\
 0 & 1
\end{gmatrix}
+
0 \cdot
\begin{gmatrix}[v]
 1 & 0 \\
 -1 & 1
\end{gmatrix}
= 1 \cdot (-1) \cdot 1
= -1 \neq 0,
\end{align*}
womit $A$ regulär ist.
Damit ist Antwort (d) falsch.
Wir untersuchen nun nacheinander die Antworten (b) und (c). Uns ist bekannt, dass der Zusammenhang
\begin{align*}
A \cdot A^{-1} = 
\begin{pmatrix}
1 & 0 & 0 \\
0 & 1 & 0 \\
0 & 0 & 1
\end{pmatrix}
\end{align*}
gelten muss. Wir überprüfen bei den Matrizen von (b) und (c), ob unpassende Einträge vorliegen.
Für Antwort (b) betrachten wir die erste Zeile von $A$ mal die erste Spalte von $A^{-1}$.
Wir erhalten
\begin{align*}
( 1 , 1 , 0 ) \cdot 
\begin{pmatrix}
1\\
1\\
-1
\end{pmatrix}
= 
1 \cdot 1 + 1 \cdot 1 + 0 \cdot -1 
= 2
\end{align*}
für den ersten Diagonaleintrag.
Somit ist Antwort (b) falsch.\\ \\
Für Antwort (c) wählen wir  die dritte Zeile von $A$ mal die zweite Spalte $A^{-1}$.
Dies ergibt den zweiten Eintrag der dritten Zeile von $A \cdot A^{-1}$. Dieser sollte gleich $0$ sein.
Wegen
\begin{align*}
(1,1,1) \cdot
\begin{pmatrix}
1\\
-1 \\
1
\end{pmatrix}
= 1 \cdot 1 + 1 \cdot (-1) + 1 \cdot 1
= 1 \neq 0
\end{align*}
ist auch Antwort (c) falsch.\\
\\
Es gibt noch zwei weitere Lösungsmöglichkeiten.
Die Erste wäre (a)-(c) nachzurechnen. 
Dies ist aber zeitaufwendig und fehleranfällig.
Die Zweite wäre das Gauß-Verfahren durchzurechnen.
Die angegebene Methode lässt sich jedoch bis auf die Determinante ohne schriftliche Rechnungen lösen, was zeitsparend ist.
\\
\ \\
Damit ist Antwort (a) richtig.
\newpage
\subsubsection*{Anwendungsbeispiel C}
Gegeben sei die Matrix
\begin{equation*}
A = 
\begin{pmatrix}
1 & t & 1 \\
2 & t & 0 \\
-1 & t & 2t
\end{pmatrix},
\end{equation*}
wobei $t \in \mathbb{R}$.
\\
\\
Für welche Werte von $t$ ist der Rang von $A$ gleich $3$?
\\
\\
\textbf{Lösung:}
\begin{mdframed}
\renewcommand{\labelenumi}{\theenumi.}
\underline{\textbf{Vorgehensweise:}}
\begin{enumerate}
\item Gebe eine Bedingung an, sodass eine quadratische Matrix den Rang $3$ besitzt.
\item Bestimme die Werte von $t$.
\end{enumerate}
\end{mdframed}
\underline{1. Gebe eine Bedingung an, sodass eine quadratische Matrix den Rang $3$ besitzt}\\
Für eine quadratische Matrix kennen wir den Zusammenhang:
\begin{align*}
\text{rg}(A) = 3 
\Leftrightarrow
A \ \text{ist regulär}
\Leftrightarrow
\det(A) \neq 0
\end{align*}
Wir müssen also nur die Determinate auf Nullstellen überprüfen.\\
\\
\underline{2. Bestimme die Werte von $t$}\\
Zunächst berechnen wir die Determinate von $A$.
Wir erhalten 
\begin{equation*}
\begin{split}
\det(A)
&= 
\left| 
\begin{pmatrix}
1 & t & 1 \\
2 & t & 0 \\
-1 & t & 2t
\end{pmatrix}
\right|\\
&=
1 \cdot t \cdot 2t +t \cdot 0 \cdot (-1) + 1 \cdot 2 \cdot t 
- (-1) \cdot t \cdot 1 - t \cdot 0  \cdot 1 - 2t \cdot 2 \cdot t\\
&=
2 t^2 + 2t + t - 4 t^2 
= -2 t^2 +3 t = -t ( 2 t - 3)
\end{split}
\end{equation*}
durch die Regel von Sarrus (Sauron).
Weiter gilt
\begin{equation*}
\det(A) = 0 
\Leftrightarrow
-t ( 2 t - 3) = 0 
\Leftrightarrow
t_1 = 0, \quad t_2 = \frac{3}{2}
\end{equation*}
für die Nullstellen der Determinante.
Wir wissen nun, dass 
\begin{align*}
\det(A) \neq 0 
\Leftrightarrow
\text{rg}(A) = 3 
\Leftrightarrow
t \in \mathbb{R} \setminus \left\lbrace 0, \frac{3}{2} \right\rbrace
\end{align*}
gilt.\\
\\
Die Matrix $A$ besitzt den Rang $3$ für alle $t \in \mathbb{R} \setminus \left\lbrace 0, \frac{3}{2} \right\rbrace$.
\newpage
\subsubsection*{Anwendungsbeispiel D}
Die folgende Tabelle beschreibt die jährlichen Payoffs zweier Wertpapiere zu identischem Ausgangspreis,
abhängig von der jeweiligen konjunkturellen Lage:
\begin{table}[H]
\centering
\begin{tabular}{lcc}
\hline 
Konjunktur & Aktie 1 & Aktie 2 \\ 
\hline 
Expansion & 1.5 & 3 \\ 
wirtschaftliche Stabilität & 1.5 & 2 \\ 
Rezession & 1.5 & 0.5 \\ 
\hline 
\end{tabular} 
\end{table}
Ermitteln Sie, ob das folgende Auszahlungsschema für den Investor möglich ist,
wenn er nur in die Aktien 1 und 2 investiert:
\begin{table}[H]
\centering
\begin{tabular}{lc}
\hline 
Konjunktur & Payoff des Investors \\ 
\hline 
Expansion & 1'500 \\ 
wirtschaftliche Stabilität & 2'000 \\ 
Rezession & 1'000 \\ 
\hline 
\end{tabular} 
\end{table}
\ \\
\textbf{Lösung:}
\begin{mdframed}
\underline{\textbf{Vorgehensweise:}}
\begin{enumerate}
\renewcommand{\labelenumi}{\theenumi.}
\item Gebe eine Bedingung an, sodass der Payoff des Investors realisiert werden kann.
\item Überprüfe diese Bedingung.
\end{enumerate}
\end{mdframed}
\underline{1. Gebe eine Bedingung an, sodass der Payoff des Investors realisiert werden kann}\\
Wir können den Payoff des Investors realisieren, falls wir $\lambda_1, \lambda_2$ finden, so dass
\begin{align*}
\begin{pmatrix}
1500\\
2000\\
1000\\
\end{pmatrix}
= \lambda_1 
\begin{pmatrix}
1.5\\
1.5\\
1.5
\end{pmatrix}
+ \lambda_2
\begin{pmatrix}
3\\
2\\ 
0.5
\end{pmatrix}
\end{align*}
gilt.
Die Payoff-Vektoren der Wertpapiere sind linear unabhängig.
Damit finden wir unsere $\lambda_1, \lambda_2$ nur, falls
\begin{align*}
\left\lbrace 
\begin{pmatrix}
1500\\
2000\\
1000\\
\end{pmatrix},
\begin{pmatrix}
1.5\\
1.5\\
1.5
\end{pmatrix},
\begin{pmatrix}
3\\
2\\ 
0.5
\end{pmatrix}
\right\rbrace
\end{align*}
linear abhängig ist.
Um uns die Rechnung zu vereinfachen, können wir die Vektoren mit geschickten Vielfachen austauschen.
Durch
\begin{align*}
\left\lbrace 
\begin{pmatrix}
3\\
4\\
2\\
\end{pmatrix},
\begin{pmatrix}
1\\
1\\
1
\end{pmatrix},
\begin{pmatrix}
6\\
4\\ 
1
\end{pmatrix}
\right\rbrace
\end{align*}
erhalten wir ein einfacheres System.
Wir wollen noch exemplarisch die Rechnung für den ersten Vektor durchführen:
\begin{align*}
\frac{1}{500}
\begin{pmatrix}
1500\\
2000\\
1000\\
\end{pmatrix}
=
\begin{pmatrix}
3\\
4\\
2\\
\end{pmatrix}
\end{align*}
Wenn die lineare Abhängigkeit nachgewiesen werden kann, wissen wir, dass
\begin{align*}
A
= 
\begin{pmatrix}
3 & 1 & 6 \\
4 & 1 & 4 \\
2 & 1 & 1
\end{pmatrix}
\end{align*}
singulär ist.\\
\\
\underline{2. Überprüfe diese Bedingung}\\
Die Matrix $A$ ist singulär genau dann, wenn
\begin{align*}
\det(A) = 0
\end{align*}
gilt.
Hier berechnen wir die Determinante
\begin{equation*}
\begin{split}
\det(A)
=
\left|
\begin{pmatrix}
3 & 1 & 6 \\
4 & 1 & 4 \\
2 & 1 & 1
\end{pmatrix}
\right|
=
3 + 8 + 24 - 12 - 12 -4
= 
7
\end{split}
\end{equation*}
mit der Regel von Sarrus.
Damit ist die Matrix $A$ regulär und das System
\begin{align*}
\left\lbrace 
\begin{pmatrix}
1500\\
2000\\
1000\\
\end{pmatrix},
\begin{pmatrix}
1.5\\
1.5\\
1.5
\end{pmatrix},
\begin{pmatrix}
3\\
2\\ 
0.5
\end{pmatrix}
\right\rbrace
\end{align*}
linear unabhängig.
Eine weitere Möglichkeit ist es, elementare Zeilenumformungen auf die Matrix $A$ anzuwenden.
Die Matrix $A$ ist regulär, wenn wir eine Zeilenstufenform erreichen.
Wir betrachten nun:
\begin{align*}
\begin{gmatrix}[p]
3 & 1 & 6 \\
4 & 1 & 4 \\
2 & 1 & 1
\rowops
\add[-1]{0}{1}
\add[-1]{0}{2}
\end{gmatrix}
\leadsto
&\begin{gmatrix}[p]
3 & 1 & 6 \\
1 & 0 & -2 \\
-1 & 0 & -5
\rowops
\add[-3]{1}{0}
\add[\cdot 1]{1}{2}
\end{gmatrix}\\
\leadsto
&\begin{gmatrix}[p]
0 & 1 & 12 \\
1 & 0 & -2 \\
0 & 0 & -7
\rowops
\mult{2}{\cdot ( -\frac{1}{7} )}
\end{gmatrix}\\
\leadsto
&\begin{gmatrix}[p]
0 & 1 & 12 \\
1 & 0 & -2 \\
0 & 0 & 1
\rowops
\add[\cdot 2]{2}{1}
\add[\cdot (12)]{2}{0}
\end{gmatrix}\\
\leadsto
&\begin{gmatrix}[p]
0 & 1 & 0 \\
1 & 0 & 0 \\
0 & 0 & 1
\rowops
\swap{0}{1}
\end{gmatrix}\\
\leadsto
&\begin{gmatrix}[p]
1 & 0 & 0 \\
0 & 1 & 0 \\
0 & 0 & 1
\end{gmatrix}\\
\end{align*}
Somit sehen wir auch hieran, dass die Matrix nicht singulär ist.
\\
\\
Das Auszahlungsschema des Investors lässt sich also nicht verwirklichen.
