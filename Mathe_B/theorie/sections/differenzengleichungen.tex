\section{Differenzengleichungen}
\subsection{Allgemeine Definitionen}
\begin{mybox}{Definition}
\index{Differenzengleichung}
Ein \textit{lineare Differenzengleichung} erster Ordnung
ist durch
\begin{align*}
a_1 y_{n+1} + a_0 y_n = b 
\end{align*}
mit $a_0,a_1,b \in \mathbb{R}$ gegeben.\\
\\
Wir bezeichnen eine Folge $y_n$ als \textit{Lösung} der Differenzengleichung,
falls diese die Gleichung für alle $n \in \mathbb{N}$ erfüllt.\\
Mit $y_0 = c$ bezeichnen wir die \textit{Anfangsbedingung} für die Differenzengleichung,
d.h. den ersten Wert den die Lösungsfolge annimmt.
\end{mybox}
\ \\
Mithilfe der Differenzengleichung können wir zu einer gegebenen Anfangsbedingung $y_0 = c$ beliebig viele Folgenglieder berechnen. Wir haben durch
\begin{align*}
a_1 y_{n+1} + a_0 y_n = b \
\Leftrightarrow \
a_1 y_{n+1} = -a_0 y_n +b
 \ \Leftrightarrow \
y_{n+1} = \frac{1}{a_{1}} \cdot ( -a_0 y_n +b ) 
\end{align*}
eine rekursive Vorschrift, d.h. wir können das $n+1$-te Folgenglied durch das vorherige berechnen.
Für $y_{100}$ müssen aber schon hundert Rechenoperationen durchgeführt werden. Dies ist sehr zeitaufwendig.
Aus diesem Grund benötigen wir eine explizite Darstellung für jedes Folgenglied.\\
Der erste Schritt hierzu ist die Normalform einer Differenzengleichung.\\
\begin{mybox}{Normalform}
\index{Differenzengleichung!Normalform}
Zu einer Differenzengleichung von der Gestalt
\begin{align*}
y_{n+1} = A y_n +B, \quad A,B \in \mathbb{R} 
\end{align*}
sagen wir, dass sie in \textit{Normalform} vorliegt.
\end{mybox}
\ \\
Jede lineare Differenzengleichung erster Ordnung lässt sich in die Normalform bringen.
Wir betrachten
\begin{align*}
a_1 y_{n+1} + a_0 y_n = b
\ \Leftrightarrow \
a_1 y_{n+1} = -a_0 y_n + b
 \ \Leftrightarrow \
y_{n+1} = -\frac{a_0}{a_1} y_n + \frac{b}{a_1} 
\end{align*}
und erhalten mit $A = -\frac{a_0}{a_1}$ und $B =\frac{b}{a_1}$ die gesuchte Normalform.
Beachte, wir dürfen die Umformungen nur durchführen, wenn $a_1 \neq 0$ ist.
Dies ist aber kein unlösbares Problem.
Angenommen es gilt $a_1 = 0$, dann folgt
\begin{align*}
a_0 y_n = b
\end{align*}
für die Differenzengleichung.\\
Wenn eine Lösung existiert, muss diese zwingend eine konstante Folge sein.
Es existiert keine Folge, welche die Differenzengleichung für $a_0 = 0 $ und $b \neq 0$ erfüllt.\\
Um zurück auf das Wesentliche zu kommen:
Wir interessieren uns für Differenzengleichungen, bei denen $a_1 \neq 0$ erfüllt ist.
Nun betrachten wir, warum die Normalform so nützlich ist.
\begin{mybox}{Allgemeine Lösung}
\index{Differenzengleichung!allgemeine Lösung}
Gegeben sei
\begin{align*}
y_{n+1} = A y_n + B
\end{align*}
in Normalform mit Anfangsbedingung $y_0$.\\
Dann ist die allgemeine Lösung der Differenzengleichung durch
\begin{align*}
y_k = 
\begin{cases}
A^k(y_0 - y^\star) 	+y^\star&, \ \text{falls} \ A \neq 1\\
y_0 + k \cdot B      &, \ \text{falls} \ A = 1 
\end{cases}
\quad
\ \text{mit} \
y^\star = \frac{B}{1-A}
\end{align*}
gegeben.
\end{mybox}
\subsection{Lösungsverhalten}
Wir wissen bereits, dass wir jede Differenzengleichung in eine Normalform überführen können.\\
Sei also zunächst $A \neq 1$.
Die allgemeine Lösung lautet
\begin{align*}
y_k = A^k(y_0 - y^\star) 	+y^\star.
\end{align*}
Hierbei sind $y_0$ und $y^\star$ konstant, weswegen nur $A^k$ für das Lösungsverhalten interessant ist.\\
\index{Differenzengleichung!Lösungsverhalten}
\begin{mybox}{Lösungsverhalten für $A \neq 1$}
\begin{align*}
A > 0 &\rightarrow \ \text{Lösung monoton}\\
A < 0 &\rightarrow \ \text{Lösung oszillierend}\\
|A| < 1  &\rightarrow \ \text{Lösung konvergent}\\
|A| > 1  &\rightarrow \ \text{Lösung divergent}
\end{align*}
\end{mybox}


Wir betrachten zunächst die zwei Fälle $ A > 0$ und $A < 0$.\\
Wenn $A > 0 $ ist, gilt
\begin{itemize}
\item
$A^n$ ist monoton fallend, wenn $A< 1$.
\begin{figure}[H]
\centering
\begin{tikzpicture} 
\begin{axis}[
axis x line=center,
axis y line=center,
ticks = none,
xtick={0,1,2},
xmin = 0,
xmax = 11,
ymin = 0] 
\addplot[only marks,mark=*,mark options={scale=1, fill=black},text mark as node=true] coordinates { 
(1,1)   
(2,0.5)
(3,0.33333)
(4,0.25)
(5,1/5)
(6,1/6)
(7,1/7)
(8,1/8)
(9,1/9)
(10,0.1 )};
\end{axis} 
\end{tikzpicture}
\caption*{monoton fallend}
\end{figure}

\newpage
\item
$A^n$ ist monoton wachsend, wenn $A> 1$.
\begin{figure}[H]
\centering
\begin{tikzpicture} 
\begin{axis}[
axis x line=center,
axis y line=center,
xtick={0,1,2},
ticks = none,
xmin = 0,
xmax = 11,
ymin = 0] 
\addplot[only marks,mark=*,mark options={scale=1, fill=black},text mark as node=true] coordinates { 
(1,1)   
(2,2)
(3,3)
(4,4)
(5,5)
(6,6)
(7,7)
(8,8)};
\end{axis} 
\end{tikzpicture}
\caption*{monoton wachsend}
\end{figure}

\end{itemize}
Insgesamt können wir sagen, dass $A^n$ \textit{monoton} ist für $A > 0$.
Anders sieht es bei $A < 0$ aus. 
Wir nutzen
\begin{align*}
A = -|A| 
\Rightarrow
A^k = (-1)^k |A|^k
\end{align*}
und sehen, dass ein Vorzeichenwechsel bei jedem Folgenglied vorliegt.
Deswegen sagen wir in diesem Fall: Die Lösung ist \textit{oszillierend}.

\begin{figure}[H]
\minipage{0.5\textwidth}
\centering
\begin{tikzpicture} 
\begin{axis}[
axis x line=center,
axis y line=center,
xtick={0,1,2},
ytick={-1,1},
xmin = 0,
xmax = 11] 
\addplot[only marks,mark=*,mark options={scale=1, fill=black},text mark as node=true] coordinates { 
(1,-1)   
(2,0.5)
(3,-0.3)
(4,0.25)
(5,-1/5)
(6,1/6)
(7,-1/7)
(8,1/8)
(9,-1/9)
(10,0.1 )};
\end{axis} 
\end{tikzpicture}
\caption*{oszillierend $|A| < 1$}
\endminipage\hfill
\minipage{0.5\textwidth}
\centering
\begin{tikzpicture} 
\begin{axis}[
axis x line=center,
axis y line=center,
xmin = 0,
xmax = 11] 
\addplot[only marks,mark=*,mark options={scale=1, fill=black},text mark as node=true] coordinates { 
(1,-1)   
(2,2)
(3,-3)
(4,4)
(5,-5)
(6,6)
(7,-7)
(8,8)
(9,-9)
(10,10 )};
\end{axis} 
\end{tikzpicture}
\caption*{oszillierend $|A| > 1$}
\endminipage
\end{figure}
Nun unterscheiden wir $|A| < 1$ und $|A| > 1$.
Hierfür können wir direkt
\begin{align*}
\lim \limits_{n\to \infty} |A|^n
= 
\begin{cases}
&0 , \quad \text{ falls} \ |A|< 1\\
&\infty, \quad \text{falls} \ | A | > 1
\end{cases}
\end{align*}
ausnutzen.
Wir sagen eine Lösung ist \textit{dämpfend/konvergent} wenn $|A|<1 $ gilt.  Analog sagen wir \textit{explosiv/divergent} falls $|A| > 1$ gilt.
Es fehlt also noch die Betrachtung von $|A| = 1$.
Hier haben wir die allgemeine Lösung
\begin{align*}
y_n = y_0 + k \cdot B.
\end{align*}
Diese ist monoton und divergent für $B \neq 0$.
Für $B= 0$ ist die Lösung konstant.
\\
\\
Wir interessieren uns aber hauptsächlich für den Fall, dass $A \neq 1$ ist.



\newpage
\subsubsection*{Anwendungsbeispiel A}
Die Folge $\lbrace y_k \rbrace_{k \in \mathbb{N}_0}$ definiert durch
\begin{align*}
y_k =  4 - \left( \frac{1}{a}\right)^k
\end{align*}
mit $a \neq 0$ löst das Anfangswertproblem
\begin{align*}
4 y_{k+1} -y_k = 12 \ \text{und} \ y_0 = 3
\end{align*}
für
\renewcommand{\labelenumi}{(\alph{enumi})}
\begin{enumerate}
\item 
$a= 1$.
\item
$a= 2$.
\item
$a= 4$.
\item
Keine der vorangehenden Antworten ist richtig.
\end{enumerate}
\ \\
\textbf{Lösung:}
\begin{mdframed}
\underline{\textbf{Vorgehensweise:}}
\renewcommand{\labelenumi}{\theenumi.}
\begin{enumerate}
\item Gebe die Normalform der Differenzengleichung an.
\item Finde die allgemeine Lösung der Differenzengleichung und löse die Aufgabe.
\end{enumerate}
\end{mdframed}

\underline{1. Gebe die Normalform der Differenzengleichung an}\\
In unserem Fall erhalten wir diese durch
\begin{align*}
4 y_{k+1} -y_k = 12 \ 
\Leftrightarrow \
4 y_{k+1} = y_k + 12
\ \Leftrightarrow \
y_{k+1} = \frac{1}{4} y_k + 3
\end{align*}
mit $A = \frac{1}{4} $ und $B = 3$.\\
\\
\underline{2. Finde die allgemeine Lösung der Differenzengleichung und löse die Aufgabe}\\
Die allgemeine Lösung einer Differenzengleichung erster Ordnung ist durch
\begin{align*}
y_k = A^k (y_0 - y^\star)  + y^\star
\end{align*}
mit
\begin{align*}
y^\star 
= \frac{B}{1 -A }, \quad A \neq 1 
\end{align*}
gegeben.
Es gilt:
\begin{align*}
y^\star 
= \frac{B}{1 -A }
= \frac{3}{1 - \frac{1}{4} }
= \frac{3}{\frac{3}{4}}
= \frac{3}{1} \cdot \frac{4}{3} = 4.
\end{align*}
Die allgemeine Lösung ist dann
\begin{align*}
y_k = A^k (y_0 - y^\star)  + y^\star
=
\left( \frac{1}{4} \right)^k(y_0 -4) +4.
\end{align*}
Mit $y_0 = 3$ erhalten wir durch
\begin{align*}
y_k = \left( \frac{1}{4} \right)^k ( 3 - 4 ) +4
=  4 - \left( \frac{1}{4} \right)^k
\end{align*}
das passende $a = 4$.
Dies können wir durch Vergleichen mit der Aufgabenstellung erkennen.\\
\\
Damit ist Antwort (c) korrekt.

\newpage
\subsubsection*{Anwendungsbeispiel B}
Die allgemeine Lösung der Differenzengleichung
\begin{align*}
5 y_{k+1} + 6 y_k = 1 , \quad k = 0,1,2,...
\end{align*}
ist
\renewcommand{\labelenumi}{(\alph{enumi})}
\begin{enumerate}
\item 
monoton und konvergent.
\item
monoton und divergent.
\item
oszillierend und konvergent.
\item
oszillierend und divergent.
\end{enumerate}
\ \\
\textbf{Lösung:}
\begin{mdframed}
\underline{\textbf{Vorgehensweise:}}
\renewcommand{\labelenumi}{\theenumi.}
\begin{enumerate}
\item Gebe die Normalform und allgemeine Lösung der Differenzengleichung an.
\item 
Bestimme die relevanten Eigenschaften.
\end{enumerate}
\end{mdframed}

\underline{1. Gebe die Normalform und allgemeine Lösung der Differenzengleichung an}\\
Wie in der vorigen Aufgabe erhalten wir durch
\begin{align*}
5 y_{k+1} + 6 y_k = 1
\ \Leftrightarrow \
5 y_{k+1} = - 6 y_k + 1
\ \Leftrightarrow \
y_{k+1} = - \frac{6}{5} y_k  + \frac{1}{5}
\end{align*}
die Normalform mit $A =  - \frac{6}{5} $ und $B = \frac{1}{5}$.
Mit 
\begin{align*}
y^\star = 
\frac{B}{1-A} 
= 
\frac{\frac{1}{5}}{1 - \left( -\frac{6}{5} \right)} 
=
\frac{\frac{1}{5}}{\frac{11}{5}}
= \frac{1}{5} \cdot \frac{5}{11} = \frac{1}{11}
\end{align*}
erhalten wir die allgemeine Lösung
\begin{align*}
y_k = A^k (y_0 - y^\star)  + y^\star
= \left( - \frac{6}{5} \right)^k \left ( y_0 - \frac{1}{11}\right) + \frac{1}{11}.
\end{align*}
\newpage
\underline{2. Bestimme die relevanten Eigenschaften}\\
Wir wissen, dass $A\neq 1$ ist.
Damit können die Fälle
\begin{align*}
A > 0 &\rightarrow \ \text{Lösung monoton}\\
A < 0 &\rightarrow \ \text{Lösung oszillierend}\\
|A| < 1  &\rightarrow \ \text{Lösung konvergent}\\
|A| > 1  &\rightarrow \ \text{Lösung divergent}
\end{align*}
eintreten.
Wegen $A = -\frac{6}{5} < 0$ wissen wir, dass die allgemeine Lösung oszillierend ist.
Zudem gilt
$
|A| = \frac{6}{5} > 1,
$
womit die allgemeine Lösung divergent ist.
Durch
\begin{align*}
\lim \limits_{k \to \infty} |A^k | 
=\lim \limits_{k \to \infty} \left| \left( - \frac{6}{5} \right)^k \right| = \infty
\end{align*}
können wir die Divergenz veranschaulichen und an
\begin{align*}
A^k = \left( - \frac{6}{5} \right)^k
= (-1)^k \cdot\left(  \frac{6}{5} \right)^k
\end{align*}
sehen wir, dass die Lösung oszilliert.
\\
\\
Somit ist Antwort (d) korrekt.

\newpage
\subsubsection*{Anwendungsbeispiel C}
Die allgemeine Lösung der Differenzengleichung
\begin{align*}
m \ y_{k+1} + y_k  \ = \ m^2, \quad k=0,1,2,...,
\end{align*}
wobei $m \in \mathbb{R} \setminus \lbrace 0 \rbrace$, ist monoton und konvergent genau dann, wenn
\renewcommand{\labelenumi}{(\alph{enumi})}
\begin{enumerate}
\item 
$m \in [-1,0)$.
\item
$m \in (0,1]$.
\item
$m < -1$.
\item
$m> 1$.
\end{enumerate}
\ \\
\textbf{Lösung:}
\begin{mdframed}
\underline{\textbf{Vorgehensweise:}}
\renewcommand{\labelenumi}{\theenumi.}
\begin{enumerate}
\item Gebe die Normalform und allgemeine Lösung der Differenzengleichung an.
\item 
Argumentiere mithilfe den bekannten Eigenschaften.
\end{enumerate}
\end{mdframed}

\underline{1. Gebe die Normalform an }\\
Wie in den vorigen Aufgaben erhalten wir mit
\begin{align*}
m  y_{k+1} + y_k   =  m^2
 \ \Leftrightarrow \
m y_{k+1} = - y_k + m^2
\ \Leftrightarrow \
y_{k+1} = - \frac{1}{m} y_k + m 
\end{align*}
die Normalform mit $A = - \frac{1}{m}$ und $B = m$.
\\
\\
\underline{2. Argumentiere mithilfe den bekannten Eigenschaften}\\
Wir wissen, dass die allgemeine Lösung genau dann monoton und konvergent ist, wenn
\begin{align*}
A > 0  \ \text{und} \ |A| < 1
\end{align*}
erfüllt ist.
Wegen $m \in \mathbb{R} \setminus \lbrace 0 \rbrace$ ist $B=0$ unmöglich.
Daraus folgt
\begin{align*}
0 < - \frac{1}{m} \ \textrm{und} \ \left| -\frac{1}{m} \right| < 1. 
\end{align*}
Wegen 
\begin{align*}
-\frac{1}{m} > 0 
\end{align*}
wissen wir direkt, dass $m < 0 $ ist.
Weiter gilt:
\begin{align*}
- \frac{1}{m} < 1 
\ \Leftrightarrow \
\frac{1}{m} >-1
 \ \overset{ m  < 0}{\Leftrightarrow }\
 1 < -m 
  \ \Leftrightarrow \
-1 > m
\end{align*}
\\

Somit ist Antwort (c) korrekt.
