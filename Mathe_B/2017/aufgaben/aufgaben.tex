%\newcommand{\ein}[2]{(#1) (#2 Punkte)}


\begin{Large}
\textbf{Teil I: Offene Aufgaben (50 Punkte)}
\end{Large}
\\
\\
\\
\textbf{Allgemeine Anweisungen für offene Fragen:}
\\
\renewcommand{\labelenumi}{(\roman{enumi})}
\begin{enumerate}
\item
Ihre Antworten müssen alle Rechenschritte enthalten,
diese müssen klar ersichtlich sein.
Verwendung korrekter mathematischer Notation wird erwartet
und fliesst in die Bewertung ein.

\item
Ihre Antworten zu den jeweiligen Teilaufgaben müssen in den dafür vorgesehenen Platz geschrie-
ben werden. Sollte dieser Platz nicht ausreichen, setzen Sie Ihre Antwort auf der Rückseite oder
dem separat zur Verfügung gestellten Papier fort. Verweisen Sie in solchen Fällen ausdrücklich
auf Ihre Fortsetzung. Bitte schreiben Sie zudem Ihren Vor- und Nachnamen auf jeden separaten
Lösungsbogen.

\item
Es werden nur Antworten im dafür vorgesehenen Platz bewertet. Antworten auf der Rückseite
oder separatem Papier werden nur bei einem vorhandenen und klaren Verweis darauf bewertet.

\item
Die Teilaufgaben werden mit den jeweils oben auf der Seite angegebenen Punkten bewertet.

\item
Ihre endgültige Lösung jeder Teilaufgabe darf nur eine einzige Version enthalten.

\item
Zwischenrechnungen und Notizen müssen auf einem getrennten Blatt gemacht werden. Diese
Blätter müssen, deutlich als Entwurf gekennzeichnet, ebenfalls abgegeben werden.
\end{enumerate}

\newpage
\section*{\hfil Aufgaben \hfil}
\vspace{1cm}
\section*{Aufgabe 1 (25 Punkte)}
\vspace{0.4cm}
\subsection*{\aufgabe{a}{7}}
Sei $ f \ : \ \mathbb{R} \times \mathbb{R} \to \mathbb{R} $ eine Funktion zweier Variablen definiert durch
\begin{align*}
f(x,y) = (x+y+a) e^x - e^y, \quad \textrm{wobei} \ a \in \mathbb{R}.
\end{align*}
Untersuchen Sie die Funktion $ f $ auf stationäre Punkte und bestimmen Sie gegebenenfalls, ob ein Maximum, ein Minimum oder ein ein Sattelpunkt vorliegt.
\\
\\
\subsection*{\aufgabe{b}{7}}
Die Funktion
\begin{align*}
f(x,y) = x^2 + y^2
\end{align*}
ist unter der Nebenbedingung
\begin{align*}
\varphi(x,y)
=
a x^2 +bxy+5y^2-16 = 0
\end{align*}
zu optimieren.\\
Bestimmen Sie die Parameter $ a \in \mathbb{R} $ und $ b \in \mathbb{R} $ so, dass in $ (1,1) $ eine mögliche Extremstelle sein könnte.\\
\\
\textbf{Bemerkung:} \\
Eine Abklärung, ob es sich um eine Extremstelle handelt und von welcher Art die Extremstelle ist (Maximum oder Minimum) wird nicht verlangt.
\\
\\
\subsection*{\aufgabe{c}{5}}
Berechnen Sie 
\begin{align*}
\int
\limits_0^{\sqrt{0.5 \ \pi}} x \ \sin(x^2) \ \left(\cos(x^2)\right)^3 dx.
\end{align*}
\ \\
\subsection*{\aufgabe{d}{6}}
Berechnen Sie 
\begin{align*}
\int_0^e |\ln(x)| dx.
\end{align*}
\ \\
\textbf{Bemerkung:} Sie dürfen das in Mathematik A bewiesene Resultat, dass gilt:
\begin{align*}
\lim \limits_{x \searrow 0} x \ \ln(x) = 0,
\end{align*}
voraussetzen.

\newpage
\section*{Aufgabe 2 (25 Punkte)}
\vspace{0.4cm}
\subsection*{\aufgabe{a}{4}}
Die quadratischen $ n \times n $ Matrizen $ A $ und $ B $ seien regulär, $ A $ sei ausserdem symmetrisch.\\
\\
Beweisen Sie:
\begin{align*}
B^\top (AB)^\top (B^{-1} A^{-1})^\top B (A B)^{-1}
=
(A^{-1} B )^\top.
\end{align*}
\\
\\
\subsection*{\aufgabe{b}{4}}
Gegeben ist die Funktion
\begin{align*}
f(x,y) = a \ln(x-2) + x \ y^2 + 8 \ y, \quad \textrm{wobei} \ a \in \mathbb{R}.
\end{align*}
Berechnen Sie den Gradienten von $ f $ an der Stelle $ (x_0,y_0) = (8,2) $.\\
\\
Wie muss der Parameter $ a \in \mathbb{R} $ gewählt werden, damit die Funktion $ f $ im Punkt $ (x_0,y_0) = (8,2) $ in der Richtung
$ \textbf{b} = \begin{pmatrix}
3 \\
4
\end{pmatrix} $ am stärksten zunimmt?\\
\\
\subsection*{\aufgabe{c}{3}}
Gegeben sind die Vektoren
\begin{align*}
\textbf{b}_1 =
\begin{pmatrix}
1 \\
t\\ 
0
\end{pmatrix},
\
\textbf{b}_2 =
\begin{pmatrix}
2t \\
4\\ 
t
\end{pmatrix},
\
\textbf{b}_3 =
\begin{pmatrix}
8 \\
t\\ 
t^2
\end{pmatrix}.
\end{align*}
Für welche Werte von $ t \in \mathbb{R} $ ist das Vektorsystem $ \{ \textbf{b}_1, \textbf{b}_2, \textbf{b}_3 \} $ \textit{keine} Basis des dreidimensionalen Raumes $ \mathbb{R}^3 $?
\\
\\
\subsection*{\aufgabe{d}{6}}
Gegeben ist die Matrix
\begin{align*}
M = 
\begin{pmatrix}
0  & 2a\\
-3a & 5a
\end{pmatrix}
,
\quad
\textrm{wobei } a \neq 0.
\end{align*}
Berechnen Sie die Eigenwerte und Eigenvektoren der Matrix $ M $.\\
\\
Berechnen Sie den Vektor $ M^n \textbf{x} $, wobei $ \textbf{x} = \begin{pmatrix}
2 \\ 2
\end{pmatrix}. $
\\
\\
\subsection*{\aufgabe{e}{8}}
Gegeben ist das Gleichungssystem
\begin{equation*}
\begin{split}
x_1 \ + \  x_2 \ + \  x_3 \ + \  x_4 \ + \ x_5 \ &= \ \ 0 \\
2x_1 \ + \ 3 x_2 \ + \ 4 x_3 \ + \ 5 x_4 \ - \ x_5 \ &= \ \ 0 \\
 x_1 \ + \ 2 x_2 \ + \ 4 x_3 \ - \  x_4 \ + \ 2x_5 \ &= \ 0
\end{split}
.
\end{equation*}
Berechnen Sie die allgemeine Lösung dieses Gleichungssystems mit dem Gauß-Verfahren.\\
\\
Für den Unterraum
\begin{align*}
W 
=
\{
\textbf{x} \in \mathbb{R}^5 | x_1 + x_2 + x_3 + x_4 + x_5 = 0 \wedge
2 x_1 + 3 x_2 +4 x_3 + 5 x_4 - x_5 = 0
\wedge 
x_1 + 2x_2 +4 x_3 - x_4 + 2x_5 = 0
\}
\end{align*}
ist eine Basis anzugeben.
\newpage


\fancyhead[C]{\normalsize\textbf{$\qquad$ Teil II: Multiple-Choice}}
\begin{Large}
\textbf{Teil II: Multiple-Choice-Fragen (50 Punkte)}
\end{Large}
\\
\\
\\
\textbf{Allgemeine Anweisungen für Multiple-Choice-Fragen:}
\\
\renewcommand{\labelenumi}{(\roman{enumi})}
\begin{enumerate}
\item
Die Antworten auf die Multiple-Choice-Fragen müssen im dafür vorgesehenen Antwortbogen ein-
getragen werden. Es werden ausschliesslich Antworten auf diesem Antwortbogen bewertet. Der
Platz unter den Fragen ist nur für Notizen vorgesehen und wird nicht korrigiert.

\item
Jede Frage hat nur eine richtige Antwort. Es muss also auch jeweils nur eine Antwort angekreuzt
werden.

\item
Falls mehrere Antworten angekreuzt sind, wird die Antwort mit 0 Punkten bewertet, auch wenn
die korrekte Antwort unter den angekreuzten ist.

\item
Bitte lesen Sie die Fragen sorgfältig.

\end{enumerate}
\newpage
\section*{Aufgabe 3 (25 Punkte)}
\vspace{0.4cm}
\subsection*{\frage{1}{3}}
Die Funktion $ f(x,y) = y  $ hat unter der Nebenbedingung $ \varphi(x,y) = \frac{x^2}{25} + \frac{y^2}{9} = 1 $
ihr Minimum in
\renewcommand{\labelenumi}{(\alph{enumi})}
\begin{enumerate}
\item $ P = (-5,0) $.
\item $ P = (0,3) $.
\item $ P = (0,0) $.
\item $ P = (0,-3) $.
\end{enumerate}
\ \\
\subsection*{\frage{2}{4}}
Gegeben ist die Funktion
\begin{align*}
f(x) 
=
\begin{cases}
ax + \frac{1}{8}, &\quad \textrm{für } 0 \leq x \leq 4\\
\quad 0					&\quad  \textrm{sonst}			
\end{cases}
\end{align*}
\renewcommand{\labelenumi}{(\alph{enumi})}
\begin{enumerate}
\item $f $ ist für alle $ a \in \mathbb{R} $ eine Dichtefunktion.
\item $f $ ist nur für $ a = \frac{1}{16} $ eine Dichtefunktion.
\item
$f $ ist nur für $ a = -\frac{1}{16} $ eine Dichtefunktion.
\item
$f $ ist für kein $ a \in \mathbb{R} $ eine Dichtefunktion.
\end{enumerate}
\ \\
\subsection*{\frage{3}{2}}
Sei $ f : [a,b] \to \mathbb{R} $ eine beliebige, auf dem Intervall $ [a,b] $ definierte Funktion.\\
\\
Welche der folgenden Aussagen ist \textit{richtig}?
\renewcommand{\labelenumi}{(\alph{enumi})}
\begin{enumerate}
\item 
Wenn das bestimmte Integral von $ f $ über $ [a,b] $ existiert, dann ist $ f $ stetig auf $ [a,b] $.
\item 
Wenn $ f $ nicht stetig ist auf $ [a,b] $, dann existiert das bestimmte Integral von $ f $ über $ [a,b] $ nicht.
\item 
Wenn $ f $ differenzierbar ist auf $ [a,b] $, dann existiert das bestimmte Integral von $ f $ über $ [a,b] $.
\item
Das bestimmte Integral von $ f $ über $ [a,b]$ existiert immer.
\end{enumerate}
\ \\
\subsection*{\frage{4}{2}}
$A$ und $B$ seien quadratische Matrizen mit
$\det(A) = 5$ und $\det(B) = 2$; die Matrix $ C $ ist definiert durch 
$C \ = \ A^{-1}B A $. 
\renewcommand{\labelenumi}{(\alph{enumi})}
\begin{enumerate}
	\item 
	Dann gilt für jedes $ n \in \mathbb{N}: $ $ \det(C^n)  = 1$.
	\item
	Dann gilt für jedes $ n \in \mathbb{N}: $ $ \det(C^n)  = 2^n$.
	\item
	Dann gilt für jedes $ n \in \mathbb{N}: $ $ \det(C^n)  = 2^n \cdot 5^n$.
	\item
	Keine der obigen Aussagen ist korrekt.
\end{enumerate}
\ \\
\subsection*{\frage{5}{4}}
Gegeben sind die Vektoren
\begin{align*}
\textbf{a}
= 
\begin{pmatrix}
1 \\ 2 \\ 3
\end{pmatrix},
\textbf{b}
=
\begin{pmatrix}
1 \\ -1 \\ 1
\end{pmatrix},
\textbf{c}
=
\begin{pmatrix}
0 \\ 0 \\ 2
\end{pmatrix},
\textbf{d}
=
\begin{pmatrix}
2 \\ 1 \\ t
\end{pmatrix}.
\end{align*}
\renewcommand{\labelenumi}{(\alph{enumi})}
\begin{enumerate}
\item 
Es ist nur für $ t = 6 $ möglich, $ \textbf{d} $ als Linearkombination von $ \textbf{a} $, $ \textbf{b} $ und $ \textbf{c} $ zu schreiben.
\item
Es ist nur für $ t = 6 $ und $ t = 0 $ möglich, $ \textbf{d} $ als Linearkombination von $ \textbf{a} $, $ \textbf{b} $ und $ \textbf{c} $ zu schreiben.
\item
Es ist für alle $ t\in \mathbb{R} $ möglich, $ \textbf{d} $ als Linearkombination von $ \textbf{a} $, $ \textbf{b} $ und $ \textbf{c} $ zu schreiben.
\item
Es ist für kein $ t\in \mathbb{R} $ möglich, $ \textbf{d} $ als Linearkombination von $ \textbf{a} $, $ \textbf{b} $ und $ \textbf{c} $ zu schreiben.
\end{enumerate}
\ \\
\subsection*{\frage{6}{2}}
$ A $ ist eine $ 6 \times 5 $ Matrix, das lineare Gleichungssystem $ A \textbf{x} = \textbf{b} $ hat unendlich viele Lösungen und der Lösungsraum hat die Dimension $ 2 $.
Dann gilt:
\renewcommand{\labelenumi}{(\alph{enumi})}
\begin{enumerate}
\item 
$ \text{rg}(A) = \text{rg}(A; \textbf{b}) = 3 $.
\item
$ \text{rg}(A) = \text{rg}(A; \textbf{b}) = 4 $.
\item
$ \text{rg}(A) < \text{rg}(A; \textbf{b}) = 3 $..
\item
Keine der obigen Aussagen ist korrekt.
\end{enumerate}
\ \\
\subsection*{\frage{7}{4}}
Das unbestimmte Integral von
\begin{align*}
\int \ln(x \ e^x ) \ dx, \ (x > 0)
\end{align*}
ist
\renewcommand{\labelenumi}{(\alph{enumi})}
\begin{enumerate}
\item 
$ x \ \ln(x) + x^2 - x + C $.
\item
$ x \ \ln(x) + \frac{x^2}{2} - x + C $.
\item
$ x \ \ln(x) + x^2  + C $.
\item
Keine der obigen Antworten ist korrekt.
\end{enumerate}
\ \\
\subsection*{\frage{8}{4}}
Gegeben ist die Matrix
\begin{align*}
A = 
\begin{pmatrix}
2 & a\\
a & 2
\end{pmatrix},
\ \textrm{wobei } a\neq 0.
\end{align*}
\renewcommand{\labelenumi}{(\alph{enumi})}
\begin{enumerate}
	\item 
	Die Matrix hat für alle $ a \neq 0 $ in $ \mathbb{R} $ zwei verschiedene reelle Eigenwerte.
	\item
	Die Matrix hat für alle $ a \neq 0 $ in $ \mathbb{R} $ genau einen reellen Eigenwert.
	
	\item
	Die Matrix hat für alle $ a \neq 0 $ in $ \mathbb{R} $ keinen reellen Eigenwert..
	\item
	Die Matrix $ A $ hat abhängig von $ a \neq 0 $ keinen, einen oder zwei reelle Eigenwerte.
\end{enumerate}


\newpage
\section*{Aufgabe 4 (25 Punkte)}
\vspace{0.4cm}

\subsection*{\frage{1}{3}}
Das bestimmte Integral
\begin{align*}
\int_0^\pi 2 \ \sin(x) \ \cos(x) \ dx
\end{align*}
hat den Wert
\renewcommand{\labelenumi}{(\alph{enumi})}
\begin{enumerate}
\item 
$0$.
\item
$1$.
\item
$2$.
\item
Keines der obigen Resultate ist korrekt.
\end{enumerate}
\ \\
\subsection*{\frage{2}{3}}
Für welchen Wert von $ t \in \mathbb{R} $ sind die Vektoren $ \textbf{u} = \begin{pmatrix}
t-2 \\ t \\ 3
\end{pmatrix} $ und $ \textbf{v} = \begin{pmatrix}
1 \\ t-1 \\ 9
\end{pmatrix} $ orthogonal?
\renewcommand{\labelenumi}{(\alph{enumi})}
\begin{enumerate}
	\item 
	$t = 5$.
	\item
	$t = 5$ oder $ t = -5 $.
	\item
	$ \textbf{u} $ und $ \textbf{v}  $ sind für kein $ t \in \mathbb{R} $ orthogonal.
	\item
	$ \textbf{u} $ und $ \textbf{v}  $ sind für alle $ t \in \mathbb{R} $ orthogonal.
\end{enumerate}
\ \\
\subsection*{\frage{3}{4}}
Die $4 \times 5$ Matrix
\begin{align*}
A
=
\begin{pmatrix}
1 & 1  & 3 & -1 & -2 \\
3 & -5  & -7 & 13 & -10 \\
-1 & 3  & 5 & -7 & 4 \\
-2 & 10  & 18 & -22 & 10  
\end{pmatrix}
\end{align*}
\renewcommand{\labelenumi}{(\alph{enumi})}
\begin{enumerate}
\item 
hat Rang $2$.
\item
hat Rang $3$.
\item
hat Rang $4$.
\item
hat Rang $5$.
\end{enumerate}
\ \\
\subsection*{\frage{4}{4}}
Gesucht ist eine Matrix $ X $, sodass
\begin{align*}
X
\begin{pmatrix}
1 & 2 \\
0 & 1 
\end{pmatrix}
=
\begin{pmatrix}
4 & 3 \\
2 & 1
\end{pmatrix}
.
\end{align*}
\renewcommand{\labelenumi}{(\alph{enumi})}
\begin{enumerate}
\item 
$X
= 
\begin{pmatrix}
1 & 3  \\
-2 & -1 
\end{pmatrix}$.
\item
$X
= 
\begin{pmatrix}
4 &-5  \\
2 & -3 
\end{pmatrix}$.
\item
$X
= 
\begin{pmatrix}
4 &-3  \\
-2 & 4 
\end{pmatrix}$.
\item
Es gibt keine Matrix $ X $, die die Gleichung erfüllt.
\end{enumerate}

\newpage
\subsection*{\frage{5}{2}}
Die $ n \times n $ Matrix habe die Eigenwerte $ \lambda_1, \lambda_2, \dots, \lambda_n $.
Dann hat die Matrix $ A^2 $
\renewcommand{\labelenumi}{(\alph{enumi})}
\begin{enumerate}
\item 
die gleichen Eigenwerte.
\item
die Eigenwerte $2 \lambda_1,2 \lambda_2, \dots,2 \lambda_n $.
\item
die Eigenwerte $ \lambda_1^2, \lambda_2^2, \dots, \lambda_n^2 $.
\item
Keine der vorangehenden Antworten ist richtig.
\end{enumerate}
\ \\
\subsection*{\frage{6}{3}}
Das Anfangswertproblem
\begin{align*}
&y_{k+1} -(1+a) y_k = a, \ \textrm{wobei } a \neq -1, a \neq 0,\\
&y_0 = 2
\end{align*}
hat die Lösung
\renewcommand{\labelenumi}{(\alph{enumi})}
\begin{enumerate}
\item 
$ y_k = 2 ( 1+a)^k $.
\item
$ y_k = 3 ( 1+a)^k  -1$.
\item
$ y_k = 4 ( 1+a)^k -1 $.
\item
$ y_k = 5 ( 1+a)^k -2$.
\end{enumerate}
\ \\
\subsection*{\frage{7}{2}}
Die allgemeine Lösung der linearen Differenzengleichung
\begin{align*}
3 (y_k - y_{k+1})+ 3 = 2 y_k - 12 
\end{align*}
ist
\renewcommand{\labelenumi}{(\alph{enumi})}
\begin{enumerate}
\item
oszillierend und konvergent.
\item
oszillierend und divergent.	
\item 
monoton und konvergent.
\item
monoton und divergent.
\end{enumerate}
\ \\
\subsection*{\frage{8}{4}}
Die allgemeine Lösung der Differenzengleichung
\begin{align*}
(2 + c) y_{k+1} + (1-c) y_k = 5
\end{align*}
wobei $c \in \mathbb{R} \setminus \lbrace -2 \rbrace$ ist, ist genau dann monoton und divergent, wenn
\renewcommand{\labelenumi}{(\alph{enumi})}
\begin{enumerate}
	\item 
	$ c < -2 $.
	\item
	$c \in (-2,0)$.
	\item
	$ c < -\frac{1}{2} $.
	\item
	Die allgemeine Lösung der obigen Differenzengleichung ist für kein $c \in \mathbb{R} \setminus \lbrace -2 \rbrace$ monoton und divergent.
\end{enumerate}