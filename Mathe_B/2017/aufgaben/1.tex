\vspace{1cm}
\fancyhead[C]{\normalsize\textbf{$\qquad$ Teil I: Offene Aufgaben}}
\renewcommand{\labelenumi}{\theenumi.}
\section*{Aufgabe 1 (25 Punkte)}
\vspace{0.4cm}
\subsection*{\aufgabe{a}{7}}
Sei $ f \ : \ \mathbb{R} \times \mathbb{R} \to \mathbb{R} $ eine Funktion zweier Variablen definiert durch
\begin{align*}
f(x,y) = (x+y+a) e^x - e^y, \quad \textrm{wobei} \ a \in \mathbb{R}.
\end{align*}
Untersuchen Sie die Funktion $ f $ auf stationäre Punkte und bestimmen Sie gegebenenfalls, ob ein Maximum, ein Minimum oder ein ein Sattelpunkt vorliegt.
\\
\\
\textbf{Lösung:}
\begin{mdframed}
\underline{\textbf{Vorgehensweise:}}
\renewcommand{\labelenumi}{\theenumi.}
\begin{enumerate}
\item 
Bestimme die partiellen Ableitungen von $ f $.
\item  
Gebe die Bedingung für einen stationären Punkt an und finde Kandidaten hierfür.
\item Entscheide, welche Art von Extremum vorliegt.
\end{enumerate}
\end{mdframed}
\underline{1. Bestimme die partiellen Ableitungen von $ f $}\\
Die partiellen Ableitungen erhalten wir durch
\begin{align*}
f_x(x,y)
&=
 1 \cdot  e^x + (x+y+a)e^x  = (x+y+a+1)e^x\\
f_y(x,y) 
&= 
e^x -e^y.
\end{align*}
\ \\
\underline{2. Gebe die notwendigen Bedingung für einen stationären Punkt an und finde Kandidaten hierfür}\\
Ein stationärer Punkt ist gegeben, falls
\begin{align*}
f_x(x,y) &= (x+y+a+1) e^x  =0\\
f_y(x,y) &= e^x - e^y =  0
\end{align*}
erfüllt ist.
Wegen $ e^x = e^y  $ muss $ x = y  $ gelten. Für die partielle Ableitung nach $ x $ erhalten wir somit
\begin{align*}
(x+y+ a +1 ) e^x = 0
\ \Leftrightarrow \
x +y +a + 1 = 0
\ \overset{x = y}{\Leftrightarrow} \
2x  + a +1 = 0
\ \Leftrightarrow \
x = -\frac{a+1}{2} = y.
\end{align*}
Dementsprechend liegt ein stationärer Punkt für
\begin{align*}
P := (x_0,y_0):= \left( - \frac{a+1}{2}, - \frac{a+1}{2}\right)
\end{align*}
vor.\\
\\
\underline{3. Entscheide, welche Art von Extremum vorliegt}\\
Für die Entscheidung benötigen wir die zweiten Ableitung:
\begin{align*}
f_{xx}(x,y)
&= 1 \cdot e^x + (x+y+a+1)e^x  
=
(x+y+a+2) e^x\\
f_{yy}(x,y)&=
-e^y\\
f_{xy}(x,y) &=
f_{yx}(x,y)
=
e^x.
\end{align*}
Damit erhalten wir 
\begin{align*}
f_{xx}(x_0,y_0)
&=
\left(
-\frac{a+1}{2} - \frac{a+1}{2} + a + 2 
\right) e^{x_0}\\
&=
\left(
-\frac{a}{2} - \frac{1}{2}  - \frac{a}{2} - \frac{1}{2} + a + 2 
\right) e^{x_0}\\
&=
(-a -1 + a + 2)e^{x_0}
= 1 \cdot e^{x_0} > 0\\
f_{yy}(x_0,y_0)
&=
-e^{y_0} < 0\\
f_{xy}(x_0,y_0) &= e^{x_0} > 0.
\end{align*}
Falls 
\begin{align*}
f_{xx}(x_0,y_0) f_{yy}(x_0,y_0)-
(f_{xy}(x_0,y_0))^2 > 0
\end{align*}
gilt, liegt eine Minimum oder Maximum in $ (x_0,y_0) $ vor. Wenn 
\begin{align*}
f_{xx}(x_0,y_0) f_{yy}(x_0,y_0)-
(f_{xy}(x_0,y_0))^2 < 0
\end{align*}
erfüllt ist, liegt ein Sattelpunkt in $ (x_0,y_0) $ vor.
Wegen
\begin{align*}
f_{xx}(x_0,y_0) f_{yy}(x_0,y_0)-
(f_{xy}(x_0,y_0))^2
=e^{x_0} (-e^{y_0} )- ( e^{x_0})^2
=-e^{2x_0} - e^{2x_0} 
= -2 e^{2x_0} < 0. 
\end{align*}
befindet sich an $ (x_0,y_0) $ ein Sattelpunkt.
\newpage

\subsection*{\aufgabe{b}{7}}
Die Funktion
\begin{align*}
f(x,y) = x^2 + y^2
\end{align*}
ist unter der Nebenbedingung
\begin{align*}
\varphi(x,y)
=
a x^2 +bxy+5y^2-16 = 0
\end{align*}
zu optimieren.\\
Bestimmen Sie die Parameter $ a \in \mathbb{R} $ und $ b \in \mathbb{R} $ so, dass in $ (1,1) $ eine mögliche Extremstelle sein könnte.\\
\\
\textbf{Bemerkung:} \\
Eine Abklärung, ob es sich um eine Extremstelle handelt und von welcher Art die Extremstelle ist (Maximum oder Minimum) wird nicht verlangt.\\

\textbf{Lösung:}
\begin{mdframed}
\underline{\textbf{Vorgehensweise:}}
\renewcommand{\labelenumi}{\theenumi.}
\begin{enumerate}
\item Stelle die Lagrangefunktion und die Bedingung für eine mögliche Extremstelle auf.

\item  Bestimme $ a $ und $ b $, sodass an $ (1,1) $ eine Extremstelle vorliegen kann.
\end{enumerate}
\end{mdframed}

\underline{1. Stelle die Lagrangefunktion und die Bedingung für eine mögliche Extremstelle auf }\\
Die Lagrangefunktion ist durch 
\begin{align*}
F(x,y,\lambda)
= f(x,y) + \lambda \varphi(x,y) 
=
x^2 +y^2 + \lambda (ax^2 +bxy+5y^2-16)
\end{align*}
gegeben.
Hieraus ergeben sich die Lagrangebedingungen
\begin{align*}
F_x(x,y,\lambda)
&=
2x +\lambda (2x a + by) = 0\\
F_y(x,y,\lambda)
&=
2y+\lambda(10 y +b x) = 0\\
F_\lambda(x,y,\lambda)
&=
\varphi(x,y) = ax^2 +bxy+5y^2 -16 = 0.
\end{align*}
Damit wir an $ (1,1) $ eine Extremstelle von $ f $ unter der Nebenbedingung $ \varphi(x,y) = 0 $ erhalten können, muss
\begin{align*}
F_x(1,1,\lambda)
&=
2 +\lambda (2 a + b) = 0\\
F_y(1,1,\lambda)
&=
2+\lambda(10  +b ) = 0\\
F_\lambda(1,1,\lambda)
&=
\varphi(1,1) = a +b+5 -16 = 0.
\end{align*}
erfüllt sein. Das heißt wir suchen nun $ a $ und $ b $, damit die Lagrangebedingungen an der Stelle $ (1,1) $ gleich Null sind.\\
\\
\underline{2. Bestimme $ a $ und $ b $, sodass an $ (1,1) $ eine Extremstelle vorliegen kann}\\
Für $ F_x(1,1,\lambda) = 0 $ gilt
\begin{align*}
2 +\lambda (2 a + b) = 0
\ \Leftrightarrow \
\lambda (2 a +b) -2
\ \Leftrightarrow \
\lambda = - \frac{2}{2a+b}.
\end{align*}
Ebenso gilt für $ F_y(1,1,\lambda) = 0 $:
\begin{align*}
2+\lambda(10  +b ) = 0
\ \Leftrightarrow \
\lambda(10  +b ) = -2
\ \Leftrightarrow \
\lambda = - \frac{2}{10  +b }.
\end{align*}
Für $ \lambda $ liefert uns dies die Darstellungen
\begin{align*}
\lambda &= - \frac{2}{2a +b}\\
\lambda &= - \frac{2}{10 + b }
\end{align*}
und durch Gleichsetzen erhalten wir
\begin{align*}
- \frac{2}{2a +b}
=
- \frac{2}{10 + b }
&\ \Leftrightarrow \
\frac{1}{2a +b} = \frac{1}{10 +b}\\
&\ \Leftrightarrow \
2a + b = 10 + b \\
&\ \Leftrightarrow \
2 a = 10\\
&\ \Leftrightarrow  \
a = 5.
\end{align*}
Mit der Nebenbedingung ergibt sich:
\begin{align*}
\varphi(1,1) = 
5  + b  +5 - 16 = 0
\ \Leftrightarrow \
b - 6 = 0
\ \Leftrightarrow \ 
b = 6.
\end{align*}
Also könnte für $ a = 5  $ und $ b = 6 $ eine Extremstelle an $ (1,1)  $ vorliegen.	
\newpage
\subsection*{\aufgabe{c}{5}}
Berechnen Sie 
\begin{align*}
\int_0^{\sqrt{0.5 \ \pi}} x \ \sin(x^2) \ \left(\cos(x^2)\right)^3 dx.
\end{align*}
\ \\
\textbf{Lösung:}
\begin{mdframed}
\underline{\textbf{Vorgehensweise:}}
\begin{enumerate}
\item Bestimme eine geeignete Substitution, um das Integral zu berechnen.
\end{enumerate}
\end{mdframed}

\underline{1. Bestimme eine geeignete Substitution, um das Integral zu berechnen}\\
Die allgemeine Formel für Integration durch Substitution ist anhand von
\begin{align*}
\int \limits_a^b f( g(x)) \cdot g^\prime(x) \  dx = 
\int \limits_{g(a)}^{g(b)} f(t) \ dt
\end{align*}
für $f$ stetig und $g$ stetig differenzierbar gegeben.
Eine passende Substitution kürzt störende Terme in einem Integral weg.
Dies Erreichen wir mit der Wahl
\begin{align*}
t = \cos(x^2),
\end{align*} 
denn es gilt
\begin{align*}
\frac{
dt}{dx} = -2x \sin(x^2)
\ \Leftrightarrow \
dx
= \frac{dt}{-2x \sin(x^2)}.
\end{align*}
Wichtig ist, auch an die Substitution der Grenzen zu denken.
Hierfür erhalten wir 
\begin{align*}
t_1 &= \cos\left( (\sqrt{0.5 \pi}^2\right)
= \cos \left(\frac{\pi }{2}\right) = 0\\
t_0 &= \cos(0) = 1.
\end{align*}
Für das Integral gilt somit
\begin{align*}
\int_0^{\sqrt{0.5 \ \pi}} x \ \sin(x^2) \ \left(\cos(x^2)\right)^3 dx.
&=
\int_1^{0}
x \ \sin(x^2) t^3 \ \frac{dt}{-2x \sin(x^2)}\\
&=
-\frac{1}{2}\int_1^{0}
t^3 \ dt
=
\frac{1}{2}\int_0^{1}
t^3 \ dt
=\frac{1}{2} 
\left[
\frac{1}{4} t^4
\right]_0^1\\
&=
\frac{1}{2} \cdot \left(\frac{1}{4} 1^4 - \frac{1}{4} 0^4\right) 
=
\frac{1}{2} \cdot \frac{1}{4} 
= \frac{1}{8}.
\end{align*}
\newpage
\subsection*{\aufgabe{d}{6}}
Berechnen Sie 
\begin{align*}
\int_0^e |\ln(x)| dx.
\end{align*}
\ \\
\textbf{Bemerkung:} Sie dürfen das in Mathematik A bewiesene Resultat, dass gilt:
\begin{align*}
\lim \limits_{x \searrow 0} x \ \ln(x) = 0,
\end{align*}
voraussetzen.\\
\\
\textbf{Lösung:}
\begin{mdframed}
\underline{\textbf{Vorgehensweise:}}
\begin{enumerate}
\item Teile das Integral auf, um den Betrag aufzulösen.
\item Bestimme die Stammfunktion mit partieller Integration.
\item Verwende die Stammfunktion, um das Integral zu berechnen.
\end{enumerate}
\end{mdframed}

\underline{1. Teile das Integral auf, um den Betrag aufzulösen}\\
Für den Logarithmus gilt
\begin{align*}
&\ln(x) < 0 \ \textrm{falls} \ x \in (0,1)\\
&\ln(x) \geq 0 \ \textrm{falls} \ x \in [1,\infty),
\end{align*}
womit für das Integral mithilfe der Definition des Betrags
\begin{align*}
\int_0^e |\ln(x) | \ dx
=
\int_0^1 |\ln(x) | \ dx
+
\int_1^e | \ln(x) | \ dx
=
\int_0^1 -\ln(x)  \ dx
+
\int_1^e  \ln(x)  \ dx
\end{align*}
folgt.\\
\\
\underline{2. Bestimme die Stammfunktion mit partieller Integration}\\
Die Formel für partielle Integration ist durch
\begin{align*}
\int u^\prime(x) \cdot v(x) \ dx = u(x) \cdot v(x) - \int u(x) v^\prime(x)\ dx
\end{align*}
für stetig differenzierbare $ u,v $ gegeben.
Durch partielle Integration erhalten wir mit
\begin{align*}
\int \ln(x) \ dx
=
\int 1 \cdot \ln(x) \ dx
= 
x \ln(x) - \int x \frac{1}{x} \ dx
=
x \ln(x) - \int 1 \ dx
= x \ln (x) - x + C
\end{align*}
die gesuchte Stammfunktion.\\
\\
\underline{3. Verwende die Stammfunktion, um das Integral zu berechnen}\\
Zuerst bestimmen wir das einfachere Integral:
\begin{align*}
\int_1^e  \ln(x)  \ dx
= \left[
x \ln(x) - x
\right]_1^e
= 
e \ln(e) - e - \ln(1)  +1 
= 
1.
\end{align*}
Für den zweiten Teil benötigen wir die Aussage
\begin{align*}
\lim \limits_{x \searrow 0} x \ \ln(x) = 0,
\end{align*}
da $ \lim_{x \searrow 0 } \ln(x) = - \infty$ gilt.
Also ist der Logarithmus für $ x = 0 $ nicht definiert und es liegt ein unbestimmtes Integral vor. 
Das Integral berechnen wir nun wie folgt:
\begin{align*}
\int_0^1 -\ln(x)  \ dx
&=
- \int_0^1 \ln(x)  \ dx
=
- \lim \limits_{a \searrow 0} \int_a^1 \ln(x) \ dx
=
- \lim \limits_{a \searrow 0} \left[
x \ln (x) - x
\right]_a^1\\
&=
- \lim \limits_{a \searrow 0} (1 \cdot \ln(1) - 1 - a \ln (a) +a )
=
1 - \lim \limits_{a \searrow 0} (-a \ln(a) +a)
=
1. 
\end{align*}
Insgesamt erhalten wir:
\begin{align*}
\int_0^e |\ln(x) | \ dx
=
\int_0^1 -\ln(x)  \ dx
+
\int_1^e  \ln(x)  \ dx
= 1 + 1 = 2.
\end{align*}
\ \\
\textit{
Alternativ lässt sich auch die partielle Integration für bestimmte Integrale verwenden, falls man die Bestimmung der Stammfunktion und die Berechnung des Integrals kombinieren möchte.
}

\newpage