\fancyhead[C]{\normalsize\textbf{$\qquad$ Teil II: Multiple-Choice}}
\section*{Aufgabe 3 (25 Punkte)}
\vspace{0.4cm}
\subsection*{\frage{1}{3}}
Die Funktion $ f(x,y) = y  $ hat unter der Nebenbedingung $ \varphi(x,y) = \frac{x^2}{25} + \frac{y^2}{9} = 1 $
ihr Minimum in
\renewcommand{\labelenumi}{(\alph{enumi})}
\begin{enumerate}
	\item $ P = (-5,0) $.
	\item $ P = (0,3) $.
	\item $ P = (0,0) $.
	\item $ P = (0,-3) $.
\end{enumerate}
\ \\
\textbf{Lösung:}
\begin{mdframed}
\underline{\textbf{Vorgehensweise:}}
\renewcommand{\labelenumi}{\theenumi.}
\begin{enumerate}
\item Überprüfe, welche Antwortbedingungen die Nebenbedingung erfüllen.
\item Bestimme die korrekte Antwort im Ausschlussverfahren.
\end{enumerate}
\end{mdframed}

\underline{1. Überprüfe, welche Antwortbedingungen die Nebenbedingung erfüllen}\\
%Der Punkt $ P = (0,0) $ erfüllt die Nebenbedingung nicht.
%Damit fällt (c) weg.
Wir setzen $ P_1 = (-5,0) $, $ P_2 = (0,3)  $, $ P_3=(0,0) $ und $ P_4 = (0,-3) $.
Dann gilt:
\begin{align*}
\varphi(P_1)
&=
\frac{(-5)^2}{25} + \frac{0^2}{9} = \frac{25}{25} + 0 = 1\\
\varphi(P_2)
&=
\frac{0^2}{25} + \frac{3^2}{9} = 0 + \frac{9}{9} = 1\\
\varphi(P_3)
&=
\frac{0^2}{25} + \frac{0^2}{9} = 0 \neq 1\\
\varphi(P_4)
&=
\frac{0^2}{25} + \frac{(-3)^2}{9} = 0 + \frac{9}{9} = 1.
\end{align*}
Damit erfüllt $ P_3 $ die Nebenbedingung nicht.\\
\\
\underline{2. Bestimme die korrekte Antwort im Ausschlussverfahren}\\
Da die anderen Punkte die Nebenbedingung erfüllen, vergleichen wir nun deren Funktionswerte:
\begin{align*}
f(P_1) &= 0\\
f(P_2) &= 3\\
f(P_4) &= -3.
\end{align*}
\ \\
Da der Funktionswert an dem Punkt $ P_4 $ am kleinsten ist, befindet sich dort das Minimum.\\
\\
Damit ist die Antwort (d) korrekt.

\newpage

\subsection*{\frage{2}{4}}
Gegeben ist die Funktion
\begin{align*}
f(x) 
=
\begin{cases}
ax + \frac{1}{8}, &\quad \textrm{für } 0 \leq x \leq 4\\
\quad 0					&\quad  \textrm{sonst}			
\end{cases}
\end{align*}
\renewcommand{\labelenumi}{(\alph{enumi})}
\begin{enumerate}
	\item $f $ ist für alle $ a \in \mathbb{R} $ eine Dichtefunktion.
	\item $f $ ist nur für $ a = \frac{1}{16} $ eine Dichtefunktion.
	\item
	$f $ ist nur für $ a = -\frac{1}{16} $ eine Dichtefunktion.
	\item
	$f $ ist für kein $ a \in \mathbb{R} $ eine Dichtefunktion.
\end{enumerate}
\ \\
\textbf{Lösung:}
\begin{mdframed}
	\underline{\textbf{Vorgehensweise:}}
	\renewcommand{\labelenumi}{\theenumi.}
	\begin{enumerate}
		\item Berechne das Integral über $ f $ in Abhängigkeit von $ a $.
		\item Verwende die Definition einer Dichtefunktion.
	\end{enumerate}
\end{mdframed}
\underline{1. Berechne das Integral über $ f $ in Abhängigkeit von $ a $}\\
Für das Integral über $ f $ gilt:
\begin{align*}
\int_{-\infty}^\infty f(x) \ dx
=
\int_0^4 ax + \frac{1}{8} \ dx
= 
\left[
a \frac{x^2}{2} + \frac{1}{8} x
\right]_0^4
=
a \frac{4^2}{2} + \frac{4}{8}- \left( a \frac{0^2}{2} + \frac{0}{8}\right)
=
8 a + \frac{1}{2}.
\end{align*}
\ \\
\underline{2. Verwende die Definition einer Dichtefunktion}\\
Eine Funktion $ f \geq 0 $ heißt Dichtefunktion auf $ (-\infty, \infty) $, falls
\begin{align*}
\int_{-\infty}^\infty f(x) \ dx = 1
\end{align*}
gilt.
Damit folgt
\begin{align*}
\int_{-\infty}^\infty f(x) \ dx
=
8a + \frac{1}{2} = 1 
\ \Leftrightarrow \
8a = \frac{1}{2}
\ \Leftrightarrow \
a =  \frac{1}{16}
\end{align*}
und wegen $ a > 0 $ erhalten wir $ f(x) \geq 0  $ für alle $ x \in \mathbb{R} $.\\
Damit ist sind die Eigenschaften einer Dichtefunktion für $ a = \frac{1}{16} $ erfüllt.\\
\\
Somit ist die Antwort (b) korrekt.
\newpage

\subsection*{\frage{3}{2}}
Sei $ f : [a,b] \to \mathbb{R} $ eine beliebige, auf dem Intervall $ [a,b] $ definierte Funktion.\\
\\
Welche der folgenden Aussagen ist \textit{richtig}?
\renewcommand{\labelenumi}{(\alph{enumi})}
\begin{enumerate}
	\item 
	Wenn das bestimmte Integral von $ f $ über $ [a,b] $ existiert, dann ist $ f $ stetig auf $ [a,b] $.
	\item 
	Wenn $ f $ nicht stetig ist auf $ [a,b] $, dann existiert das bestimmte Integral von $ f $ über $ [a,b] $ nicht.
	\item 
	Wenn $ f $ differenzierbar ist auf $ [a,b] $, dann existiert das bestimmte Integral von $ f $ über $ [a,b] $.
	\item
	Das bestimmte Integral von $ f $ über $ [a,b]$ existiert immer.
\end{enumerate}
\ \\
\textbf{Lösung:}
\begin{mdframed}
\underline{\textbf{Vorgehensweise:}}
\renewcommand{\labelenumi}{\theenumi.}
\begin{enumerate}
\item Eliminiere Lösungen mithilfe des Verständnisses von Differenzierbarkeit und Integrierbarkeit.
\end{enumerate}
\end{mdframed}

\underline{1. Eliminiere Lösungen mithilfe des Verständnisses von Differenzierbarkeit und Integrierbarkeit.}\\
Folgende Grafik veranschaulicht eine Sprungfunktion.
Das Integral über diese Funktion existiert auf dem Intervall $ [0,5] $, jedoch ist diese offensichtlich nicht stetig.
Damit ist (a) und (b) falsch.
\begin{center}
	\begin{tikzpicture}
	\begin{axis}[
	domain=0:5,
	xmin=0, xmax=5,
	ymin=0, ymax=2,
	samples=5,
	axis y line=center,
	axis x line=middle,
	]
	\addplot+[jump mark left] { 0.5 * floor(x)};
	%\addplot+[mark=none] { sin(deg(2*x)};
	%\legend{$ \sin(2x) $}
	\end{axis}
	\end{tikzpicture}
\end{center}

Wir betrachten nun die Funktion
\begin{align*}
f : [a,b] \to \mathbb{R},
x \mapsto 
\begin{cases}
\quad 1  &\quad \textrm{falls} \ x \ \textrm{rational}\\
 \quad 0  &\quad \textrm{falls} \ x \ \textrm{irrational}
\end{cases}.
\end{align*}
Das bestimmte Integral über $ [a,b] $ existiert in diesem Fall nicht.
Jedoch würde die Begründung unseren Rahmen sprengen.
Damit ist (d) auch falsch und es bleibt nur (c) übrig.\\
\\
Wenn $ f $ differenzierbar auf $ [a,b] $ ist, ist $ f $ auch stetig auf $ [a,b] $.
Stetige Funktionen auf $ [a,b] $ sind immer integrierbar.\\
\\
Damit ist die Antwort (c) korrekt.

\newpage

\subsection*{\frage{4}{2}}
$A$ und $B$ seien quadratische Matrizen mit
$\det(A) = 5$ und $\det(B) = 2$; die Matrix $ C $ ist definiert durch 
$C \ = \ A^{-1}B A $. 
\renewcommand{\labelenumi}{(\alph{enumi})}
\begin{enumerate}
\item 
Dann gilt für jedes $ n \in \mathbb{N}: $ $ \det(C^n)  = 1$.
\item
Dann gilt für jedes $ n \in \mathbb{N}: $ $ \det(C^n)  = 2^n$.
\item
Dann gilt für jedes $ n \in \mathbb{N}: $ $ \det(C^n)  = 2^n \cdot 5^n$.
\item
Keine der obigen Aussagen ist korrekt.
\end{enumerate}
\ \\
\textbf{Lösung:}
\begin{mdframed}
\underline{\textbf{Vorgehensweise:}}
\renewcommand{\labelenumi}{\theenumi.}
\begin{enumerate}
\item Verwende die Rechenregeln der Determinante.
\end{enumerate}
\end{mdframed}

\underline{1. Verwende die Rechenregeln der Determinante}\\
Für quadratische Matrizen $ X $ und $ Y $ gilt
\begin{align*}
\det (X\cdot Y) = \det(X) \cdot \det(Y).
\end{align*}
Insbesondere gilt dann auch
\begin{align*}
\det(C^n) = \underbrace{\det(C) \cdots \det(C)}_{
	 n -\textrm{mal}
}
=\left(\det(C)\right)^n.
\end{align*}
Außerdem benötigen wir noch
\begin{align*}
\det (A^{-1}) = \frac{1}{\det(A)}.
\end{align*}
Wir erhalten
\begin{align*}
\det(C) = \det(A^{-1} B A)
=
\frac{1}{\det(A)} \det(B) \det(A)
=
\det(B) = 2
\
\Rightarrow 
\
\det(C^n) = 2^n.
\end{align*}
\ \\
Damit ist Antwort (b) korrekt.
\newpage

\subsection*{\frage{5}{4}}
Gegeben sind die Vektoren
\begin{align*}
\textbf{a}
= 
\begin{pmatrix}
1 \\ 2 \\ 3
\end{pmatrix},
\textbf{b}
=
\begin{pmatrix}
1 \\ -1 \\ 1
\end{pmatrix},
\textbf{c}
=
\begin{pmatrix}
0 \\ 0 \\ 2
\end{pmatrix},
\textbf{d}
=
\begin{pmatrix}
2 \\ 1 \\ t
\end{pmatrix}.
\end{align*}
\renewcommand{\labelenumi}{(\alph{enumi})}
\begin{enumerate}
	\item 
	Es ist nur für $ t = 6 $ möglich, $ \textbf{d} $ als Linearkombination von $ \textbf{a} $, $ \textbf{b} $ und $ \textbf{c} $ zu schreiben.
	\item
	Es ist nur für $ t = 6 $ und $ t = 0 $ möglich, $ \textbf{d} $ als Linearkombination von $ \textbf{a} $, $ \textbf{b} $ und $ \textbf{c} $ zu schreiben.
	\item
	Es ist für alle $ t\in \mathbb{R} $ möglich, $ \textbf{d} $ als Linearkombination von $ \textbf{a} $, $ \textbf{b} $ und $ \textbf{c} $ zu schreiben.
	\item
	Es ist für kein $ t\in \mathbb{R} $ möglich, $ \textbf{d} $ als Linearkombination von $ \textbf{a} $, $ \textbf{b} $ und $ \textbf{c} $ zu schreiben.
\end{enumerate}
\ \\
\textbf{Lösung:}
\begin{mdframed}
\underline{\textbf{Vorgehensweise:}}
\renewcommand{\labelenumi}{\theenumi.}
\begin{enumerate}
\item Überlege dir, ob die Vektoren $ \textbf{a} $, $ \textbf{b} $ und $ \textbf{c} $ linear unabhängig sind.
\end{enumerate}
\end{mdframed}

\underline{1. Überlege dir, ob die Vektoren $ \textbf{a} $, $ \textbf{b} $ und $ \textbf{c} $ linear unabhängig sind}\\
Die Vektoren $ \textbf{a} $, $ \textbf{b} $ und $ \textbf{c} $
sind linear unabhängig genau dann, wenn
$ \det( \textbf{a} ,  \textbf{b}, \textbf{c} ) \neq 0$ ist.
Wir entwickeln nach der dritten Spalte, da dort zwei Nullen auftreten.
Durch die Entwicklung nach der dritten Spalte erhalten wir:
\begin{align*}
\det( \textbf{a} ,  \textbf{b}, \textbf{c} )
=
\det
\begin{pmatrix}
1 & 1 & 0\\
2 & -1 & 0\\
3 & 1 & 2
\end{pmatrix}
= 2 \cdot \det 
\begin{pmatrix}
1 & 1\\
2 & -1
\end{pmatrix}
= 
2 \cdot (1 \cdot (-1) - 2 \cdot 1)
=
2 \cdot (-3) = 6.
\end{align*}
Damit bilden die Vektoren $ \textbf{a} $, $ \textbf{b} $ und $ \textbf{c} $ eine Basis des $ \mathbb{R}^3 $.
Insbesondere lässt sich mit diesen jeder beliebige Vektor des $ \mathbb{R}^3 $ darstellen.\\
\\
Damit ist die Antwort (c) korrekt.
\newpage

\subsection*{\frage{6}{2}}
$ A $ ist eine $ 6 \times 5 $ Matrix, das lineare Gleichungssystem $ A \textbf{x} = \textbf{b} $ hat unendlich viele Lösungen und der Lösungsraum hat die Dimension $ 2 $.
Dann gilt:
\renewcommand{\labelenumi}{(\alph{enumi})}
\begin{enumerate}
	\item 
	$ \text{rg}(A) = \text{rg}(A; \textbf{b}) = 3 $.
	\item
	$ \text{rg}(A) = \text{rg}(A; \textbf{b}) = 4 $.
	\item
	$ \text{rg}(A) < \text{rg}(A; \textbf{b}) = 3 $..
	\item
	Keine der obigen Aussagen ist korrekt.
\end{enumerate}
\ \\
\textbf{Lösung:}
\begin{mdframed}
\underline{\textbf{Vorgehensweise:}}
\renewcommand{\labelenumi}{\theenumi.}
\begin{enumerate}
\item Berechne den Rang der Matrix $ A $.
\end{enumerate}
\end{mdframed}

\underline{1. Berechne den Rang der Matrix $ A $}\\
Da das lineare Gleichungssystem $ A \textbf{x} = \textbf{b} $ lösbar ist, gilt
\begin{align*}
rg(A) = rg(Ab).
\end{align*}
Sie $ L $ der Lösungsraum des LGS $ A \textbf{x} = \textbf{b} $ und $ n = 5 $ die Anzahl der Variablen. Dann gilt
\begin{align*}
\dim \ L = n - rg(A)
\ \Leftrightarrow \
2 = 5 - rg(A)
\ \Leftrightarrow \
rg(A) = 3.
\end{align*}
\ \\
Damit ist die Antwort (a) korrekt.
\newpage
\subsection*{\frage{7}{4}}
Das unbestimmte Integral von
\begin{align*}
\int \ln(x \ e^x ) \ dx, \ (x > 0)
\end{align*}
ist
\renewcommand{\labelenumi}{(\alph{enumi})}
\begin{enumerate}
	\item 
	$ x \ \ln(x) + x^2 - x + C $.
	\item
	$ x \ \ln(x) + \frac{x^2}{2} - x + C $.
	\item
	$ x \ \ln(x) + x^2  + C $.
	\item
	Keine der obigen Antworten ist korrekt.
\end{enumerate}
\ \\
\textbf{Lösung:}
\begin{mdframed}
\underline{\textbf{Vorgehensweise:}}
\renewcommand{\labelenumi}{\theenumi.}
\begin{enumerate}
\item Vereinfache den Integranden.
\item Bestimme die Stammfunktion.
\end{enumerate}
\end{mdframed}

\underline{1. Vereinfache den Integranden und bestimme die Stammfunktion}\\
Für den Integranden gilt
\begin{align*}
f(x) := \ln(x e^x) = \ln(x) + \ln(e^x) = \ln(x) + x.
\end{align*}
Mit der Vorüberlegung erhalten wir
\begin{align*}
\int f(x) \ dx 
=
\int \ln(x) \ dx + \int x \ dx
= 
\int \ln(x) \ dx + \frac{x^2}{2} + C,
\end{align*}
wodurch wir nur noch die Stammfunktion von $ \ln(x) $ bestimmen müssen.\\
\\
\underline{2. Bestimme die Stammfunktion}\\
Aufgrund der Stammfunktion von $ x $ liegt jedoch die Vermutung nahe, dass die Antworten (a) und (c) falsch sind, da in beiden Fällen der Faktor $ \frac{1}{2} $ fehlt.
Mit partieller Integration erhalten wir
\begin{align*}
\int 1 \cdot \ln(x) \ dx
= x \ln(x) - \int x  \cdot\frac{1}{x} \ dx
= x\ln(x) - x + C.
\end{align*}
Insgesamt folgt dann
\begin{align*}
\int f(x) \ dx = 
x \ln(x) -x + \frac{x^2}{2} + C.
\end{align*}
\ \\
Damit ist die Antwort (b) korrekt.\\
\\
\textit{Alternativ lässt sich die korrekte Antwort durch Differenzieren der Möglichkeiten (a)-(c) herleiten.
}


\newpage

\subsection*{\frage{8}{4}}
Gegeben ist die Matrix
\begin{align*}
A = 
\begin{pmatrix}
2 & a\\
a & 2
\end{pmatrix},
\ \textrm{wobei } a\neq 0.
\end{align*}
\renewcommand{\labelenumi}{(\alph{enumi})}
\begin{enumerate}
\item 
Die Matrix hat für alle $ a \neq 0 $ in $ \mathbb{R} $ zwei verschiedene reelle Eigenwerte.
\item
Die Matrix hat für alle $ a \neq 0 $ in $ \mathbb{R} $ genau einen reellen Eigenwert.

\item
Die Matrix hat für alle $ a \neq 0 $ in $ \mathbb{R} $ keinen reellen Eigenwert..
\item
Die Matrix $ A $ hat abhängig von $ a \neq 0 $ keinen, einen oder zwei reelle Eigenwerte.
\end{enumerate}
\ \\
\textbf{Lösung:}
\begin{mdframed}
\underline{\textbf{Vorgehensweise:}}
\renewcommand{\labelenumi}{\theenumi.}
\begin{enumerate}
\item Bestimme die Eigenwerte der Matrix $ A $.
\end{enumerate}
\end{mdframed}

\underline{1. Bestimme die Eigenwerte der Matrix $ A $}\\
Die Eigenwerte der Matrix $ A $ sind die Lösungen der Gleichung
\begin{align*}
\det(A - \lambda I) = 0.
\end{align*}
Es gilt
\begin{align*}
\det(A - \lambda I)=
\det
\begin{pmatrix}
2- \lambda & a \\
a & 2 - \lambda
\end{pmatrix}
= (2- \lambda)^2 - a^2
= 4 - 4 \lambda + \lambda^2 - a^2.
\end{align*}
Die Nullstellen hiervon erhalten wir durch
\begin{align*}
\lambda_{\nicefrac{1}{2}}
=
\frac{4 \pm \sqrt{(-4)^2 - 4 \cdot 1 \cdot (4 - a^2)}}{2}
=
\frac{4 \pm \sqrt{4 a^2}}{2}
=
\frac{4 \pm 2 |a|}{2}
=
2 \pm |a|.
\end{align*}
Damit hat $ A $ zwei reelle Eigenwerte.
Alternativ lassen sich die Nullstellen auch wie folgt bestimmen:
\begin{align*}
(2- \lambda)^2 - a^2 = 0 
\ \Leftrightarrow \
(2- \lambda)^2 = a^2 
\ \Leftrightarrow \
2 - \lambda = \pm|a|
\ \Leftrightarrow \
\lambda = 2 \pm |a|.
\end{align*}
\\
\\
Somit ist Antwort (a) korrekt.

\newpage