\section*{Aufgabe 2 (25 Punkte)}
\vspace{0.4cm}
\subsection*{\aufgabe{a}{4}}
Die quadratischen $ n \times n $ Matrizen $ A $ und $ B $ seien regulär, $ A $ sei ausserdem symmetrisch.\\
\\
Beweisen Sie:
\begin{align*}
B^\top (AB)^\top (B^{-1} A^{-1})^\top B (A B)^{-1}
=
(A^{-1} B )^\top.
\end{align*}
\\
\\
\textbf{Lösung:}
\begin{mdframed}
\renewcommand{\labelenumi}{\theenumi.}
\underline{\textbf{Vorgehensweise:}}
\begin{enumerate}
\item Rufe dir die notwendigen Rechengesetze in Erinnerung.
\item Beweise die Aussage.
\end{enumerate}
\end{mdframed}


\underline{1. Rufe dir die notwendigen Rechengesetze in Erinnerung }\\
Wir rufen uns die nötigen Rechengesetze in Erinnerung.
Eine Matrix $ A $ heißt symmetrisch, falls
\begin{align*}
A = A^\top
\end{align*}
gilt. Dementsprechend erhalten wir
\begin{align*}
A^{-1} = (A^\top)^{-1} = (A^{-1})^\top.
\end{align*}
Außerdem gelten:
\begin{align*}
(A B)^{-1} &= B^{-1} A^{-1}\\
(A B)^\top &= B^\top A^\top.
\end{align*}
\ \\
\underline{2. Beweise die Aussage}\\
Durch die Umformungen
\begin{align*}
B^\top (AB)^\top (B^{-1} A^{-1})^\top B (A B)^{-1}
&=
B^\top B^\top A^\top (A^{-1})^\top (B^{-1})^\top B B^{-1} A^{-1}\\
&=
B^\top B^\top A^\top (A^\top)^{-1} (B^{-1})^\top B B^{-1} A^{-1}\\
&=
B^\top B^\top I (B^{-1})^\top B B^{-1} A^{-1}\\
&=
B^\top B^\top I (B^\top)^{-1} B B^{-1} A^{-1}\\
&=
B^\top I B B^{-1} A^{-1}\\
&=
B^\top A^{(-1)}\\
&=
((A^{-1})^\top B)^\top\\
&= (A^{-1} B)^\top
\end{align*}
erhalten wir die Aussage.
\newpage

\subsection*{\aufgabe{b}{4}}
Gegeben ist die Funktion
\begin{align*}
f(x,y) = a \ln(x-2) + x \ y^2 + 8 \ y, \quad \textrm{wobei} \ a \in \mathbb{R}.
\end{align*}
Berechnen Sie den Gradienten von $ f $ an der Stelle $ (x_0,y_0) = (8,2) $.\\
\\
Wie muss der Parameter $ a \in \mathbb{R} $ gewählt werden, damit die Funktion $ f $ im Punkt $ (x_0,y_0) = (8,2) $ in der Richtung
$ \textbf{b} = \begin{pmatrix}
3 \\
4
\end{pmatrix} $ am stärksten zunimmt?\\
\\
\textbf{Lösung:}
\begin{mdframed}
\underline{\textbf{Vorgehensweise:}}
\begin{enumerate}
\renewcommand{\labelenumi}{\theenumi.}
\item Berechne den Gradienten an der Stelle $ (x_0,y_0) $.
\item Richte den Gradienten nach $ \textbf{b} $ aus.
\end{enumerate}
\end{mdframed}
\underline{1. Berechne den Gradienten an der Stelle $ (x_0,y_0) $}\\
Zunächst berechnen wir die partiellen Ableitungen von $ f $. Hierfür erhalten wir
\begin{align*}
f_x(x,y) &=  \frac{a}{x-2}+ y^2 \\
f_y(x,y) &= 2 xy + 8,
\end{align*}
wodurch für den Gradienten
\begin{align*}
\textbf{grad}f(x,y)
= \begin{pmatrix}
\frac{a}{x-2} +y^2\\
2xy +8
\end{pmatrix}
\ \Rightarrow \
\textbf{grad}f(8,2) =
\begin{pmatrix}
\frac{a}{6} + 4\\
40
\end{pmatrix}
\end{align*}
gilt.\\
\\
\underline{2. Richte den Gradienten nach $ \textbf{b} $ aus}\\
Der Gradient an der Stelle $ (x_0,y_0) $ zeigt in die Richtung des steilsten Anstiegs. Dementsprechend muss
\begin{align*}
\lambda \textbf{b} = \textbf{grad} f(8,2)
\ \Leftrightarrow \
\lambda 
\begin{pmatrix}
3 \\ 4
\end{pmatrix}
= 
\begin{pmatrix}
\frac{a}{6} + 4\\
40
\end{pmatrix}
\end{align*}
erfüllt sein. Aus der zweiten Komponente erhalten wir $ \lambda = 10 $.
Die erste Komponente liefert dann mit
\begin{align*}
10 \cdot 3 = \frac{a}{6} + 4
\ \Leftrightarrow \
30  - 4 = 26 =  \frac{a}{6}
\ \Leftrightarrow \
a = 6 \cdot 26 = 156
\end{align*}
das passende $ a $.


\newpage

\subsection*{\aufgabe{c}{3}}
Gegeben sind die Vektoren
\begin{align*}
\textbf{b}_1 =
\begin{pmatrix}
1 \\
t\\ 
0
\end{pmatrix},
\
\textbf{b}_2 =
\begin{pmatrix}
2t \\
4\\ 
t
\end{pmatrix},
\
\textbf{b}_3 =
\begin{pmatrix}
8 \\
t\\ 
t^2
\end{pmatrix}.
\end{align*}
Für welche Werte von $ t \in \mathbb{R} $ ist das Vektorsystem $ \{ \textbf{b}_1, \textbf{b}_2, \textbf{b}_3 \} $ \textit{keine} Basis des dreidimensionalen Raumes $ \mathbb{R}^3 $?
\\
\\
\textbf{Lösung:}
\begin{mdframed}
\underline{\textbf{Vorgehensweise:}}
\renewcommand{\labelenumi}{\theenumi.}
\begin{enumerate}
\item Überlege dir den Zusammenhang von linearer Abhängigkeit und der Determinante einer Matrix.
\end{enumerate}
\end{mdframed}

\underline{1. Überlege dir den Zusammenhang von linearer Abhängigkeit und der Determinante einer Matrix}\\
Die drei Vektoren $ \textbf{b}_1 $, $ \textbf{b}_2 $ und $ \textbf{b}_3$ bilden keine Basis, wenn diese linear abhängig sind. 
$ \textbf{b}_1 $, $ \textbf{b}_2 $ und $ \textbf{b}_3$ sind linear abhängig genau dann, wenn
\begin{align*}
\det \begin{pmatrix}
\textbf{b}_1,\textbf{b}_2,\textbf{b}_3
\end{pmatrix}
= 0
\end{align*}
gilt. 
Wir bestimmen nun die Determinante abhängig von $ t $. Wir werden nach der ersten Spalte entwickeln, da dort eine Null vorkommt. Mit der Entwicklung nach der ersten Spalte gilt:
\begin{align*}
\det \begin{pmatrix}
\textbf{b}_1,\textbf{b}_2,\textbf{b}_3
\end{pmatrix}
= 
\det 
\begin{pmatrix}
1 & 2t & 8\\
t & 4 &t \\
0 & t & t^2
\end{pmatrix}
&=
1 \cdot \det \begin{pmatrix}
4 & t \\
t & t^2
\end{pmatrix}
- t \det
\begin{pmatrix}
2t & 8 \\
t & t^2
\end{pmatrix}\\
&=
4t^2 - t^2 - t(2t^3 - 8t)
= 
3t^2 - 2t^4 + 8 t^2 \\
&= -2t^4 + 11 t^2
=
t^2(11 - 2 t^2).
\end{align*}
Für $ t = 0 $ ist diese Determinante gleich null.
Außerdem gilt
\begin{align*}
11 - 2t^2 = 0
\ \Leftrightarrow \
t^2 = \frac{ 11}{2}
\ \Leftrightarrow \
t = \pm \sqrt{\frac{11}{2}}.
\end{align*} 
Damit sind die Vektoren $ \textbf{b}_1 $, $ \textbf{b}_2 $ und $ \textbf{b}_3$ für 
\begin{align*}
\left\lbrace - \sqrt{\frac{11}{2}} , 0 ,  \sqrt{\frac{11}{2}} \right\rbrace
\end{align*}
linear abhängig.
\newpage
\subsection*{\aufgabe{d}{6}}
Gegeben ist die Matrix
\begin{align*}
M = 
\begin{pmatrix}
0  & 2a\\
-3a & 5a
\end{pmatrix}
,
\quad
\textrm{wobei } a \neq 0.
\end{align*}
Berechnen Sie die Eigenwerte und Eigenvektoren der Matrix $ M $.\\
\\
Berechnen Sie den Vektor $ M^n \textbf{x} $, wobei $ \textbf{x} = \begin{pmatrix}
2 \\ 2
\end{pmatrix}. $
\\
\\
\textbf{Lösung:}
\begin{mdframed}
\underline{\textbf{Vorgehensweise:}}
\renewcommand{\labelenumi}{\theenumi.}
\begin{enumerate}
\item Berechne die Eigenwerte und zugehörige Eigenvektoren.
\item Bestimme den gesuchten Vektor.
\end{enumerate}
\end{mdframed}

\underline{1. Berechne die Eigenwerte und zugehörige Eigenvektoren }\\
Die Eigenwerte der Matrix $ M $ bestimmen wir, indem wir die Gleichung
\begin{align*}
\det ( M - \lambda I) = 0
\end{align*}
lösen.
Es gilt
\begin{align*}
\det 
\begin{pmatrix}
- \lambda & 2a\\
-3a & 5a - \lambda
\end{pmatrix}
=
- \lambda (5a - \lambda) - (-3a) \cdot 2a
=
- \lambda (5a - \lambda) + 6 a^2
=
\lambda^2 - 5a \lambda + 6 a^2,
\end{align*}
womit wir 
\begin{align*}
\lambda_{\nicefrac{1}{2} }
&=
\frac{5a \pm \sqrt{(-5a)^2- 4 \cdot 6 a^2}}{2}
=
\frac{5a \pm \sqrt{25 a^2 - 24 a^2}}{2}
=
\frac{5a \pm \sqrt{a^2}}{2}
= \frac{5a \pm a}{2}\\
\ \Leftrightarrow \
\lambda_1 &= 2 a, \quad \lambda_2 = 3a
\end{align*}
erhalten.
Zuerst bestimmen wir die Eigenvektoren zu $ \lambda_1 $. Hierfür lösen wir
\begin{align*}
(M - \lambda_1 I ) x = 0.
\end{align*}
Wir erhalten 
\begin{align*}
\begin{pmatrix}
0 - \lambda_1 & 2a \\
-3a & 5a - \lambda_1
\end{pmatrix}
=
\begin{pmatrix}
0 - 2a & 2a \\
-3a & 5a - 2a
\end{pmatrix}
=
\begin{gmatrix}[p]
-2a & 2a \\
-3a & 3a
\rowops
\mult{0}{\cdot ( \frac{1}{2a} )}
\mult{1}{\cdot ( \frac{1}{3a} )}
\end{gmatrix}
\leadsto 
\begin{gmatrix}[p]
-1 & 1 \\
-1 &1
\end{gmatrix}
\leadsto
\begin{pmatrix}
-1 & 1 \\
0 & 0
\end{pmatrix},
\end{align*}
wobei wir die rechte Seite bewusst weglassen. Damit gilt insbesondere 
\begin{align*}
-x_1 + x_2 = 0 \ \Leftrightarrow \ x_1 = x_2, 
\end{align*}
wodurch wir die Eigenvektoren
\begin{align*}
\textbf{v}_1 = t \cdot \begin{pmatrix}
1 \\ 1
\end{pmatrix}, \quad t \in \mathbb{R}
\end{align*}
zu $ \lambda_1  $ erhalten.
Für $ \lambda_2 $ lösen wir 
\begin{align*}
(M-\lambda_2I)x = 0.
\end{align*}
Die Zeilenumformungen 
\begin{align*}
\begin{pmatrix}
0 - \lambda_2 & 2a \\
-3a & 5a - \lambda_2
\end{pmatrix}
=
\begin{pmatrix}
-3a - 2a & 2a \\
-3a & 5a - 3a
\end{pmatrix}
=
\begin{gmatrix}[p]
-3a & 2a \\
-3a & 2a
\rowops
\add[-1]{0}{1} 
\end{gmatrix}
&\leadsto
\begin{gmatrix}[p]
-3a & 2a \\
0 & 0
\rowops
\mult{0}{\cdot \frac{1}{3} a}
\end{gmatrix}\\
&\leadsto 
\begin{pmatrix}
-1 & \frac{2}{3} \\
0 & 0
\end{pmatrix}
\end{align*}
führen auf
\begin{align*}
-x_1 + \frac{2}{3} x_2 = 0 
\ \Leftrightarrow \
x_1 = \frac{2}{3} x_2 
\end{align*}
Dies liefert uns die Eigenvektoren
\begin{align*}
\textbf{v}_2
= s \cdot \begin{pmatrix}
2 \\ 3
\end{pmatrix}, \quad s \in \mathbb{R}
\end{align*}
für den Eigenwert $ \lambda_2 $.\\
\\
\underline{2. Bestimme den gesuchten Vektor}\\
Der Wert $\lambda$ heißt \textit{Eigenwert} zum Eigenvektor $\textbf{x} \neq 0$, falls
\begin{align*}
M \cdot \textbf{x} = \lambda \cdot \textbf{x}
\end{align*}
gilt.
Der Vektor $ \textbf{x} = \begin{pmatrix}
2 \\ 2
\end{pmatrix} $ ist ein Vielfaches des Vektors
\begin{align*}
\begin{pmatrix}
1 \\ 1
\end{pmatrix}
\end{align*}
und damit ein Eigenvektor zum Eigenwert $ \lambda_1 = 2a $. 
Dementsprechend gilt
\begin{align*}
M^n \textbf{x } = 
\lambda_1 M^{n-1} \textbf{x}
= \dots 
= 
\lambda_1^n \textbf{x}
= 
(2a)^n 
\begin{pmatrix}
2\\2
\end{pmatrix}
=
2^{n+1} a^n 
\begin{pmatrix}
1\\1	
\end{pmatrix}.
\end{align*}
\newpage

\subsection*{\aufgabe{e}{8}}
Gegeben ist das Gleichungssystem
\begin{equation*}
\begin{split}
x_1 \ + \  x_2 \ + \  x_3 \ + \  x_4 \ + \ x_5 \ &= \ \ 0 \\
2x_1 \ + \ 3 x_2 \ + \ 4 x_3 \ + \ 5 x_4 \ - \ x_5 \ &= \ \ 0 \\
x_1 \ + \ 2 x_2 \ + \ 4 x_3 \ - \  x_4 \ + \ 2x_5 \ &= \ 0
\end{split}
.
\end{equation*}
Berechnen Sie die allgemeine Lösung dieses Gleichungssystems mit dem Gauß-Verfahren.\\
\\
Für den Unterraum
\begin{align*}
W 
=
\{
\textbf{x} \in \mathbb{R}^5 | x_1 + x_2 + x_3 + x_4 + x_5 = 0 \wedge
2 x_1 + 3 x_2 +4 x_3 + 5 x_4 - x_5 = 0
\wedge
x_1 + 2x_2 +4 x_3 - x_4 + 2x_5 = 0
\}
\end{align*}
ist eine Basis anzugeben.\\
\\
\textbf{Lösung:}
\begin{mdframed}
\underline{\textbf{Vorgehensweise:}}
\renewcommand{\labelenumi}{\theenumi.}
\begin{enumerate}
\item Stelle die erweiterte Koeffizientenmatrix auf.
\item Löse das System und gebe die Basis von $ W $ an.
\end{enumerate}
\end{mdframed}

\underline{1. Stelle die erweiterte Koeffizientenmatrix auf}\\
Die erweiterte Koeffizientenmatrix ist durch
\begin{align*}
	(A| \textbf{b})
	=
	\begin{gmatrix}[p]
	1 & 1 & 1 & 1 & 1 &  \BAR & 0 \\ 
	2 & 3 & 4 & 5 & -1 &  \BAR & 0 \\
	1 & 2 & 4 & -1 & 2 &  \BAR & 0 
	\end{gmatrix}
\end{align*}
gegeben. Da die rechte Seite null ist, können wir diese bei weiteren Umformungen ignorieren.\\
\\
\underline{2. Löse das System und gebe die Basis von $ W $ an}\\
Die Lösung des Systems erhalten wir durch elementare Zeilenumformungen. Diese Umformungen liefern
\begin{align*}
\begin{gmatrix}[p]
1 & 1 & 1 & 1 & 1 \\ 
2 & 3 & 4 & 5 & -1  \\
1 & 2 & 4 & -1 & 2 
\rowops
\add[-2]{0}{1}
\add[-1]{0}{2} 
\end{gmatrix}
&\leadsto
\begin{gmatrix}[p]
1 & 1 & 1 & 1 & 1 \\ 
0 & 1 & 2 & 3 & -3  \\
0 & 1 & 3 & -2 & 1 
\rowops
\add[-1]{1}{0}
\add[-1]{1}{2} 
\end{gmatrix}\\
&\leadsto
\begin{gmatrix}[p]
1 & 0 & -1 & -2 & 4 \\ 
0 & 1 & 2 & 3 & -3  \\
0 & 0 & 1 & -5 & 4 
\rowops
\add[-2]{2}{1}
\add[1]{2}{0} 
\end{gmatrix}\\
&\leadsto
\begin{gmatrix}[p]
1 & 0 & 0 & -7 & 8 \\ 
0 & 1 & 0 & 13 & -11  \\
0 & 0 & 1 & -5 & 4 
\end{gmatrix},
\end{align*}
wodurch wir die Gleichungen
\begin{align*}
x_1 -7x_4 +8x_5 = 0 
\ &\Leftrightarrow \
x_1 = 7x_4 - 8 x_5\\
x_2 + 13 x_4 -11 x_5
\ &\Leftrightarrow \
x_2 = -13 x_4 + 11 x_5\\
x_3 -5x_4 + 4 x_5
\ &\Leftrightarrow \
x_3 = 5x_4 -4 x_5 
\end{align*}
extrahieren können.
Die $ x_1 $, $ x_2 $ und $ x_3 $ Komponente hängen also von $ x_4 $ und $ x_5 $ ab.
Dies bedeutet, dass sich die allgemeine Lösung durch
\begin{align*}
\begin{pmatrix}
x_1 \\ x_2 \\ x_3 \\ x_4 \\ x_5
\end{pmatrix}
=
\begin{pmatrix}
7 x_4 - 8 x_5\\
-13 x_4 + 11 x_5\\
5x_4 - 4 x_5\\
x_4 \\
x_5
\end{pmatrix}
=
x_4 \cdot
\begin{pmatrix}
7 \\ -13 \\ 5 \\1 \\ 0
\end{pmatrix}
+
x_5
\begin{pmatrix}
-8 \\ 11 \\ -  4 \\ 0 \\ 1
\end{pmatrix}
\end{align*}
darstellen lässt. Damit gilt für die Lösungsmenge
\begin{align*}
W = 
\left\lbrace 
\textbf{x} \in \mathbb{R} \ | \
\textbf{x} = 
t \cdot
\begin{pmatrix}
7 \\ -13 \\ 5 \\1 \\ 0
\end{pmatrix}
+
s \cdot
\begin{pmatrix}
-8 \\ 11 \\ -  4 \\ 0 \\ 1
\end{pmatrix},
\ s,t \in \mathbb{R}
\right\rbrace.
\end{align*}
Die Lösungsmenge $ W $ ist gleichzeitig der gesuchte Unterraum und die Basis ist durch
\begin{align*}
\left\lbrace
\begin{pmatrix}
7 \\ -13 \\ 5 \\1 \\ 0
\end{pmatrix},
\begin{pmatrix}
-8 \\ 11 \\ -  4 \\ 0 \\ 1
\end{pmatrix}
\right\rbrace
\end{align*}
gegeben.


\newpage