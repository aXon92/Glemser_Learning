\documentclass[a4paper,11pt,german]{report}

\usepackage[ngerman]{babel}   % deutsche Sprachanpassung
\usepackage[utf8]{inputenc} % erlaubt Umlaute in der tex-Datei
\usepackage[T1]{fontenc}      % Trennung bei Wörtern mit Umlauten
\usepackage[DIV15]{typearea}  % Vergrößerung des Textbereichs
\usepackage{amsmath}
\usepackage{amsfonts}
\usepackage{amssymb}
\usepackage{amsthm}
\usepackage{units}	
\usepackage{esvect}
\usepackage{ wasysym }
\usepackage{wrapfig}
\usepackage{graphicx} %graphiken einfügen
\usepackage{floatflt} % graphik als fließtext
\usepackage[framemethod=tikz]{mdframed} %Rahmen für Boxen
\usepackage{ulem} %Unterschtreichen 
\usepackage{enumitem}
\usepackage{float}
\usepackage{gauss}
\usepackage{tikz}
\usepackage{pgfplots}
\usepackage{multirow}
\usepgfplotslibrary{fillbetween}

%\usepackage{tabular}
%%%%%%%%%%%%%%%%%%%% KOPF-ZEILE %%%%%%%%%%%%%%%%%%%%		

\makeatletter
\renewcommand*\env@matrix[1][*\c@MaxMatrixCols c]{%
  \hskip -\arraycolsep
  \let\@ifnextchar\new@ifnextchar
  \array{#1}}
\makeatother

\usepackage{etoolbox}
\makeatletter
\patchcmd\g@matrix
 {\vbox\bgroup}
 {\vbox\bgroup\normalbaselines}% restore the standard baselineskip
 {}{}
\makeatother


\newcommand{\BAR}{%
  \hspace{-\arraycolsep}%
  \strut\vrule % the `\vrule` is as high and deep as a strut
  \hspace{-\arraycolsep}%
}

%Kopf- und Fußzeile
\usepackage{fancyhdr}
\pagestyle{fancy}
%Kopfzeile links bzw. innen
\fancyhead[L]{\normalsize\textbf{Testklausur 2021}}
%Kopfzeile mittig
\fancyhead[C]{\normalsize\textbf{Teil I: Offene Aufgaben}}
%Kopfzeile rechts bzw. außen
\fancyhead[R]{\normalsize\textbf{Glemser Learning}}
%Linie oben
\renewcommand{\headrulewidth}{0.5pt}
\cfoot{Testklausur 2021 - \thepage}
%%%%%%%%%%%%%%%%%%%%%%%%%%%%%%%%%%%%%%%%%%%%%%%%%%%%%%%
%%%% KEIN EINRÜCKEN %%%%%%%%%%%%%%%%%%%%
\setlength{\parindent}{0cm}	
%%%%%%%%%%%%%%%%%%%%%%%%%%%%%%%%%%%%%%				




%\cfoot{Testklausur 2019 - Seite \thepage}
\setcounter{page}{1}



\newcommand{\td}[1]{\text{\rmfamily d} #1}
\newcommand{\aufgabe}[2]{(#1) (#2 Punkte)}
\newcommand{\frage}[2]{Frage #1 (#2 Punkte)}			
\allowdisplaybreaks	
%% \begin{document}	- Textanfang!	
\begin{document}


%\newcommand{\ein}[2]{(#1) (#2 Punkte)}


\begin{Large}
\textbf{Teil I: Offene Aufgaben (50 Punkte)}
\end{Large}
\\
\\
\\
\textbf{Allgemeine Anweisungen für offene Fragen:}
\\
\renewcommand{\labelenumi}{(\roman{enumi})}
\begin{enumerate}
\item
Ihre Antworten müssen alle Rechenschritte enthalten,
diese müssen klar ersichtlich sein.
Verwendung korrekter mathematischer Notation wird erwartet
und fliesst in die Bewertung ein.

\item
Ihre Antworten zu den jeweiligen Teilaufgaben müssen in den dafür vorgesehenen Platz geschrie-
ben werden. Sollte dieser Platz nicht ausreichen, setzen Sie Ihre Antwort auf der Rückseite oder
dem separat zur Verfügung gestellten Papier fort. Verweisen Sie in solchen Fällen ausdrücklich
auf Ihre Fortsetzung. Bitte schreiben Sie zudem Ihren Vor- und Nachnamen auf jeden separaten
Lösungsbogen.

\item
Es werden nur Antworten im dafür vorgesehenen Platz bewertet. Antworten auf der Rückseite
oder separatem Papier werden nur bei einem vorhandenen und klaren Verweis darauf bewertet.

\item
Die Teilaufgaben werden mit den jeweils oben auf der Seite angegebenen Punkten bewertet.

\item
Ihre endgültige Lösung jeder Teilaufgabe darf nur eine einzige Version enthalten.

\item
Zwischenrechnungen und Notizen müssen auf einem getrennten Blatt gemacht werden. Diese
Blätter müssen, deutlich als Entwurf gekennzeichnet, ebenfalls abgegeben werden.
\end{enumerate}

\newpage
\section*{\hfil Aufgaben \hfil}
\vspace{1cm}
\section*{Aufgabe 1 (25 Punkte)}
\vspace{0.4cm}
\subsection*{\aufgabe{a}{7}}
Sei $ f \ : \ \mathbb{R} \times \mathbb{R} \to \mathbb{R} $ eine Funktion zweier Variablen definiert durch
\begin{align*}
f(x,y) = (x+y+a) e^x - e^y, \quad \textrm{wobei} \ a \in \mathbb{R}.
\end{align*}
Untersuchen Sie die Funktion $ f $ auf stationäre Punkte und bestimmen Sie gegebenenfalls, ob ein Maximum, ein Minimum oder ein ein Sattelpunkt vorliegt.
\\
\\
\subsection*{\aufgabe{b}{7}}
Die Funktion
\begin{align*}
f(x,y) = x^2 + y^2
\end{align*}
ist unter der Nebenbedingung
\begin{align*}
\varphi(x,y)
=
a x^2 +bxy+5y^2-16 = 0
\end{align*}
zu optimieren.\\
Bestimmen Sie die Parameter $ a \in \mathbb{R} $ und $ b \in \mathbb{R} $ so, dass in $ (1,1) $ eine mögliche Extremstelle sein könnte.\\
\\
\textbf{Bemerkung:} \\
Eine Abklärung, ob es sich um eine Extremstelle handelt und von welcher Art die Extremstelle ist (Maximum oder Minimum) wird nicht verlangt.
\\
\\
\subsection*{\aufgabe{c}{5}}
Berechnen Sie 
\begin{align*}
\int
\limits_0^{\sqrt{0.5 \ \pi}} x \ \sin(x^2) \ \left(\cos(x^2)\right)^3 dx.
\end{align*}
\ \\
\subsection*{\aufgabe{d}{6}}
Berechnen Sie 
\begin{align*}
\int_0^e |\ln(x)| dx.
\end{align*}
\ \\
\textbf{Bemerkung:} Sie dürfen das in Mathematik A bewiesene Resultat, dass gilt:
\begin{align*}
\lim \limits_{x \searrow 0} x \ \ln(x) = 0,
\end{align*}
voraussetzen.

\newpage
\section*{Aufgabe 2 (25 Punkte)}
\vspace{0.4cm}
\subsection*{\aufgabe{a}{4}}
Die quadratischen $ n \times n $ Matrizen $ A $ und $ B $ seien regulär, $ A $ sei ausserdem symmetrisch.\\
\\
Beweisen Sie:
\begin{align*}
B^\top (AB)^\top (B^{-1} A^{-1})^\top B (A B)^{-1}
=
(A^{-1} B )^\top.
\end{align*}
\\
\\
\subsection*{\aufgabe{b}{4}}
Gegeben ist die Funktion
\begin{align*}
f(x,y) = a \ln(x-2) + x \ y^2 + 8 \ y, \quad \textrm{wobei} \ a \in \mathbb{R}.
\end{align*}
Berechnen Sie den Gradienten von $ f $ an der Stelle $ (x_0,y_0) = (8,2) $.\\
\\
Wie muss der Parameter $ a \in \mathbb{R} $ gewählt werden, damit die Funktion $ f $ im Punkt $ (x_0,y_0) = (8,2) $ in der Richtung
$ \textbf{b} = \begin{pmatrix}
3 \\
4
\end{pmatrix} $ am stärksten zunimmt?\\
\\
\subsection*{\aufgabe{c}{3}}
Gegeben sind die Vektoren
\begin{align*}
\textbf{b}_1 =
\begin{pmatrix}
1 \\
t\\ 
0
\end{pmatrix},
\
\textbf{b}_2 =
\begin{pmatrix}
2t \\
4\\ 
t
\end{pmatrix},
\
\textbf{b}_3 =
\begin{pmatrix}
8 \\
t\\ 
t^2
\end{pmatrix}.
\end{align*}
Für welche Werte von $ t \in \mathbb{R} $ ist das Vektorsystem $ \{ \textbf{b}_1, \textbf{b}_2, \textbf{b}_3 \} $ \textit{keine} Basis des dreidimensionalen Raumes $ \mathbb{R}^3 $?
\\
\\
\subsection*{\aufgabe{d}{6}}
Gegeben ist die Matrix
\begin{align*}
M = 
\begin{pmatrix}
0  & 2a\\
-3a & 5a
\end{pmatrix}
,
\quad
\textrm{wobei } a \neq 0.
\end{align*}
Berechnen Sie die Eigenwerte und Eigenvektoren der Matrix $ M $.\\
\\
Berechnen Sie den Vektor $ M^n \textbf{x} $, wobei $ \textbf{x} = \begin{pmatrix}
2 \\ 2
\end{pmatrix}. $
\\
\\
\subsection*{\aufgabe{e}{8}}
Gegeben ist das Gleichungssystem
\begin{equation*}
\begin{split}
x_1 \ + \  x_2 \ + \  x_3 \ + \  x_4 \ + \ x_5 \ &= \ \ 0 \\
2x_1 \ + \ 3 x_2 \ + \ 4 x_3 \ + \ 5 x_4 \ - \ x_5 \ &= \ \ 0 \\
 x_1 \ + \ 2 x_2 \ + \ 4 x_3 \ - \  x_4 \ + \ 2x_5 \ &= \ 0
\end{split}
.
\end{equation*}
Berechnen Sie die allgemeine Lösung dieses Gleichungssystems mit dem Gauß-Verfahren.\\
\\
Für den Unterraum
\begin{align*}
W 
=
\{
\textbf{x} \in \mathbb{R}^5 | x_1 + x_2 + x_3 + x_4 + x_5 = 0 \wedge
2 x_1 + 3 x_2 +4 x_3 + 5 x_4 - x_5 = 0
\wedge 
x_1 + 2x_2 +4 x_3 - x_4 + 2x_5 = 0
\}
\end{align*}
ist eine Basis anzugeben.
\newpage


\fancyhead[C]{\normalsize\textbf{$\qquad$ Teil II: Multiple-Choice}}
\begin{Large}
\textbf{Teil II: Multiple-Choice-Fragen (50 Punkte)}
\end{Large}
\\
\\
\\
\textbf{Allgemeine Anweisungen für Multiple-Choice-Fragen:}
\\
\renewcommand{\labelenumi}{(\roman{enumi})}
\begin{enumerate}
\item
Die Antworten auf die Multiple-Choice-Fragen müssen im dafür vorgesehenen Antwortbogen ein-
getragen werden. Es werden ausschliesslich Antworten auf diesem Antwortbogen bewertet. Der
Platz unter den Fragen ist nur für Notizen vorgesehen und wird nicht korrigiert.

\item
Jede Frage hat nur eine richtige Antwort. Es muss also auch jeweils nur eine Antwort angekreuzt
werden.

\item
Falls mehrere Antworten angekreuzt sind, wird die Antwort mit 0 Punkten bewertet, auch wenn
die korrekte Antwort unter den angekreuzten ist.

\item
Bitte lesen Sie die Fragen sorgfältig.

\end{enumerate}
\newpage
\section*{Aufgabe 3 (25 Punkte)}
\vspace{0.4cm}
\subsection*{\frage{1}{3}}
Die Funktion $ f(x,y) = y  $ hat unter der Nebenbedingung $ \varphi(x,y) = \frac{x^2}{25} + \frac{y^2}{9} = 1 $
ihr Minimum in
\renewcommand{\labelenumi}{(\alph{enumi})}
\begin{enumerate}
\item $ P = (-5,0) $.
\item $ P = (0,3) $.
\item $ P = (0,0) $.
\item $ P = (0,-3) $.
\end{enumerate}
\ \\
\subsection*{\frage{2}{4}}
Gegeben ist die Funktion
\begin{align*}
f(x) 
=
\begin{cases}
ax + \frac{1}{8}, &\quad \textrm{für } 0 \leq x \leq 4\\
\quad 0					&\quad  \textrm{sonst}			
\end{cases}
\end{align*}
\renewcommand{\labelenumi}{(\alph{enumi})}
\begin{enumerate}
\item $f $ ist für alle $ a \in \mathbb{R} $ eine Dichtefunktion.
\item $f $ ist nur für $ a = \frac{1}{16} $ eine Dichtefunktion.
\item
$f $ ist nur für $ a = -\frac{1}{16} $ eine Dichtefunktion.
\item
$f $ ist für kein $ a \in \mathbb{R} $ eine Dichtefunktion.
\end{enumerate}
\ \\
\subsection*{\frage{3}{2}}
Sei $ f : [a,b] \to \mathbb{R} $ eine beliebige, auf dem Intervall $ [a,b] $ definierte Funktion.\\
\\
Welche der folgenden Aussagen ist \textit{richtig}?
\renewcommand{\labelenumi}{(\alph{enumi})}
\begin{enumerate}
\item 
Wenn das bestimmte Integral von $ f $ über $ [a,b] $ existiert, dann ist $ f $ stetig auf $ [a,b] $.
\item 
Wenn $ f $ nicht stetig ist auf $ [a,b] $, dann existiert das bestimmte Integral von $ f $ über $ [a,b] $ nicht.
\item 
Wenn $ f $ differenzierbar ist auf $ [a,b] $, dann existiert das bestimmte Integral von $ f $ über $ [a,b] $.
\item
Das bestimmte Integral von $ f $ über $ [a,b]$ existiert immer.
\end{enumerate}
\ \\
\subsection*{\frage{4}{2}}
$A$ und $B$ seien quadratische Matrizen mit
$\det(A) = 5$ und $\det(B) = 2$; die Matrix $ C $ ist definiert durch 
$C \ = \ A^{-1}B A $. 
\renewcommand{\labelenumi}{(\alph{enumi})}
\begin{enumerate}
	\item 
	Dann gilt für jedes $ n \in \mathbb{N}: $ $ \det(C^n)  = 1$.
	\item
	Dann gilt für jedes $ n \in \mathbb{N}: $ $ \det(C^n)  = 2^n$.
	\item
	Dann gilt für jedes $ n \in \mathbb{N}: $ $ \det(C^n)  = 2^n \cdot 5^n$.
	\item
	Keine der obigen Aussagen ist korrekt.
\end{enumerate}
\ \\
\subsection*{\frage{5}{4}}
Gegeben sind die Vektoren
\begin{align*}
\textbf{a}
= 
\begin{pmatrix}
1 \\ 2 \\ 3
\end{pmatrix},
\textbf{b}
=
\begin{pmatrix}
1 \\ -1 \\ 1
\end{pmatrix},
\textbf{c}
=
\begin{pmatrix}
0 \\ 0 \\ 2
\end{pmatrix},
\textbf{d}
=
\begin{pmatrix}
2 \\ 1 \\ t
\end{pmatrix}.
\end{align*}
\renewcommand{\labelenumi}{(\alph{enumi})}
\begin{enumerate}
\item 
Es ist nur für $ t = 6 $ möglich, $ \textbf{d} $ als Linearkombination von $ \textbf{a} $, $ \textbf{b} $ und $ \textbf{c} $ zu schreiben.
\item
Es ist nur für $ t = 6 $ und $ t = 0 $ möglich, $ \textbf{d} $ als Linearkombination von $ \textbf{a} $, $ \textbf{b} $ und $ \textbf{c} $ zu schreiben.
\item
Es ist für alle $ t\in \mathbb{R} $ möglich, $ \textbf{d} $ als Linearkombination von $ \textbf{a} $, $ \textbf{b} $ und $ \textbf{c} $ zu schreiben.
\item
Es ist für kein $ t\in \mathbb{R} $ möglich, $ \textbf{d} $ als Linearkombination von $ \textbf{a} $, $ \textbf{b} $ und $ \textbf{c} $ zu schreiben.
\end{enumerate}
\ \\
\subsection*{\frage{6}{2}}
$ A $ ist eine $ 6 \times 5 $ Matrix, das lineare Gleichungssystem $ A \textbf{x} = \textbf{b} $ hat unendlich viele Lösungen und der Lösungsraum hat die Dimension $ 2 $.
Dann gilt:
\renewcommand{\labelenumi}{(\alph{enumi})}
\begin{enumerate}
\item 
$ \text{rg}(A) = \text{rg}(A; \textbf{b}) = 3 $.
\item
$ \text{rg}(A) = \text{rg}(A; \textbf{b}) = 4 $.
\item
$ \text{rg}(A) < \text{rg}(A; \textbf{b}) = 3 $..
\item
Keine der obigen Aussagen ist korrekt.
\end{enumerate}
\ \\
\subsection*{\frage{7}{4}}
Das unbestimmte Integral von
\begin{align*}
\int \ln(x \ e^x ) \ dx, \ (x > 0)
\end{align*}
ist
\renewcommand{\labelenumi}{(\alph{enumi})}
\begin{enumerate}
\item 
$ x \ \ln(x) + x^2 - x + C $.
\item
$ x \ \ln(x) + \frac{x^2}{2} - x + C $.
\item
$ x \ \ln(x) + x^2  + C $.
\item
Keine der obigen Antworten ist korrekt.
\end{enumerate}
\ \\
\subsection*{\frage{8}{4}}
Gegeben ist die Matrix
\begin{align*}
A = 
\begin{pmatrix}
2 & a\\
a & 2
\end{pmatrix},
\ \textrm{wobei } a\neq 0.
\end{align*}
\renewcommand{\labelenumi}{(\alph{enumi})}
\begin{enumerate}
	\item 
	Die Matrix hat für alle $ a \neq 0 $ in $ \mathbb{R} $ zwei verschiedene reelle Eigenwerte.
	\item
	Die Matrix hat für alle $ a \neq 0 $ in $ \mathbb{R} $ genau einen reellen Eigenwert.
	
	\item
	Die Matrix hat für alle $ a \neq 0 $ in $ \mathbb{R} $ keinen reellen Eigenwert..
	\item
	Die Matrix $ A $ hat abhängig von $ a \neq 0 $ keinen, einen oder zwei reelle Eigenwerte.
\end{enumerate}


\newpage
\section*{Aufgabe 4 (25 Punkte)}
\vspace{0.4cm}

\subsection*{\frage{1}{3}}
Das bestimmte Integral
\begin{align*}
\int_0^\pi 2 \ \sin(x) \ \cos(x) \ dx
\end{align*}
hat den Wert
\renewcommand{\labelenumi}{(\alph{enumi})}
\begin{enumerate}
\item 
$0$.
\item
$1$.
\item
$2$.
\item
Keines der obigen Resultate ist korrekt.
\end{enumerate}
\ \\
\subsection*{\frage{2}{3}}
Für welchen Wert von $ t \in \mathbb{R} $ sind die Vektoren $ \textbf{u} = \begin{pmatrix}
t-2 \\ t \\ 3
\end{pmatrix} $ und $ \textbf{v} = \begin{pmatrix}
1 \\ t-1 \\ 9
\end{pmatrix} $ orthogonal?
\renewcommand{\labelenumi}{(\alph{enumi})}
\begin{enumerate}
	\item 
	$t = 5$.
	\item
	$t = 5$ oder $ t = -5 $.
	\item
	$ \textbf{u} $ und $ \textbf{v}  $ sind für kein $ t \in \mathbb{R} $ orthogonal.
	\item
	$ \textbf{u} $ und $ \textbf{v}  $ sind für alle $ t \in \mathbb{R} $ orthogonal.
\end{enumerate}
\ \\
\subsection*{\frage{3}{4}}
Die $4 \times 5$ Matrix
\begin{align*}
A
=
\begin{pmatrix}
1 & 1  & 3 & -1 & -2 \\
3 & -5  & -7 & 13 & -10 \\
-1 & 3  & 5 & -7 & 4 \\
-2 & 10  & 18 & -22 & 10  
\end{pmatrix}
\end{align*}
\renewcommand{\labelenumi}{(\alph{enumi})}
\begin{enumerate}
\item 
hat Rang $2$.
\item
hat Rang $3$.
\item
hat Rang $4$.
\item
hat Rang $5$.
\end{enumerate}
\ \\
\subsection*{\frage{4}{4}}
Gesucht ist eine Matrix $ X $, sodass
\begin{align*}
X
\begin{pmatrix}
1 & 2 \\
0 & 1 
\end{pmatrix}
=
\begin{pmatrix}
4 & 3 \\
2 & 1
\end{pmatrix}
.
\end{align*}
\renewcommand{\labelenumi}{(\alph{enumi})}
\begin{enumerate}
\item 
$X
= 
\begin{pmatrix}
1 & 3  \\
-2 & -1 
\end{pmatrix}$.
\item
$X
= 
\begin{pmatrix}
4 &-5  \\
2 & -3 
\end{pmatrix}$.
\item
$X
= 
\begin{pmatrix}
4 &-3  \\
-2 & 4 
\end{pmatrix}$.
\item
Es gibt keine Matrix $ X $, die die Gleichung erfüllt.
\end{enumerate}

\newpage
\subsection*{\frage{5}{2}}
Die $ n \times n $ Matrix habe die Eigenwerte $ \lambda_1, \lambda_2, \dots, \lambda_n $.
Dann hat die Matrix $ A^2 $
\renewcommand{\labelenumi}{(\alph{enumi})}
\begin{enumerate}
\item 
die gleichen Eigenwerte.
\item
die Eigenwerte $2 \lambda_1,2 \lambda_2, \dots,2 \lambda_n $.
\item
die Eigenwerte $ \lambda_1^2, \lambda_2^2, \dots, \lambda_n^2 $.
\item
Keine der vorangehenden Antworten ist richtig.
\end{enumerate}
\ \\
\subsection*{\frage{6}{3}}
Das Anfangswertproblem
\begin{align*}
&y_{k+1} -(1+a) y_k = a, \ \textrm{wobei } a \neq -1, a \neq 0,\\
&y_0 = 2
\end{align*}
hat die Lösung
\renewcommand{\labelenumi}{(\alph{enumi})}
\begin{enumerate}
\item 
$ y_k = 2 ( 1+a)^k $.
\item
$ y_k = 3 ( 1+a)^k  -1$.
\item
$ y_k = 4 ( 1+a)^k -1 $.
\item
$ y_k = 5 ( 1+a)^k -2$.
\end{enumerate}
\ \\
\subsection*{\frage{7}{2}}
Die allgemeine Lösung der linearen Differenzengleichung
\begin{align*}
3 (y_k - y_{k+1})+ 3 = 2 y_k - 12 
\end{align*}
ist
\renewcommand{\labelenumi}{(\alph{enumi})}
\begin{enumerate}
\item
oszillierend und konvergent.
\item
oszillierend und divergent.	
\item 
monoton und konvergent.
\item
monoton und divergent.
\end{enumerate}
\ \\
\subsection*{\frage{8}{4}}
Die allgemeine Lösung der Differenzengleichung
\begin{align*}
(2 + c) y_{k+1} + (1-c) y_k = 5
\end{align*}
wobei $c \in \mathbb{R} \setminus \lbrace -2 \rbrace$ ist, ist genau dann monoton und divergent, wenn
\renewcommand{\labelenumi}{(\alph{enumi})}
\begin{enumerate}
	\item 
	$ c < -2 $.
	\item
	$c \in (-2,0)$.
	\item
	$ c < -\frac{1}{2} $.
	\item
	Die allgemeine Lösung der obigen Differenzengleichung ist für kein $c \in \mathbb{R} \setminus \lbrace -2 \rbrace$ monoton und divergent.
\end{enumerate}
\newpage
\section*{\hfil Lösungen \hfil}
\vspace{1cm}
\fancyhead[C]{\normalsize\textbf{$\qquad$ Teil I: Offene Aufgaben}}
\renewcommand{\labelenumi}{\theenumi.}
\section*{Aufgabe 1 (40 Punkte)}
\vspace{0.4cm}
\subsection*{\aufgabe{a}{12}}
Estis lang ersehnter Traum ist endlich wahr geworden
- 
sie hat sich für den Ironman Hawaii qualifiziert.
Allerdings weiss sie, dass sie ihr Training optimieren muss, um ihre bestmögliche Leistung erbringen zu können.
Esti beschliesst, als Steuerungsparameter ihren $VO_2max$-Wert zu nutzen.
Dieser misst die maximale Menge an Sauerstoff, die ihr Körper während der Belastung nutzen kann, und stellt somit eine hilfreiche Grösse dar, um die Verbesserung ihres Fitnesslevels abzubilden.
Sie möchte die optimale Verteilung ihrer Trainingszeit zwischen
Schwimmen, Radfahren und Laufen finden, um ihren $VO_2max$-Wert zu maximieren.\\
Estis derzeitiger $VO_2max$-Wert beträgt $48$ und ihre Trainingszeit ist auf $15$ Stunden pro Woche begrenzt.
Sei $x$ die Anzahl an Stunden pro Woche, die sie für Schwimmen und Radfahren zusammen aufwendet und $y$ die Anzahl an Stunden pro Woche, die sie mit Laufen verbringt.
Esti folgt der Empfehlung ihres Trainers und trainiert pro Woche viermal so viel Radfahren wie Schwimmen. Ausserdem nimmt sie an, dass sich der Trainingseffekt bezüglich ihres, am Tag des Ironman-Events gemessenen, $VO_2max$-Werts wie folgt modellieren lässt:
\begin{align*}
	f(x,y) = 1.8 x^{\frac{2}{3}} y^{\frac{1}{3}}.
\end{align*}
\begin{enumerate}
	\item[\textbf{(a1)}]
	Formulieren Sie das Optimierungsproblem, das Esti lösen muss, um ihren $VO_2max$-Wert am Tag des Events zu optimieren.
	\item[\textbf{(a2)}] 
	Lösen Sie das Optimierungsproblem und bestimmen Sie die optimale Anzahl an Stunden pro Woche, die Esti für Schwimmen, Radfahren und Laufen aufwenden sollte, sowie den $VO_2max$-Wert am Tag des Ironman-Events, wenn sie diesem Trainingsplan folgt.\\
	\\
	\textit{Anmerkung: Es ist nicht notwendig, die hinreichende Bedingung für einen Extremwert zu überprüfen.}
\end{enumerate}
 
\textbf{Lösung:}
\begin{mdframed}
\underline{\textbf{Vorgehensweise:}}
\renewcommand{\labelenumi}{\theenumi.}
\begin{enumerate}
\item[\textbf{(a1)}] Formuliere das Optimierungsproblem.
\item[\textbf{(a2)}] Wende die Lagrangemethode an.
\begin{enumerate}
	\item[1.] Stelle die Lagrangefunktion und die Lagrangebedingungen auf. 
	\item[2.] Löse die Lagrangebedingungen auf.
\end{enumerate}
\end{enumerate}
\end{mdframed}
Bevor wir uns in die Aufgabe stürzen, müssen wir den mathematischen Rahmen aufbauen.
Das Ziel von Esti den $VO_2max$-Wert zu maximieren deutet auf ein Optimierungsproblem hin. Bei diesen Problemen müssen wir prüfen, ob Nebenbedingungen vorliegt. Nebenbedingungen schränkten die möglichen Variablenwerte ein.
Da Esti $15$ Stunden pro Woche trainieren sind die Stunden für das Radfahren/Schwimmen $x$ und die Stunden für das Laufen $y$ durch
\begin{align*}
	x + y = 15
\end{align*} 
eingeschränkt.
\newpage

\underline{\textbf{(a1)} Formuliere das Optimierungsproblem }\\
Das Ziel von Esti ist, dass der $VO_2max$-Wert zum Ironman-Tag maximal ist.
Wegen Estis aktuellem $VO_2max$-Wert von $48$ und der Modellierung des Trainingseffekts durch $f(x,y) = 1.8 x^{\frac{2}{3}} y^{\frac{1}{3}}$ erhalten wir das folgende Optimierungsproblem:
\begin{align*}
	\max_{x,y} V(x,y)
	=
	48 + 1.8 x^{\frac{2}{3}} y^{\frac{1}{3}},
	\ \text{unter der Bedingung} \ x + y = 5 \ \Leftrightarrow \ \varphi(x,y) := x + y - 15 = 0.
\end{align*}

\underline{\textbf{(a2)} 1. Stelle die Lagrangefunktion und die Lagrangebedingungen auf}\\
Die Lagrangefunktion $L(x,y, \lambda)$ ist gegeben durch
\begin{align*}
	L(x,y, \lambda) 
	=
	V(x,y) + \lambda \varphi(x,y)
	= 
	48 + 1.8 x^{\frac{2}{3}} y^{\frac{1}{3}} + \lambda(x+y-15).
\end{align*}
Die partiellen Ableitungen sind gegeben durch:
\begin{align*}
	&\frac{\partial L }{\partial x}(x,y,\lambda) = 
	1.8 \frac{2}{3} x^{\frac{2}{3} - 1 } y^{\frac{1}{3}} + \lambda
	= 
	\frac{18}{10} \cdot \frac{2}{3} x^{-\frac{1}{3}} y^{\frac{1}{3}} + \lambda
	= 
	\frac{6}{5}  x^{-\frac{1}{3}} y^{\frac{1}{3}} + \lambda
	=
	= 
	1.2  x^{-\frac{1}{3}} y^{\frac{1}{3}} + \lambda
	\\
	&\frac{\partial L }{\partial y}(x,y,\lambda) 
	=
	1.8 \frac{1}{3} x^{\frac{2}{3}  } y^{\frac{1}{3}- 1} + \lambda
	=
	\frac{6}{10} x^{\frac{2}{3}  } y^{-\frac{2}{3}} + \lambda
	=
	0.6 x^{\frac{2}{3}  } y^{-\frac{2}{3}} + \lambda
	\\
	&\frac{\partial L }{\partial \lambda}(x,y,\lambda) = x + y -15
\end{align*}
Hieraus ergeben sich die Lagrangebedingungen:
\begin{align*}
	\mathrm{(I)}& \
	\frac{\partial L }{\partial x}(x,y,\lambda) = 0
	\ \Leftrightarrow \
	1.2  x^{-\frac{1}{3}} y^{\frac{1}{3}} + \lambda = 0
	\\
	\mathrm{(II)}& \
	\frac{\partial L }{\partial y}(x,y,\lambda) = 0
	\ \Leftrightarrow \
	0.6 x^{\frac{2}{3}  } y^{-\frac{2}{3}} + \lambda = 0
	\\
	\mathrm{(III)}& \
	\frac{\partial L }{\partial \lambda}(x,y,\lambda) = 0
	\ \Leftrightarrow \
	 x + y -15 = 0.
\end{align*}
\ \\
\underline{\textbf{(a2)} 2. Löse die Lagrangebedingungen auf}\\
Durch Gleichsetzen der Gleichungen (I) und (II) erhält man 
\begin{align*}
	1.2  x^{-\frac{1}{3}} y^{\frac{1}{3}} + \lambda
	=
	0.6 x^{\frac{2}{3}  } y^{-\frac{2}{3}} + \lambda
	\ &\Leftrightarrow \
	\frac{12}{10} x^{-\frac{1}{3}} y^{\frac{1}{3}}
	=
	\frac{6}{10} x^{\frac{2}{3}  } y^{-\frac{2}{3}} \\
	\ &\Leftrightarrow \
	\frac{12}{10} \cdot \frac{10}{6} x^{-\frac{1}{3}} y^{\frac{1}{3}}
	= 
	x^{\frac{2}{3}  } y^{-\frac{2}{3}} 
	\ \Leftrightarrow \
	2  y^{\frac{1}{3}}
	= 
	x^{\frac{2}{3} + \frac{1}{3} } y^{-\frac{2}{3}} \\
	\ &\Leftrightarrow \
	2  y
	= 
	x.
\end{align*}
Mit der Gleichung (III) folgt somit
\begin{align*}
	x + y = = 2y + y = 3y =  15
	\ \Leftrightarrow \
	y = 5
	\ \Rightarrow \ 2y = 10 = x.
\end{align*}
Damit gilt $x = 10 $ (Stunden Schwimmen und Radfahren) und $y= 5$ (Stunden Laufen).
Da Esti laut Aufgabenstellen viermal mehr Rad fährt als schwimmt, 
Aus $x = 10$ folgt, dass Esti $2 $ Stunden schwimmt und $8$ Stunden Rad fährt.
Dies gilt, da Esti laut Aufgabenstellen viermal mehr Rad fährt als schwimmt.
Der $VO_2max$-Wert zum Ironman-Tag ist für $5$ Stunden Laufen, $8$ Stunden Radfahren und $2$ Stunden Schwimmen gegeben durch:
\begin{align*}
	V(10,5)
	= 
	48 + 1.8 10^{\frac{2}{3} } 5^{\frac{1}{3}}
	=
	62.29.
\end{align*}
\newpage

\subsection*{\aufgabe{b}{14}}
Vier Werkstücke $P_1$, $P_2$, $P_3$ und $P_4$ müssen in einem gegebenen Produktionsprozess jeweils drei Maschinen $M_1$, $M_2$ und $M_3$ durchlaufen.
Die für die einzelnen Werkstücke pro Einheit benötigten Bearbeitungszeiten in Minuten sind in der folgenden Tabelle angegeben: 
\begin{table}[H]
	\centering
	%
	\begin{tabular}{c | c c c c}
		$ $  & $P_1$  &  $P_2$ &  $P_3$ & $P_4$ \\ 
		\hline
		$ M_1 $ & $ 10 $ & $ 20 $ & $ 20 $ & $15$  \\ 
		$ M_2 $ & $ 10 $ & $ 30 $ & $ 16 $ & $5$ \\
		$ M_3 $ & $ 20 $ & $ 20 $ & $ 10 $ & $25$
	\end{tabular}%
\end{table}
Jede Maschine läuft genau $8$ Stunden pro Tag.\\
\\
Der Nettogewinn in Schweizer Franken für die vier Werkstücke beträgt pro Einheit der Reihe nach
\begin{align*}
	q_1 = 5; \quad
	q_2=  10; \quad
	q_3=12; \quad 
	q_4=m.
\end{align*}
Der angestrebte Nettogewinn aus der Gesamtproduktion beträgt $250$ Schweizer Franken pro Tag.
\begin{enumerate}
	\item[\textbf{(b1)}]
	Beschreiben Sie die Beziehung zwischen Produktionszeiten, Laufzeit je Maschine pro Tag, Nettogewinn und der Anzahl der produzierten Werkstücke pro Tag in einem Gleichungssystem.
	
	\item[\textbf{(b2)}] 
	Für welche(s) $m > 0$ existiert ein realisierbarer Produktionsplan?
	\item[\textbf{(b3)}]
	Bestimmen Sie den Produktionsplan, d.h. die Anzahl produzierter Einheiten jedes Werkstücks unter Berücksichtigung der verfügbaren Kapazitäten der Maschinen und des angestrebten Nettogewinns, für $m = 5$.
\end{enumerate}
\ \\
\textbf{Lösung:}
\begin{mdframed}
\underline{\textbf{Vorgehensweise:}}
\renewcommand{\labelenumi}{\theenumi.}
\begin{enumerate}
\item[\textbf{(b1)}] 
\item[\textbf{(b2)}] 
\begin{enumerate}
	\item[1.] .
	\item[2.] 
\end{enumerate}
\end{enumerate}
\end{mdframed}


\newpage
\subsection*{\aufgabe{c}{14}}
Nach der letzten Preiserhöhung bei Strom möchten Sie in Solarpanele investieren und den erzeugten Strom verkaufen. 
Sie zahlen $10'000$ Schweizer Franken für den Kauf und die
Installation der Solarpanele.
Unter Berücksichtigung des kontinuierlich schlechter werdenden Wirkungsgrads nehmen Sie konservativ an, dass sich die Leistungsabgabe der Solarpanele, gemessen in Kilowattstunden pro Jahr, mit der Zeit wie folgt entwickelt:
\begin{align*}
	f(t) = 10'0000 e^{-0.1 t},
\end{align*}
wobei $t$ in Jahren gemessen wird.
Sie gehen davon aus, dass die Lebensdauer der Solarpanele $20$ Jahre beträgt. Nach diesen $20$ Jahren müssen Sie die Solarpanele zu einem Preis von $10'000$ Schweizer Franken entsorgen. Ausserdem nehmen Sie an, dass sich der Marktpreis in
Schweizer Franken für eine Kilowattstunde mit der Zeit wie folgt entwickelt:
\begin{align*}
	p(t)
	=
	0.3 e^{0.01t}.
\end{align*}
Derzeit sind die Zinssätze eher hoch, aber Sie sind sich sicher, dass sie nach einer Weile wieder niedriger werden. Sie gehen folglich für die nächsten $5$ Jahre von einer kontinuierlichen
Verzinsung zu einem Satz von $4 \%$ p.a. und anschliessend zu einem Satz von $2\%$ p.a. aus.

\begin{enumerate}
	\item[\textbf{(c1)}] 
	Bestimmen Sie die insgesamt erzeugte Energiemenge.
	\item[\textbf{(c2)}]
	Berechnen Sie den Nettobarwert Ihrer Investition.
\end{enumerate}
\ \\
\textbf{Lösung:}
\begin{mdframed}
\underline{\textbf{Vorgehensweise:}}
\begin{enumerate}
\item[\textbf{(c1)}] .
\item[\textbf{(c2)}]
\begin{enumerate}
	\item[1.] 
\end{enumerate}
\end{enumerate}
\end{mdframed}




\newpage


%\section*{Aufgabe 2 (24 Punkte)}
\vspace{0.4cm}
\subsection*{\aufgabe{a1}{5}}
Sei $ a_k = \ln \left( 1 + \left( \frac{1}{2}\right)^k \right) $ für $ k = 1,2,... $\\
\\
Verwenden Sie das Taylorpolynom $ P_2 $ zweiter Ordnung der Funktion
\begin{align*}
f \ : \ D_f \to \mathbb{R}, \ x \mapsto y = f(x) = \ln(1+x)
\end{align*}
im Punkt $ x_0 = 0 $, um einen Näherungswert für 
$ \sum_{k=1}^\infty a_k $ zu bestimmen.
\\
\\
\textbf{Lösung:}
\begin{mdframed}
\renewcommand{\labelenumi}{\theenumi.}
\underline{\textbf{Vorgehensweise:}}
\begin{enumerate}
\item Bestimme allgemein das Taylorpolynom zweiter Ordnung an $ P_2 $.
\item Berechne $ P_2 $.
\item Approximiere die Reihe.
\end{enumerate}
\end{mdframed}

\underline{1. Bestimme allgemein das Taylorpolynom $ P_2 $ zweiter Ordnung an}\\
Im Allgemeinen berechnen wir das $ n $-te Taylorpolynom im Entwicklungspunkt $ x_0 $ durch
\begin{align*}
P_n(x) = \sum \limits_{k=0}^n \frac{f^{k}(x_0)}{k!} (x-x_0)^k.
\end{align*}
Allgemein müssen wir in unserem Fall wegen $ x_0 = 0 $
\begin{align*}
P_2(x) 
%= f(x_0 ) + f^\prime(x_0) (x - x_0) + \frac{f^{\prime \prime}(x_0)}{2}(x - x_0)^2
= 
f(0) + f^{\prime}(0) x + \frac{f^{\prime \prime}(0)}{2} x^2
\end{align*}
berechnen.
\\
\\
\underline{2. Berechne $ P_2 $}\\
Wir bestimmen zunächst mithilfe der Kettenregel die Ableitungen.
Es gilt
\begin{align*}
f(x) &= \ln(1+x)\\
f^\prime(x) &= \frac{1}{1+ x} = (1+x)^{-1}\\
f^{\prime \prime}(x) &=-1 \cdot (1+x)^{-2} = - \frac{1}{(1+x)^2}
\end{align*}
womit 
\begin{align*}
f(0) &= 0 \\
f^\prime(0) &= 1\\
f^{\prime \prime}(0 ) &= -1
\end{align*}
folgt.
Damit erhalten wir 
\begin{align*}
P_2(x) = x - \frac{x^2}{2}.
\end{align*}

\underline{3. Approximiere die Reihe}\\
Nun ist $ P_2 $ eine Approximation unserer Funktion $ f $, daher gilt
\begin{align*}
a_k = \ln \left( 1 + \left( \frac{1}{2}\right)^k \right)
\approx P_2 \left( \left( \frac{1}{2}\right)^k\right)
= 
\left(\frac{1}{2}\right)^k - \frac{1}{2} \left(\left(\frac{1}{2}\right)^k \right)^2
=
\left(\frac{1}{2}\right)^k - \frac{1}{2} \left(\frac{1}{4}\right)^k. 
\end{align*} 
Unter Verwendung der geometrischen Reihe erhalten wir eine Approximation der Reihe:
\begin{align*}
\sum \limits_{k=1}^\infty a_k
&\approx 
\sum \limits_{k=1}^\infty
\left(\left(\frac{1}{2}\right)^k - \frac{1}{2} \left(\frac{1}{4}\right)^k\right)
= 
\sum \limits_{k=1}^\infty \left(\frac{1}{2}\right)^k
- \frac{1}{2} \sum \limits_{k=1}^\infty \left(\frac{1}{4}\right)^k
=
\frac{1}{2} \sum \limits_{k=0}^\infty \left(\frac{1}{2}\right)^k
\frac{1}{2\cdot 4} \sum \limits_{k=0}^\infty \left(\frac{1}{4}\right)^k\\
&=\frac{1}{2} \frac{1}{1 - \frac{1}{2}} - \frac{1}{8} \frac{1}{1 - \frac{1}{4}}
= \frac{1}{2} \cdot 2 - \frac{1}{8} \frac{4}{3}
= 1 - \frac{1}{6} = \frac{5}{6}.
\end{align*}      
\newpage          
                  
\subsection*{\aufgabe{a2}{4}}
Gegeben ist die Funktion
\begin{align*}
f \ : \ D_f \to \mathbb{R}, \ x \mapsto y = f(x) = \ln(1+x).
\end{align*}
$ R_2 $ bezeichne das Restglied zweiter Ordnung von $ f  $ in $ x_0 = 0 $.\\
\\
Zeigen Sie
\begin{align*}
\sum \limits_{k=1}^\infty R_2 \left( \left( \frac{1}{2}\right)^k \right) \leq \frac{1}{21}.
\end{align*}
\\
\textbf{Lösung:}
\begin{mdframed}
\underline{\textbf{Vorgehensweise:}}
\begin{enumerate}
\renewcommand{\labelenumi}{\theenumi.}
\item Gebe die allgemeine Formel für das Restglied an.
\item Forme das Restglied um.
\item Zeige die Ungleichung.
\end{enumerate}
\end{mdframed}

\underline{1. Gebe die allgemeine Formel für das Restglied an}\\
Das Restglied $ n $-ter Ordnung für $ f $ im Punkt $ x_0 $ ist durch
\begin{align*}
R_n(x) = \frac{f^{n+1}(\xi)}{(n+1)!}(x - x_0)^{(n+1)} 
\end{align*}
für $ \xi \in [x_0,x] $ gegeben. Für unseren Fall ergibt sich also
\begin{align*}
R_2(x) = \frac{f^{(3)}(\xi)}{3!} x^3
\end{align*}
für $ \xi \in [0,x] $.
\\
\\

\underline{2. Forme das Restglied um}\\
Aus der letzten Aufgabe erhalten wir
\begin{align*}
f^{\prime \prime}(x) = - \frac{1}{(1+x)^2}
\
\Rightarrow
\
f^{(3)}(x) = \frac{2}{(1+x)^3}
\end{align*}
und damit gilt
\begin{align*}
R_2(x) = \frac{f^{(3)}(\xi)}{3!} x^3 = \frac{1}{6} \frac{2}{(1+\xi)^3} x^3
=
\frac{1}{3} \frac{1}{(1+\xi)^3} x^3
\Rightarrow
R_2(x) = \frac{1}{3} \underbrace{\frac{1}{(1+\xi)^3}}_{\leq 1} x^3
\leq 
\frac{x^3}{3}
\end{align*}
für $ x \in [0,1] $.
\\

\underline{3. Zeige die Ungleichung}\\
Mit unseren Überlegungen erhalten wir durch
\begin{align*}
\sum \limits_{k=1}^\infty R_2 \left( \left( \frac{1}{2}\right)^k \right)
\leq 
\sum \limits_{k=1}^\infty \frac{1}{3} \left( \frac{1}{2}\right)^{3k}
=
\frac{1}{3} \sum \limits_{k=1}^\infty  \left( \frac{1}{8}\right)^{k}
=
\frac{1}{3 \cdot 8} 
\sum \limits_{k=0}^\infty  \left( \frac{1}{8}\right)^{k}
=
\frac{1}{24} \cdot \frac{1}{1 - \frac{1}{8}}
=
\frac{1}{24} \cdot \frac{8}{7} = \frac{1}{21}
\end{align*} 
die gesuchte Ungleichung.


\subsection*{\aufgabe{b}{4}}
Gegeben ist die Funktion 
\begin{align*}
f(x,y)
=
\frac{\ln(9 - 9 x^2 - y^2)}{(x-y) \sqrt{4 - x^2 - y^2}}.
\end{align*}

Ermitteln Sie den Definitionsbereich $ D_f $ von $ f $ und stellen Sie diesen graphisch dar.
\\
\\

\textbf{Lösung:}
\begin{mdframed}
\underline{\textbf{Vorgehensweise:}}
\renewcommand{\labelenumi}{\theenumi.}
\begin{enumerate}
\item Überlege dir, für welche Werte die Funktion definiert ist.
\item Gebe den Definitionsbereich graphisch an.
\end{enumerate}
\end{mdframed}

\underline{1. Überlege dir, für welche Werte die Funktion definiert ist}\\
Wir betrachten zunächst den Zähler $ \ln(9 - 9 x^2 - y^2) $. 
Dieser ist definiert, falls der Ausdruck im Logarithmus größer Null ist. Daher muss
\begin{align*}
9 - 9x^2 -y^2  > 0 
\ 
\Leftrightarrow \
9 > 9x^2 + y^2
\
\Leftrightarrow
\
1 > x^2 + \frac{y^2}{9} = x^2 + \frac{y^2}{3^2}
\end{align*}
gelten. Die Gleichung
\begin{align*}
x^2  + \frac{y^2}{3^2} < 1
\end{align*}
beschreibt die Fläche innerhalb einer Ellipse mit Mittelpunkt $ (0,0) $ und den Halbachsen $ a = 1 $ und $ b = 3 $.
Nun betrachten wir den Nenner $ (x-y) \sqrt{4 - x^2 - y^2} $. Zunächst muss $ x \neq y $ gelten. 
Für die Wurzel muss noch
\begin{align*}
4 -x^2 - y^2 > 0
\
\Leftrightarrow
\
x^2 + y^2 < 4 = 2^2
\end{align*}
erfüllt sein.
Beachte, dass die Wurzelfunktion definiert ist, falls der enthaltene Ausdruck größer gleich Null ist.
Da die Wurzel im Nenner steht, müssen wir den gleich Null Fall ausschließen.
Hierdurch wird eine Kreisfläche mit Mittelpunkt $ (0,0) $ und Radius $ 2 $ beschrieben.
\\
\\
\underline{2. Gebe den Definitionsbereich graphisch an}\\
Zusammengefasst erhalten wir die drei Bedingungen
\begin{align*}
x \in D_f \
\Leftrightarrow
\
\begin{cases}
x \neq y\\
x^2  + \frac{y^2}{3^2} < 1\\
x^2 + y^2 <  2^2.
\end{cases}
\end{align*}
Diese lassen sich durch die Skizze
\begin{center}
	\includegraphics[width=0.5\textwidth]{pictures/auf2_b.png}
\end{center}
veranschaulichen.

\newpage
\subsection*{\aufgabe{c}{5}}
Gegeben sei die Nutzenfunktion
\begin{align*}
u(c_1,c_2) = c_1^\alpha c_2^{1-\alpha}
\end{align*}
für $ \alpha \in (0,1) $, wobei $ c_1,c_2 $ die konsumierten Mengen der Güter 1 und 2 sind, und die Budgetrestriktion
\begin{align*}
C \ : \ p_1 c_1 + p_2 c_2 = 10
\end{align*}
für Preise $ p_1 > 0  $ und $ p_2 > 0 $.\\
\\
Für welche Werte der Parameter $ \alpha, p_1, p_2 $ berührt die Niveaulinie (Indifferenzkurve) $ u(c_1,c_2) = \sqrt{2} $ die Budgetlinie $ C $ im Konsumgüterbündel $ (c_1^\star,c_2^\star)  = (1,2)$?
\\
\\
\textbf{Lösung:}
\begin{mdframed}
\underline{\textbf{Vorgehensweise:}}
\renewcommand{\labelenumi}{\theenumi.}
\begin{enumerate}
\item Gebe die mathematische Formulierung des Problems an.
\item Wende das Theorem für implizite Funktionen an, um das Problem zu lösen.
\end{enumerate}
\end{mdframed}

\underline{1. Gebe die mathematische Formulierung des Problems an}\\
Das Konsumgüterbündel $ (c_1^\star,c_2^\star)  = (1,2)$ eingesetzt in die Budgetrestriktion ergibt die Budgetgerade
\begin{align*}
 p_1 + 2 p_2 = 10
 \
 \Leftrightarrow
 \
 p_1 = 10 - 2 p_2.
\end{align*}
Weiter muss wegen
\begin{align*}
u(1,2) = 1^\alpha 2^{1-\alpha} = \sqrt{2}
\end{align*}
$ \alpha = 0.5 $ gelten.
Die letzte Bedingung ist, dass sich die beiden Kurven 
\begin{align*}
u(c_1,c_2) = \sqrt{2 } 
 \ \Leftrightarrow \ 
f(c_1,c_2) &:= u(c_1,c_2) - \sqrt{2} = c_1^{0.5}c_2^{0.5} - \sqrt{2} =  0\\
p_1 c_1 + p_2 c_2 = 10
\ \Leftrightarrow \
\varphi(c_1,c_2) &:= p_1 c_1 + p_2 c_2 - 10 = 0
\end{align*}
in $ (c_1^\star,c_2^\star)  = (1,2)$ berühren.
\\
\\
\underline{2. Wende das Theorem für implizite Funktionen an, um das Problem zu lösen}\\ 
Das Theorem für implizite Funktionen liefert uns den Zusammenhang
\begin{align*}
- \frac{f_{c_1}(1,2)}{f_{c_2}(1,2)}
=
- \frac{\varphi_{c_1}(1,2)}{\varphi_{c_2}(1,2)}
\end{align*}
und es gelten 
\begin{align*}
f_{c_1}(c_1,c_2) = 0.5 c_1^{-0.5} c_2^{0.5}
&\Rightarrow 
f_{c_1}(1,2) = 0.5 \sqrt{2}\\
f_{c_2}(c_1,c_2) = 0.5 c_1^{0.5} c_2^{-0.5}
&\Rightarrow 
f_{c_1}(1,2) = 0.5 \frac{1}{\sqrt{2}}\\
\varphi_{c_1}(1,2) = p_1, &\qquad 
\varphi_{c_2}(1,2) = p_2.
\end{align*}
Insgesamt erhalten wir die Gleichung
\begin{align*}
- \frac{0.5 \sqrt{2}}{0.5 \frac{1}{\sqrt{2}}} = - \frac{p_1}{p_2}
\
\Leftrightarrow
\
2 p_2 = p_1
\end{align*}
und mit $ p_1 = 10 - 2 p_2 $ folgt
\begin{align*}
2 p_2 = 10 - 2 p_2 \ 
\Leftrightarrow
\
4 p_2 = 10 
\ \Leftrightarrow \
p_2 = \frac{10}{4} = \frac{5}{2} = 2.5.
\end{align*}
Wegen $ p_1 = 2 p_2 $ gilt dann auch $ p_2 = 2 \cdot 2.5 = 5 $.
Das Endresultat ist also durch
\begin{align*}
\alpha = 0.5 , \ p_1 = 5, \ p_2 =  \frac{5}{2}
\end{align*}
gegeben.

\newpage


\subsection*{\aufgabe{d}{6}}
Die Funktionen $ f $ und $ g $ sind auf $ \mathbb{R}^2_{++} $ definiert und haben den Wertebereich $ \mathbb{R}_{++} $.
Außerdem ist die Funktion $ f $ homogen vom Grad $ r $ und die Funktion $ g $ homogen vom Grad $ r -2 $.\\
Für die Funktion $ h $ gilt:
\begin{align*}
h(x,y) &= \frac{f(x,y)}{g(x,y)}\\
h_y(x,y) &= x  - \frac{3}{2} x^{0.5} y^{0.5}
\end{align*}
und
\begin{align*}
\varepsilon_{h,x}(x,y)
= 
\frac{xy - \frac{1}{2}x^{0.5} y^{1.5}}{xy - x^{0.5} y^{1.5}}
\end{align*}
Ermitteln Sie $ h(x,y) $ und vereinfachen Sie den Funktionsterm.
\\
\\
\textbf{Lösung:}
\begin{mdframed}
\underline{\textbf{Vorgehensweise:}}
\renewcommand{\labelenumi}{\theenumi.}
\begin{enumerate}
\item Überlege dir, was du aus der Homogenität von $ f $ und $ g $ schließen kannst.
\item Verwende das Resultat, um die Aufgabe zu lösen.
\end{enumerate}
\end{mdframed}

\underline{1. Überlege dir, was du aus der Homogenität von $ f $ und $ g $ schließen kannst}\\
Nach Voraussetzung ist $ f $ homogen vom Grad $ r $ und $ g $ homogen vom Grad $ r-2 $, d.h.
\begin{align*}
f(\lambda x , \lambda y) &= \lambda^r f(x,y)\\
g(\lambda x , \lambda y) &= \lambda^{r-2} g(x,y).
\end{align*}
Dies liefert uns mit
\begin{align*}
h(\lambda x , \lambda y)
= 
\frac{f(\lambda x , \lambda y)}{g(\lambda x , \lambda y)}
=
\frac{\lambda^{r} f(x,y)}{\lambda^{r-2} g(x,y)}
=
\lambda^2 \frac{f(x,y)}{g(x,y)}
=
\lambda^2 h(x,y)
\end{align*}
die Homogenität vom Grad $ 2 $ für $ h $.
Wir betrachten die partielle Elastizität
\begin{align*}
\varepsilon_{h,x}(x,y) = x \frac{h_x(x,y)}{h(x,y)}
\
\Leftrightarrow
\
\varepsilon_{h,x}(x,y) h(x,y) = x h_x(x,y)
\end{align*}
und die eulerschen Relation
\begin{align*}
x h_x(x,y) + y h_y(x,y) = 2 h(x,y).
\end{align*}
Die Kombination beider Aussagen liefert
\begin{align*}
&\quad \ \ \varepsilon_{h,x}(x,y) h(x,y) + y h_y(x,y) = 2 h(x,y)\\
&\Leftrightarrow
y h_y(x,y) = 2 h(x,y)- \varepsilon_{h,x}(x,y) h(x,y) =
h(x,y) ( 2 - \varepsilon_{h,x}(x,y)) \\
&\Leftrightarrow
h(x,y) = \frac{y h_y(x,y)}{2 - \varepsilon_{h,x}(x,y)} 
\end{align*}
wodurch wir eine Darstellung für $ h $ gefunden haben.
Durch Einsetzen von $ h_y $ und $ \varepsilon_{h,x} $ erhalten wir nun die Lösung.
\\
\\
\underline{2. Verwende das Resultat, um die Aufgabe zu lösen}\\
Es gilt
\begin{align*}
h(x,y) &= \frac{y h_y(x,y)}{2 - \varepsilon_{h,x}(x,y)} 
=
\frac{y \left( x  - \frac{3}{2} x^{0.5} y^{0.5} \right)}{2 -  
	\frac{xy - \frac{1}{2}x^{0.5} y^{1.5}}{xy - x^{0.5} y^{1.5}}
	}
=
\frac{y \left( x  - \frac{3}{2} x^{0.5} y^{0.5} \right)}
{\frac{2 (xy - x^{0.5} y^{1.5})}{xy - x^{0.5} y^{1.5}} -  
	\frac{xy - \frac{1}{2}x^{0.5} y^{1.5}}{xy - x^{0.5} y^{1.5}}
}\\
&=
\frac{y \left( x  - \frac{3}{2} x^{0.5} y^{0.5} \right)}
{2 (xy - x^{0.5} y^{1.5}) -  
	xy + \frac{1}{2}x^{0.5} y^{1.5}
}
(xy - x^{0.5} y^{1.5})
=
\frac{  xy  - \frac{3}{2} x^{0.5} y^{1.5} }
{xy  -  
	 \frac{3}{2}x^{0.5} y^{1.5}
}
(xy - x^{0.5} y^{1.5})\\
&=
xy - x^{0.5} y^{1.5}, 
\end{align*}
womit 
\begin{align*}
h(x,y) = xy - x^{0.5} y^{1.5}
\end{align*}
folgt.
\newpage
\section*{Aufgabe 2 (24 Punkte)}
\vspace{0.4cm}
\subsection*{\aufgabe{a1}{5}}
Sei $ a_k = \ln \left( 1 + \left( \frac{1}{2}\right)^k \right) $ für $ k = 1,2,... $\\
\\
Verwenden Sie das Taylorpolynom $ P_2 $ zweiter Ordnung der Funktion
\begin{align*}
f \ : \ D_f \to \mathbb{R}, \ x \mapsto y = f(x) = \ln(1+x)
\end{align*}
im Punkt $ x_0 = 0 $, um einen Näherungswert für 
$ \sum_{k=1}^\infty a_k $ zu bestimmen.
\\
\\
\textbf{Lösung:}
\begin{mdframed}
\renewcommand{\labelenumi}{\theenumi.}
\underline{\textbf{Vorgehensweise:}}
\begin{enumerate}
\item Bestimme allgemein das Taylorpolynom zweiter Ordnung an $ P_2 $.
\item Berechne $ P_2 $.
\item Approximiere die Reihe.
\end{enumerate}
\end{mdframed}

\underline{1. Bestimme allgemein das Taylorpolynom $ P_2 $ zweiter Ordnung an}\\
Im Allgemeinen berechnen wir das $ n $-te Taylorpolynom im Entwicklungspunkt $ x_0 $ durch
\begin{align*}
P_n(x) = \sum \limits_{k=0}^n \frac{f^{k}(x_0)}{k!} (x-x_0)^k.
\end{align*}
Allgemein müssen wir in unserem Fall wegen $ x_0 = 0 $
\begin{align*}
P_2(x) 
%= f(x_0 ) + f^\prime(x_0) (x - x_0) + \frac{f^{\prime \prime}(x_0)}{2}(x - x_0)^2
= 
f(0) + f^{\prime}(0) x + \frac{f^{\prime \prime}(0)}{2} x^2
\end{align*}
berechnen.
\\
\\
\underline{2. Berechne $ P_2 $}\\
Wir bestimmen zunächst mithilfe der Kettenregel die Ableitungen.
Es gilt
\begin{align*}
f(x) &= \ln(1+x)\\
f^\prime(x) &= \frac{1}{1+ x} = (1+x)^{-1}\\
f^{\prime \prime}(x) &=-1 \cdot (1+x)^{-2} = - \frac{1}{(1+x)^2}
\end{align*}
womit 
\begin{align*}
f(0) &= 0 \\
f^\prime(0) &= 1\\
f^{\prime \prime}(0 ) &= -1
\end{align*}
folgt.
Damit erhalten wir 
\begin{align*}
P_2(x) = x - \frac{x^2}{2}.
\end{align*}

\underline{3. Approximiere die Reihe}\\
Nun ist $ P_2 $ eine Approximation unserer Funktion $ f $, daher gilt
\begin{align*}
a_k = \ln \left( 1 + \left( \frac{1}{2}\right)^k \right)
\approx P_2 \left( \left( \frac{1}{2}\right)^k\right)
= 
\left(\frac{1}{2}\right)^k - \frac{1}{2} \left(\left(\frac{1}{2}\right)^k \right)^2
=
\left(\frac{1}{2}\right)^k - \frac{1}{2} \left(\frac{1}{4}\right)^k. 
\end{align*} 
Unter Verwendung der geometrischen Reihe erhalten wir eine Approximation der Reihe:
\begin{align*}
\sum \limits_{k=1}^\infty a_k
&\approx 
\sum \limits_{k=1}^\infty
\left(\left(\frac{1}{2}\right)^k - \frac{1}{2} \left(\frac{1}{4}\right)^k\right)
= 
\sum \limits_{k=1}^\infty \left(\frac{1}{2}\right)^k
- \frac{1}{2} \sum \limits_{k=1}^\infty \left(\frac{1}{4}\right)^k
=
\frac{1}{2} \sum \limits_{k=0}^\infty \left(\frac{1}{2}\right)^k
\frac{1}{2\cdot 4} \sum \limits_{k=0}^\infty \left(\frac{1}{4}\right)^k\\
&=\frac{1}{2} \frac{1}{1 - \frac{1}{2}} - \frac{1}{8} \frac{1}{1 - \frac{1}{4}}
= \frac{1}{2} \cdot 2 - \frac{1}{8} \frac{4}{3}
= 1 - \frac{1}{6} = \frac{5}{6}.
\end{align*}      
\newpage          
                  
\subsection*{\aufgabe{a2}{4}}
Gegeben ist die Funktion
\begin{align*}
f \ : \ D_f \to \mathbb{R}, \ x \mapsto y = f(x) = \ln(1+x).
\end{align*}
$ R_2 $ bezeichne das Restglied zweiter Ordnung von $ f  $ in $ x_0 = 0 $.\\
\\
Zeigen Sie
\begin{align*}
\sum \limits_{k=1}^\infty R_2 \left( \left( \frac{1}{2}\right)^k \right) \leq \frac{1}{21}.
\end{align*}
\\
\textbf{Lösung:}
\begin{mdframed}
\underline{\textbf{Vorgehensweise:}}
\begin{enumerate}
\renewcommand{\labelenumi}{\theenumi.}
\item Gebe die allgemeine Formel für das Restglied an.
\item Forme das Restglied um.
\item Zeige die Ungleichung.
\end{enumerate}
\end{mdframed}

\underline{1. Gebe die allgemeine Formel für das Restglied an}\\
Das Restglied $ n $-ter Ordnung für $ f $ im Punkt $ x_0 $ ist durch
\begin{align*}
R_n(x) = \frac{f^{n+1}(\xi)}{(n+1)!}(x - x_0)^{(n+1)} 
\end{align*}
für $ \xi \in [x_0,x] $ gegeben. Für unseren Fall ergibt sich also
\begin{align*}
R_2(x) = \frac{f^{(3)}(\xi)}{3!} x^3
\end{align*}
für $ \xi \in [0,x] $.
\\
\\

\underline{2. Forme das Restglied um}\\
Aus der letzten Aufgabe erhalten wir
\begin{align*}
f^{\prime \prime}(x) = - \frac{1}{(1+x)^2}
\
\Rightarrow
\
f^{(3)}(x) = \frac{2}{(1+x)^3}
\end{align*}
und damit gilt
\begin{align*}
R_2(x) = \frac{f^{(3)}(\xi)}{3!} x^3 = \frac{1}{6} \frac{2}{(1+\xi)^3} x^3
=
\frac{1}{3} \frac{1}{(1+\xi)^3} x^3
\Rightarrow
R_2(x) = \frac{1}{3} \underbrace{\frac{1}{(1+\xi)^3}}_{\leq 1} x^3
\leq 
\frac{x^3}{3}
\end{align*}
für $ x \in [0,1] $.
\\

\underline{3. Zeige die Ungleichung}\\
Mit unseren Überlegungen erhalten wir durch
\begin{align*}
\sum \limits_{k=1}^\infty R_2 \left( \left( \frac{1}{2}\right)^k \right)
\leq 
\sum \limits_{k=1}^\infty \frac{1}{3} \left( \frac{1}{2}\right)^{3k}
=
\frac{1}{3} \sum \limits_{k=1}^\infty  \left( \frac{1}{8}\right)^{k}
=
\frac{1}{3 \cdot 8} 
\sum \limits_{k=0}^\infty  \left( \frac{1}{8}\right)^{k}
=
\frac{1}{24} \cdot \frac{1}{1 - \frac{1}{8}}
=
\frac{1}{24} \cdot \frac{8}{7} = \frac{1}{21}
\end{align*} 
die gesuchte Ungleichung.


\subsection*{\aufgabe{b}{4}}
Gegeben ist die Funktion 
\begin{align*}
f(x,y)
=
\frac{\ln(9 - 9 x^2 - y^2)}{(x-y) \sqrt{4 - x^2 - y^2}}.
\end{align*}

Ermitteln Sie den Definitionsbereich $ D_f $ von $ f $ und stellen Sie diesen graphisch dar.
\\
\\

\textbf{Lösung:}
\begin{mdframed}
\underline{\textbf{Vorgehensweise:}}
\renewcommand{\labelenumi}{\theenumi.}
\begin{enumerate}
\item Überlege dir, für welche Werte die Funktion definiert ist.
\item Gebe den Definitionsbereich graphisch an.
\end{enumerate}
\end{mdframed}

\underline{1. Überlege dir, für welche Werte die Funktion definiert ist}\\
Wir betrachten zunächst den Zähler $ \ln(9 - 9 x^2 - y^2) $. 
Dieser ist definiert, falls der Ausdruck im Logarithmus größer Null ist. Daher muss
\begin{align*}
9 - 9x^2 -y^2  > 0 
\ 
\Leftrightarrow \
9 > 9x^2 + y^2
\
\Leftrightarrow
\
1 > x^2 + \frac{y^2}{9} = x^2 + \frac{y^2}{3^2}
\end{align*}
gelten. Die Gleichung
\begin{align*}
x^2  + \frac{y^2}{3^2} < 1
\end{align*}
beschreibt die Fläche innerhalb einer Ellipse mit Mittelpunkt $ (0,0) $ und den Halbachsen $ a = 1 $ und $ b = 3 $.
Nun betrachten wir den Nenner $ (x-y) \sqrt{4 - x^2 - y^2} $. Zunächst muss $ x \neq y $ gelten. 
Für die Wurzel muss noch
\begin{align*}
4 -x^2 - y^2 > 0
\
\Leftrightarrow
\
x^2 + y^2 < 4 = 2^2
\end{align*}
erfüllt sein.
Beachte, dass die Wurzelfunktion definiert ist, falls der enthaltene Ausdruck größer gleich Null ist.
Da die Wurzel im Nenner steht, müssen wir den gleich Null Fall ausschließen.
Hierdurch wird eine Kreisfläche mit Mittelpunkt $ (0,0) $ und Radius $ 2 $ beschrieben.
\\
\\
\underline{2. Gebe den Definitionsbereich graphisch an}\\
Zusammengefasst erhalten wir die drei Bedingungen
\begin{align*}
x \in D_f \
\Leftrightarrow
\
\begin{cases}
x \neq y\\
x^2  + \frac{y^2}{3^2} < 1\\
x^2 + y^2 <  2^2.
\end{cases}
\end{align*}
Diese lassen sich durch die Skizze
\begin{center}
	\includegraphics[width=0.5\textwidth]{pictures/auf2_b.png}
\end{center}
veranschaulichen.

\newpage
\subsection*{\aufgabe{c}{5}}
Gegeben sei die Nutzenfunktion
\begin{align*}
u(c_1,c_2) = c_1^\alpha c_2^{1-\alpha}
\end{align*}
für $ \alpha \in (0,1) $, wobei $ c_1,c_2 $ die konsumierten Mengen der Güter 1 und 2 sind, und die Budgetrestriktion
\begin{align*}
C \ : \ p_1 c_1 + p_2 c_2 = 10
\end{align*}
für Preise $ p_1 > 0  $ und $ p_2 > 0 $.\\
\\
Für welche Werte der Parameter $ \alpha, p_1, p_2 $ berührt die Niveaulinie (Indifferenzkurve) $ u(c_1,c_2) = \sqrt{2} $ die Budgetlinie $ C $ im Konsumgüterbündel $ (c_1^\star,c_2^\star)  = (1,2)$?
\\
\\
\textbf{Lösung:}
\begin{mdframed}
\underline{\textbf{Vorgehensweise:}}
\renewcommand{\labelenumi}{\theenumi.}
\begin{enumerate}
\item Gebe die mathematische Formulierung des Problems an.
\item Wende das Theorem für implizite Funktionen an, um das Problem zu lösen.
\end{enumerate}
\end{mdframed}

\underline{1. Gebe die mathematische Formulierung des Problems an}\\
Das Konsumgüterbündel $ (c_1^\star,c_2^\star)  = (1,2)$ eingesetzt in die Budgetrestriktion ergibt die Budgetgerade
\begin{align*}
 p_1 + 2 p_2 = 10
 \
 \Leftrightarrow
 \
 p_1 = 10 - 2 p_2.
\end{align*}
Weiter muss wegen
\begin{align*}
u(1,2) = 1^\alpha 2^{1-\alpha} = \sqrt{2}
\end{align*}
$ \alpha = 0.5 $ gelten.
Die letzte Bedingung ist, dass sich die beiden Kurven 
\begin{align*}
u(c_1,c_2) = \sqrt{2 } 
 \ \Leftrightarrow \ 
f(c_1,c_2) &:= u(c_1,c_2) - \sqrt{2} = c_1^{0.5}c_2^{0.5} - \sqrt{2} =  0\\
p_1 c_1 + p_2 c_2 = 10
\ \Leftrightarrow \
\varphi(c_1,c_2) &:= p_1 c_1 + p_2 c_2 - 10 = 0
\end{align*}
in $ (c_1^\star,c_2^\star)  = (1,2)$ berühren.
\\
\\
\underline{2. Wende das Theorem für implizite Funktionen an, um das Problem zu lösen}\\ 
Das Theorem für implizite Funktionen liefert uns den Zusammenhang
\begin{align*}
- \frac{f_{c_1}(1,2)}{f_{c_2}(1,2)}
=
- \frac{\varphi_{c_1}(1,2)}{\varphi_{c_2}(1,2)}
\end{align*}
und es gelten 
\begin{align*}
f_{c_1}(c_1,c_2) = 0.5 c_1^{-0.5} c_2^{0.5}
&\Rightarrow 
f_{c_1}(1,2) = 0.5 \sqrt{2}\\
f_{c_2}(c_1,c_2) = 0.5 c_1^{0.5} c_2^{-0.5}
&\Rightarrow 
f_{c_1}(1,2) = 0.5 \frac{1}{\sqrt{2}}\\
\varphi_{c_1}(1,2) = p_1, &\qquad 
\varphi_{c_2}(1,2) = p_2.
\end{align*}
Insgesamt erhalten wir die Gleichung
\begin{align*}
- \frac{0.5 \sqrt{2}}{0.5 \frac{1}{\sqrt{2}}} = - \frac{p_1}{p_2}
\
\Leftrightarrow
\
2 p_2 = p_1
\end{align*}
und mit $ p_1 = 10 - 2 p_2 $ folgt
\begin{align*}
2 p_2 = 10 - 2 p_2 \ 
\Leftrightarrow
\
4 p_2 = 10 
\ \Leftrightarrow \
p_2 = \frac{10}{4} = \frac{5}{2} = 2.5.
\end{align*}
Wegen $ p_1 = 2 p_2 $ gilt dann auch $ p_2 = 2 \cdot 2.5 = 5 $.
Das Endresultat ist also durch
\begin{align*}
\alpha = 0.5 , \ p_1 = 5, \ p_2 =  \frac{5}{2}
\end{align*}
gegeben.

\newpage


\subsection*{\aufgabe{d}{6}}
Die Funktionen $ f $ und $ g $ sind auf $ \mathbb{R}^2_{++} $ definiert und haben den Wertebereich $ \mathbb{R}_{++} $.
Außerdem ist die Funktion $ f $ homogen vom Grad $ r $ und die Funktion $ g $ homogen vom Grad $ r -2 $.\\
Für die Funktion $ h $ gilt:
\begin{align*}
h(x,y) &= \frac{f(x,y)}{g(x,y)}\\
h_y(x,y) &= x  - \frac{3}{2} x^{0.5} y^{0.5}
\end{align*}
und
\begin{align*}
\varepsilon_{h,x}(x,y)
= 
\frac{xy - \frac{1}{2}x^{0.5} y^{1.5}}{xy - x^{0.5} y^{1.5}}
\end{align*}
Ermitteln Sie $ h(x,y) $ und vereinfachen Sie den Funktionsterm.
\\
\\
\textbf{Lösung:}
\begin{mdframed}
\underline{\textbf{Vorgehensweise:}}
\renewcommand{\labelenumi}{\theenumi.}
\begin{enumerate}
\item Überlege dir, was du aus der Homogenität von $ f $ und $ g $ schließen kannst.
\item Verwende das Resultat, um die Aufgabe zu lösen.
\end{enumerate}
\end{mdframed}

\underline{1. Überlege dir, was du aus der Homogenität von $ f $ und $ g $ schließen kannst}\\
Nach Voraussetzung ist $ f $ homogen vom Grad $ r $ und $ g $ homogen vom Grad $ r-2 $, d.h.
\begin{align*}
f(\lambda x , \lambda y) &= \lambda^r f(x,y)\\
g(\lambda x , \lambda y) &= \lambda^{r-2} g(x,y).
\end{align*}
Dies liefert uns mit
\begin{align*}
h(\lambda x , \lambda y)
= 
\frac{f(\lambda x , \lambda y)}{g(\lambda x , \lambda y)}
=
\frac{\lambda^{r} f(x,y)}{\lambda^{r-2} g(x,y)}
=
\lambda^2 \frac{f(x,y)}{g(x,y)}
=
\lambda^2 h(x,y)
\end{align*}
die Homogenität vom Grad $ 2 $ für $ h $.
Wir betrachten die partielle Elastizität
\begin{align*}
\varepsilon_{h,x}(x,y) = x \frac{h_x(x,y)}{h(x,y)}
\
\Leftrightarrow
\
\varepsilon_{h,x}(x,y) h(x,y) = x h_x(x,y)
\end{align*}
und die eulerschen Relation
\begin{align*}
x h_x(x,y) + y h_y(x,y) = 2 h(x,y).
\end{align*}
Die Kombination beider Aussagen liefert
\begin{align*}
&\quad \ \ \varepsilon_{h,x}(x,y) h(x,y) + y h_y(x,y) = 2 h(x,y)\\
&\Leftrightarrow
y h_y(x,y) = 2 h(x,y)- \varepsilon_{h,x}(x,y) h(x,y) =
h(x,y) ( 2 - \varepsilon_{h,x}(x,y)) \\
&\Leftrightarrow
h(x,y) = \frac{y h_y(x,y)}{2 - \varepsilon_{h,x}(x,y)} 
\end{align*}
wodurch wir eine Darstellung für $ h $ gefunden haben.
Durch Einsetzen von $ h_y $ und $ \varepsilon_{h,x} $ erhalten wir nun die Lösung.
\\
\\
\underline{2. Verwende das Resultat, um die Aufgabe zu lösen}\\
Es gilt
\begin{align*}
h(x,y) &= \frac{y h_y(x,y)}{2 - \varepsilon_{h,x}(x,y)} 
=
\frac{y \left( x  - \frac{3}{2} x^{0.5} y^{0.5} \right)}{2 -  
	\frac{xy - \frac{1}{2}x^{0.5} y^{1.5}}{xy - x^{0.5} y^{1.5}}
	}
=
\frac{y \left( x  - \frac{3}{2} x^{0.5} y^{0.5} \right)}
{\frac{2 (xy - x^{0.5} y^{1.5})}{xy - x^{0.5} y^{1.5}} -  
	\frac{xy - \frac{1}{2}x^{0.5} y^{1.5}}{xy - x^{0.5} y^{1.5}}
}\\
&=
\frac{y \left( x  - \frac{3}{2} x^{0.5} y^{0.5} \right)}
{2 (xy - x^{0.5} y^{1.5}) -  
	xy + \frac{1}{2}x^{0.5} y^{1.5}
}
(xy - x^{0.5} y^{1.5})
=
\frac{  xy  - \frac{3}{2} x^{0.5} y^{1.5} }
{xy  -  
	 \frac{3}{2}x^{0.5} y^{1.5}
}
(xy - x^{0.5} y^{1.5})\\
&=
xy - x^{0.5} y^{1.5}, 
\end{align*}
womit 
\begin{align*}
h(x,y) = xy - x^{0.5} y^{1.5}
\end{align*}
folgt.
\newpage
\fancyhead[C]{\normalsize\textbf{$\qquad$ Teil II: Multiple-Choice}}
\section*{Aufgabe 3 (25 Punkte)}
\vspace{0.4cm}
\subsection*{\frage{1}{3}}
Die Funktion $ f(x,y) = y  $ hat unter der Nebenbedingung $ \varphi(x,y) = \frac{x^2}{25} + \frac{y^2}{9} = 1 $
ihr Minimum in
\renewcommand{\labelenumi}{(\alph{enumi})}
\begin{enumerate}
	\item $ P = (-5,0) $.
	\item $ P = (0,3) $.
	\item $ P = (0,0) $.
	\item $ P = (0,-3) $.
\end{enumerate}
\ \\
\textbf{Lösung:}
\begin{mdframed}
\underline{\textbf{Vorgehensweise:}}
\renewcommand{\labelenumi}{\theenumi.}
\begin{enumerate}
\item Überprüfe, welche Antwortbedingungen die Nebenbedingung erfüllen.
\item Bestimme die korrekte Antwort im Ausschlussverfahren.
\end{enumerate}
\end{mdframed}

\underline{1. Überprüfe, welche Antwortbedingungen die Nebenbedingung erfüllen}\\
%Der Punkt $ P = (0,0) $ erfüllt die Nebenbedingung nicht.
%Damit fällt (c) weg.
Wir setzen $ P_1 = (-5,0) $, $ P_2 = (0,3)  $, $ P_3=(0,0) $ und $ P_4 = (0,-3) $.
Dann gilt:
\begin{align*}
\varphi(P_1)
&=
\frac{(-5)^2}{25} + \frac{0^2}{9} = \frac{25}{25} + 0 = 1\\
\varphi(P_2)
&=
\frac{0^2}{25} + \frac{3^2}{9} = 0 + \frac{9}{9} = 1\\
\varphi(P_3)
&=
\frac{0^2}{25} + \frac{0^2}{9} = 0 \neq 1\\
\varphi(P_4)
&=
\frac{0^2}{25} + \frac{(-3)^2}{9} = 0 + \frac{9}{9} = 1.
\end{align*}
Damit erfüllt $ P_3 $ die Nebenbedingung nicht.\\
\\
\underline{2. Bestimme die korrekte Antwort im Ausschlussverfahren}\\
Da die anderen Punkte die Nebenbedingung erfüllen, vergleichen wir nun deren Funktionswerte:
\begin{align*}
f(P_1) &= 0\\
f(P_2) &= 3\\
f(P_4) &= -3.
\end{align*}
\ \\
Da der Funktionswert an dem Punkt $ P_4 $ am kleinsten ist, befindet sich dort das Minimum.\\
\\
Damit ist die Antwort (d) korrekt.

\newpage

\subsection*{\frage{2}{4}}
Gegeben ist die Funktion
\begin{align*}
f(x) 
=
\begin{cases}
ax + \frac{1}{8}, &\quad \textrm{für } 0 \leq x \leq 4\\
\quad 0					&\quad  \textrm{sonst}			
\end{cases}
\end{align*}
\renewcommand{\labelenumi}{(\alph{enumi})}
\begin{enumerate}
	\item $f $ ist für alle $ a \in \mathbb{R} $ eine Dichtefunktion.
	\item $f $ ist nur für $ a = \frac{1}{16} $ eine Dichtefunktion.
	\item
	$f $ ist nur für $ a = -\frac{1}{16} $ eine Dichtefunktion.
	\item
	$f $ ist für kein $ a \in \mathbb{R} $ eine Dichtefunktion.
\end{enumerate}
\ \\
\textbf{Lösung:}
\begin{mdframed}
	\underline{\textbf{Vorgehensweise:}}
	\renewcommand{\labelenumi}{\theenumi.}
	\begin{enumerate}
		\item Berechne das Integral über $ f $ in Abhängigkeit von $ a $.
		\item Verwende die Definition einer Dichtefunktion.
	\end{enumerate}
\end{mdframed}
\underline{1. Berechne das Integral über $ f $ in Abhängigkeit von $ a $}\\
Für das Integral über $ f $ gilt:
\begin{align*}
\int_{-\infty}^\infty f(x) \ dx
=
\int_0^4 ax + \frac{1}{8} \ dx
= 
\left[
a \frac{x^2}{2} + \frac{1}{8} x
\right]_0^4
=
a \frac{4^2}{2} + \frac{4}{8}- \left( a \frac{0^2}{2} + \frac{0}{8}\right)
=
8 a + \frac{1}{2}.
\end{align*}
\ \\
\underline{2. Verwende die Definition einer Dichtefunktion}\\
Eine Funktion $ f \geq 0 $ heißt Dichtefunktion auf $ (-\infty, \infty) $, falls
\begin{align*}
\int_{-\infty}^\infty f(x) \ dx = 1
\end{align*}
gilt.
Damit folgt
\begin{align*}
\int_{-\infty}^\infty f(x) \ dx
=
8a + \frac{1}{2} = 1 
\ \Leftrightarrow \
8a = \frac{1}{2}
\ \Leftrightarrow \
a =  \frac{1}{16}
\end{align*}
und wegen $ a > 0 $ erhalten wir $ f(x) \geq 0  $ für alle $ x \in \mathbb{R} $.\\
Damit ist sind die Eigenschaften einer Dichtefunktion für $ a = \frac{1}{16} $ erfüllt.\\
\\
Somit ist die Antwort (b) korrekt.
\newpage

\subsection*{\frage{3}{2}}
Sei $ f : [a,b] \to \mathbb{R} $ eine beliebige, auf dem Intervall $ [a,b] $ definierte Funktion.\\
\\
Welche der folgenden Aussagen ist \textit{richtig}?
\renewcommand{\labelenumi}{(\alph{enumi})}
\begin{enumerate}
	\item 
	Wenn das bestimmte Integral von $ f $ über $ [a,b] $ existiert, dann ist $ f $ stetig auf $ [a,b] $.
	\item 
	Wenn $ f $ nicht stetig ist auf $ [a,b] $, dann existiert das bestimmte Integral von $ f $ über $ [a,b] $ nicht.
	\item 
	Wenn $ f $ differenzierbar ist auf $ [a,b] $, dann existiert das bestimmte Integral von $ f $ über $ [a,b] $.
	\item
	Das bestimmte Integral von $ f $ über $ [a,b]$ existiert immer.
\end{enumerate}
\ \\
\textbf{Lösung:}
\begin{mdframed}
\underline{\textbf{Vorgehensweise:}}
\renewcommand{\labelenumi}{\theenumi.}
\begin{enumerate}
\item Eliminiere Lösungen mithilfe des Verständnisses von Differenzierbarkeit und Integrierbarkeit.
\end{enumerate}
\end{mdframed}

\underline{1. Eliminiere Lösungen mithilfe des Verständnisses von Differenzierbarkeit und Integrierbarkeit.}\\
Folgende Grafik veranschaulicht eine Sprungfunktion.
Das Integral über diese Funktion existiert auf dem Intervall $ [0,5] $, jedoch ist diese offensichtlich nicht stetig.
Damit ist (a) und (b) falsch.
\begin{center}
	\begin{tikzpicture}
	\begin{axis}[
	domain=0:5,
	xmin=0, xmax=5,
	ymin=0, ymax=2,
	samples=5,
	axis y line=center,
	axis x line=middle,
	]
	\addplot+[jump mark left] { 0.5 * floor(x)};
	%\addplot+[mark=none] { sin(deg(2*x)};
	%\legend{$ \sin(2x) $}
	\end{axis}
	\end{tikzpicture}
\end{center}

Wir betrachten nun die Funktion
\begin{align*}
f : [a,b] \to \mathbb{R},
x \mapsto 
\begin{cases}
\quad 1  &\quad \textrm{falls} \ x \ \textrm{rational}\\
 \quad 0  &\quad \textrm{falls} \ x \ \textrm{irrational}
\end{cases}.
\end{align*}
Das bestimmte Integral über $ [a,b] $ existiert in diesem Fall nicht.
Jedoch würde die Begründung unseren Rahmen sprengen.
Damit ist (d) auch falsch und es bleibt nur (c) übrig.\\
\\
Wenn $ f $ differenzierbar auf $ [a,b] $ ist, ist $ f $ auch stetig auf $ [a,b] $.
Stetige Funktionen auf $ [a,b] $ sind immer integrierbar.\\
\\
Damit ist die Antwort (c) korrekt.

\newpage

\subsection*{\frage{4}{2}}
$A$ und $B$ seien quadratische Matrizen mit
$\det(A) = 5$ und $\det(B) = 2$; die Matrix $ C $ ist definiert durch 
$C \ = \ A^{-1}B A $. 
\renewcommand{\labelenumi}{(\alph{enumi})}
\begin{enumerate}
\item 
Dann gilt für jedes $ n \in \mathbb{N}: $ $ \det(C^n)  = 1$.
\item
Dann gilt für jedes $ n \in \mathbb{N}: $ $ \det(C^n)  = 2^n$.
\item
Dann gilt für jedes $ n \in \mathbb{N}: $ $ \det(C^n)  = 2^n \cdot 5^n$.
\item
Keine der obigen Aussagen ist korrekt.
\end{enumerate}
\ \\
\textbf{Lösung:}
\begin{mdframed}
\underline{\textbf{Vorgehensweise:}}
\renewcommand{\labelenumi}{\theenumi.}
\begin{enumerate}
\item Verwende die Rechenregeln der Determinante.
\end{enumerate}
\end{mdframed}

\underline{1. Verwende die Rechenregeln der Determinante}\\
Für quadratische Matrizen $ X $ und $ Y $ gilt
\begin{align*}
\det (X\cdot Y) = \det(X) \cdot \det(Y).
\end{align*}
Insbesondere gilt dann auch
\begin{align*}
\det(C^n) = \underbrace{\det(C) \cdots \det(C)}_{
	 n -\textrm{mal}
}
=\left(\det(C)\right)^n.
\end{align*}
Außerdem benötigen wir noch
\begin{align*}
\det (A^{-1}) = \frac{1}{\det(A)}.
\end{align*}
Wir erhalten
\begin{align*}
\det(C) = \det(A^{-1} B A)
=
\frac{1}{\det(A)} \det(B) \det(A)
=
\det(B) = 2
\
\Rightarrow 
\
\det(C^n) = 2^n.
\end{align*}
\ \\
Damit ist Antwort (b) korrekt.
\newpage

\subsection*{\frage{5}{4}}
Gegeben sind die Vektoren
\begin{align*}
\textbf{a}
= 
\begin{pmatrix}
1 \\ 2 \\ 3
\end{pmatrix},
\textbf{b}
=
\begin{pmatrix}
1 \\ -1 \\ 1
\end{pmatrix},
\textbf{c}
=
\begin{pmatrix}
0 \\ 0 \\ 2
\end{pmatrix},
\textbf{d}
=
\begin{pmatrix}
2 \\ 1 \\ t
\end{pmatrix}.
\end{align*}
\renewcommand{\labelenumi}{(\alph{enumi})}
\begin{enumerate}
	\item 
	Es ist nur für $ t = 6 $ möglich, $ \textbf{d} $ als Linearkombination von $ \textbf{a} $, $ \textbf{b} $ und $ \textbf{c} $ zu schreiben.
	\item
	Es ist nur für $ t = 6 $ und $ t = 0 $ möglich, $ \textbf{d} $ als Linearkombination von $ \textbf{a} $, $ \textbf{b} $ und $ \textbf{c} $ zu schreiben.
	\item
	Es ist für alle $ t\in \mathbb{R} $ möglich, $ \textbf{d} $ als Linearkombination von $ \textbf{a} $, $ \textbf{b} $ und $ \textbf{c} $ zu schreiben.
	\item
	Es ist für kein $ t\in \mathbb{R} $ möglich, $ \textbf{d} $ als Linearkombination von $ \textbf{a} $, $ \textbf{b} $ und $ \textbf{c} $ zu schreiben.
\end{enumerate}
\ \\
\textbf{Lösung:}
\begin{mdframed}
\underline{\textbf{Vorgehensweise:}}
\renewcommand{\labelenumi}{\theenumi.}
\begin{enumerate}
\item Überlege dir, ob die Vektoren $ \textbf{a} $, $ \textbf{b} $ und $ \textbf{c} $ linear unabhängig sind.
\end{enumerate}
\end{mdframed}

\underline{1. Überlege dir, ob die Vektoren $ \textbf{a} $, $ \textbf{b} $ und $ \textbf{c} $ linear unabhängig sind}\\
Die Vektoren $ \textbf{a} $, $ \textbf{b} $ und $ \textbf{c} $
sind linear unabhängig genau dann, wenn
$ \det( \textbf{a} ,  \textbf{b}, \textbf{c} ) \neq 0$ ist.
Wir entwickeln nach der dritten Spalte, da dort zwei Nullen auftreten.
Durch die Entwicklung nach der dritten Spalte erhalten wir:
\begin{align*}
\det( \textbf{a} ,  \textbf{b}, \textbf{c} )
=
\det
\begin{pmatrix}
1 & 1 & 0\\
2 & -1 & 0\\
3 & 1 & 2
\end{pmatrix}
= 2 \cdot \det 
\begin{pmatrix}
1 & 1\\
2 & -1
\end{pmatrix}
= 
2 \cdot (1 \cdot (-1) - 2 \cdot 1)
=
2 \cdot (-3) = 6.
\end{align*}
Damit bilden die Vektoren $ \textbf{a} $, $ \textbf{b} $ und $ \textbf{c} $ eine Basis des $ \mathbb{R}^3 $.
Insbesondere lässt sich mit diesen jeder beliebige Vektor des $ \mathbb{R}^3 $ darstellen.\\
\\
Damit ist die Antwort (c) korrekt.
\newpage

\subsection*{\frage{6}{2}}
$ A $ ist eine $ 6 \times 5 $ Matrix, das lineare Gleichungssystem $ A \textbf{x} = \textbf{b} $ hat unendlich viele Lösungen und der Lösungsraum hat die Dimension $ 2 $.
Dann gilt:
\renewcommand{\labelenumi}{(\alph{enumi})}
\begin{enumerate}
	\item 
	$ \text{rg}(A) = \text{rg}(A; \textbf{b}) = 3 $.
	\item
	$ \text{rg}(A) = \text{rg}(A; \textbf{b}) = 4 $.
	\item
	$ \text{rg}(A) < \text{rg}(A; \textbf{b}) = 3 $..
	\item
	Keine der obigen Aussagen ist korrekt.
\end{enumerate}
\ \\
\textbf{Lösung:}
\begin{mdframed}
\underline{\textbf{Vorgehensweise:}}
\renewcommand{\labelenumi}{\theenumi.}
\begin{enumerate}
\item Berechne den Rang der Matrix $ A $.
\end{enumerate}
\end{mdframed}

\underline{1. Berechne den Rang der Matrix $ A $}\\
Da das lineare Gleichungssystem $ A \textbf{x} = \textbf{b} $ lösbar ist, gilt
\begin{align*}
rg(A) = rg(Ab).
\end{align*}
Sie $ L $ der Lösungsraum des LGS $ A \textbf{x} = \textbf{b} $ und $ n = 5 $ die Anzahl der Variablen. Dann gilt
\begin{align*}
\dim \ L = n - rg(A)
\ \Leftrightarrow \
2 = 5 - rg(A)
\ \Leftrightarrow \
rg(A) = 3.
\end{align*}
\ \\
Damit ist die Antwort (a) korrekt.
\newpage
\subsection*{\frage{7}{4}}
Das unbestimmte Integral von
\begin{align*}
\int \ln(x \ e^x ) \ dx, \ (x > 0)
\end{align*}
ist
\renewcommand{\labelenumi}{(\alph{enumi})}
\begin{enumerate}
	\item 
	$ x \ \ln(x) + x^2 - x + C $.
	\item
	$ x \ \ln(x) + \frac{x^2}{2} - x + C $.
	\item
	$ x \ \ln(x) + x^2  + C $.
	\item
	Keine der obigen Antworten ist korrekt.
\end{enumerate}
\ \\
\textbf{Lösung:}
\begin{mdframed}
\underline{\textbf{Vorgehensweise:}}
\renewcommand{\labelenumi}{\theenumi.}
\begin{enumerate}
\item Vereinfache den Integranden.
\item Bestimme die Stammfunktion.
\end{enumerate}
\end{mdframed}

\underline{1. Vereinfache den Integranden und bestimme die Stammfunktion}\\
Für den Integranden gilt
\begin{align*}
f(x) := \ln(x e^x) = \ln(x) + \ln(e^x) = \ln(x) + x.
\end{align*}
Mit der Vorüberlegung erhalten wir
\begin{align*}
\int f(x) \ dx 
=
\int \ln(x) \ dx + \int x \ dx
= 
\int \ln(x) \ dx + \frac{x^2}{2} + C,
\end{align*}
wodurch wir nur noch die Stammfunktion von $ \ln(x) $ bestimmen müssen.\\
\\
\underline{2. Bestimme die Stammfunktion}\\
Aufgrund der Stammfunktion von $ x $ liegt jedoch die Vermutung nahe, dass die Antworten (a) und (c) falsch sind, da in beiden Fällen der Faktor $ \frac{1}{2} $ fehlt.
Mit partieller Integration erhalten wir
\begin{align*}
\int 1 \cdot \ln(x) \ dx
= x \ln(x) - \int x  \cdot\frac{1}{x} \ dx
= x\ln(x) - x + C.
\end{align*}
Insgesamt folgt dann
\begin{align*}
\int f(x) \ dx = 
x \ln(x) -x + \frac{x^2}{2} + C.
\end{align*}
\ \\
Damit ist die Antwort (b) korrekt.\\
\\
\textit{Alternativ lässt sich die korrekte Antwort durch Differenzieren der Möglichkeiten (a)-(c) herleiten.
}


\newpage

\subsection*{\frage{8}{4}}
Gegeben ist die Matrix
\begin{align*}
A = 
\begin{pmatrix}
2 & a\\
a & 2
\end{pmatrix},
\ \textrm{wobei } a\neq 0.
\end{align*}
\renewcommand{\labelenumi}{(\alph{enumi})}
\begin{enumerate}
\item 
Die Matrix hat für alle $ a \neq 0 $ in $ \mathbb{R} $ zwei verschiedene reelle Eigenwerte.
\item
Die Matrix hat für alle $ a \neq 0 $ in $ \mathbb{R} $ genau einen reellen Eigenwert.

\item
Die Matrix hat für alle $ a \neq 0 $ in $ \mathbb{R} $ keinen reellen Eigenwert..
\item
Die Matrix $ A $ hat abhängig von $ a \neq 0 $ keinen, einen oder zwei reelle Eigenwerte.
\end{enumerate}
\ \\
\textbf{Lösung:}
\begin{mdframed}
\underline{\textbf{Vorgehensweise:}}
\renewcommand{\labelenumi}{\theenumi.}
\begin{enumerate}
\item Bestimme die Eigenwerte der Matrix $ A $.
\end{enumerate}
\end{mdframed}

\underline{1. Bestimme die Eigenwerte der Matrix $ A $}\\
Die Eigenwerte der Matrix $ A $ sind die Lösungen der Gleichung
\begin{align*}
\det(A - \lambda I) = 0.
\end{align*}
Es gilt
\begin{align*}
\det(A - \lambda I)=
\det
\begin{pmatrix}
2- \lambda & a \\
a & 2 - \lambda
\end{pmatrix}
= (2- \lambda)^2 - a^2
= 4 - 4 \lambda + \lambda^2 - a^2.
\end{align*}
Die Nullstellen hiervon erhalten wir durch
\begin{align*}
\lambda_{\nicefrac{1}{2}}
=
\frac{4 \pm \sqrt{(-4)^2 - 4 \cdot 1 \cdot (4 - a^2)}}{2}
=
\frac{4 \pm \sqrt{4 a^2}}{2}
=
\frac{4 \pm 2 |a|}{2}
=
2 \pm |a|.
\end{align*}
Damit hat $ A $ zwei reelle Eigenwerte.
Alternativ lassen sich die Nullstellen auch wie folgt bestimmen:
\begin{align*}
(2- \lambda)^2 - a^2 = 0 
\ \Leftrightarrow \
(2- \lambda)^2 = a^2 
\ \Leftrightarrow \
2 - \lambda = \pm|a|
\ \Leftrightarrow \
\lambda = 2 \pm |a|.
\end{align*}
\\
\\
Somit ist Antwort (a) korrekt.

\newpage


\end{document} 