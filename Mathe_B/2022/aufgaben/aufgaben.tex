%\newcommand{\ein}[2]{(#1) (#2 Punkte)}


\begin{Large}
\textbf{Teil I: Offene Fragen (40 Punkte)}
\end{Large}
\\
\\
\\
\textbf{Allgemeine Anweisungen für offene Fragen:}
\\
\renewcommand{\labelenumi}{(\roman{enumi})}
\begin{enumerate}
\item
Ihre Antworten müssen alle Rechenschritte enthalten,
diese müssen klar ersichtlich sein.
Verwendung korrekter mathematischer Notation wird erwartet
und fliesst in die Bewertung ein.

\item
Ihre Antworten zu den jeweiligen Teilaufgaben müssen in den dafür vorgesehenen Platz geschrie-
ben werden. Sollte dieser Platz nicht ausreichen, setzen Sie Ihre Antwort auf der Rückseite oder
dem separat zur Verfügung gestellten Papier fort. Verweisen Sie in solchen Fällen ausdrücklich
auf Ihre Fortsetzung. Bitte schreiben Sie zudem Ihren Vor- und Nachnamen auf jeden separaten
Lösungsbogen.

\item
Es werden nur Antworten im dafür vorgesehenen Platz bewertet. Antworten auf der Rückseite
oder separatem Papier werden nur bei einem vorhandenen und klaren Verweis darauf bewertet.

\item
Die Teilaufgaben werden mit den jeweils oben auf der Seite angegebenen Punkten bewertet.

\item
Ihre endgültige Lösung jeder Teilaufgabe darf nur eine einzige Version enthalten.

\item
Zwischenrechnungen und Notizen müssen auf einem getrennten Blatt gemacht werden. Diese
Blätter müssen, deutlich als Entwurf gekennzeichnet, ebenfalls abgegeben werden.
\end{enumerate}

\newpage
\section*{\hfil Aufgaben \hfil}
\vspace{1cm}
\section*{Aufgabe 1 (40 Punkte)}
\vspace{0.4cm}
\subsection*{\aufgabe{a}{12}}
Eine Rederei muss für ihre Schiffe Container herstellen, um darin Güter um die ganze Welt verschiffen zu können. Die Container müssen alle identisch sein und sollen, mathematisch formuliert, die Form eines Quaders  mit Länge $ x $, Höhe $ y $ und Breite $ z $ besitzen. Jeder Container muss ein Fassungsvermögen von 27 Volumeneinheiten haben. 
\begin{enumerate}
	\item[\textbf{(a1)}]
	Bestimmen Sie $ z $ als Funktion von $ x $ und $ y $ mithilfe der Volumenvorgabe jedes Containers.
	\item[\textbf{(a2)}] 
	Bestimmen Sie die Länge $ x $ und Höhe $ y $ (und damit die Breite $ z $ mit der Hilfe von Aufgabenteil (a1)), sodass die Container die minimale Oberfläche besitzen (und folglich die geringsten Herstellungskosten).
\end{enumerate}
\ \\
\subsection*{\aufgabe{b}{14}}
Die Renditen der Aktien der Unternehmen BEST, SUCCESS und WINNER hängen von einem von vier möglichen Szenarien ab, wie sich die Finanzmärkte entwickeln. Die möglichen Szenarien und Renditen der jeweiligen Aktie werden in folgender Tabelle beschrieben:
\begin{table}[H]
	\centering
	%
	\begin{tabular}{c c c c}
		\hline
		Szenario & BEST  &  SUCCESS &  WINNER \\ 
		\hline
		$ s_1 $ & $ 3 $ & $ 1 $ & $ 2 $  \\ 
		$ s_2 $ & $ 2 $ & $ 2 $ & $ 2 $ \\
		$ s_3 $ & $ 3 $ & $ 3 $ & $ 1 $ \\
		$ s_4 $ & $ -1 $ & $ 1 $ & $ 0 $\\
		\hline
	\end{tabular}%
\end{table}
Eine Investorin möchte in die drei Aktien investieren. Ihr Ziel ist es folgende Rendite, abhängig vom jeweiligen Szenario, zu generieren:
\begin{table}[H]
	\centering
	%
	\begin{tabular}{c c}
		\hline
		Szenario & Rendite  \\
		\hline
		$ s_1 $ & $ 5 $ \\ 
		$ s_2 $ & $ 3 $  \\
		$ s_3 $ & $ a $  \\
		$ s_4 $ & $ -2 $ \\
		\hline
	\end{tabular}%
\end{table}
wobei $ a \in \mathbb{R} $.
\begin{enumerate}
	\item[\textbf{(b1)}]
	Bestimmen Sie das lineare Gleichungssystem, welches das Allokationsproblem der Investorin beschreibt.
\end{enumerate}
Verwenden Sie den Gauß-Algorithmus für die folgenden Aufgaben.
\begin{enumerate}
	\item[\textbf{(b2)}] 
	Für welche Werte von $ a $ existiert eine Lösung des Investitionsproblems?
	\item[\textbf{(b3)}]
	Bestimmen Sie für $ a \ = \ -3.5 $ die Investitionsstrategie der Investorin, welche die entsprechenden Renditen je nach Szenario erwirtschaftet.
\end{enumerate}

\ \\
\subsection*{\aufgabe{c}{14}}
Ein Konsument deckt seine Nachfrage nach zwei Gütern. Dabei beschreibt $ c_1 $ die Menge von Gut $ 1 $ und $ c_2 $ die Menge von Gut $ 2 $. Der Konsument besitzt die Nutzenfunktion $ u(c_1,c_2) $, welche als Kehrwert der geometrischen Distanz zwischen dem Konsumbündel $ (c_1,c_2) $ und einem angestrebten Konsumbündel $ (3,3) $ definiert ist. Der Preis für Gut $ 1 $ beträgt $ p_1 \ = \ 2 $ (CHF) und für Gut $ 2 $ $ p_2 \ = \ 3 $ (CHF). Der Konsument besitzt das Einkommen von $ e = 10 $ (CHF).
\begin{enumerate}
	\item[\textbf{(c1)}] 
	Bestimmen Sie die Nutzenfunktion des Konsumenten sowie dessen Budgetrestriktion.
	\item[\textbf{(c2)}]
	Vereinfachen Sie das Optimierungsproblem, indem Sie es geeignet transformieren und bestimmen Sie das optimale Konsumbündel für den Konsumenten.
\end{enumerate}
\ \\
\textit{Bemerkungen: Der Kehrwert eines mathematischen Ausdrucks $ x $ ist definiert als $ \frac{1}{x} $. Es ist \textbf{nicht} nötig zu zeigen, dass es sich um ein Maximum handelt.}

\newpage


\fancyhead[C]{\normalsize\textbf{$\qquad$ Teil II: Multiple-Choice}}
\begin{Large}
\textbf{Teil II: Multiple-Choice-Fragen (60 Punkte)}
\end{Large}
\\
\\
\\
\textbf{Allgemeine Anweisungen für Multiple-Choice-Fragen:}
\\
\renewcommand{\labelenumi}{(\roman{enumi})}
\begin{enumerate}
\item
Die Antworten auf die Multiple-Choice-Fragen müssen im dafür vorgesehenen Antwortbogen ein-
getragen werden. Es werden ausschliesslich Antworten auf diesem Antwortbogen bewertet. Der
Platz unter den Fragen ist nur für Notizen vorgesehen und wird nicht korrigiert.

\item
Jede Frage hat nur eine richtige Antwort. Es muss also auch jeweils nur eine Antwort angekreuzt
werden.

\item
Falls mehrere Antworten angekreuzt sind, wird die Antwort mit 0 Punkten bewertet, auch wenn
die korrekte Antwort unter den angekreuzten ist.

\item
Bitte lesen Sie die Fragen sorgfältig.

\end{enumerate}
\newpage
\section*{Aufgabe 2 (32 Punkte)}
\vspace{0.4cm}
\subsection*{\frage{1}{4}}
Die Funktion $ f(x,y) \ = \ (x-2)^2 -y^2 $ hat einen Sattelpunkt beim Punkt $ P \ = \ (2,0) $. Welche der folgenden Aussagen ist korrekt?
 \renewcommand{\labelenumi}{(\alph{enumi})}
\begin{enumerate}
\item Die Funktion $ f $ hat unter der Nebenbedingung $ 2x \ - y \ - \ 4 \ = \ 0 $ ein lokales Minimum beim Punkt $ P $.
\item Die Funktion $ f $ hat unter der Nebenbedingung $ 2x \ - y \ - \ 4 \ = \ 0 $ einen Sattelpunkt beim Punkt $ P $.
\item Die Funktion $ f $ hat unter der Nebenbedingung $ -x \ - 2y \ - \ 4 \ = \ 0 $ einen Sattelpunkt beim Punkt $ P $.
\item Die Funktion $ f $ hat unter der Nebenbedingung $ -\frac{1}{2}x \ - y \ + \ 1 \ = \ 0 $ ein lokales Minimum beim Punkt $ P $.
\item Die Funktion $ f $ hat unter der Nebenbedingung $ -\frac{1}{2}x \ - y \ + \ 1 \ = \ 0 $ ein lokales Maximum beim Punkt $ P $.
\end{enumerate}
\ \\
\subsection*{\frage{2}{3}}
Sei $ f \ : \ D_f \to \mathbb{R} $ eine positive Funktion zweier reeller Variablen, welche ein lokales Maximum beim Punkt $ P = (5,2)  $ hat.
Sei $ h  \ : \ D_f \to \mathbb{R} $ die Funktion definiert durch
\begin{align*}
	h(x,y) = e^{\sqrt{f(x,-y)}}.
\end{align*} 
Welche der folgenden Aussagen ist korrekt?
\renewcommand{\labelenumi}{(\alph{enumi})}
\begin{enumerate}
\item Die Funktion $ h $ hat ein lokales Maximum beim Punkt $ Q = (-5,2). $
\item Die Funktion $ h $ hat ein lokales Maximum beim Punkt $ Q = (5,-2). $
\item Die Funktion $ h $ hat ein lokales Minimum beim Punkt $ Q = (-5,2). $
\item Die Funktion $ h $ hat ein lokales Minimum beim Punkt $ Q = (5,-2). $
\item Wir können die Extremstellen von $ h $ nicht analysieren, ohne mehr Informationen zur Funktion $ f $ zu haben.
\end{enumerate}
\newpage
\subsection*{\frage{3}{3}}
Sei $ F $ die Stammfunktion der stetigen Funktion $ f $ einer reellen Variable $ x $ und sei $ g $ eine lineare Funktion, d.h. $ g(x) = ax +b $ für die Parameter $ a,b \in \mathbb{R} $. Das unbestimmte Integral
\begin{align*}
  \int f(x) g(x) \ dx.
\end{align*}
ist gleich:
\renewcommand{\labelenumi}{(\alph{enumi})}
\begin{enumerate}
\item 
$ a x F(x) + b F(x) - a \int F(x) dx$.
\item 
$ g(x) F(x) - a \int f(x) dx$.
\item 
$ \left(\frac{1}{2} a x^2 +b x\right) F(x) - \int a F(x) dx$.
\item
$ \left(\frac{1}{2} a x^2 +b x\right) F(x) - \int  F(x) g(x) dx$.
\item 
Wir können das Integral nicht transformieren, ohne mehr Informationen zur Funktion $ f $ zu haben.
\end{enumerate}
\ \\
\subsection*{\frage{4}{3}}
Sei $ f $ eine \textit{gerade} Funktion, welche die folgenden (Integral-) Gleichungen erfüllt:
\begin{align*}
	\int_{2}^3 f(x) \ dx = 10, \ \textrm{und} \
	\int_1^3 f(x) \ dx = 15.
\end{align*}
Das bestimmte Integral
\begin{align*}
	\int_{-2}^{-1} (2 - f(x) ) \ dx
\end{align*}  
ist gleich
\renewcommand{\labelenumi}{(\alph{enumi})}
\begin{enumerate}
	\item 
	$ -1 $.
	\item
	$ -2 $.
	\item
	$ -3 $.
	\item
	$ -4 $.
	\item
	Wir haben nicht genügend Informationen, um diesen Wert auszurechnen.
\end{enumerate}
\ \\
\subsection*{\frage{5}{4}}
Gegeben seien die Matrizen $ A_{n \times m} $ und $ B_{m \times n} $ für $ m,n \in \mathbb{N} $.\\
\\
Es folgt:
\renewcommand{\labelenumi}{(\alph{enumi})}
\begin{enumerate}
\item 
Die Matrix $ (AB)^\top $ hat die Dimensionen $ n \times m $.
\item 
Die Matrix $ (AB)^\top $ ist symmetrisch, falls $ A $ symmetrisch ist.
\item 
Die Matrix $ (AB)^\top $ ist invertierbar, falls $ \mathrm{rg}(A) = \mathrm{rg}(B) = n $.
\item 
Die Matrix $ (AB)^\top $ ist invertierbar, falls $ BA $ invertierbar ist.
\item 
Die Matrix $ (AB)^\top $ ist invertierbar, falls sie $ n $ reelle Eigenwerte besitzt, welche ungleich null sind.
\end{enumerate}
\ \\
\subsection*{\frage{6}{3}}
Gegeben sei die $ (n \times n ) $-dimensionale, reguläre Matrix $ A $ mit einem reellen Eigenwert $ \lambda \neq 0 $.\\
\\
Es folgt, dass die Matrix
\begin{align*}
	A^{-4} = A^{-1} \cdot A^{-1} \cdot A^{-1} \cdot A^{-1}
\end{align*}
den folgenden Eigenwert hat:
\renewcommand{\labelenumi}{(\alph{enumi})}
\begin{enumerate}
	\item 
	$ \frac{1}{\lambda^4} $.
	\item 
	$ \frac{4}{\lambda} $.
	\item
	$ \lambda$.
	\item
	$\lambda^3 $.
	\item 
	Es ist nicht möglich, eine Aussage bezüglich der Eigenwerte von $ A^{-4} $ zu treffen.
\end{enumerate}
\ \\
\subsection*{\frage{7}{3}}
Gegeben sei das lineare Gleichungssystem
\begin{align*}
	A \textbf{x} = \textbf{b}.
\end{align*}
\textit{Keine} notwendige Annahme, um das lineare Gleichungssystem mithilfe der Cramerschen Regel zu lösen zu können, ist:
\renewcommand{\labelenumi}{(\alph{enumi})}
\begin{enumerate}
\item 
$ A $ ist quadratisch.
\item
$ A $ hat vollen Rang.
\item
$ \det(A) \neq 0$.
\item
$ \textbf{b} \neq \textbf{0} $.
\item
$ \mathrm{rg}(A) = \mathrm{rg}([A,\textbf{b}])$.
\end{enumerate}
\newpage
\subsection*{\frage{8}{3}}
Sei $ A $ eine $ (1 \times 5) $-Matrix definiert als
\begin{align*}
	A
	=
	\left(
	\textbf{a}_1,
	\textbf{a}_2,
	\textbf{a}_3,
	\textbf{a}_4,
	\textbf{a}_5
	\right).
\end{align*}
für $ a_i \in \mathbb{R}, \ i = 1,2,...,5 $.\\
\\
Es folgt, dass:
\renewcommand{\labelenumi}{(\alph{enumi})}
\begin{enumerate}
	\item 
	$ \mathrm{rg}(A) = 5 $, falls $ a_i \neq 0 $ für alle $ i \in \{1,2,3,4,5\} $ und $ a_i \neq a_j $ für $ i \neq j $.
	\item
	$ \mathrm{rg}(A) = 5 $, falls $ a_i \neq 0 $ für mindestens ein $ i \in \{1,2,3,4,5\} $.
	\item
	$ \mathrm{rg}(A) = 1 $, falls $ a_i \neq 0 $ für mindestens ein $ i \in \{1,2,3,4,5\} $.
	\item
	$ \mathrm{rg}(A) = 0 $, falls $ a_i = 0 $ für genau ein $ i \in \{1,2,3,4,5\} $.
	\item
	$1 <  \mathrm{rg}(A)\leq 5 $, falls $ a_i \neq 0 $ für mindestens ein $ i \in \{1,2,3,4,5\} $.
	\item 
	Es ist nicht möglich, mit den gegebenen Informationen eine Aussage zum Rang von $ A $ zu machen.
\end{enumerate}
\ \\
\subsection*{\frage{9}{3}}
Sei $ A $ eine $ (4 \times 2) $-dimensionale Matrix mit Rang $ 2 $ und $ B $ eine Matrix mit den Dimensionen $ (4 \times 3) $ mit Rang $ 1 $.\\
\\
Wir definieren die Matrix $ C $ als die Aneinanderreihung der Spalten von $ A $ und $ B $ wie folgt:
\begin{align*}
	C = [A, B, A].
\end{align*}
Es folgt, dass:
\renewcommand{\labelenumi}{(\alph{enumi})}
\begin{enumerate}
	\item 
	$ \mathrm{rg}(C) = 2 $.
	\item
	$ \mathrm{rg}(C) = 3 $.
	\item
	$ \mathrm{rg}(C)\in \{2,3\} $.
	\item
	$ \mathrm{rg}(C) = 4 $.
	\item
	Die gegebenen Informationen sind nicht ausreichend, um eine Aussage über den Rang von $ C $ treffen zu können.
\end{enumerate}
\ \\
\subsection*{\frage{10}{3}}
Sei $ A \in \mathbb{R}^{n\times n} $ eine quadratische obere Dreiecksmatrix, d.h. eine quadratische Matrix deren Elemente unterhalb der Diagonale null sind.\\
\\
Welche der folgenden Aussagen ist \textit{nicht} korrekt bezüglich der Matrix $ A $?
\renewcommand{\labelenumi}{(\alph{enumi})}
\begin{enumerate}
	\item 
	Die Determinante ist das Produkt der Elemente auf der Diagonale.
	\item
	Die Matrix $ A + A^\top $ ist symmetrisch.
	\item
	Falls die Inverse $ A^{-1} $ existiert, so ist sie ebenfalls eine obere Dreiecksmatrix.
	\item
	Die Eigenwerte von $ A $ entsprechen den Elementen auf der Diagonale
	\item 
	Die Eigenvektoren von $ A $ entsprechen den Zeilen von $ A $.
\end{enumerate}

\newpage
\section*{Aufgabe 3 (28 Punkte)}
\vspace{0.4cm}

\subsection*{\frage{1}{3}}
Das bestimmte Integral
\begin{align*}
	\int_0^{2\pi} \cos \left(\frac{x}{t}\right) e^{- \sin(\frac{x}{t})} dx
\end{align*}
mit $ t \in \mathbb{N} $ ist gleich:

\renewcommand{\labelenumi}{(\alph{enumi})}
\begin{enumerate}
\item 
$ \frac{1}{t} -1  $.
\item
$ (1 - e^{-\sin\left(\frac{2\pi}{t}\right)})t $
\item
$ t + e^{\sin\left(\frac{2\pi}{t}\right)} $
\item 
Keine der obigen Antworten ist korrekt.
\end{enumerate}
\ \\
\subsection*{\frage{2}{4}}
Sei $ f \ : \ \mathbb{R}^2 \to \mathbb{R} $ eine partiell differenzierbare Funktion zweier reeller Variablen, definiert als 
\begin{align*}
	f(x,y) = e^{ax^2 +a^2 xy^2}
\end{align*}
mit $ a \in \mathbb{R}_{++} $. Für welchen Wert $ a $ ist die Richtung des steilsten Anstiegs von $ f $ im Punkt $ P = (1,0.5) $ gegeben durch den Vektor
\begin{align*}
	\textbf{v} =
	\begin{pmatrix}
		e\\
		e
	\end{pmatrix}?
\end{align*}
\renewcommand{\labelenumi}{(\alph{enumi})}
\begin{enumerate}
	\item 
	$0 $.
	\item
	$\frac{8}{3}$.
	\item
	$\frac{1}{2}$.
	\item
	Es existiert kein solches $ a \in \mathbb{R}_{++} $.
\end{enumerate}
\newpage
\subsection*{\frage{3}{3}}
Gegeben sein eine Zufallsvariable $ X $ mit der Dichtefunktion
\begin{align*}
	f(x)
	=
	\begin{cases}
		0 \quad &x \geq 1\\
		\frac{3}{2} \sqrt{x} \quad &0 \leq x < 1\\
		0 \quad &x<0
	\end{cases}.
\end{align*}
Der Erwartungswert von $ X $ ist gleich:
\renewcommand{\labelenumi}{(\alph{enumi})}
\begin{enumerate}
	\item 
	$\mathbb{E}[X] = \frac{3}{5}$.
	\item
	$\mathbb{E}[X] = \sqrt{2}$.
	\item
	$\mathbb{E}[X] = \frac{6}{7}$.
	\item
	$\mathbb{E}[X] = \sqrt{\frac{6}{7}}$.
\end{enumerate}
\ \\
\subsection*{\frage{4}{4}}
Gegeben sei die Funktion $ f $ definiert als
\begin{align*}
	f(x)
	=
	\begin{cases}
		0 \quad &x \geq 1\\
		\frac{c}{2x} \quad &0.5 \leq x < 1\\
		6x^2 \quad &0 \leq x < 0.5\\
		0 \quad &x<0
	\end{cases}.
\end{align*}
Für welche Werte von $ c \in \mathbb{R} $ ist $ f $ eine Dichtefunktion?
\renewcommand{\labelenumi}{(\alph{enumi})}
\begin{enumerate}
\item 
$ c = 10$.
\item 
$ c= -\frac{3}{2 \ln(0.5)} $.
\item
$ c= -\frac{2}{3 \ln(2)} $.
\item
$ c = \ln \left(\frac{2}{3}\right) $.
\item
$ c = e^{\frac{2}{3}} $.
\end{enumerate}

\newpage
\subsection*{\frage{5}{3}}
Gegeben sei die Matrix
\begin{align*}
	A =
	\begin{pmatrix}
		2 & 1 & 1 & 1 & 1 \\
		3 & 2 & 1 & 1 & 1 \\
		6 & 0 & 6 & 6 & 6 \\ 
		8 & 1 & 7 & 8 & 7
	\end{pmatrix}.
\end{align*}
Es folgt, dass
\renewcommand{\labelenumi}{(\alph{enumi})}
\begin{enumerate}
	\item 
	$ \mathrm{rg}(A) = 1 $.
	\item 
	$ \mathrm{rg}(A) = 2 $.
	\item
	$ \mathrm{rg}(A) = 3 $.
	\item
	$ \mathrm{rg}(A) = 4 $.
	\item
	$ \mathrm{rg}(A) = 5 $.
\end{enumerate}
\ \\

\subsection*{\frage{6}{3}}
Gegeben sei die Matrix
\begin{align*}
	A =
	\begin{pmatrix}
		-7 & 0 & 0 \\
		1 & 2 & 0 \\
		-5m & 5 & 4
	\end{pmatrix}
	, 
	\quad 
	\textrm{mit } m \in \mathbb{R}.
\end{align*}
Für den Eigenwert $ \lambda = 4 $ ist der Eigenraum $ W $ gleich
\renewcommand{\labelenumi}{(\alph{enumi})}
\begin{enumerate}
	\item 
	$ W
	=
	\left\{
	\textbf{x} \in \mathbb{R}^3
	:
	\textbf{x}
	=
	s 
	\begin{pmatrix}
		0\\ 0 \\ -1
	\end{pmatrix},
	s \in \mathbb{R}
	\right\}
	 $
	.
	\item 
	$ W
	=
	\left\{
	\textbf{x} \in \mathbb{R}^3
	:
	\textbf{x}
	=
	s	 
	\begin{pmatrix}
		7m \\ -1 \\ m-1
	\end{pmatrix},
	s \in \mathbb{R}
	\right\}
	$
	.
	\item
	$ W
	=
	\left\{
	\textbf{x} \in \mathbb{R}^3
	:
	\textbf{x}
	=
	s	 
	\begin{pmatrix}
		3 \\ -7 \\ 1 - 5m
	\end{pmatrix},
	s \in \mathbb{R}
	\right\}
	$
	.
	\item
	$ W
	=
	\left\{
	\textbf{x} \in \mathbb{R}^3
	:
	\textbf{x}
	=
	s	 
	\begin{pmatrix}
		1 \\ -3 \\ 1 - 3m
	\end{pmatrix},
	s \in \mathbb{R}
	\right\}
	$
	.
	\item
	$ \lambda = 4 $ ist kein Eigenwert von $ A $.
\end{enumerate}
\ \\
\subsection*{\frage{7}{4}}
Welche der folgenden Differenzengleichungen wird durch die Folge
\begin{align*}
	y_{k} = \left(\frac{1}{4}\right)^{k+1} + 4, \quad k \in \mathbb{N},
\end{align*}
gelöst?
\renewcommand{\labelenumi}{(\alph{enumi})}
\begin{enumerate}
\item
$ 6 y_{k+1}  - 3 y_k = 12$.
\item
$ 4 y_{k+1} - 32 y_k = -26 $.	
\item 
$ 8 y_{k+1} - 2 y_k = 24 $
\item
Keine der obigen Antworten ist korrekt.
\end{enumerate}
\ \\
\subsection*{\frage{8}{4}}
Gegeben sei die Differenzengleichung
\begin{align*}
6y_{k+1} + 2 y_k +7 = 6
\quad (k = 0,1,2,...).
\end{align*}
Es folgt, dass:
\renewcommand{\labelenumi}{(\alph{enumi})}
\begin{enumerate}
	\item
	$ (y_k)_{k \in \mathbb{N}} $ oszillierend und divergent ist.
	\item
	$ (y_k)_{k \in \mathbb{N}} $ monoton und divergent ist.	
	\item 
	$ (y_k)_{k \in \mathbb{N}} $ monoton und konvergent ist mit $ \lim \limits_{k \to \infty} y_k = -\frac{1}{7} $.
	\item
	$ (y_k)_{k \in \mathbb{N}} $ oszillierend und konvergent ist mit $ \lim \limits_{k \to \infty} y_k = \frac{1}{7} $.
	\item 
	$ (y_k)_{k \in \mathbb{N}} $ oszillierend und konvergent ist mit $ \lim \limits_{k \to \infty} y_k = -\frac{1}{8} $.
	\item 
	$ (y_k)_{k \in \mathbb{N}} $ monoton und konvergent ist mit $ \lim \limits_{k \to \infty} y_k = \frac{1}{8} $.
\end{enumerate}