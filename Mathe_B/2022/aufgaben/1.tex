\vspace{1cm}
\fancyhead[C]{\normalsize\textbf{$\qquad$ Teil I: Offene Aufgaben}}
\renewcommand{\labelenumi}{\theenumi.}
\section*{Aufgabe 1 (40 Punkte)}
\vspace{0.4cm}
\subsection*{\aufgabe{a}{12}}
Eine Rederei muss für ihre Schiffe Container herstellen, um darin Güter um die ganze Welt verschiffen zu können. Die Container müssen alle identisch sein und sollen, mathematisch formuliert, die Form eines Quaders  mit Länge $ x $, Höhe $ y $ und Breite $ z $ besitzen. Jeder Container muss ein Fassungsvermögen von 27 Volumeneinheiten haben. 
\begin{enumerate}
	\item[\textbf{(a1)}]
	Bestimmen Sie $ z $ als Funktion von $ x $ und $ y $ mithilfe der Volumenvorgabe jedes Containers.
	\item[\textbf{(a2)}] 
	Bestimmen Sie die Länge $ x $ und Höhe $ y $ (und damit die Breite $ z $ mit der Hilfe von Aufgabenteil (a1)), sodass die Container die minimale Oberfläche besitzen (und folglich die geringsten Herstellungskosten).
\end{enumerate}
\ \\
\textbf{Lösung:}
\begin{mdframed}
\underline{\textbf{Vorgehensweise:}}
\renewcommand{\labelenumi}{\theenumi.}
\begin{enumerate}
\item[\textbf{(a1)}] Stelle die Funktion mithilfe des Quadervolumens auf.
\item[\textbf{(a2)}] 
\end{enumerate}
\end{mdframed}
Die Verwendung von \glqq minimal\grqq \ lässt vermuten, dass ein Optimierungsproblem vorliegt.
Hierbei müssen wir zunächst prüfen ob und welche Nebenbedingungen vorliegen. Diese beschränken die Variablen. In unseren Fall müssen wir $ x $,  $ y $ und $ z $ so wählen, dass das Volumen des quaderförmigen Containers $ 27 $ Einheiten beträgt.
Es liegt also ein Minimierungsproblem mit Nebenbedingung vor.\\
\\
\underline{\textbf{(a1)} Stelle die Funktion mithilfe des Quadervolumens auf }\\
Ein Quader(Container) mit Länge $ x $, Höhe $ y $ und Breite $ z $ hat das Volumen $ V $
\begin{align*}
	V(x,y,z) = xyz.
\end{align*}
In unserem Fall soll das Volumen des Containers $ 27 $ Volumeneinheiten betragen.
Damit erhalten wir die Breite $ z $ als Funktion
\begin{align*}
	V(x,y,z) = xyz = 27  
	\ \Leftrightarrow \
	z(x,y) := z = \frac{27}{xy}
\end{align*}
für $ x,y > 0 $.





\newpage

\subsection*{\aufgabe{b}{14}}
Die Renditen der Aktien der Unternehmen BEST, SUCCESS und WINNER hängen von einem von vier möglichen Szenarien ab, wie sich die Finanzmärkte entwickeln. Die möglichen Szenarien und Renditen der jeweiligen Aktie werden in folgender Tabelle beschrieben:
\begin{table}[H]
	\centering
	%
	\begin{tabular}{c c c c}
		\hline
		Szenario & BEST  &  SUCCESS &  WINNER \\ 
		\hline
		$ s_1 $ & $ 3 $ & $ 1 $ & $ 2 $  \\ 
		$ s_2 $ & $ 2 $ & $ 2 $ & $ 2 $ \\
		$ s_3 $ & $ 3 $ & $ 3 $ & $ 1 $ \\
		$ s_4 $ & $ -1 $ & $ 1 $ & $ 0 $\\
		\hline
	\end{tabular}%
\end{table}
Eine Investorin möchte in die drei Aktien investieren. Ihr Ziel ist es folgende Rendite, abhängig vom jeweiligen Szenario, zu generieren:
\begin{table}[H]
	\centering
	%
	\begin{tabular}{c c}
		\hline
		Szenario & Rendite  \\
		\hline
		$ s_1 $ & $ 5 $ \\ 
		$ s_2 $ & $ 3 $  \\
		$ s_3 $ & $ a $  \\
		$ s_4 $ & $ -2 $ \\
		\hline
	\end{tabular}%
\end{table}
wobei $ a \in \mathbb{R} $.
\begin{enumerate}
	\item[\textbf{(b1)}]
	Bestimmen Sie das lineare Gleichungssystem, welches das Allokationsproblem der Investorin beschreibt.
\end{enumerate}
Verwenden Sie den Gauß-Algorithmus für die folgenden Aufgaben.
\begin{enumerate}
	\item[\textbf{(b2)}] 
	Für welche Werte von $ a $ existiert eine Lösung des Investitionsproblems?
	\item[\textbf{(b3)}]
	Bestimmen Sie für $ a \ = \ -3.5 $ die Investitionsstrategie der Investorin, welche die entsprechenden Renditen je nach Szenario erwirtschaftet.
\end{enumerate}
\ \\
\textbf{Lösung:}
\begin{mdframed}
\underline{\textbf{Vorgehensweise:}}
\renewcommand{\labelenumi}{\theenumi.}
\begin{enumerate}
\item[\textbf{(b1)}] Stelle das lineare Gleichungssystem auf.
\item[\textbf{(b2)}] 
\begin{enumerate}
	\item[1.] Wende das elementare Zeilenumformungen an (Gauß-Verfahren).
	\item[2.] Überlege ob Lösungen existieren und prüfe deren Eindeutigkeit.
\end{enumerate}

\item[\textbf{(b3)}] Löse das LGS für $a=-3.5 $.
\end{enumerate}
\end{mdframed}

In dieser Aufgabe liegt ein lineares Gleichungssystem vor.
Dies müssen wir zunächst aus der Aufgabenstellung extrahieren und darauf dessen Struktur und Eigenschaften untersuchen.
Wir schreiben die Renditen der einzelnen Szenarien zu einer Aktie als Vektor und linear kombinieren diese zu den Gesamtrenditen der einzelnen Szenarien:
\begin{align*}
	x_1
	\begin{pmatrix}
		3 \\ 2 \\ 3 \\ -1
	\end{pmatrix}
	+
	x_2
	\begin{pmatrix}
		1 \\ 2 \\ 3 \\ 1
	\end{pmatrix}
	+
	x_3
	\begin{pmatrix}
		2 \\ 2 \\ 1 \\ 0
	\end{pmatrix}
	=
	\begin{pmatrix}
		5 \\ 3 \\ a \\ -2
	\end{pmatrix}
\end{align*}

\underline{\textbf{(b1)} Stelle das lineare Gleichungssystem auf}\\
In der Einleitung haben wir das Aufstellen des Gleichungssystems
\begin{align*}
	s_1 : 3x_1 +  \ x_2 + 2 x_3 \ &=\ 5\\
	s_2 : 2x_2 + 2 x_2 + 2 x_3 \ &= \ 3\\
	s_3 : 3x_1 + 3x_2 + \ \ x_3 \ &= \ a \ \\
	s_4 : -x_1 + x_2 \qquad \quad \ \ &= -2
\end{align*}
für die einzelen Szenarien $s_i$ mit $i = 1,...,4$ vorwegenommen.
Übersetzt in eine erweiterte Koeffizientenmatrix erhalten wir:
\begin{align*}
	(A,\mathbf{b})
	=
	\begin{pmatrix}
		3 & 1 & 2 & \BAR & 5\\
		2 & 2 & 2 & \BAR & 3\\
		3 & 3 & 1 & \BAR & a \\
		-1 & 1 & 0 & \BAR & -2
	\end{pmatrix}.
\end{align*}
\ \\
\underline{\textbf{(b2)} 1. Wende das elementare Zeilenumformungen an (Gauß-Verfahren)}\\
Durch elementare Zeilenumformungen erhalten wir:
\begin{align*}
	\begin{gmatrix}[p]
		3 & 1 & 2 & \BAR & 5\\
		2 & 2 & 2 & \BAR & 3\\
		3 & 3 & 1 & \BAR & a \\
		-1 & 1 & 0 & \BAR & -2
		\rowops
		\add[ \cdot 3]{3}{2}
		\add[ \cdot 2]{3}{1}
		\add[ \cdot 3]{3}{0}
	\end{gmatrix}
	&\leadsto
	\begin{gmatrix}[p]
		0 & 4 & 2 & \BAR & -1\\
		0 & 4 & 2 & \BAR & -1\\
		0 & 6 & 1 & \BAR & a - 6\\
		-1 & 1 & 0 & \BAR & -2
		\rowops
		\add[ \cdot (-1)]{0}{2}
	\end{gmatrix}
	\leadsto
	\begin{gmatrix}[p]
		0 & 4 & 2 & \BAR & -1\\
		0 & 0 & 0 & \BAR & 0\\
		0 & 6 & 1 & \BAR & a - 6\\
		-1 & 1 & 0 & \BAR & -2
	\end{gmatrix}\\
	&\leadsto
	\begin{gmatrix}[p]
		-1 & 1 & 0 & \BAR & -2\\
		0 & 4 & 2 & \BAR & -1\\
		0 & 6 & 1 & \BAR & a - 6\\
		0 & 0 & 0 & \BAR & 0
		\rowops
		\mult{1}{ \cdot (3)}
		\mult{2}{ \cdot (2)}
	\end{gmatrix}
	\leadsto
	\begin{gmatrix}[p]
		-1 & 1 & 0 & \BAR & -2\\
		0 & 12 & 6 & \BAR & -3\\
		0 & 12 & 2 & \BAR & 2a - 12\\
		0 & 0 & 0 & \BAR & 0
		\rowops
		\mult{1}{ \cdot (3)}
		\mult{2}{ \cdot (2)}
	\end{gmatrix}\\
	&\leadsto
	\begin{gmatrix}[p]
		-1 & 1 & 0 & \BAR & -2\\
		0 & 12 & 6 & \BAR & -3\\
		0 & 12 & 2 & \BAR & 2a - 12\\
		0 & 0 & 0 & \BAR & 0
		\rowops
		\add[ \cdot (-1)]{1}{2}
	\end{gmatrix}
	\leadsto
	\begin{gmatrix}[p]
		-1 & 1 & 0 & \BAR & -2\\
		0 & 12 & 6 & \BAR & -3\\
		0 & 0 & 4 & \BAR & 2a - 9\\
		0 & 0 & 0 & \BAR & 0
		\rowops
		\mult{1}{ \cdot (\frac{1}{3})}
	\end{gmatrix}\\
	&\leadsto
	\begin{gmatrix}[p]
		-1 & 1 & 0 & \BAR & -2\\
		0 & 4 & 2 & \BAR & -1\\
		0 & 0 & -4 & \BAR & 2a - 9\\
		0 & 0 & 0 & \BAR & 0
	\end{gmatrix}
\end{align*}
\ \\
\underline{\textbf{(b2)} 2. Überlege ob Lösungen existieren und prüfe deren Eindeutigkeit}\\
Damit ein Gleichungssystem $ A \mathbf{x} = \mathbf{b} $ mindestens eine Lösung besitzt, muss $ \mathrm{rg}(A) = \mathrm{rg}( A,\mathbf{b} ) $ gelten.
Mit $n$ bezeichnen wir die Anzahl der Unbekannten.
Falls mindestens eine Lösung existiert, können zwei Fälle eintreten:
\begin{description}
	\item[Fall 1 $\mathbf{\mathrm{rg}(A) = \mathrm{rg}(A,b) = n}$:] Es existiert genau eine (eindeutige) Lösung.
	\item[Fall 2 $\mathbf{\mathrm{rg}(A) = \mathrm{rg}(A,b) < n }$:] Es existieren unendlich viele Lösungen.
\end{description}
In unserem Fall gilt unabhängig von $a$ $\mathrm{rg}(A) = \mathrm{rg}(A,\mathbf{b}) = 3 = n$. Damit existiert für alle $a \in \mathbb{R}$ eine eindeutige Lösung für das Investitionsproblem.\\
\\
\underline{\textbf{(b3)} Löse das LGS für $a=-3.5 $}\\
Wir setzen $a = -3.5 = - \frac{7	}{2}$ in die erweiterte Koeffizientenmatrix ein und erhalten durch elementare Zeilenumformungen:
\begin{align*}
	\begin{gmatrix}[p]
		-1 & 1 & 0 & \BAR & -2\\
		0 & 4 & 2 & \BAR & -1\\
		0 & 0 & -4 & \BAR & -16\\
		0 & 0 & 0 & \BAR & 0
		\rowops
		\mult{2}{ \cdot (\frac{1}{4})}
	\end{gmatrix}
	&\leadsto
	\begin{gmatrix}[p]
		-1 & 1 & 0 & \BAR & -2\\
		0 & 4 & 2 & \BAR & -1\\
		0 & 0 & -1 & \BAR & -4\\
		0 & 0 & 0 & \BAR & 0
		\rowops
		\add[ \cdot 2]{2}{1}
		\mult{0}{ \cdot 4}
	\end{gmatrix}
	\leadsto
	\begin{gmatrix}[p]
		-4 & 4 & 0 & \BAR & -8\\
		0 & 4 & 0 & \BAR & -9\\
		0 & 0 & 1 & \BAR & 4\\
		0 & 0 & 0 & \BAR & 0
		\rowops
		\add[ \cdot (-1)]{1}{0}
		\mult{2}{ \cdot (-1)}
	\end{gmatrix}\\
	&\leadsto
	\begin{gmatrix}[p]
		-4 & 0 & 0 & \BAR & 1\\
		0 & 4 & 0 & \BAR & -9\\
		0 & 0 & 1 & \BAR & 4\\
		0 & 0 & 0 & \BAR & 0
		\rowops
		\mult{0}{ \cdot (-\frac{1}{4})}
		\mult{1}{ \cdot \frac{1}{4}}
	\end{gmatrix}
	\leadsto
	\begin{gmatrix}[p]
		1 & 0 & 0 & \BAR & -\frac{1}{4}\\
		0 & 1 & 0 & \BAR & - \frac{9}{4}\\
		0 & 0 & 1 & \BAR & 4\\
		0 & 0 & 0 & \BAR & 0
	\end{gmatrix}.
\end{align*}
Hieran lässt sich die Lösung 
\begin{align*}
	x_1 = - \frac{1}{4} = - 0.25, \quad
	x_2 = -\frac{9}{4} = - 2.25, \quad
	x_3 = 4
\end{align*}
direkt ablesen.\\
\\
Alternativ lässt sich die obere Dreiecksform nach den elementaren Zeilenumformungen ausnutzen. Die erweiterte Koeffizientenmatrix
\begin{align*}
	\begin{gmatrix}[p]
		-1 & 1 & 0 & \BAR & -2\\
		0 & 4 & 2 & \BAR & -1\\
		0 & 0 & -4 & \BAR & -16\\
		0 & 0 & 0 & \BAR & 0
	\end{gmatrix}
	&\leadsto
	\begin{gmatrix}[p]
		-1 & 1 & 0 & \BAR & -2\\
		0 & 4 & 2 & \BAR & -1\\
		0 & 0 & 1 & \BAR & 4\\
		0 & 0 & 0 & \BAR & 0
	\end{gmatrix}
\end{align*}
enthält die Gleichungen:
\begin{align*}
	-x_1 + x_2 &= -2 \\
	4 x_2 + 2 x_3 &= 1 \\
	x_3 &= 4.
\end{align*}
Durch Rückwärtseinsetzen erhalten wir ebenso die Lösung.
\newpage
\subsection*{\aufgabe{c}{14}}
Ein Konsument deckt seine Nachfrage nach zwei Gütern. Dabei beschreibt $ c_1 $ die Menge von Gut $ 1 $ und $ c_2 $ die Menge von Gut $ 2 $. Der Konsument besitzt die Nutzenfunktion $ u(c_1,c_2) $, welche als Kehrwert der geometrischen Distanz zwischen dem Konsumbündel $ (c_1,c_2) $ und einem angestrebten Konsumbündel $ (3,3) $ definiert ist. Der Preis für Gut $ 1 $ beträgt $ p_1 \ = \ 2 $ (CHF) und für Gut $ 2 $ $ p_2 \ = \ 3 $ (CHF). Der Konsument besitzt das Einkommen von $ e = 10 $ (CHF).
\begin{enumerate}
	\item[\textbf{(c1)}] 
	Bestimmen Sie die Nutzenfunktion des Konsumenten sowie dessen Budgetrestriktion.
	\item[\textbf{(c2)}]
	Vereinfachen Sie das Optimierungsproblem, indem Sie es geeignet transformieren und bestimmen Sie das optimale Konsumbündel für den Konsumenten.
\end{enumerate}
\ \\
\textit{Bemerkungen: Der Kehrwert eines mathematischen Ausdrucks $ x $ ist definiert als $ \frac{1}{x} $. Es ist \textbf{nicht} nötig zu zeigen, dass es sich um ein Maximum handelt.}
\\ \\
\textbf{Lösung:}
\begin{mdframed}
\underline{\textbf{Vorgehensweise:}}
\begin{enumerate}
\item[\textbf{(c1)}] 
\item[\textbf{(c2)}] 
\end{enumerate}
\end{mdframed}

\underline{\textbf{(c1)} }\\



\newpage

