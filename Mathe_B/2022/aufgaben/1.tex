\vspace{1cm}
\fancyhead[C]{\normalsize\textbf{$\qquad$ Teil I: Offene Aufgaben}}
\renewcommand{\labelenumi}{\theenumi.}
\section*{Aufgabe 1 (40 Punkte)}
\vspace{0.4cm}
\subsection*{\aufgabe{a}{12}}
Eine Rederei muss für ihre Schiffe Container herstellen, um darin Güter um die ganze Welt verschiffen zu können. Die Container müssen alle identisch sein und sollen, mathematisch formuliert, die Form eines Quaders  mit Länge $ x $, Höhe $ y $ und Breite $ z $ besitzen. Jeder Container muss ein Fassungsvermögen von 27 Volumeneinheiten haben. 
\begin{enumerate}
	\item[\textbf{(a1)}]
	Bestimmen Sie $ z $ als Funktion von $ x $ und $ y $ mithilfe der Volumenvorgabe jedes Containers.
	\item[\textbf{(a2)}] 
	Bestimmen Sie die Länge $ x $ und Höhe $ y $ (und damit die Breite $ z $ mit der Hilfe von Aufgabenteil (a1)), sodass die Container die minimale Oberfläche besitzen (und folglich die geringsten Herstellungskosten).
\end{enumerate}
\ \\
\textbf{Lösung:}
\begin{mdframed}
\underline{\textbf{Vorgehensweise:}}
\renewcommand{\labelenumi}{\theenumi.}
\begin{enumerate}
\item[\textbf{(a1)}] 
\begin{enumerate}
	\item[1.] .
	\item[2.] .
\end{enumerate} 
\item[\textbf{(a2)}] 
\end{enumerate}
\end{mdframed}
\underline{\textbf{(a1)} 1. }\\






\newpage

\subsection*{\aufgabe{b}{14}}
Die Renditen der Aktien der Unternehmen BEST, SUCCESS und WINNER hängen von einem von vier möglichen Szenarien ab, wie sich die Finanzmärkte entwickeln. Die möglichen Szenarien und Renditen der jeweiligen Aktie werden in folgender Tabelle beschrieben:
\begin{table}[H]
	\centering
	%
	\begin{tabular}{c c c c}
		\hline
		Szenario & BEST  &  SUCCESS &  WINNER \\ 
		\hline
		$ s_1 $ & $ 3 $ & $ 1 $ & $ 2 $  \\ 
		$ s_2 $ & $ 2 $ & $ 2 $ & $ 2 $ \\
		$ s_3 $ & $ 3 $ & $ 3 $ & $ 1 $ \\
		$ s_4 $ & $ -1 $ & $ 1 $ & $ 0 $\\
		\hline
	\end{tabular}%
\end{table}
Eine Investorin möchte in die drei Aktien investieren. Ihr Ziel ist es folgende Rendite, abhängig vom jeweiligen Szenario, zu generieren:
\begin{table}[H]
	\centering
	%
	\begin{tabular}{c c}
		\hline
		Szenario & Rendite  \\
		\hline
		$ s_1 $ & $ 5 $ \\ 
		$ s_2 $ & $ 3 $  \\
		$ s_3 $ & $ a $  \\
		$ s_4 $ & $ -2 $ \\
		\hline
	\end{tabular}%
\end{table}
wobei $ a \in \mathbb{R} $.
\begin{enumerate}
	\item[\textbf{(b1)}]
	Bestimmen Sie das lineare Gleichungssystem, welches das Allokationsproblem der Investorin beschreibt.
\end{enumerate}
Verwenden Sie den Gauß-Algorithmus für die folgenden Aufgaben.
\begin{enumerate}
	\item[\textbf{(b2)}] 
	Für welche Werte von $ a $ existiert eine Lösung des Investitionsproblems?
	\item[\textbf{(b3)}]
	Bestimmen Sie für $ a \ = \ -3.5 $ die Investitionsstrategie der Investorin, welche die entsprechenden Renditen je nach Szenario erwirtschaftet.
\end{enumerate}
\ \\
\textbf{Lösung:}
\begin{mdframed}
\underline{\textbf{Vorgehensweise:}}
\renewcommand{\labelenumi}{\theenumi.}
\begin{enumerate}
\item[\textbf{(b1)}] 
\item[\textbf{(b2)}] 
\item[\textbf{(b3)}] 
\end{enumerate}
\end{mdframed}

\underline{\textbf{(b1)} }\\

\newpage
\subsection*{\aufgabe{c}{14}}
Ein Konsument deckt seine Nachfrage nach zwei Gütern. Dabei beschreibt $ c_1 $ die Menge von Gut $ 1 $ und $ c_2 $ die Menge von Gut $ 2 $. Der Konsument besitzt die Nutzenfunktion $ u(c_1,c_2) $, welche als Kehrwert der geometrischen Distanz zwischen dem Konsumbündel $ (c_1,c_2) $ und einem angestrebten Konsumbündel $ (3,3) $ definiert ist. Der Preis für Gut $ 1 $ beträgt $ p_1 \ = \ 2 $ (CHF) und für Gut $ 2 $ $ p_2 \ = \ 3 $ (CHF). Der Konsument besitzt das Einkommen von $ e = 10 $ (CHF).
\begin{enumerate}
	\item[\textbf{(c1)}] 
	Bestimmen Sie die Nutzenfunktion des Konsumenten sowie dessen Budgetrestriktion.
	\item[\textbf{(c2)}]
	Vereinfachen Sie das Optimierungsproblem, indem Sie es geeignet transformieren und bestimmen Sie das optimale Konsumbündel für den Konsumenten.
\end{enumerate}
\ \\
\textit{Bemerkungen: Der Kehrwert eines mathematischen Ausdrucks $ x $ ist definiert als $ \frac{1}{x} $. Es ist \textbf{nicht} nötig zu zeigen, dass es sich um ein Maximum handelt.}
\\ \\
\textbf{Lösung:}
\begin{mdframed}
\underline{\textbf{Vorgehensweise:}}
\begin{enumerate}
\item[\textbf{(c1)}] 
\item[\textbf{(c2)}] 
\end{enumerate}
\end{mdframed}

\underline{\textbf{(c1)} }\\



\newpage

