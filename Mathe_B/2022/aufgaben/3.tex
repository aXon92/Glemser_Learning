\section*{Aufgabe 3 (28 Punkte)}
\vspace{0.4cm}
\subsection*{\frage{1}{3}}
Das bestimmte Integral
\begin{align*}
	\int_0^{2\pi} \cos \left(\frac{x}{t}\right) e^{- \sin(\frac{x}{t})} dx
\end{align*}
mit $ t \in \mathbb{N} $ ist gleich:

\renewcommand{\labelenumi}{(\alph{enumi})}
\begin{enumerate}
	\item 
	$ \frac{1}{t} -1  $.
	\item
	$ (1 - e^{-\sin\left(\frac{2\pi}{t}\right)})t $
	\item
	$ t + e^{\sin\left(\frac{2\pi}{t}\right)} $
	\item 
	Keine der obigen Antworten ist korrekt.
\end{enumerate}
\ \\
\textbf{Lösung:}
\begin{mdframed}
\underline{\textbf{Vorgehensweise:}}
\renewcommand{\labelenumi}{\theenumi.}
\begin{enumerate}
\item .
\end{enumerate}
\end{mdframed}

\underline{1.}\\


 
\newpage

\subsection*{\frage{2}{4}}
Sei $ f \ : \ \mathbb{R}^2 \to \mathbb{R} $ eine partiell differenzierbare Funktion zweier reeller Variablen, definiert als 
\begin{align*}
	f(x,y) = e^{ax^2 +a^2 xy^2}
\end{align*}
mit $ a \in \mathbb{R}_{++} $. Für welchen Wert $ a $ ist die Richtung des steilsten Anstiegs von $ f $ im Punkt $ P = (1,0.5) $ gegeben durch den Vektor
\begin{align*}
	\textbf{v} =
	\begin{pmatrix}
		e\\
		e
	\end{pmatrix}?
\end{align*}
\renewcommand{\labelenumi}{(\alph{enumi})}
\begin{enumerate}
	\item 
	$0 $.
	\item
	$\frac{8}{3}$.
	\item
	$\frac{1}{2}$.
	\item
	Es existiert kein solches $ a \in \mathbb{R}_{++} $.
\end{enumerate}
\ \\
\textbf{Lösung:}
\begin{mdframed}
\underline{\textbf{Vorgehensweise:}}
\renewcommand{\labelenumi}{\theenumi.}
\begin{enumerate}
\item 
\end{enumerate}
\end{mdframed}

\underline{1. }\\


\newpage
\subsection*{\frage{3}{3}}
Gegeben sein eine Zufallsvariable $ X $ mit der Dichtefunktion
\begin{align*}
	f(x)
	=
	\begin{cases}
		0 \quad &x \geq 1\\
		\frac{3}{2} \sqrt{x} \quad &0 \leq x < 1\\
		0 \quad &x<0
	\end{cases}.
\end{align*}
Der Erwartungswert von $ X $ ist gleich:
\renewcommand{\labelenumi}{(\alph{enumi})}
\begin{enumerate}
	\item 
	$\mathbb{E}[X] = \frac{3}{5}$.
	\item
	$\mathbb{E}[X] = \sqrt{2}$.
	\item
	$\mathbb{E}[X] = \frac{6}{7}$.
	\item
	$\mathbb{E}[X] = \sqrt{\frac{6}{7}}$.
\end{enumerate}
\ \\
\textbf{Lösung:}
\begin{mdframed}
\underline{\textbf{Vorgehensweise:}}
\renewcommand{\labelenumi}{\theenumi.}
\begin{enumerate}
\item 
\end{enumerate}
\end{mdframed}

\underline{1.}\\

\newpage

\newpage
\subsection*{\frage{4}{4}}
Gegeben sei die Funktion $ f $ definiert als
\begin{align*}
	f(x)
	=
	\begin{cases}
		0 \quad &x \geq 1\\
		\frac{c}{2x} \quad &0.5 \leq x < 1\\
		6x^2 \quad &0 \leq x < 0.5\\
		0 \quad &x<0
	\end{cases}.
\end{align*}
Für welche Werte von $ c \in \mathbb{R} $ ist $ f $ eine Dichtefunktion?
\renewcommand{\labelenumi}{(\alph{enumi})}
\begin{enumerate}
	\item 
	$ c = 10$.
	\item 
	$ c= -\frac{3}{2 \ln(0.5)} $.
	\item
	$ c= -\frac{2}{3 \ln(2)} $.
	\item
	$ c = \ln \left(\frac{2}{3}\right) $.
	\item
	$ c = e^{\frac{2}{3}} $.
\end{enumerate}
\ \\
\textbf{Lösung:}
\begin{mdframed}
	\underline{\textbf{Vorgehensweise:}}
	\renewcommand{\labelenumi}{\theenumi.}
	\begin{enumerate}
		\item 
	\end{enumerate}
\end{mdframed}

\underline{1.}\\

\newpage

\subsection*{\frage{5}{3}}
Gegeben sei die Matrix
\begin{align*}
	A =
	\begin{pmatrix}
		2 & 1 & 1 & 1 & 1 \\
		3 & 2 & 1 & 1 & 1 \\
		6 & 0 & 6 & 6 & 6 \\ 
		8 & 1 & 7 & 8 & 7
	\end{pmatrix}.
\end{align*}
Es folgt, dass
\renewcommand{\labelenumi}{(\alph{enumi})}
\begin{enumerate}
	\item 
	$ \mathrm{rg}(A) = 1 $.
	\item 
	$ \mathrm{rg}(A) = 2 $.
	\item
	$ \mathrm{rg}(A) = 3 $.
	\item
	$ \mathrm{rg}(A) = 4 $.
	\item
	$ \mathrm{rg}(A) = 5 $.
\end{enumerate}
\ \\
\textbf{Lösung:}
\begin{mdframed}
\underline{\textbf{Vorgehensweise:}}
\renewcommand{\labelenumi}{\theenumi.}
\begin{enumerate}
\item 
\end{enumerate}
\end{mdframed}

\underline{1. }\\

\newpage

\subsection*{\frage{6}{3}}
Gegeben sei die Matrix
\begin{align*}
	A =
	\begin{pmatrix}
		-7 & 0 & 0 \\
		1 & 2 & 0 \\
		-5m & 5 & 4
	\end{pmatrix}
	, 
	\quad 
	\textrm{mit } m \in \mathbb{R}.
\end{align*}
Für den Eigenwert $ \lambda = 4 $ ist der Eigenraum $ W $ gleich
\renewcommand{\labelenumi}{(\alph{enumi})}
\begin{enumerate}
	\item 
	$ W
	=
	\left\{
	\textbf{x} \in \mathbb{R}^3
	:
	\textbf{x}
	=
	s 
	\begin{pmatrix}
		0\\ 0 \\ -1
	\end{pmatrix},
	s \in \mathbb{R}
	\right\}
	$
	.
	\item 
	$ W
	=
	\left\{
	\textbf{x} \in \mathbb{R}^3
	:
	\textbf{x}
	=
	s	 
	\begin{pmatrix}
		7m \\ -1 \\ m-1
	\end{pmatrix},
	s \in \mathbb{R}
	\right\}
	$
	.
	\item
	$ W
	=
	\left\{
	\textbf{x} \in \mathbb{R}^3
	:
	\textbf{x}
	=
	s	 
	\begin{pmatrix}
		3 \\ -7 \\ 1 - 5m
	\end{pmatrix},
	s \in \mathbb{R}
	\right\}
	$
	.
	\item
	$ W
	=
	\left\{
	\textbf{x} \in \mathbb{R}^3
	:
	\textbf{x}
	=
	s	 
	\begin{pmatrix}
		1 \\ -3 \\ 1 - 3m
	\end{pmatrix},
	s \in \mathbb{R}
	\right\}
	$
	.
	\item
	$ \lambda = 4 $ ist kein Eigenwert von $ A $.
\end{enumerate}
\ \\
\textbf{Lösung:}
\begin{mdframed}
\underline{\textbf{Vorgehensweise:}}
\renewcommand{\labelenumi}{\theenumi.}
\begin{enumerate}
\item Verwende die Definition des Eigenwerts.
\end{enumerate}
\end{mdframed}

\underline{1. Verwende die Definition des Eigenwerts}\\
Der Vektor $ \textbf{x} $ heißt Eigenvektor zum Eigenwert $ -1 $, falls
\begin{align*}
	A \textbf{x} = - \textbf{x}
\end{align*}
für $ \textbf{x} \neq 0 $ gilt. Mit $ \textbf{x}_a $,..., $ \textbf{x}_d $ bezeichnen wir die Vektoren der Antwortmöglichkeiten mit $ s = 1 $. 


\newpage

\subsection*{\frage{7}{4}}
Welche der folgenden Differenzengleichungen wird durch die Folge
\begin{align*}
	y_{k} = \left(\frac{1}{4}\right)^{k+1} + 4, \quad k \in \mathbb{N},
\end{align*}
gelöst?
\renewcommand{\labelenumi}{(\alph{enumi})}
\begin{enumerate}
	\item
	$ 6 y_{k+1}  - 3 y_k = 12$.
	\item
	$ 4 y_{k+1} - 32 y_k = -26 $.	
	\item 
	$ 8 y_{k+1} - 2 y_k = 24 $
	\item
	Keine der obigen Antworten ist korrekt.
\end{enumerate}
\ \\
\textbf{Lösung:}
\begin{mdframed}
\underline{\textbf{Vorgehensweise:}}
\renewcommand{\labelenumi}{\theenumi.}
\begin{enumerate}
\item 
\end{enumerate}
\end{mdframed}

\underline{1. }\\

\newpage

\subsection*{\frage{8}{4}}
Gegeben sei die Differenzengleichung
\begin{align*}
	6y_{k+1} + 2 y_k +7 = 6
	\quad (k = 0,1,2,...).
\end{align*}
Es folgt, dass:
\renewcommand{\labelenumi}{(\alph{enumi})}
\begin{enumerate}
	\item
	$ (y_k)_{k \in \mathbb{N}} $ oszillierend und divergent ist.
	\item
	$ (y_k)_{k \in \mathbb{N}} $ monoton und divergent ist.	
	\item 
	$ (y_k)_{k \in \mathbb{N}} $ monoton und konvergent ist mit $ \lim \limits_{k \to \infty} y_k = -\frac{1}{7} $.
	\item
	$ (y_k)_{k \in \mathbb{N}} $ oszillierend und konvergent ist mit $ \lim \limits_{k \to \infty} y_k = \frac{1}{7} $.
	\item 
	$ (y_k)_{k \in \mathbb{N}} $ oszillierend und konvergent ist mit $ \lim \limits_{k \to \infty} y_k = -\frac{1}{8} $.
	\item 
	$ (y_k)_{k \in \mathbb{N}} $ monoton und konvergent ist mit $ \lim \limits_{k \to \infty} y_k = \frac{1}{8} $.
\end{enumerate}
\ \\
\textbf{Lösung:}
\begin{mdframed}
\underline{\textbf{Vorgehensweise:}}
\renewcommand{\labelenumi}{\theenumi.}
\begin{enumerate}
\item 
\end{enumerate}
\end{mdframed}

\underline{1. }\\
