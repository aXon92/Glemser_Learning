\fancyhead[C]{\normalsize\textbf{$\qquad$ Teil II: Multiple-Choice}}
\section*{Aufgabe 2 (32 Punkte)}
\vspace{0.4cm}
\subsection*{\frage{1}{4}}
Die Funktion $ f(x,y) \ = \ (x-2)^2 -y^2 $ hat einen Sattelpunkt beim Punkt $ P \ = \ (2,0) $. Welche der folgenden Aussagen ist korrekt?
\renewcommand{\labelenumi}{(\alph{enumi})}
\begin{enumerate}
	\item Die Funktion $ f $ hat unter der Nebenbedingung $ 2x \ - y \ - \ 4 \ = \ 0 $ ein lokales Minimum beim Punkt $ P $.
	\item Die Funktion $ f $ hat unter der Nebenbedingung $ 2x \ - y \ - \ 4 \ = \ 0 $ einen Sattelpunkt beim Punkt $ P $.
	\item Die Funktion $ f $ hat unter der Nebenbedingung $ -x \ - 2y \ - \ 4 \ = \ 0 $ einen Sattelpunkt beim Punkt $ P $.
	\item Die Funktion $ f $ hat unter der Nebenbedingung $ -\frac{1}{2}x \ - y \ + \ 1 \ = \ 0 $ ein lokales Minimum beim Punkt $ P $.
	\item Die Funktion $ f $ hat unter der Nebenbedingung $ -\frac{1}{2}x \ - y \ + \ 1 \ = \ 0 $ ein lokales Maximum beim Punkt $ P $.
\end{enumerate}
\ \\
\textbf{Lösung:}
\begin{mdframed}
\underline{\textbf{Vorgehensweise:}}
\renewcommand{\labelenumi}{\theenumi.}
\begin{enumerate}
\item 
\end{enumerate}
\end{mdframed}

\underline{1. }\\


\newpage

\subsection*{\frage{2}{3}}
Sei $ f \ : \ D_f \to \mathbb{R} $ eine positive Funktion zweier reeller Variablen, welche ein lokales Maximum beim Punkt $ P = (5,2)  $ hat.
Sei $ h  \ : \ D_f \to \mathbb{R} $ die Funktion definiert durch
\begin{align*}
	h(x,y) = e^{\sqrt{f(x,-y)}}.
\end{align*} 
Welche der folgenden Aussagen ist korrekt?
\renewcommand{\labelenumi}{(\alph{enumi})}
\begin{enumerate}
	\item Die Funktion $ h $ hat ein lokales Maximum beim Punkt $ Q = (-5,2). $
	\item Die Funktion $ h $ hat ein lokales Maximum beim Punkt $ Q = (5,-2). $
	\item Die Funktion $ h $ hat ein lokales Minimum beim Punkt $ Q = (-5,2). $
	\item Die Funktion $ h $ hat ein lokales Minimum beim Punkt $ Q = (5,-2). $
	\item Wir können die Extremstellen von $ h $ nicht analysieren, ohne mehr Informationen zur Funktion $ f $ zu haben.
\end{enumerate}
\ \\
\textbf{Lösung:}
\begin{mdframed}
	\underline{\textbf{Vorgehensweise:}}
	\renewcommand{\labelenumi}{\theenumi.}
	\begin{enumerate}
		\item 
	\end{enumerate}
\end{mdframed}
\underline{1. }\\

\newpage
\subsection*{\frage{3}{3}}
Sei $ F $ die Stammfunktion der stetigen Funktion $ f $ einer reellen Variable $ x $ und sei $ g $ eine lineare Funktion, d.h. $ g(x) = ax +b $ für die Parameter $ a,b \in \mathbb{R} $. Das unbestimmte Integral
\begin{align*}
	\int f(x) g(x) \ dx.
\end{align*}
ist gleich:
\renewcommand{\labelenumi}{(\alph{enumi})}
\begin{enumerate}
	\item 
	$ a x F(x) + b F(x) - a \int F(x) dx$.
	\item 
	$ g(x) F(x) - a \int f(x) dx$.
	\item 
	$ \left(\frac{1}{2} a x^2 +b x\right) F(x) - \int a F(x) dx$.
	\item
	$ \left(\frac{1}{2} a x^2 +b x\right) F(x) - \int  F(x) g(x) dx$.
	\item 
	Wir können das Integral nicht transformieren, ohne mehr Informationen zur Funktion $ f $ zu haben.
\end{enumerate}
\ \\
\textbf{Lösung:}
\begin{mdframed}
\underline{\textbf{Vorgehensweise:}}
\renewcommand{\labelenumi}{\theenumi.}
\begin{enumerate}
\item 
\end{enumerate}
\end{mdframed}

\underline{1. }\\


\newpage

\subsection*{\frage{4}{3}}
Sei $ f $ eine \textit{gerade} Funktion, welche die folgenden (Integral-) Gleichungen erfüllt:
\begin{align*}
	\int_{2}^3 f(x) \ dx = 10, \ \textrm{und} \
	\int_1^3 f(x) \ dx = 15.
\end{align*}
Das bestimmte Integral
\begin{align*}
	\int_{-2}^{-1} (2 - f(x) ) \ dx
\end{align*}  
gleich
\renewcommand{\labelenumi}{(\alph{enumi})}
\begin{enumerate}
	\item 
	$ -1 $.
	\item
	$ -2 $.
	\item
	$ -3 $.
	\item
	$ -4 $.
	\item
	Wir haben nicht genügend Informationen, um diesen Wert auszurechnen.
\end{enumerate}
\ \\
\textbf{Lösung:}
\begin{mdframed}
\underline{\textbf{Vorgehensweise:}}
\renewcommand{\labelenumi}{\theenumi.}
\begin{enumerate}
\item 
\end{enumerate}
\end{mdframed}

\underline{1. }\\

\newpage
\subsection*{\frage{5}{4}}
Gegeben seien die Matrizen $ A_{n \times m} $ und $ B_{m \times n} $ für $ m,n \in \mathbb{N} $.\\
\\
Es folgt:
\renewcommand{\labelenumi}{(\alph{enumi})}
\begin{enumerate}
	\item 
	Die Matrix $ (AB)^\top $ hat die Dimensionen $ n \times m $.
	\item 
	Die Matrix $ (AB)^\top $ ist symmetrisch, falls $ A $ symmetrisch ist.
	\item 
	Die Matrix $ (AB)^\top $ ist invertierbar, falls $ \mathrm{rg}(A) = \mathrm{rg}(B) = n $.
	\item 
	Die Matrix $ (AB)^\top $ ist invertierbar, falls $ BA $ invertierbar ist.
	\item 
	Die Matrix $ (AB)^\top $ ist invertierbar, falls sie $ n $ reelle Eigenwerte besitzt, welche ungleich null sind.
\end{enumerate}
\ \\
\textbf{Lösung:}
\begin{mdframed}
\underline{\textbf{Vorgehensweise:}}
\renewcommand{\labelenumi}{\theenumi.}
\begin{enumerate}
\item 
\end{enumerate}
\end{mdframed}

\underline{1. }\\

\newpage

\subsection*{\frage{6}{3}}
Gegeben sei die $ (n \times n ) $-dimensionale, reguläre Matrix $ A $ mit einem reellen Eigenwert $ \lambda \neq 0 $.\\
\\
Es folgt, dass die Matrix
\begin{align*}
	A^{-4} = A^{-1} \cdot A^{-1} \cdot A^{-1} \cdot A^{-1}
\end{align*}
den folgenden Eigenwert hat:
\renewcommand{\labelenumi}{(\alph{enumi})}
\begin{enumerate}
	\item 
	$ \frac{1}{\lambda^4} $.
	\item 
	$ \frac{4}{\lambda} $.
	\item
	$ \lambda$.
	\item
	$\lambda^3 $.
	\item 
	Es ist nicht möglich, eine Aussage bezüglich der Eigenwerte von $ A^{-4} $ zu treffen.
\end{enumerate}
\ \\
\textbf{Lösung:}
\begin{mdframed}
\underline{\textbf{Vorgehensweise:}}
\renewcommand{\labelenumi}{\theenumi.}
\begin{enumerate}
\item 
\end{enumerate}
\end{mdframed}

\underline{1. }\\


\newpage
\subsection*{\frage{7}{3}}
Gegeben sei das lineare Gleichungssystem
\begin{align*}
	A \textbf{x} = \textbf{b}.
\end{align*}
\textit{Keine} notwendige Annahme, um das lineare Gleichungssystem mithilfe der Cramerschen Regel zu lösen zu können, ist:
\renewcommand{\labelenumi}{(\alph{enumi})}
\begin{enumerate}
	\item 
	$ A $ ist quadratisch.
	\item
	$ A $ hat vollen Rang.
	\item
	$ \det(A) \neq 0$.
	\item
	$ \textbf{b} \neq \textbf{0} $.
	\item
	$ \mathrm{rg}(A) = \mathrm{rg}([A,\textbf{b}])$.
\end{enumerate}
\ \\
\textbf{Lösung:}
\begin{mdframed}
\underline{\textbf{Vorgehensweise:}}
\renewcommand{\labelenumi}{\theenumi.}
\begin{enumerate}
\item 
\end{enumerate}
\end{mdframed}

\underline{1. }\\

\newpage

\subsection*{\frage{8}{3}}
Sei $ A $ eine $ (1 \times 5) $-Matrix definiert als
\begin{align*}
	A
	=
	\left(
	\textbf{a}_1,
	\textbf{a}_2,
	\textbf{a}_3,
	\textbf{a}_4,
	\textbf{a}_5
	\right).
\end{align*}
für $ a_i \in \mathbb{R}, \ i = 1,2,...,5 $.\\
\\
Es folgt, dass:
\renewcommand{\labelenumi}{(\alph{enumi})}
\begin{enumerate}
	\item 
	$ \mathrm{rg}(A) = 5 $, falls $ a_i \neq 0 $ für alle $ i \in \{1,2,3,4,5\} $ und $ a_i \neq a_j $ für $ i \neq j $.
	\item
	$ \mathrm{rg}(A) = 5 $, falls $ a_i \neq 0 $ für mindestens ein $ i \in \{1,2,3,4,5\} $.
	\item
	$ \mathrm{rg}(A) = 1 $, falls $ a_i \neq 0 $ für mindestens ein $ i \in \{1,2,3,4,5\} $.
	\item
	$ \mathrm{rg}(A) = 0 $, falls $ a_i = 0 $ für genau ein $ i \in \{1,2,3,4,5\} $.
	\item
	$1 <  \mathrm{rg}(A)\leq 5 $, falls $ a_i \neq 0 $ für mindestens ein $ i \in \{1,2,3,4,5\} $.
	\item 
	Es ist nicht möglich, mit den gegebenen Informationen eine Aussage zum Rang von $ A $ zu machen.
\end{enumerate}
\ \\
\textbf{Lösung:}
\begin{mdframed}
\underline{\textbf{Vorgehensweise:}}
\renewcommand{\labelenumi}{\theenumi.}
\begin{enumerate}
\item Verwende die Definition des Ranges.
\end{enumerate}
\end{mdframed}

\underline{1. Verwende die Definition des Ranges}\\
Der Rang einer Matrix $ A $ ist definiert durch
\begin{align*}
	\mathrm{rg}(A) := \ \text{Anzahl der linear unabhängigen Spalten von $ A $}.
\end{align*}
Da $ A $ eine $ (1 \times 5) $ Matrix ist, sind die einzelnen Spalten Skalare.
Damit kann der Rang höchstens $ 1 $ sein und ist genau dann $ 1 $, wenn mindestens einer dieser Skalare ungleich $ 0 $ ist.\\
\\
Also ist die Antwort (c) korrekt.\\
\\
Man kann den Rang einer Matrix auch als die Anzahl der linear unabhängigen Zeilen definieren. Diese ist äquivalent zu der Definition über die linear unabhängigen Spalten. Sei $ \textbf{v} := A^\top $ der Zeilenvektor von $ A $. Die Definition der lineare Unabhängigkeit über einen Vektor ist: 
\begin{align*}
	\lambda \textbf{v} = \textbf{0} \ \Rightarrow \ \lambda = 0.
\end{align*}
Diese kann nur erfüllt sein, falls mindestens ein Eintrag von $ \textbf{v} $ ungleich $ 0 $ ist. 


\newpage
\subsection*{\frage{9}{3}}
Sei $ A $ eine $ (4 \times 2) $-dimensionale Matrix mit Rang $ 2 $ und $ B $ eine Matrix mit den Dimensionen mit Rang $ 1 $.\\
\\
Wir definieren die Matrix $ C $ als die Aneinanderreihung der Spalten von $ A $ und $ B $ wie folgt:
\begin{align*}
	C = [A, B, A].
\end{align*}
Es folgt, dass:
\renewcommand{\labelenumi}{(\alph{enumi})}
\begin{enumerate}
	\item 
	$ \mathrm{rg}(C) = 2 $.
	\item
	$ \mathrm{rg}(C) = 3 $.
	\item
	$ \mathrm{rg}(C)\in \{2,3\} $.
	\item
	$ \mathrm{rg}(C) = 4 $.
	\item
	Die gegebenen Informationen sind nicht ausreichend, um eine Aussage über den Rang von $ C $ treffen zu können.
\end{enumerate}
\ \\
\textbf{Lösung:}
\begin{mdframed}
	\underline{\textbf{Vorgehensweise:}}
	\renewcommand{\labelenumi}{\theenumi.}
	\begin{enumerate}
		\item 
	\end{enumerate}
\end{mdframed}

\underline{1. }\\



\newpage
\subsection*{\frage{10}{3}}
Sei $ A \in \mathbb{R}^{n\times n} $ eine quadratische obere Dreiecksmatrix, d.h. eine quadratische Matrix deren Elemente unterhalb der Diagonale null sind.\\
\\
Welche der folgenden Aussagen ist \textit{nicht} korrekt bezüglich der Matrix $ A $?
\renewcommand{\labelenumi}{(\alph{enumi})}
\begin{enumerate}
	\item 
	Die Determinante ist das Produkt der Elemente auf der Diagonale.
	\item
	Die Matrix $ A + A^\top $ ist symmetrisch.
	\item
	Falls die Inverse $ A^{-1} $ existiert, so ist sie ebenfalls eine obere Dreiecksmatrix.
	\item
	Die Eigenwerte von $ A $ entsprechen den Elementen auf der Diagonale
	\item 
	Die Eigenvektoren von $ A $ entsprechen den Zeilen von $ A $.
\end{enumerate}
\ \\
\textbf{Lösung:}
\begin{mdframed}
	\underline{\textbf{Vorgehensweise:}}
	\renewcommand{\labelenumi}{\theenumi.}
	\begin{enumerate}
		\item 
	\end{enumerate}
\end{mdframed}

\underline{1. }\\

