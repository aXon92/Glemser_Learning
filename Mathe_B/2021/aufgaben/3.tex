\section*{Aufgabe 3 (30 Punkte)}
\vspace{0.4cm}
\subsection*{\frage{1}{3}}
Die Funktion
\begin{align*}
	f(x,y) = 2^{xy}
\end{align*}
unter der Nebenbedingung
\begin{align*}
	\varphi(x,y)
	=
	x - 2 y - 2
	= 
	0
\end{align*}
hat
\renewcommand{\labelenumi}{(\alph{enumi})}
\begin{enumerate}
	\item 
	ein Minimum $ P = \left(1,-\frac{1}{2} \right) $.
	\item
	ein Maximum $ P = \left(1,-\frac{1}{2} \right) $.
	\item
	einen Sattelpunkt $ P = \left(1,-\frac{1}{2} \right) $.
	\item 
	Keine der obigen Aussagen ist richtig.
\end{enumerate}
\ \\
\textbf{Lösung:}
\begin{mdframed}
\underline{\textbf{Vorgehensweise:}}
\renewcommand{\labelenumi}{\theenumi.}
\begin{enumerate}
\item .
\end{enumerate}
\end{mdframed}

\underline{1. }\\


 
\newpage

\subsection*{\frage{2}{4}}
Sei $ a \in \mathbb{R}_{++} $. Das bestimmte Integral
\begin{align*}
	\int_0^1
	\frac{1}{x^2} e^{\frac{a}{x}} dx
\end{align*}
ist gleich
\renewcommand{\labelenumi}{(\alph{enumi})}
\begin{enumerate}
	\item 
	$a e^{-a} $.
	\item
	$\frac{1}{a^2} e^{-a}$.
	\item
	$e^{-a} $.
	\item
	$\frac{1}{a} e^{-a}$.	
	\item
	Das Integral existiert nicht.
\end{enumerate}
\ \\
\textbf{Lösung:}
\begin{mdframed}
\underline{\textbf{Vorgehensweise:}}
\renewcommand{\labelenumi}{\theenumi.}
\begin{enumerate}
\item .
\end{enumerate}
\end{mdframed}

\underline{1. }\\

\newpage
\subsection*{\frage{3}{4}}
Sei $ n \in \mathbb{N} $.
Das bestimmte Integral
\begin{align*}
	\int_0^1
	x(1-x)^n
	\ \td{x}
\end{align*}
ist gleich
\renewcommand{\labelenumi}{(\alph{enumi})}
\begin{enumerate}
	\item 
	$\frac{1}{n(n+1)}$.
	\item
	$\frac{n}{(n+1)(n+2)}$.
	\item
	$\frac{n}{n+1}$.
	\item
	$\frac{1}{n(n+2)}$.
	\item
	$\frac{1}{(n+1)(n+2)}$.	
\end{enumerate}
\ \\
\textbf{Lösung:}
\begin{mdframed}
\underline{\textbf{Vorgehensweise:}}
\renewcommand{\labelenumi}{\theenumi.}
\begin{enumerate}
\item .
\end{enumerate}
\end{mdframed}
%\allowdisplaybreaks
\underline{1. }\\
\newpage

\subsection*{\frage{4}{4}}
Gegeben ist die Matrix
\begin{align*}
	A
	=
	\begin{pmatrix}
		2 & a & -4 \\
		-1 & 3 & 4\\
		1 & -2 & -3
	\end{pmatrix}.
\end{align*}
Für welche Werte von $ a \in \mathbb{R} $ ist die Matrix $ A $
\textit{idempotent}, d.h., es gilt $ A^2 = A $?
\renewcommand{\labelenumi}{(\alph{enumi})}
\begin{enumerate}
	\item 
	$ a= 0 $.
	\item 
	$ a= -2 $.
	\item
	$ a= 2 $ oder $ a= -1 $.
	\item
	Es gibt kein $ a \in \mathbb{R} $, sodass $ A $ idempotent ist.
	\item
	$ A $ ist idempotent für jedes $ a \in \mathbb{R} $.
\end{enumerate}
\ \\
\textbf{Lösung:}
\begin{mdframed}
\underline{\textbf{Vorgehensweise:}}
\renewcommand{\labelenumi}{\theenumi.}
\begin{enumerate}
\item .
\end{enumerate}
\end{mdframed}

\underline{1. }\\
\newpage

\subsection*{\frage{5}{4}}
Gegeben ist die Matrix
\begin{align*}
	A =
	\begin{pmatrix}
		1 & 0 & 2 & 3 \\
		2 & 1 & r & 7 \\
		3 & 2 & 4 &  s 
	\end{pmatrix}.
\end{align*}
Für welche Werte von $ r \in \mathbb{R} $ und $ s \in \mathbb{R} $ besitzt die Matrix $ A $ den Rang $ 2 $?
\renewcommand{\labelenumi}{(\alph{enumi})}
\begin{enumerate}
	\item 
	$ r = 10 $ und $ s = 2 $.
	\item 
	$ r =  2 $ und $ s = 10 $.
	\item
	$ r = 2 $ und $ s $ beliebig.
	\item
	$ r = 3 $ und $ s = 11 $.
	\item
	Die Matrix $ A $ hat für alle $ (r,s) \in \mathbb{R}^2 $ \textit{nicht} den Rang $ 2 $.
\end{enumerate}
\ \\
\textbf{Lösung:}
\begin{mdframed}
\underline{\textbf{Vorgehensweise:}}
\renewcommand{\labelenumi}{\theenumi.}
\begin{enumerate}
\item .
\end{enumerate}
\end{mdframed}

\underline{1. }\\
\newpage

\subsection*{\frage{6}{4}}
Gegeben ist die Matrix
\begin{align*}
	A =
	\begin{pmatrix}
		0 & 3 & 0 \\
		1 & 2 & 0 \\
		-5m & 5 & 4
	\end{pmatrix}
	, 
	\quad 
	\textrm{mit } m \in \mathbb{R}.
\end{align*}
Für den Eigenwert $ \lambda = -1 $ ist der Eigenraum $ W $ gegeben durch
\renewcommand{\labelenumi}{(\alph{enumi})}
\begin{enumerate}
	\item 
	$ W
	=
	\left\{
	\textbf{x} \in \mathbb{R}^3
	:
	\textbf{x}
	=
	s 
	\begin{pmatrix}
		3 \\ -1 \\ 1 - 3m
	\end{pmatrix},
	s \in \mathbb{R}
	\right\}
	$
	.
	\item 
	$ W
	=
	\left\{
	\textbf{x} \in \mathbb{R}^3
	:
	\textbf{x}
	=
	s	 
	\begin{pmatrix}
		-3 \\ 1 \\ -1 - 3m
	\end{pmatrix},
	s \in \mathbb{R}
	\right\}
	$
	.
	\item
	$ W
	=
	\left\{
	\textbf{x} \in \mathbb{R}^3
	:
	\textbf{x}
	=
	s	 
	\begin{pmatrix}
		1 \\ -3 \\ 1 - 3m
	\end{pmatrix},
	s \in \mathbb{R}
	\right\}
	$
	.
	\item
	$ W
	=
	\left\{
	\textbf{x} \in \mathbb{R}^3
	:
	\textbf{x}
	=
	s	 
	\begin{pmatrix}
		3m \\ -1 \\ m-1
	\end{pmatrix},
	s \in \mathbb{R}
	\right\}
	$
	.
	\item
	$ \lambda = -1 $ ist kein Eigenwert von $ A $.
\end{enumerate}
\ \\
\textbf{Lösung:}
\begin{mdframed}
\underline{\textbf{Vorgehensweise:}}
\renewcommand{\labelenumi}{\theenumi.}
\begin{enumerate}
\item .
\end{enumerate}
\end{mdframed}

\underline{1. }\\
\newpage

\subsection*{\frage{7}{3}}
Von der Lösung $ \{y_k\}_{k \in \mathbb{N}_0} $ der Differenzengleichung
\begin{align*}
	y_{k+1} = A y_k + B \quad (k = 0,1,2,...)
\end{align*}
sind die Werte
\begin{align*}
	y_0 = 2, y_1 = 1 \ \textrm{und} \ y_2 = 3
\end{align*}
bekannt.\\
\\
Bestimmen Sie $ A $ und $ B $.
\renewcommand{\labelenumi}{(\alph{enumi})}
\begin{enumerate}
	\item
	$ A=1 $ und $ B=2 $.
	\item
	$ A=2 $ und $ B=1 $.	
	\item 
	$ A=-1 $ und $ B=4 $.
	\item
	$ A=1 $ und $ B=2 $.
	\item
	$ A=-2 $ und $ B=5 $.
\end{enumerate}
\ \\
\textbf{Lösung:}
\begin{mdframed}
\underline{\textbf{Vorgehensweise:}}
\renewcommand{\labelenumi}{\theenumi.}
\begin{enumerate}
\item Stelle ein lineares Gleichungssystem auf.
\end{enumerate}
\end{mdframed}

\underline{1. Stelle ein lineares Gleichungssystem auf}\\
Die bekannten Werte ergeben die beiden Gleichungen
\begin{align*}
	 y_1 = A y_0 + B 
	\ &\Leftrightarrow \
	1 = 2 A + B\\
	y_2 = A y_1 + B 
	\ &\Leftrightarrow \
	3 = A + B.
\end{align*}
Dies ist ein lineares Gleichungssystem in den Variablen $ A $ und $ B $. Das Aufstellen der Koeffizientenmatrix und Anwenden elementarer Zeilenumforungen ergibt:

\begin{align*}
	\begin{pmatrix}
		2 & 1 & \BAR & 1\\
		1 & 1 & \BAR & 3
	\end{pmatrix}
	\leadsto
	\begin{pmatrix}
		1 & 0 & \BAR & -2\\
		1 & 1 & \BAR & 3
	\end{pmatrix}
	\leadsto
	\begin{pmatrix}
		1 & 0 & \BAR & -2\\
		0 & 1 & \BAR & 5
	\end{pmatrix}
	\ \Rightarrow A = -2, B = 5.
\end{align*}
Damit ist die Anwort (e) korrekt.\\
\\
Alternativ lässt sich das LGS auch durch Einsetzen lösen.
\newpage

\subsection*{\frage{8}{4}}
Die allgemeine Lösung der  Differenzengleichung
\begin{align*}
	(1+a) y_{k+1} + y_k -1 = 0
	\quad (k = 0,1,2,...)
\end{align*}
mit $ a \in \mathbb{R} \setminus \{-1,-2\} $, ist genau dann monoton und konvergent, wenn
\renewcommand{\labelenumi}{(\alph{enumi})}
\begin{enumerate}
	\item
	$ a > 0 $.
	\item
	$ a >0 $ oder $ a < -1 $.	
	\item 
	$ -2 < a < 0 $.
	\item
	$ a < -2 $.
	\item 
	Die allgemeine Lösung $ \{y_k\}_{k \in \mathbb{N}_0} $ ist nie monoton und konvergent.
\end{enumerate}
\ \\
\textbf{Lösung:}
\begin{mdframed}
\underline{\textbf{Vorgehensweise:}}
\renewcommand{\labelenumi}{\theenumi.}
\begin{enumerate}
\item .
\end{enumerate}
\end{mdframed}

\underline{1. }\\
