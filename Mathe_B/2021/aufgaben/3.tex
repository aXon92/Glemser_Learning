\section*{Aufgabe 3 (30 Punkte)}
\vspace{0.4cm}
\subsection*{\frage{1}{3}}
Die Funktion
\begin{align*}
	f(x,y) = 2^{xy}
\end{align*}
unter der Nebenbedingung
\begin{align*}
	\varphi(x,y)
	=
	x - 2 y - 2
	= 
	0
\end{align*}
hat
\renewcommand{\labelenumi}{(\alph{enumi})}
\begin{enumerate}
	\item 
	ein Minimum $ P = \left(1,-\frac{1}{2} \right) $.
	\item
	ein Maximum $ P = \left(1,-\frac{1}{2} \right) $.
	\item
	einen Sattelpunkt $ P = \left(1,-\frac{1}{2} \right) $.
	\item 
	Keine der obigen Aussagen ist richtig.
\end{enumerate}
\ \\
\textbf{Lösung:}
\begin{mdframed}
\underline{\textbf{Vorgehensweise:}}
\renewcommand{\labelenumi}{\theenumi.}
\begin{enumerate}
\item Forme die Nebenbedingung um und verwende die Monotonie der Exponentialfunktion.
\end{enumerate}
\end{mdframed}

\underline{1. Forme die Nebenbedingung um und verwende die Monotonie der Exponentialfunktion}\\
Die Nebenbedingung lässt sich durch
\begin{align*}
	\varphi(x,y) = x - 2y -2 = 0
	\ \Leftrightarrow \
	x = 2 y  +2 
\end{align*}
umformen. Hiermit lässt sich die Funktion $ f $ unter der Nebenbedingung $ \varphi $ folgendermaßen darstellen:
\begin{align*}
	F(y) = f(2y+2,y) = 2^{(2y +2) y }
	=2^{2y^2 + 2y}.
\end{align*}
Wir setzen $ g(y) := 2y^2 +2y $. Da die Exponentialfunktion $ 2^{(\cdot)} $ streng monoton wachsend ist, bleiben Extremstellen von $ g $ durch $ 2^{g(y)} $ erhalten.\\
Die Funktion $ g $ ist eine nach oben geöffnete Parabel. Demnach besitzt diese ein Minimum an ihrem Scheitelpunkt.
Wegen 
\begin{align*}
	g(y) = 2 y^2 +2 y
	= 2 y (y +1)
\end{align*}
besitzt $ g $ die Nullstellen $ -1 $ und $ 0 $. Wegen der Symmetrie der Parabel liegt der Scheitelpunkt und somit das Minimum an $ y^\star = -\frac{1}{2} $. Das Einsetzten von $ y^\star $ in die Nebenbedingung liefert:
\begin{align*}
	x = 2 y^\star + 2 = -1 + 2 = 1.
\end{align*}
Somit hat $ f $ unter der Nebenbedingung $ \varphi $ ein Minimum am Punkt $ P = \left(1, -\frac{1}{2}\right) $.\\
\\
Also ist die Antwort (a) korrekt.


 
\newpage

\subsection*{\frage{2}{4}}
Sei $ a \in \mathbb{R}_{++} $. Das bestimmte Integral
\begin{align*}
	\int_0^1
	\frac{1}{x^2} e^{\frac{a}{x}} dx
\end{align*}
ist gleich
\renewcommand{\labelenumi}{(\alph{enumi})}
\begin{enumerate}
	\item 
	$a e^{-a} $.
	\item
	$\frac{1}{a^2} e^{-a}$.
	\item
	$e^{-a} $.
	\item
	$\frac{1}{a} e^{-a}$.	
	\item
	Das Integral existiert nicht.
\end{enumerate}
\ \\
\textbf{Lösung:}
\begin{mdframed}
\underline{\textbf{Vorgehensweise:}}
\renewcommand{\labelenumi}{\theenumi.}
\begin{enumerate}
\item Bestimme die Stammfunktion durch Substitution.
\item Berechne das bestimmte Integral.
\end{enumerate}
\end{mdframed}

\underline{1. Bestimme die Stammfunktion durch Substitution}\\
Wir formen das zugehörige unbestimmte Integral zunächst um:
\begin{align*}
	\int \frac{1}{x^2} e^{-\frac{a}{x}} \td{x}
	=
	\frac{1}{a} \int \frac{a}{x^2} e^{-\frac{a}{x}} \td{x}.
\end{align*}
Wir substituieren nun $ t = -\frac{a}{x} $. Hierfür gilt:
\begin{align*}
	\frac{\td{t}}{\td{x}} = \frac{a}{x^2} \ \Rightarrow
	\ 
	\td{t} =\frac{a}{x^2} \td{x} .
\end{align*}
Damit erhalten wir
\begin{align*}
	\frac{1}{a} \int \frac{a}{x^2} e^{-\frac{a}{x}} \td{x}
	=
	\frac{1}{a} \int e^{t} \td{t}
	=
	\frac{1}{a} e^t + C
	=
	\frac{1}{a} e^{- \frac{a}{x}} +C.
\end{align*}
\ \\
\underline{2. Berechne das bestimmte Integral}\\
Das bestimmte Integral berechnet sich nun wegen $ a > 0  $ durch:
\begin{align*}
	\int_0^1
	\frac{1}{x^2} e^{\frac{a}{x}} dx
	=
	\lim
	\limits_{N\to 0^+}
	\int_N^1
	\frac{1}{x^2} e^{\frac{a}{x}} dx
	=
	\lim 
	\limits_{N\to 0^+}
	\left[\frac{1}{a} e^{- \frac{a}{x}}\right]_N^1
	=
	\frac{1}{a}
	\lim
	\limits_{N\to 0^+} \left(e^{- \frac{a}{1}}  - \underbrace{e^{- \frac{a}{N}}}_{\to 0}\right)
	=
	\frac{1}{a}  e^{-a}.
\end{align*}
Damit ist die Antwort (d) korrekt.

\newpage
\subsection*{\frage{3}{4}}
Sei $ n \in \mathbb{N} $.
Das bestimmte Integral
\begin{align*}
	\int_0^1
	x(1-x)^n
	\ \td{x}
\end{align*}
ist gleich
\renewcommand{\labelenumi}{(\alph{enumi})}
\begin{enumerate}
	\item 
	$\frac{1}{n(n+1)}$.
	\item
	$\frac{n}{(n+1)(n+2)}$.
	\item
	$\frac{n}{n+1}$.
	\item
	$\frac{1}{n(n+2)}$.
	\item
	$\frac{1}{(n+1)(n+2)}$.	
\end{enumerate}
\ \\
\textbf{Lösung:}
\begin{mdframed}
\underline{\textbf{Vorgehensweise:}}
\renewcommand{\labelenumi}{\theenumi.}
\begin{enumerate}
\item .
\end{enumerate}
\end{mdframed}
%\allowdisplaybreaks
\underline{1. }\\
\newpage

\subsection*{\frage{4}{4}}
Gegeben ist die Matrix
\begin{align*}
	A
	=
	\begin{pmatrix}
		2 & a & -4 \\
		-1 & 3 & 4\\
		1 & -2 & -3
	\end{pmatrix}.
\end{align*}
Für welche Werte von $ a \in \mathbb{R} $ ist die Matrix $ A $
\textit{idempotent}, d.h., es gilt $ A^2 = A $?
\renewcommand{\labelenumi}{(\alph{enumi})}
\begin{enumerate}
	\item 
	$ a= 0 $.
	\item 
	$ a= -2 $.
	\item
	$ a= 2 $ oder $ a= -1 $.
	\item
	Es gibt kein $ a \in \mathbb{R} $, sodass $ A $ idempotent ist.
	\item
	$ A $ ist idempotent für jedes $ a \in \mathbb{R} $.
\end{enumerate}
\ \\
\textbf{Lösung:}
\begin{mdframed}
\underline{\textbf{Vorgehensweise:}}
\renewcommand{\labelenumi}{\theenumi.}
\begin{enumerate}
\item Bestimme $ A^2 $ und überlege, wann $ A $ idempotent ist.
\end{enumerate}
\end{mdframed}

\underline{1. Bestimme $ A^2 $ und überlege, wann $ A $ idempotent ist}\\
Es gilt
\begin{align*}
	A^2 = A \cdot A
	&=
	\begin{pmatrix}
		2 & a & -4 \\
		-1 & 3 & 4\\
		1 & -2 & -3
	\end{pmatrix}
	\cdot
	\begin{pmatrix}
		2 & a & -4 \\
		-1 & 3 & 4\\
		1 & -2 & -3
	\end{pmatrix}\\
	&=
	\begin{pmatrix}
		4-a-4 & 2a +3a+8 & -8+4a +12\\
		-2-3+4 & -a+9 -8 & 4 +12 -12\\
		2 +2 -3 & a-6 +6 & -4-8 + 9
	\end{pmatrix}
	=
	\begin{pmatrix}
		-a & 5a+8 & 4a +4\\
		-1 & -a+1 & 4 \\
		1 & a & -3
	\end{pmatrix}. 
\end{align*}
Damit $ A $ idempotent ist, muss
\begin{align*}
	\begin{pmatrix}
		2 & a & -4 \\
		-1 & 3 & 4\\
		1 & -2 & -3
	\end{pmatrix}
	=
	\begin{pmatrix}
		-a & 5a+8 & 4a +4\\
		-1 & -a+1 & 4 \\
		1 & a & -3
	\end{pmatrix}
\end{align*}
erfüllt sein. Wegen dem ersten Eintrag der ersten Spalte und dem dritten Eintrag der zweiten Spalte, muss $ a = -2 $ gelten. Für die Idempotenz müssen wir noch die weiteren Bedingungen überprüfen:
\begin{align*}
	a &= 5a +8 \ \overset{a=-2}{\Rightarrow} \
	-2 = 5 (-2) +8 = -10 +8 = -2\\
	-4 &= 4a +4 \ \overset{a=-2}{\Rightarrow} \
	-4 = 4 (-2) + 4 = -8 +4 = -4\\
	3 &= -a +1 \ \overset{a=-2}{\Rightarrow} \
	3 = -(-2) +1 = 2 +1 = 3.
\end{align*}
Damit ist die Matrix $ A $ für $ a=-2 $ idempotent.\\
\\
Also ist die Antwort (b) korrekt.
\newpage

\subsection*{\frage{5}{4}}
Gegeben ist die Matrix
\begin{align*}
	A =
	\begin{pmatrix}
		1 & 0 & 2 & 3 \\
		2 & 1 & r & 7 \\
		3 & 2 & 4 &  s 
	\end{pmatrix}.
\end{align*}
Für welche Werte von $ r \in \mathbb{R} $ und $ s \in \mathbb{R} $ besitzt die Matrix $ A $ den Rang $ 2 $?
\renewcommand{\labelenumi}{(\alph{enumi})}
\begin{enumerate}
	\item 
	$ r = 10 $ und $ s = 2 $.
	\item 
	$ r =  2 $ und $ s = 10 $.
	\item
	$ r = 2 $ und $ s $ beliebig.
	\item
	$ r = 3 $ und $ s = 11 $.
	\item
	Die Matrix $ A $ hat für alle $ (r,s) \in \mathbb{R}^2 $ \textit{nicht} den Rang $ 2 $.
\end{enumerate}
\ \\
\textbf{Lösung:}
\begin{mdframed}
\underline{\textbf{Vorgehensweise:}}
\renewcommand{\labelenumi}{\theenumi.}
\begin{enumerate}
\item Bringe die Matrix in Zeilenstufenform.
\end{enumerate}
\end{mdframed}

\underline{1. Bringe die Matrix in Zeilenstufenform}\\
Das Anwenden elementarer Zeilenoperationen ergibt:
\begin{align*}
	\begin{gmatrix}[p]
		1 & 0 & 2 & 3 \\
		2 & 1 & r & 7 \\
		3 & 2 & 4 &  s
		\rowops
		\add[ \cdot (-2)]{0}{1}
		\add[ \cdot (-3)]{0}{2}	
	\end{gmatrix}
	&\leadsto
	\begin{gmatrix}[p]
		1 & 0 & 2 & 3 \\
		0 & 1 & r-4 & 1 \\
		0 & 2 & -2 &  s -9
		\rowops
		\add[ \cdot (-2)]{1}{2}
	\end{gmatrix}\\
	&\leadsto
	\begin{gmatrix}[p]
		1 & 0 & 2 & 3 \\
		0 & 1 & r-4 & 1 \\
		0 & 0 & -2r  + 6 &  s -11
	\end{gmatrix}.
\end{align*}
Damit $ \mathrm{rg}(A) = 2 $ gilt, muss $ A $ eine Nullzeile besitzen. Dies ist erfüllt, falls 
\begin{align*}
	-2 r + 6 = 0 \ &\Leftrightarrow \ r = 3\\
	s - 11 = 0 \ &\Leftrightarrow \ s = 11
\end{align*}
gilt.\\
\\
Also ist die Antwort (d) korrekt.
\newpage

\subsection*{\frage{6}{4}}
Gegeben ist die Matrix
\begin{align*}
	A =
	\begin{pmatrix}
		0 & 3 & 0 \\
		1 & 2 & 0 \\
		-5m & 5 & 4
	\end{pmatrix}
	, 
	\quad 
	\textrm{mit } m \in \mathbb{R}.
\end{align*}
Für den Eigenwert $ \lambda = -1 $ ist der Eigenraum $ W $ gegeben durch
\renewcommand{\labelenumi}{(\alph{enumi})}
\begin{enumerate}
	\item 
	$ W
	=
	\left\{
	\textbf{x} \in \mathbb{R}^3
	:
	\textbf{x}
	=
	s 
	\begin{pmatrix}
		3 \\ -1 \\ 1 - 3m
	\end{pmatrix},
	s \in \mathbb{R}
	\right\}
	$
	.
	\item 
	$ W
	=
	\left\{
	\textbf{x} \in \mathbb{R}^3
	:
	\textbf{x}
	=
	s	 
	\begin{pmatrix}
		-3 \\ 1 \\ -1 - 3m
	\end{pmatrix},
	s \in \mathbb{R}
	\right\}
	$
	.
	\item
	$ W
	=
	\left\{
	\textbf{x} \in \mathbb{R}^3
	:
	\textbf{x}
	=
	s	 
	\begin{pmatrix}
		1 \\ -3 \\ 1 - 3m
	\end{pmatrix},
	s \in \mathbb{R}
	\right\}
	$
	.
	\item
	$ W
	=
	\left\{
	\textbf{x} \in \mathbb{R}^3
	:
	\textbf{x}
	=
	s	 
	\begin{pmatrix}
		3m \\ -1 \\ m-1
	\end{pmatrix},
	s \in \mathbb{R}
	\right\}
	$
	.
	\item
	$ \lambda = -1 $ ist kein Eigenwert von $ A $.
\end{enumerate}
\ \\
\textbf{Lösung:}
\begin{mdframed}
\underline{\textbf{Vorgehensweise:}}
\renewcommand{\labelenumi}{\theenumi.}
\begin{enumerate}
\item Verwende die Definition des Eigenwerts.
\end{enumerate}
\end{mdframed}

\underline{1. Verwende die Definition des Eigenwerts}\\
Der Vektor $ \textbf{x} $ heißt Eigenvektor zum Eigenwert $ -1 $, falls
\begin{align*}
	A \textbf{x} = - \textbf{x}
\end{align*}
für $ \textbf{x} \neq 0 $ gilt. Mit $ \textbf{x}_a $,..., $ \textbf{x}_d $ bezeichnen wir die Vektoren der Antwortmöglichkeiten mit $ s = 1 $. 
Es gilt:
\begin{align*}
	A \cdot \textbf{x}_a
	&=
	\begin{pmatrix}
		0 & 3 & 0 \\
		1 & 2 & 0 \\
		-5m & 5 & 4
	\end{pmatrix} 
	\cdot 
	\begin{pmatrix}
		3 \\ -1 \\ 1-3m
	\end{pmatrix}
	=
	\begin{pmatrix}
		-3 \\ 1 \\ -15m - 5 +4 -12m
	\end{pmatrix}
	=
	\begin{pmatrix}
		-3 \\ 1 \\ -1 -27m
	\end{pmatrix}
	\neq - \textbf{x}_a
	\\
	A \cdot \textbf{x}_b
	&=
	\begin{pmatrix}
		0 & 3 & 0 \\
		1 & 2 & 0 \\
		-5m & 5 & 4
	\end{pmatrix} 
	\cdot 
	\begin{pmatrix}
		-3 \\ 1 \\ -1-3m
	\end{pmatrix}
	=
	\begin{pmatrix}
		3 \\ -1 \\ 15 m +5 -4 -12m
	\end{pmatrix}
	=
	\begin{pmatrix}
		3 \\ -1 \\ 1 +3m
	\end{pmatrix}
	= - \textbf{x}_b.
\end{align*} 
Damit ist $ \textbf{x}_b $ ein Eigenvektor zum Eigenwert $ -1 $.\\
\\
Also ist die Antwort (b) korrekt.\\
\\
Alternativ lässt sich die korrekte Antwort auch über das Lösen des LGS $ (A - (-1) \cdot I) = 0 $ finden.

\newpage

\subsection*{\frage{7}{3}}
Von der Lösung $ \{y_k\}_{k \in \mathbb{N}_0} $ der Differenzengleichung
\begin{align*}
	y_{k+1} = A y_k + B \quad (k = 0,1,2,...)
\end{align*}
sind die Werte
\begin{align*}
	y_0 = 2, y_1 = 1 \ \textrm{und} \ y_2 = 3
\end{align*}
bekannt.\\
\\
Bestimmen Sie $ A $ und $ B $.
\renewcommand{\labelenumi}{(\alph{enumi})}
\begin{enumerate}
	\item
	$ A=1 $ und $ B=2 $.
	\item
	$ A=2 $ und $ B=1 $.	
	\item 
	$ A=-1 $ und $ B=4 $.
	\item
	$ A=1 $ und $ B=2 $.
	\item
	$ A=-2 $ und $ B=5 $.
\end{enumerate}
\ \\
\textbf{Lösung:}
\begin{mdframed}
\underline{\textbf{Vorgehensweise:}}
\renewcommand{\labelenumi}{\theenumi.}
\begin{enumerate}
\item Stelle ein lineares Gleichungssystem auf.
\end{enumerate}
\end{mdframed}

\underline{1. Stelle ein lineares Gleichungssystem auf}\\
Die bekannten Werte ergeben die beiden Gleichungen
\begin{align*}
	 y_1 = A y_0 + B 
	\ &\Leftrightarrow \
	1 = 2 A + B\\
	y_2 = A y_1 + B 
	\ &\Leftrightarrow \
	3 = A + B.
\end{align*}
Dies ist ein lineares Gleichungssystem in den Variablen $ A $ und $ B $. Das Aufstellen der Koeffizientenmatrix und Anwenden elementarer Zeilenumforungen ergibt:

\begin{align*}
	\begin{pmatrix}
		2 & 1 & \BAR & 1\\
		1 & 1 & \BAR & 3
	\end{pmatrix}
	\leadsto
	\begin{pmatrix}
		1 & 0 & \BAR & -2\\
		1 & 1 & \BAR & 3
	\end{pmatrix}
	\leadsto
	\begin{pmatrix}
		1 & 0 & \BAR & -2\\
		0 & 1 & \BAR & 5
	\end{pmatrix}
	\ \Rightarrow A = -2, B = 5.
\end{align*}
Damit ist die Anwort (e) korrekt.\\
\\
Alternativ lässt sich das LGS auch durch Einsetzen lösen.
\newpage

\subsection*{\frage{8}{4}}
Die allgemeine Lösung der  Differenzengleichung
\begin{align*}
	(1+a) y_{k+1} + y_k -1 = 0
	\quad (k = 0,1,2,...)
\end{align*}
mit $ a \in \mathbb{R} \setminus \{-1,-2\} $, ist genau dann monoton und konvergent, wenn
\renewcommand{\labelenumi}{(\alph{enumi})}
\begin{enumerate}
	\item
	$ a > 0 $.
	\item
	$ a >0 $ oder $ a < -1 $.	
	\item 
	$ -2 < a < 0 $.
	\item
	$ a < -2 $.
	\item 
	Die allgemeine Lösung $ \{y_k\}_{k \in \mathbb{N}_0} $ ist nie monoton und konvergent.
\end{enumerate}
\ \\
\textbf{Lösung:}
\begin{mdframed}
\underline{\textbf{Vorgehensweise:}}
\renewcommand{\labelenumi}{\theenumi.}
\begin{enumerate}
\item .
\end{enumerate}
\end{mdframed}

\underline{1. }\\
