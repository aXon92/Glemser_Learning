\fancyhead[C]{\normalsize\textbf{$\qquad$ Teil II: Multiple-Choice}}
\section*{Aufgabe 2 (33 Punkte)}
\vspace{0.4cm}
\subsection*{\frage{1}{3}}
In welchem Punkt hat die Funktion
\begin{align*}
	f(x,y) = 5x
\end{align*}
unter der Nebenbedingung
\begin{align*}
	\varphi(x,y) = 2 y^2 - x - 1 = 0
\end{align*}
ein Minimum?
\renewcommand{\labelenumi}{(\alph{enumi})}
\begin{enumerate}
	\item $ P= (-2,\sqrt{0.5}) $.
	\item $ P= (0,\sqrt{0.5}) $.
	\item $ P= (7,2) $.
	\item $ P= (-1,0) $.
	\item Keiner der oben gegebenen Punkte ist ein Minimum der Funktion $ f $ unter der Nebenbedingung $ \varphi(x,y)= 0 $.
\end{enumerate}
\ \\
\textbf{Lösung:}
\begin{mdframed}
\underline{\textbf{Vorgehensweise:}}
\renewcommand{\labelenumi}{\theenumi.}
\begin{enumerate}
\item Überlege dir, welche Punkte die Nebenbedingung erfüllen.
\item Nutze die Funktion (Darstellung einer Ellipse).
\end{enumerate}
\end{mdframed}

\underline{1. Überlege dir, welche Punkte die Nebenbedingung erfüllen}\\


\newpage

\subsection*{\frage{2}{3}}
Sei $ f $ eine zweimal stetig partiell differenzierbare Funktion zweier Variablen. Für den stationären Punkt $ (x_0,y_0) $ gilt:
\begin{align*}
	f_{xx}(x_0,y_0) < f_{yy}(x_0,y_0) < f_{xy}(x_0,y_0) < 0.
\end{align*} 
Dann gilt:
\renewcommand{\labelenumi}{(\alph{enumi})}
\begin{enumerate}
	\item $ f $ hat in $ (x_0,y_0) $ ein lokales Minimum.
	\item $ f $ hat in $ (x_0,y_0) $ ein lokales Maximum.
	\item $ f $ hat in $ (x_0,y_0) $ einen Sattelpunkt.
	\item Die Information genügt nicht, um zu entscheiden, ob $ f $ in $ (x_0,y_0) $ einen Sattelpunkt oder ein lokales Extremum hat.
\end{enumerate}
\ \\
\textbf{Lösung:}
\begin{mdframed}
	\underline{\textbf{Vorgehensweise:}}
	\renewcommand{\labelenumi}{\theenumi.}
	\begin{enumerate}
		\item .
	\end{enumerate}
\end{mdframed}
\underline{1. }\\
\newpage
\subsection*{\frage{3}{4}}
Die Funktion $ f $ sei gegeben durch
\begin{align*}
	f(x) = \int_2^{x^2} e^{\sin(t)} \ dt.
\end{align*}
Welcher der folgenden Ausdrücke gibt die Ableitung $ f^\prime(x) $ wieder?
\renewcommand{\labelenumi}{(\alph{enumi})}
\begin{enumerate}
	\item 
	$ f^\prime(x) = e^{\sin(x^2)} \sin(x^2)$.
	\item 
	$ f^\prime(x) = 2 x e^{\sin(x^2)}$.
	\item 
	$ f^\prime(x) = e^{\sin(x^2)}$.
	\item
	$ f^\prime(x) = 2 x e^{\sin(x^2)} \sin(x^2)$.
	\item 
	$ f^\prime(x) = x^2 e^{\sin(x^2)}$ 
	\item 
	Keiner der obigen Ausdrücke ist korrekt.
\end{enumerate}
\ \\
\textbf{Lösung:}
\begin{mdframed}
\underline{\textbf{Vorgehensweise:}}
\renewcommand{\labelenumi}{\theenumi.}
\begin{enumerate}
\item .
\end{enumerate}
\end{mdframed}

\underline{1. }\\

\newpage

\subsection*{\frage{4}{3}}
Sei $ f $ eine \textit{ungerade} Funktion, welche die folgenden drei (Integral-) Gleichungen erfüllt:
\begin{align*}
	\int_{-1}^9 f(x) \ dx = 15, \
	\int_4^9 f(x) \ dx = 5 \ \textrm{und} \
	\int_1^6 f(x) \ dx = 7.
\end{align*}
Dann ist der Wert des bestimmten Integrals
\begin{align*}
	\int_4^6 f(x) \ dx
\end{align*}  
gleich
\renewcommand{\labelenumi}{(\alph{enumi})}
\begin{enumerate}
	\item 
	$ 0 $.
	\item
	$ 3 $.
	\item
	$ 2 $.
	\item
	$ -3 $.
	\item
	Die Information genügt nicht, um den Wert des Integrals zu bestimmen.
\end{enumerate}\ \\
\textbf{Lösung:}
\begin{mdframed}
\underline{\textbf{Vorgehensweise:}}
\renewcommand{\labelenumi}{\theenumi.}
\begin{enumerate}
\item .
\end{enumerate}
\end{mdframed}

\underline{1. }\\
 

\newpage
\subsection*{\frage{5}{4}}
Gegeben ist die Funktion
\begin{align*}
	f(x) =
	\begin{cases}
		2 x c e^{1 - x^2}, \ &x \geq c  \geq 0\\
		\qquad	\ 0 , \ &x < c
	\end{cases}.
\end{align*}
$ f $ ist eine Dichtefunktion für
\renewcommand{\labelenumi}{(\alph{enumi})}
\begin{enumerate}
	\item 
	$ c = 0 $.
	\item 
	$ c = \frac{1}{e} $.
	\item 
	$ c= 1 $.
	\item 
	$ c = 2e $.
	\item
	alle $ c \in \mathbb{R}_{++} $.
\end{enumerate}
\ \\
\textbf{Lösung:}
\begin{mdframed}
\underline{\textbf{Vorgehensweise:}}
\renewcommand{\labelenumi}{\theenumi.}
\begin{enumerate}
\item .
\end{enumerate}
\end{mdframed}

\underline{1. }\\
 

\newpage

\subsection*{\frage{6}{3}}
Die quadratischen Matrizen $ A_{n \times n} $ und $ B_{n \times n} $ seien regulär.
Der Ausdruck
\begin{align*}
	(A B)^T(B^{-1} A^{-1})^T B (AB)^{-1}
\end{align*}
ist gleich
\renewcommand{\labelenumi}{(\alph{enumi})}
\begin{enumerate}
	\item 
	$ A^{-1} B $.
	\item 
	$ I $, die $ (n \times n) $ Einheitsmatrix.
	\item
	$ AB^{-1} $.
	\item
	$ B $.
	\item 
	$ A^{-1} $
\end{enumerate}
\ \\
\textbf{Lösung:}
\begin{mdframed}
\underline{\textbf{Vorgehensweise:}}
\renewcommand{\labelenumi}{\theenumi.}
\begin{enumerate}
\item .
\end{enumerate}
\end{mdframed}

\underline{1. }\\


\newpage
\subsection*{\frage{7}{2}}
\begin{center}
	\includegraphics[width=0.7\textwidth]{pictures/aufgabe2_7}
\end{center}
Der Flächeninhalt $ A $ des Parallelogramms in der obigen Figur ist gleich
\renewcommand{\labelenumi}{(\alph{enumi})}
\begin{enumerate}
	\item 
	$ A = 10 $.
	\item
	$ A = 14 $.
	\item
	$ A = 5 \cdot  \sqrt{17} $.
	\item
	$ A = 16 $.
	\item
	$ A = 13 $.
\end{enumerate}
\ \\
\textbf{Lösung:}
\begin{mdframed}
\underline{\textbf{Vorgehensweise:}}
\renewcommand{\labelenumi}{\theenumi.}
\begin{enumerate}
\item .
\end{enumerate}
\end{mdframed}

\underline{1. }\\

\newpage

\subsection*{\frage{8}{3}}
$ A $ sei eine $ (5 \times 5) $-Matrix gegeben durch
\begin{align*}
	A
	=
	\left[
	\textbf{a}_1,
	\textbf{a}_2,
	\textbf{a}_3,
	\textbf{a}_4,
	\textbf{a}_5
	\right].
\end{align*}
Sei 
\begin{align*}
	B =
	\left[
	\textbf{a}_1 - 3 \textbf{a}_2,
	\textbf{a}_2,
	\textbf{a}_3 + \textbf{a}_2 - \textbf{a}_5,
	-2 \textbf{a}_4,
	\textbf{a}_5
	\right].
\end{align*}
Dann gilt
\renewcommand{\labelenumi}{(\alph{enumi})}
\begin{enumerate}
	\item 
	$ \det(B) = -2 \det(A) $.
	\item
	$ \det(B) = \det(A) $.
	\item
	$ \det(B) = 6 \det(A) $.
	\item
	$ \det(B) = -\det(A) $.
	\item
	Es gibt im Allgemeinen keine Relation zwischen $ \det(A) $ und $ \det(B) $.
\end{enumerate}
\ \\
\textbf{Lösung:}
\begin{mdframed}
\underline{\textbf{Vorgehensweise:}}
\renewcommand{\labelenumi}{\theenumi.}
\begin{enumerate}
\item .
\end{enumerate}
\end{mdframed}

\underline{1. }\\

\newpage
\subsection*{\frage{9}{2}}
Die Matrix $ A $ ist eine $ (7 \times 4) $-Matrix mit Rang $ 4 $.
Man bekommt die Matrix $ B $, indem man bei $ A $ die dritte Spalte streicht.\\
\\
Dann folgt:
\renewcommand{\labelenumi}{(\alph{enumi})}
\begin{enumerate}
	\item 
	$ \mathrm{rg}(B) = 4 $.
	\item
	$ \mathrm{rg}(B) = 3 $.
	
	\item
	$ \mathrm{rg}(B) < 3 $.
	
	\item
	Die Informationen reichen nicht aus, den Rang von $ B $ zu bestimmen.
\end{enumerate}
\ \\
\textbf{Lösung:}
\begin{mdframed}
	\underline{\textbf{Vorgehensweise:}}
	\renewcommand{\labelenumi}{\theenumi.}
	\begin{enumerate}
		\item . 
	\end{enumerate}
\end{mdframed}

\underline{1. }\\

\newpage
\subsection*{\frage{10}{3}}
Welche der folgenden Aussagen ist \textit{falsch}?
\renewcommand{\labelenumi}{(\alph{enumi})}
\begin{enumerate}
	\item 
	Die Summe von zwei regulären Matrizen ist regulär
	\item
	Das Produkt von zwei singulären Matrizen ist singulär.
	
	\item
	Der Rang einer Matrix ist gleich dem Rang der Transponierten.
	\item
	Der Rang einer $ (n \times m) $-Matrix ist kleiner gleich $ n $, unabhängig davon, was $ m \in \mathbb{N} $ ist.
	\item 
	Die Summe zweier singulärer Matrizen kann regulär sein.
\end{enumerate}
\ \\
\textbf{Lösung:}
\begin{mdframed}
	\underline{\textbf{Vorgehensweise:}}
	\renewcommand{\labelenumi}{\theenumi.}
	\begin{enumerate}
		\item Bestimme die richtige Lösung mit dem Eliminationsverfahren. 
	\end{enumerate}
\end{mdframed}

\underline{1. Bestimme die richtige Lösung mit dem Eliminationsverfahren}\\
Die Differenzengleichung ist in Normalform.
Wegen $ |A| < 1 $ konvergiert die Folge gegen $ y^\star $.
Damit ist Antwort (a) falsch.
Wir setzen $ B = 0 $. Dann konvergiert $ y_k $ gegen $ 0 $ und es gilt $ y_k = A^k y_0 $.
Damit hängt das Vorzeichen von der Anfangsbedingung $ y_0 $ ab.
Für $ y_0 < 0  $ ist (b) falsch und für $ y_0 > 0 $ ist (c) falsch.
Also bleibt Antwort (d) übrig.\\
\\
Somit ist die Antwort (d) korrekt.
