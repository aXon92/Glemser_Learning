\vspace{1cm}
\fancyhead[C]{\normalsize\textbf{$\qquad$ Teil I: Offene Aufgaben}}
\renewcommand{\labelenumi}{\theenumi.}
\section*{Aufgabe 1 (40 Punkte)}
\vspace{0.4cm}
\subsection*{\aufgabe{a}{10}}
Ein Monopolist stellt zwei Güter her.
In Abhängigkeit von den Preisen $ p_1 $ und $ p_2 $ je Mengeneinheit von Gut $ 1 $ bzw. Gut $ 2 $ lautet die
\begin{align*}
	&\textrm{Absatzmenge für Gut $ 1 $:} \quad 
	x = 30 - 3 p_1 + 2 p_2\\
	&\textrm{Absatzmenge für Gut $ 2 $:} \quad
	y = 20 + p_1 - p_2.
\end{align*}
\begin{enumerate}
	\item[\textbf{(a1)}]
	Stellen Sie den Gesamtumsatz $ U(x,y) $ nur durch $ x $ und $ y  $ dar.
	\item[\textbf{(a2)}] 
	Die Produktionskosten lauten
	\begin{align*}
		K(x,y) = x^2 + xy + y^2.
	\end{align*}
	Bei welchen Produktionsmengen $ x^\ast  $ und $ y^\ast $ wird der Reingewinn $ G(x,y) $ maximal?\\
	Wie gross ist der maximale Reingewinn?
\end{enumerate}
\ \\

\textbf{Lösung:}
\begin{mdframed}
\underline{\textbf{Vorgehensweise:}}
\renewcommand{\labelenumi}{\theenumi.}
\begin{enumerate}
\item  
\end{enumerate}
\end{mdframed}
\underline{1. }\\






\newpage

\subsection*{\aufgabe{b}{14}}
Je nach Konjunktur (gut, mittel, schlecht) haben drei Aktien unterschiedliche zu erwartende Auszahlungen.
\begin{table}[H]
	\centering
	%
	\begin{tabular}{|l |c |c |c|}
		\hline
		\multirow{3}{*}{Konjunktur}
		& erwartete Auszahlung	& erwartete Auszahlung & erwartete Auszahlung\\
		& pro Einheit von & pro Einheit von & pro Einheit von\\
		& Aktie 1 & Aktie 2	& Aktie 3\\
		\hline
		gut & $ 2 $  &  $ a $ &  $ 5 $ \\ 
		\hline
		mittel & $ 1 $ & $ 1 $ & $ 1 $  \\ 
		\hline
		schlecht & $ 0 $ & $ 1 $ & $ a $ \\
		\hline
	\end{tabular}%
\end{table}
Gesucht sind geeignete Kombinationen dieser Geldanlagen (Käufe, respektive Leerverkäufe), die bei guter Konjunktur die Auszahlung $ 7'000 $, bei mittlerer Konjunktur die Auszahlung $ 5'000 $ und bei schlechter Konjunktur die Auszahlung $ 1'000 $ ergeben.
\begin{enumerate}
	\item[\textbf{(b1)}]
	Bestimmen Sie ein lineares Gleichungssysteme für die benötigten Einheiten $ x_i \in \mathbb{R} $ der Anteile an Aktie $ i \in \{1,2,3\} $, die man haben muss, um bei jeder Konjunkturlage die gewünschte Auszahlung erreichen zu können.
	\item[\textbf{(b2)}] 
	Für welche Werte von $ a \in \mathbb{R} $ gibt es eine eindeutige Kombination?
	Für welche Werte von $ a \in \mathbb{R} $ gibt es keine mögliche Kombination?
	Für welche Werte von $ a \in \mathbb{R} $ gibt es unendlich viele mögliche Kombinationen?
	\item[\textbf{(b3)}]
	Berechnen Sie im Falle einer eindeutigen Lösung die Lösung $ (x_1,x_2,x_3) $ (in Abhängigkeit von $ a $).
	\item[(b4)]
	Beschreiben Sie die Lösungsgesamtkeit der möglichen Kombinationen, wenn es unendlich viele Lösungen gibt.
\end{enumerate}
\ \\
\textbf{Lösung:}
\begin{mdframed}
\underline{\textbf{Vorgehensweise:}}
\renewcommand{\labelenumi}{\theenumi.}
\begin{enumerate}
\item 
\end{enumerate}
\end{mdframed}

\underline{1. }\\


\newpage
\subsection*{\aufgabe{c}{10}}
Für einen elliptischen Tisch mit den Halbachsen $ a > 0  $ und $ b > 0 $ ist eine flächenmässig möglichst grosse rechteckige Auflage anzufertigen, die an keiner Stelle übersteht.
\begin{enumerate}
	\item[\textbf(c1)]
	Welche Ausmasse (Länge und Breite) hat die rechteckige Auflage?
	\item[\textbf(c2)]
	Wie gross ist die maximale Fläche?
\end{enumerate}
\textit{Wichtige Bemerkung}: Eine Abklärung, ob es sich wirklich um ein Maximum handelt, wird \textbf{nicht} verlangt, weil es offensichtlich ein Maximum geben muss.
\\ \\
\textbf{Lösung:}
\begin{mdframed}
\underline{\textbf{Vorgehensweise:}}
\begin{enumerate}
\item 
\end{enumerate}
\end{mdframed}

\underline{1. }\\

\newpage
\subsection*{\aufgabe{d}{6}}
\begin{enumerate}
	\item[\textbf{(d1)}]
	Gesucht sind alle (positiven) Funktionen $ f(x) $, welche für alle $ x > 0 $ die Elastizität
	\begin{align*}
		\varepsilon_f(x) = \frac{1}{x}
	\end{align*}
	besitzen.\\
	\textit{Tipp:} Erinnern Sie sich, dass für positive Funktonen $ f(x) $ gilt:
	\begin{align*}
		\frac{d}{dx} \ln(f(x)) = \rho_f(x) 
		\ \textrm{und} \
		\varepsilon_f(x) = x \cdot \rho_f(x),
	\end{align*}
	wobei $ \rho_f $ die Wachstumsrate von $ f $ ist.
	\item[\textbf{(d2)}]
	Welche positive Funktion $ f(x) $ mit $ f(1) = 1 $ besitzt die Elastizität $ \varepsilon_f(x) = \frac{1}{x} $ für alle alle $ x >0 $.
\end{enumerate}
\ \\
\textbf{Lösung:}
\begin{mdframed}
\underline{\textbf{Vorgehensweise:}}
\begin{enumerate}
\item[]
\end{enumerate}
\end{mdframed}

\underline{1. }\\
\newpage

