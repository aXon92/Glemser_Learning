\section*{Aufgabe 3 (28 Punkte)}
\vspace{0.4cm}
\subsection*{\frage{1}{3}}
Sei $t > 0$. Das bestimmte Integral
\begin{align*}
	\int_1^e \frac{1}{x} \ln\left((tx)^2\right) \ dx
\end{align*}
entspricht:
\renewcommand{\labelenumi}{(\alph{enumi})}
\begin{enumerate}
	\item 
	$ \ln(t^2)  +1  $.
	\item
	$ 1 $.
	\item
	$ \ln(t^2) -1 $.
	\item 
	$\frac{1}{2} \ln(t^2)$.
	\item $4$.
\end{enumerate}
\ \\
\textbf{Lösung:}
\begin{mdframed}
\underline{\textbf{Vorgehensweise:}}
\renewcommand{\labelenumi}{\theenumi.}
\begin{enumerate}
\item 
\end{enumerate}
\end{mdframed}

\underline{1. }\\


 
\newpage

\subsection*{\frage{2}{4}}
Gegeben sei die Funktion
\begin{align*}
	f(x,y) = 2^x +  a e^{y^2}
\end{align*}
Für welches $a \in \mathbb{R}$ ist die Länge des Gradienten im Punkt $(x_0,y_0) = (0,1)$ gleich $4$?
\renewcommand{\labelenumi}{(\alph{enumi})}
\begin{enumerate}
	\item 
	$a = \sqrt{\frac{4 - \ln(2)^2}{4 e^2}} $.
	\item
	$a = \sqrt{\frac{16 - \ln(2)^2}{4 e^2}}$.
	\item
	$a = \sqrt{16 - \ln(2) + 4 e^2}$.
	\item
	$a = \frac{2 - \ln(2)}{2 e }$.
	\item 
	$a = \frac{4 - \ln(2)}{4e}$.
\end{enumerate}
\ \\
\textbf{Lösung:}
\begin{mdframed}
\underline{\textbf{Vorgehensweise:}}
\renewcommand{\labelenumi}{\theenumi.}
\begin{enumerate}
\item 
\end{enumerate}
\end{mdframed}

\underline{1. }\\



\newpage
\subsection*{\frage{3}{3}}
Gegeben sei die Funktion $f $ definiert als 
\begin{align*}
	f(x)
	=
	\begin{cases}
		0 \quad & x \geq 2 \pi\\
		a | \sin(x)  | \quad &0 \leq x < 2 \pi\\
		0 \quad &x<0
	\end{cases}.
\end{align*}
Für welchen Wert von $a \in \mathbb{R}$ ist $f$ eine Dichtefunktion?
\renewcommand{\labelenumi}{(\alph{enumi})}
\begin{enumerate}
	\item 
	$a = 0.5$.
	\item
	$a =0.25$.
	\item
	$a = 0.5 \pi $.
	\item
	$a = \pi$.
\end{enumerate}
\ \\
\textbf{Lösung:}
\begin{mdframed}
\underline{\textbf{Vorgehensweise:}}
\renewcommand{\labelenumi}{\theenumi.}
\begin{enumerate}
\item Verwende die Definition der Dichtefunktion.
\end{enumerate}
\end{mdframed}

\underline{1. Verwende die Definition der Dichtefunktion}\\
Eine nicht negative Funktion $f : \ \mathbb{R} \to \ \mathbb{R}$ heißt Dichtefunktion, falls
\begin{align*}
	\int \limits_{- \infty}^\infty f(x) \ dx = 1
\end{align*}
gilt. Für unsere gegebene Funktion $f$ erhalten wir den Ansatz 
\begin{align*}
	\int \limits_{- \infty}^\infty f(x) \ dx
	=
	\int \limits_{- \infty}^{0} \underbrace{f(x)}_{= 0} \ dx
	+
	\int \limits_{0}^{2 \pi} f(x) \ dx
	+
	\int \limits_{2 \pi}^\infty \underbrace{f(x)}_{= 0} \ dx
	=
	\int \limits_{0}^{2 \pi} a | \sin(x) | \ dx.
\end{align*}
Da für $f$ nicht negativ sein darf, d.h. $f(x) \geq 0$ für alle $x \in D_f$, muss wegen des Betrags auch $a \geq 0 $ gelten.
Wegen der Punktsymmetrie an $x_0  = \pi$ für $\sin(x)$ gilt
\begin{align*}
	\int \limits_{0}^{2 \pi} a | \sin(x) | \ dx
	&=
	a \int \limits_{0}^{2 \pi}  | \sin(x) | \ dx
	=
	a \cdot 2 \cdot  \int \limits_{0}^\pi \sin(x) \ dx\\
	&=
	2a \left[- \cos (x)\right]_{0}^\pi
	= 
	2a ( -\cos(\pi) - (- \cos(0)) )
	=
	2a ( 1 + 1 ) = 4a. 
\end{align*} 
Damit $f$ eine Dichtefunktion ist muss noch gelten: 
\begin{align*}
	\int \limits_{- \infty}^\infty f(x) \ dx  = 4a = 1  
	\ \Leftrightarrow \
	a = \frac{1}{4} = 0.25.
\end{align*}

Somit ist Antwort (b) korrekt.


\newpage

\newpage
\subsection*{\frage{4}{4}}
Die Funktion $g$ erfüllt $g(x) > -4$ für alle $x \in \mathbb{R}$ und $\int_{- \infty}^\infty g(x) \ dx = 3$.\\
Welche der folgenden Funktionen ist eine Dichtefunktionen?
\renewcommand{\labelenumi}{(\alph{enumi})}
\begin{enumerate}
	\item 
	$ a(x) = g(x) + 4$.
	\item 
	$ b(x) = \frac{g(x)}{3}$.
	\item
	$ c(x) = \frac{g(x) + 4}{3} $.
	\item
	$ d(x) = \frac{3}{4} g(x) + \frac{1}{4} $.
	\item
	Keine der obigen Antworten ist korrekt.
\end{enumerate}
\ \\
\textbf{Lösung:}
\begin{mdframed}
	\underline{\textbf{Vorgehensweise:}}
	\renewcommand{\labelenumi}{\theenumi.}
	\begin{enumerate}
		\item Verwende die Definition der Dichtefunktion
	\end{enumerate}
\end{mdframed}

\underline{1. Verwende die Definition der Dichtefunktion}\\
Eine nicht negative Funktion $f : \ \mathbb{R} \to \ \mathbb{R}$ heißt Dichtefunktion, falls
\begin{align*}
	\int \limits_{- \infty}^\infty f(x) \ dx = 1
\end{align*}
gilt.\\
\\
Die Funktion $g$ verletzt mit $g(x) > - 4$ (ggf.) die Nicht-Negativität einer Dichtefunktion. Deswegen können wir ohne weitere Informationen über $g$ keine Aussage treffen.\\
\\
Somit ist Antwort (e) korrekt.\\
\\
\textit{Anmerkung:} Das Ziel der Aufgabe ist die potentielle Verletzung der Nicht-Negativität zu erkennen. Für eine beliebige Dichtefunktion $f$ wäre die Aussage $f(x) \geq 0 > -4 $ für $x \in \mathbb{R}$ auch korrekt.\\
\\
\textit{Zusatzaufgabe:} Welche Antwort ist korrekt, falls $g $ eine Dichtefunktion ist?



\newpage

\subsection*{\frage{5}{3}}
Gegeben sei die Matrix
\begin{align*}
	A =
	\begin{pmatrix}
		2 & 1 & 0 & 4 & 0 \\
		3 & 2 & 1 & 9 & x \\
		6 & 0 & 6 & 18 & 6 
	\end{pmatrix}.
\end{align*}
Es folgt, dass
\renewcommand{\labelenumi}{(\alph{enumi})}
\begin{enumerate}
	\item 
	$ \mathrm{rg}(A) = 1 $.
	\item 
	$ \mathrm{rg}(A) = 2 $.
	\item
	$ \mathrm{rg}(A) = 3 $.
	\item
	der Rang von $A$ von $x$ abhängt.
	\item
	Keine der obigen Antworten ist korrekt.
\end{enumerate}
\ \\
\textbf{Lösung:}
\begin{mdframed}
\underline{\textbf{Vorgehensweise:}}
\renewcommand{\labelenumi}{\theenumi.}
\begin{enumerate}
\item Prüfe geeignete Spalten auf lineare Unabhängigkeit.
\end{enumerate}
\end{mdframed}

\underline{1. Prüfe geeignete Spalten auf lineare Unabhängigkeit}\\
Da die Matrix $A $ $3$ Zeilen und $5$ Spalten besitzt kann diese maximal einen Rang von $3$ besitzen.
Wegen 
\begin{align*}
	\det
	\begin{pmatrix}
		2 & 1 & 0\\
		3 & 2 & 1\\
		6 & 0 & 6
	\end{pmatrix}
		=
	2 \ 
	\det
	\begin{pmatrix}
		2 & 1 \\
		0 & 6
	\end{pmatrix}
	-
	\det
	\begin{pmatrix}
	3 & 1 \\
	6 & 6
	\end{pmatrix}
	= 
	2 \cdot 12 - ( 18 - 6 )
	=
	12 \neq 0
\end{align*}
sind die ersten drei Spalten der Matrix $A$ linear unabhängig und es gilt $\mathrm{rg}(A) = 3$.\\
\\
Damit ist Antwort (c) korrekt.



\newpage

\subsection*{\frage{6}{3}}
Gegeben sei die Matrix
\begin{align*}
	A =
	\begin{pmatrix}
		2 & -1 & 0 \\
		-1 & 2 & 0 \\
		-1 & 1 & 1
	\end{pmatrix}
	.
\end{align*}
Welcher der folgenden Vektoren ist \textit{kein} Eigenvektor von $A$?
\renewcommand{\labelenumi}{(\alph{enumi})}
\begin{enumerate}
	\item 
	$ 
	\mathbf{v}
	=
	\begin{pmatrix}
		0\\ 0 \\ 1
	\end{pmatrix}
	$
	.
	\item 
	$ 
	\mathbf{v}
	=
	\begin{pmatrix}
		1\\ 1 \\ 0
	\end{pmatrix}
	$
	.
	\item
	$ 
	\mathbf{v}
	=
	\begin{pmatrix}
		-1\\ 1 \\ 1
	\end{pmatrix}
	$
	.
	\item
	$ 
	\mathbf{v}
	=
	\begin{pmatrix}
		0\\ -1 \\ 1
	\end{pmatrix}
	$
	.
	\item
	$ \lambda = 4 $ ist kein Eigenwert von $ A $.
\end{enumerate}
\ \\
\textbf{Lösung:}
\begin{mdframed}
\underline{\textbf{Vorgehensweise:}}
\renewcommand{\labelenumi}{\theenumi.}
\begin{enumerate}
\item .
\end{enumerate}
\end{mdframed}

\underline{1. }\\



\newpage

\subsection*{\frage{7}{4}}
Maximilian zahlt $100'000$ Schweizer Franken auf ein Sparkonto ein, welches jährlich $2 \%$ Zins für die ersten $20'000$ Schweizer Franken und $1 \%$ Zins für den restlichen Betrag zahlt.\\
\\
Wie hoch sind seine Ersparnisse nach $10 $ Jahren?
\renewcommand{\labelenumi}{(\alph{enumi})}
\begin{enumerate}
	\item
	$108,769.80$ Schweizer Franken.
	\item
	$110,462.20$ Schweizer Franken.	
	\item 
	$111,242.25$ Schweizer Franken.
	\item
	$112,554.65$ Schweizer Franken.
	\item
	$112,749.70$ Schweizer Franken.
\end{enumerate}
\ \\
\textbf{Lösung:}
\begin{mdframed}
\underline{\textbf{Vorgehensweise:}}
\renewcommand{\labelenumi}{\theenumi.}
\begin{enumerate}
\item 
\end{enumerate}
\end{mdframed}

\underline{1. }\\

\newpage

\subsection*{\frage{8}{4}}
Gegeben sei die Differenzengleichung
\begin{align*}
	\frac{1}{3} y_k
	-
	2 y_{k+1}
	-3
	=0
	\quad (k = 0,1,2,...).
\end{align*}
Es folgt:
\renewcommand{\labelenumi}{(\alph{enumi})}
\begin{enumerate}
	\item
	$ (y_k)_{k \in \mathbb{N}} $ ist oszillierend und divergent.
	\item
	$ (y_k)_{k \in \mathbb{N}} $ ist monoton und divergent.	
	\item 
	$ (y_k)_{k \in \mathbb{N}} $ ist monoton und konvergent mit $ \lim \limits_{k \to \infty} y_k = -1.8$.
	\item
	$ (y_k)_{k \in \mathbb{N}} $ ist oszillierend und konvergent mit $ \lim \limits_{k \to \infty} y_k = 1.8$.
	\item 
	$ (y_k)_{k \in \mathbb{N}} $ ist oszillierend und konvergent mit $ \lim \limits_{k \to \infty} y_k = \frac{4}{5}$.
	\item 
	$ (y_k)_{k \in \mathbb{N}} $ ist monoton und konvergent mit $ \lim \limits_{k \to \infty} y_k = -\frac{4}{5}$.
\end{enumerate}
\ \\
\textbf{Lösung:}
\begin{mdframed}
\underline{\textbf{Vorgehensweise:}}
\renewcommand{\labelenumi}{\theenumi.}
\begin{enumerate}
\item Bringe die Differenzengleichung in die Normalform.
\end{enumerate}
\end{mdframed}

\underline{1. Bringe die Differenzengleichung in die Normalform}\\
Zunächst bringen wir die Differenzengleichung durch
\begin{align*}
	\frac{1}{3} y_k
	-
	2 y_{k+1}
	-3
	=0
	\ \Leftrightarrow \
	2 y_{k+1} 
	= 
	\frac{1}{3} y_k
	-3
	\ \Leftrightarrow \
	 y_{k+1} 
	= 
	\frac{1}{6} y_k
	-\frac{3}{2}
\end{align*}
in die Normalform. Damit liegt eine Differenzengleichung mit $A = \frac{1}{6}$ und $B = - \frac{3}{2}$ vor.
Wegen $0 < A < 1$ ist die Lösung der Differenzengleichung monoton und konvergent mit
\begin{align*}
	\lim \limits_{k \to \infty}
	y_k
	= 
	\frac{B}{1 -A} 
	=
	\frac{- \frac{3}{2}}{1 - \frac{1}{6}}
	=
	\frac{-\frac{3}{2}}{\frac{5}{6}}
	=
	- \frac{3}{2} \cdot \frac{6}{5}
	= 
	- \frac{3^2}{5}
	= 
	- \frac{9}{5} = - \frac{18}{10} = 1.8.
\end{align*}
Damit ist Antwort (c) korrekt.