%\newcommand{\ein}[2]{(#1) (#2 Punkte)}


\begin{Large}
\textbf{Teil I: Offene Fragen (40 Punkte)}
\end{Large}
\\
\\
\\
\textbf{Allgemeine Anweisungen für offene Fragen:}
\\
\renewcommand{\labelenumi}{(\roman{enumi})}
\begin{enumerate}
\item
Ihre Antworten müssen alle Rechenschritte enthalten,
diese müssen klar ersichtlich sein.
Verwendung korrekter mathematischer Notation wird erwartet
und fliesst in die Bewertung ein.

\item
Ihre Antworten zu den jeweiligen Teilaufgaben müssen in den dafür vorgesehenen Platz geschrie-
ben werden. Sollte dieser Platz nicht ausreichen, setzen Sie Ihre Antwort auf der Rückseite oder
dem separat zur Verfügung gestellten Papier fort. Verweisen Sie in solchen Fällen ausdrücklich
auf Ihre Fortsetzung. Bitte schreiben Sie zudem Ihren Vor- und Nachnamen auf jeden separaten
Lösungsbogen.

\item
Es werden nur Antworten im dafür vorgesehenen Platz bewertet. Antworten auf der Rückseite
oder separatem Papier werden nur bei einem vorhandenen und klaren Verweis darauf bewertet.

\item
Die Teilaufgaben werden mit den jeweils oben auf der Seite angegebenen Punkten bewertet.

\item
Ihre endgültige Lösung jeder Teilaufgabe darf nur eine einzige Version enthalten.

\item
Zwischenrechnungen und Notizen müssen auf einem getrennten Blatt gemacht werden. Diese
Blätter müssen, deutlich als Entwurf gekennzeichnet, ebenfalls abgegeben werden.
\end{enumerate}

\newpage
\section*{\hfil Aufgaben \hfil}
\vspace{1cm}
\section*{Aufgabe 1 (40 Punkte)}
\vspace{0.4cm}
\subsection*{\aufgabe{a}{12}}
Estis lang ersehnter Traum ist endlich wahr geworden
- 
sie hat sich für den Ironman Hawaii qualifiziert.
Allerdings weiss sie, dass sie ihr Training optimieren muss, um ihre bestmögliche Leistung erbringen zu können.
Esti beschliesst, als Steuerungsparameter ihren $VO_2max$-Wert zu nutzen.
Dieser misst die maximale Menge an Sauerstoff, die ihr Körper während der Belastung nutzen kann, und stellt somit eine hilfreiche Grösse dar, um die Verbesserung ihres Fitnesslevels abzubilden.
Sie möchte die optimale Verteilung ihrer Trainingszeit zwischen
Schwimmen, Radfahren und Laufen finden, um ihren $VO_2max$-Wert zu maximieren.\\
Estis derzeitiger $VO_2max$-Wert beträgt $48$ und ihre Trainingszeit ist auf $15$ Stunden pro Woche begrenzt.
Sei $x$ die Anzahl an Stunden pro Woche, die sie für Schwimmen und Radfahren zusammen aufwendet und $y$ die Anzahl an Stunden pro Woche, die sie mit Laufen verbringt.
Esti folgt der Empfehlung ihres Trainers und trainiert pro Woche viermal so viel Radfahren wie Schwimmen. Ausserdem nimmt sie an, dass sich der Trainingseffekt bezüglich ihres, am Tag des Ironman-Events gemessenen, $VO_2max$-Werts wie folgt modellieren lässt:
\begin{align*}
	f(x,y) = 1.8 x^{\frac{2}{3}} y^{\frac{1}{3}}.
\end{align*}
\begin{enumerate}
	\item[\textbf{(a1)}]
	Formulieren Sie das Optimierungsproblem, das Esti lösen muss, um ihren $VO_2max$-Wert am Tag des Events zu optimieren.
	\item[\textbf{(a2)}] 
	Lösen Sie das Optimierungsproblem und bestimmen Sie die optimale Anzahl an Stunden pro Woche, die Esti für Schwimmen, Radfahren und Laufen aufwenden sollte, sowie den $VO_2max$-Wert am Tag des Ironman-Events, wenn sie diesem Trainingsplan folgt.\\
	\\
	\textit{Anmerkung: Es ist nicht notwendig, die hinreichende Bedingung für einen Extremwert zu überprüfen.}
\end{enumerate}
\ \\
\subsection*{\aufgabe{b}{14}}
Vier Werkstücke $P_1$, $P_2$, $P_3$ und $P_4$ müssen in einem gegebenen Produktionsprozess jeweils drei Maschinen $M_1$, $M_2$ und $M_3$ durchlaufen.
Die für die einzelnen Werkstücke pro Einheit benötigten Bearbeitungszeiten in Minuten sind in der folgenden Tabelle angegeben: 
\begin{table}[H]
	\centering
	%
	\begin{tabular}{c | c c c c}
		$ $  & $P_1$  &  $P_2$ &  $P_3$ & $P_4$ \\ 
		\hline
		$ M_1 $ & $ 10 $ & $ 20 $ & $ 20 $ & $15$  \\ 
		$ M_2 $ & $ 10 $ & $ 30 $ & $ 16 $ & $5$ \\
		$ M_3 $ & $ 20 $ & $ 20 $ & $ 10 $ & $25$
	\end{tabular}%
\end{table}
Jede Maschine läuft genau $8$ Stunden pro Tag.\\
\\
Der Nettogewinn in Schweizer Franken für die vier Werkstücke beträgt pro Einheit der Reihe nach
\begin{align*}
	q_1 = 5; \quad
	q_2=  10; \quad
	q_3=12; \quad 
	q_4=m.
\end{align*}
Der angestrebte Nettogewinn aus der Gesamtproduktion beträgt $250$ Schweizer Franken pro Tag.
\begin{enumerate}
	\item[\textbf{(b1)}]
	Beschreiben Sie die Beziehung zwischen Produktionszeiten, Laufzeit je Maschine pro Tag, Nettogewinn und der Anzahl der produzierten Werkstücke pro Tag in einem Gleichungssystem.

	\item[\textbf{(b2)}] 
	Für welche(s) $m > 0$ existiert ein realisierbarer Produktionsplan?
	\item[\textbf{(b3)}]
	Bestimmen Sie den Produktionsplan, d.h. die Anzahl produzierter Einheiten jedes Werkstücks unter Berücksichtigung der verfügbaren Kapazitäten der Maschinen und des angestrebten Nettogewinns, für $m = 5$.
\end{enumerate}
\ \\
\subsection*{\aufgabe{c}{14}}
Nach der letzten Preiserhöhung bei Strom möchten Sie in Solarpanele investieren und den erzeugten Strom verkaufen. 
Sie zahlen $10'000$ Schweizer Franken für den Kauf und die
Installation der Solarpanele.
Unter Berücksichtigung des kontinuierlich schlechter werdenden Wirkungsgrads nehmen Sie konservativ an, dass sich die Leistungsabgabe der Solarpanele, gemessen in Kilowattstunden pro Jahr, mit der Zeit wie folgt entwickelt:
\begin{align*}
	f(t) = 10'0000 e^{-0.1 t},
\end{align*}
wobei $t$ in Jahren gemessen wird.
Sie gehen davon aus, dass die Lebensdauer der Solarpanele $20$ Jahre beträgt. Nach diesen $20$ Jahren müssen Sie die Solarpanele zu einem Preis von $10'000$ Schweizer Franken entsorgen. Ausserdem nehmen Sie an, dass sich der Marktpreis in
Schweizer Franken für eine Kilowattstunde mit der Zeit wie folgt entwickelt:
\begin{align*}
	p(t)
	=
	0.3 e^{0.01t}.
\end{align*}
Derzeit sind die Zinssätze eher hoch, aber Sie sind sich sicher, dass sie nach einer Weile wieder niedriger werden. Sie gehen folglich für die nächsten $5$ Jahre von einer kontinuierlichen
Verzinsung zu einem Satz von $4 \%$ p.a. und anschliessend zu einem Satz von $2\%$ p.a. aus.

\begin{enumerate}
	\item[\textbf{(c1)}] 
	Bestimmen Sie die insgesamt erzeugte Energiemenge.
	\item[\textbf{(c2)}]
	Berechnen Sie den Nettobarwert Ihrer Investition.
\end{enumerate}

\newpage


\fancyhead[C]{\normalsize\textbf{$\qquad$ Teil II: Multiple-Choice}}
\begin{Large}
\textbf{Teil II: Multiple-Choice-Fragen (60 Punkte)}
\end{Large}
\\
\\
\\
\textbf{Allgemeine Anweisungen für Multiple-Choice-Fragen:}
\\
\renewcommand{\labelenumi}{(\roman{enumi})}
\begin{enumerate}
\item
Die Antworten auf die Multiple-Choice-Fragen müssen im dafür vorgesehenen Antwortbogen ein-
getragen werden. Es werden ausschliesslich Antworten auf diesem Antwortbogen bewertet. Der
Platz unter den Fragen ist nur für Notizen vorgesehen und wird nicht korrigiert.

\item
Jede Frage hat nur eine richtige Antwort. Es muss also auch jeweils nur eine Antwort angekreuzt
werden.

\item
Falls mehrere Antworten angekreuzt sind, wird die Antwort mit 0 Punkten bewertet, auch wenn
die korrekte Antwort unter den angekreuzten ist.

\item
Bitte lesen Sie die Fragen sorgfältig.

\end{enumerate}
\newpage
\section*{Aufgabe 2 (32 Punkte)}
\vspace{0.4cm}
\subsection*{\frage{1}{3}}
Welcher der folgenden mathematischen Ausdrücke beschreibt das nachfolgende Problem:
\begin{displayquote}
	Finde den Punkt $P = (x,y)$ auf der Hyperbel $x^2 - 9 y^2 = 16$ mit dem kleinsten
	Abstand zum Punkt $P_0 = (0,-3)$.
\end{displayquote}
 \renewcommand{\labelenumi}{(\alph{enumi})}
\begin{enumerate}
\item 
$\min_{x,y} D(x,y) = x^2 - 9y^2 - 16$ so, dass
$\sqrt{x^2 + (y-3)^2} = 0$.
\item 
$\min_{x,y} D(x,y) = \sqrt{x^2 + ( y + 3)^2}$ so, dass
$x^2 - 9 y^2 - 16  = 0$.
\item 
$\max_{x,y} D(x,y) = 16 - x^2 + 9y^2 $ so, dass
$\sqrt{x^2 + (y+3)^2} = 0$.
\item 
$\min_{x,y} D(x,y) = \sqrt{x^2 + ( y - 3)^2}$ so, dass
$x^2 - 9 y^2 - 16  = 0$.
\item 
$\min_{x,y} D(x,y) = x^2 - 9y^2 - 16$ so, dass
$\sqrt{x^2 + (y+3)^2} = 0$.
\item
$\max_{x,y} D(x,y) = \sqrt{x^2 + ( y - 3)^2}$ so, dass
$x^2 - 9 y^2 - 16  = 0$.
\end{enumerate}
\ \\
\subsection*{\frage{2}{4}}
Sei $ f \ : \ D_f \to \mathbb{R} $ eine ungerade Funktion zweier reellen Variablen, d.h., $f(x,y) = - f(-x,-y)$ für alle $(x,y) \in D_f$, mit einem lokalen Maximum, welches nicht am Nullpunkt $(0,0)$ liegt.\\
\\ 
Welche der folgenden Aussagen ist korrekt?
\renewcommand{\labelenumi}{(\alph{enumi})}
\begin{enumerate}
\item Die Funktion $f$ hat mindestens ein lokales Minimum.
\item Die Funktion $f$ hat mindestens einen Sattelpunkt.
\item Die Funktion $f$ hat mindestens zwei lokale Maxima.
\item Die Funktion $f$ hat mindestens ein globales Maximum.
\item Die Funktion $f$ hat kein globales Maximum.
\end{enumerate}
\newpage
\subsection*{\frage{3}{3}}
Sei $f$ eine stetige reellwertige Funktion auf einem abgeschlossenen Intervall $[a,b]$.\\
\\
Was ist eine der Aussagen des Hauptsatzes der Differential- und Integralrechnung?
\renewcommand{\labelenumi}{(\alph{enumi})}
\begin{enumerate}
\item 
$\int_{-\infty}^b f(x) \ dx = 
\lim_{a \to -\infty} \int_{a}^b f(x) \ dx
$.
\item 
$\int f(g(x)) g^\prime(x) \ dx
= \int f(y) \ d(y)
= F(g(x) ) + C
$, wobei $F$ eine beliebige Stammfunktion von $f$ ist.
\item 
$\int u(x) v^\prime(x) \ dx = u(x) v(x)  - \int u^\prime(x) v(x) \ dx$.
\item 
$\int_a^b f(x) \ dx
=\left[F(x)\right]_a^b= F(b) +F(a)
$, wobei $F$ eine beliebige Stammfunktion von $f$ ist.
\item 
Sei die Funktion $F$ definiert als $F(x) = \int_a^x f(u) \ du$.
Dann ist $F$ eine Stammfunktion von $f$.
\end{enumerate}
\ \\
\subsection*{\frage{4}{3}}
Sei $ f $ eine gerade Funktion.
Für $a > 0$ ist $\int_{-a}^a f(x) \ dx$ gleich:
\renewcommand{\labelenumi}{(\alph{enumi})}
\begin{enumerate}
	\item 
	$ 0 $.
	\item
	$ 2 \int_a^0 f(x) \ dx $.
	\item
	$ -2  \int_a^0 f(x) \ dx$.
	\item
	$ 1 $.
	\item
	Keine der obigen Antworten ist korrekt.
\end{enumerate}
\ \\
\subsection*{\frage{5}{4}}
$A$ ist eine $(m \times n)$-dimensionale und invertierbare Matrix mit $m,n \in \mathbb{N}$.\\
\\
Es folgt:
\renewcommand{\labelenumi}{(\alph{enumi})}
\begin{enumerate}
\item 
Der Rang von $A$ ist $\mathrm{rg}(A) = m < n$.
\item 
Wenn $A$ eine Diagonalmatrix ist, dann gilt $\det(A) = \det(A^{-1})$.
\item 
Wenn $A$ eine idempotente Matrix ist, d.h. $A = A^2$, dann gilt $\det(A) = 1$.
\item 
Es existiert ein Vektor $\mathbf{x} \neq 0$ so, dass $A \mathbf{x} = 0$.
\item 
$A$ ist nicht symmetrisch  für $m=n$. 
\end{enumerate}
\ \\
\subsection*{\frage{6}{3}}
Sei $\mathbf{u} $ ein Eigenvektor einer Matrix $A$ mit zugehörigem Eigenwert $\lambda$.
\\
Es folgt, dass 
\begin{align*}
	AA \mathbf{u}
\end{align*}
gleich:
\renewcommand{\labelenumi}{(\alph{enumi})}
\begin{enumerate}
	\item 
	$ \frac{1}{\lambda} \mathbf{u} $ ist.
	\item 
	$ \lambda \mathbf{u} $ ist.
	\item
	$ \lambda^2 \mathbf{u}$ ist.
	\item
	$ 2\lambda \mathbf{u} $ ist.
	\item 
	$ \lambda \mathbf{u}^2 $ ist.
\end{enumerate}
\ \\
\subsection*{\frage{7}{3}}
Seien $A$ und $B$ zwei reguläre $(n \times n)$-dimensionale Matrizen mit $\det(A) = \det(B) = n$.\\
Es folgt, dass $\det(2AA^{-1} B B^{-1} B A)$:
\renewcommand{\labelenumi}{(\alph{enumi})}
\begin{enumerate}
\item 
gleich $0$ ist.
\item
gleich $6n$ ist.
\item
gleich $n^6$ ist.
\item
gleich $2 n^2$ ist.
\item
gleich $2^n n^2$ ist.
\end{enumerate}
\newpage
\subsection*{\frage{8}{3}}
Sei $ A $ eine $ (3 \times 1) $-Matrix mit $A = (a_1, a_2, a_3)^\top$, wobei $ a_i \in \mathbb{R}, \ i = 1,2,3$.\\
\\
Sei $ B $ eine $ (3 \times 1) $-Matrix mit $B = (b_1, b_2, b_3)^\top$, wobei $ b_i \in \mathbb{R}, \ i = 1,2,3$.\\
\\
Sei $ C $ eine $ (3 \times 1) $-Matrix mit $C = (c_1, c_2, c_3)^\top$, wobei $ c_i \in \mathbb{R}, \ i = 1,2,3$.\\

Wir definieren die Matrix $D$ durch die folgende Zusammensetzung der Spalten von $A$, $B$ und $C$:
\begin{align*}
	D = [A,B,C]
\end{align*}
Es folgt, dass:
\renewcommand{\labelenumi}{(\alph{enumi})}
\begin{enumerate}
	\item 
	$ \mathrm{rg}(D) = 3 $, wenn $a_1 = a_2$, $b_1 = b_2$ und $c_1 = c_2$.
	\item
	$ \mathrm{rg}(D) = 1 $, wenn $ a_i = 0 $ für alle $ i \in \{1,2,3\} $.
	\item
	$ \mathrm{rg}(D) \geq  2 $, wenn $ a_i \neq 0 $ für mindestens ein $ i \in \{1,2,3\} $.
	\item
	$ \mathrm{rg}(D) < 3 $, wenn $a_3 = 0$, $b_3 = 0$ und $a_1 b_2 = a_2 b_1$. 
	\item
	$ \mathrm{rg}(D) = 2 $, wenn $a_1 = a_2$, $b_1 = b_2$ und $c_1 = c_2$.
	\item 
	Keine der obigen Antworten ist korrekt.
\end{enumerate}
\ \\
\subsection*{\frage{9}{3}}
Sei $ A $ eine $ (5 \times 3) $-dimensionale Matrix mit Rang $ 2 $ und $ B $ eine andere Matrix mit den Dimensionen $ (5 \times 2) $ mit Rang $ 2 $.\\
\\
Wir definieren die Matrix $ C $ durch die folgende Zusammensetzung der Spalten von $ A $ und $ B $:
\begin{align*}
	C = [A, B].
\end{align*}
Es folgt:
\renewcommand{\labelenumi}{(\alph{enumi})}
\begin{enumerate}
	\item 
	$ \det(C) = 4$.
	\item
	$ \mathrm{rg}(C) < 4 $.
	\item
	$ C$ hat vollen Rang.
	\item
	Die Inverse von $C$ existiert nicht.
	\item
	$\det(C) = \det(A) \det(B)$.
\end{enumerate}
\ \\
\subsection*{\frage{10}{3}}
Sei $A$ eine $(n \times n)$-dimensionale, nicht-invertierbare Matrix.\\
\\
Welche der folgenden Aussagen trifft auf die Matrix $A$ zu?
\renewcommand{\labelenumi}{(\alph{enumi})}
\begin{enumerate}
	\item 
	$A$ muss negative Eigenwerte haben.
	\item
	Alle Eigenwerte sind positiv.
	\item
	$\det(A)$ ist negativ.
	\item
	$0$ ist ein Eigenwert von $A$.
	\item 
	$\lambda_1 \cdot \lambda_2 \cdot \dots \cdot \lambda_n > 0 $, wobei $\lambda_i$ für $i = 1,2,\dots,n$ die Eigenwerte von $A$ sind.
\end{enumerate}

\newpage
\section*{Aufgabe 3 (28 Punkte)}
\vspace{0.4cm}

\subsection*{\frage{1}{3}}
Sei $t > 0$. Das bestimmte Integral
\begin{align*}
	\int_1^e \frac{1}{x} \ln\left((tx)^2\right) \ dx
\end{align*}
entspricht:
\renewcommand{\labelenumi}{(\alph{enumi})}
\begin{enumerate}
\item 
$ \ln(t^2)  +1  $.
\item
$ 1 $.
\item
$ \ln(t^2) -1 $.
\item 
$\frac{1}{2} \ln(t^2)$.
\item $4$.
\end{enumerate}
\ \\
\subsection*{\frage{2}{4}}
Gegeben sei die Funktion
\begin{align*}
	f(x,y) = 2^x +  a e^{y^2}
\end{align*}
Für welches $a \in \mathbb{R}$ ist die Länge des Gradienten im Punkt $(x_0,y_0) = (0,1)$ gleich $4$?
\renewcommand{\labelenumi}{(\alph{enumi})}
\begin{enumerate}
	\item 
	$a = \sqrt{\frac{4 - \ln(2)^2}{4 e^2}} $.
	\item
	$a = \sqrt{\frac{16 - \ln(2)^2}{4 e^2}}$.
	\item
	$a = \sqrt{16 - \ln(2) + 4 e^2}$.
	\item
	$a = \frac{2 - \ln(2)}{2 e }$.
	\item 
	$a = \frac{4 - \ln(2)}{4e}$.
\end{enumerate}
\newpage
\subsection*{\frage{3}{3}}
Gegeben sei die Funktion $f $ definiert als 
\begin{align*}
	f(x)
	=
	\begin{cases}
		0 \quad & x \geq 2 \pi\\
		a | \sin(x)  | \quad &0 \leq x < 2 \pi\\
		0 \quad &x<0
	\end{cases}.
\end{align*}
Für welchen Wert von $a \in \mathbb{R}$ ist $f$ eine Dichtefunktion?
\renewcommand{\labelenumi}{(\alph{enumi})}
\begin{enumerate}
	\item 
	$a = 0.5$.
	\item
	$a =0.25$.
	\item
	$a = 0.5 \pi $.
	\item
	$a = \pi$.
\end{enumerate}
\ \\
\subsection*{\frage{4}{4}}
Die Funktion $g$ erfüllt $g(x) > -4$ für alle $x \in \mathbb{R}$ und $\int_{- \infty}^\infty g(x) \ dx = 3$.\\
Welche der folgenden Funktionen ist eine Dichtefunktionen?
\renewcommand{\labelenumi}{(\alph{enumi})}
\begin{enumerate}
\item 
$ a(x) = g(x) + 4$.
\item 
$ b(x) = \frac{g(x)}{3}$.
\item
$ c(x) = \frac{g(x) + 4}{3} $.
\item
$ d(x) = \frac{3}{4} g(x) + \frac{1}{4} $.
\item
Keine der obigen Antworten ist korrekt.
\end{enumerate}

\newpage
\subsection*{\frage{5}{3}}
Gegeben sei die Matrix
\begin{align*}
	A =
	\begin{pmatrix}
		2 & 1 & 0 & 4 & 0 \\
		3 & 2 & 1 & 9 & x \\
		6 & 0 & 6 & 18 & 6 
	\end{pmatrix}.
\end{align*}
Es folgt, dass
\renewcommand{\labelenumi}{(\alph{enumi})}
\begin{enumerate}
	\item 
	$ \mathrm{rg}(A) = 1 $.
	\item 
	$ \mathrm{rg}(A) = 2 $.
	\item
	$ \mathrm{rg}(A) = 3 $.
	\item
	der Rang von $A$ von $x$ abhängt.
	\item
	Keine der obigen Antworten ist korrekt.
\end{enumerate}
\ \\

\subsection*{\frage{6}{3}}
Gegeben sei die Matrix
\begin{align*}
	A =
	\begin{pmatrix}
		2 & -1 & 0 \\
		-1 & 2 & 0 \\
		-1 & 1 & 1
	\end{pmatrix}
	.
\end{align*}
Welcher der folgenden Vektoren ist \textit{kein} Eigenvektor von $A$?
\renewcommand{\labelenumi}{(\alph{enumi})}
\begin{enumerate}
	\item 
	$ 
	\mathbf{v}
	=
	\begin{pmatrix}
		0\\ 0 \\ 1
	\end{pmatrix}
	$
	.
	\item 
		$ 
	\mathbf{v}
	=
	\begin{pmatrix}
		1\\ 1 \\ 0
	\end{pmatrix}
	$
	.
	\item
	$ 
	\mathbf{v}
	=
	\begin{pmatrix}
		-1\\ 1 \\ 1
	\end{pmatrix}
	$
	.
	\item
		$ 
	\mathbf{v}
	=
	\begin{pmatrix}
		0\\ -1 \\ 1
	\end{pmatrix}
	$
	.
\end{enumerate}
\ \\
\subsection*{\frage{7}{4}}
Maximilian zahlt $100'000$ Schweizer Franken auf ein Sparkonto ein, welches jährlich $2 \%$ Zins für die ersten $20'000$ Schweizer Franken und $1 \%$ Zins für den restlichen Betrag zahlt.\\
\\
Wie hoch sind seine Ersparnisse nach $10 $ Jahren?
\renewcommand{\labelenumi}{(\alph{enumi})}
\begin{enumerate}
\item
$108,769.80$ Schweizer Franken.
\item
$110,462.20$ Schweizer Franken.	
\item 
$111,242.25$ Schweizer Franken.
\item
$112,554.65$ Schweizer Franken.
\item
$112,749.70$ Schweizer Franken.
\end{enumerate}
\ \\
\subsection*{\frage{8}{4}}
Gegeben sei die Differenzengleichung
\begin{align*}
\frac{1}{3} y_k
-
2 y_{k+1}
-3
=0
\quad (k = 0,1,2,...).
\end{align*}
Es folgt:
\renewcommand{\labelenumi}{(\alph{enumi})}
\begin{enumerate}
	\item
	$ (y_k)_{k \in \mathbb{N}} $ ist oszillierend und divergent.
	\item
	$ (y_k)_{k \in \mathbb{N}} $ ist monoton und divergent.	
	\item 
	$ (y_k)_{k \in \mathbb{N}} $ ist monoton und konvergent mit $ \lim \limits_{k \to \infty} y_k = -1.8$.
	\item
	$ (y_k)_{k \in \mathbb{N}} $ ist oszillierend und konvergent mit $ \lim \limits_{k \to \infty} y_k = 1.8$.
	\item 
	$ (y_k)_{k \in \mathbb{N}} $ ist oszillierend und konvergent mit $ \lim \limits_{k \to \infty} y_k = \frac{4}{5}$.
	\item 
	$ (y_k)_{k \in \mathbb{N}} $ ist monoton und konvergent mit $ \lim \limits_{k \to \infty} y_k = -\frac{4}{5}$.
\end{enumerate}