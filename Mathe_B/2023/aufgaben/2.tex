\fancyhead[C]{\normalsize\textbf{$\qquad$ Teil II: Multiple-Choice}}
\section*{Aufgabe 2 (32 Punkte)}
\vspace{0.4cm}
\subsection*{\frage{1}{3}}
Welcher der folgenden mathematischen Ausdrücke beschreibt das nachfolgende Problem:
\begin{displayquote}
	Finde den Punkt $P = (x,y)$ auf der Hyperbel $x^2 - 9 y^2 = 16$ mit dem kleinsten
	Abstand zum Punkt $P_0 = (0,-3)$.
\end{displayquote}
\renewcommand{\labelenumi}{(\alph{enumi})}
\begin{enumerate}
	\item 
	$\min_{x,y} D(x,y) = x^2 - 9y^2 - 16$ so, dass
	$\sqrt{x^2 + (y-3)^2} = 0$.
	\item 
	$\min_{x,y} D(x,y) = \sqrt{x^2 + ( y + 3)^2}$ so, dass
	$x^2 - 9 y^2 - 16  = 0$.
	\item 
	$\max_{x,y} D(x,y) = 16 - x^2 + 9y^2 $ so, dass
	$\sqrt{x^2 + (y+3)^2} = 0$.
	\item 
	$\min_{x,y} D(x,y) = \sqrt{x^2 + ( y - 3)^2}$ so, dass
	$x^2 - 9 y^2 - 16  = 0$.
	\item 
	$\min_{x,y} D(x,y) = x^2 - 9y^2 - 16$ so, dass
	$\sqrt{x^2 + (y+3)^2} = 0$.
	\item
	$\max_{x,y} D(x,y) = \sqrt{x^2 + ( y - 3)^2}$ so, dass
	$x^2 - 9 y^2 - 16  = 0$.
\end{enumerate}
\ \\
\textbf{Lösung:}
\begin{mdframed}
\underline{\textbf{Vorgehensweise:}}
\renewcommand{\labelenumi}{\theenumi.}
\begin{enumerate}
\item 
\end{enumerate}
\end{mdframed}

\underline{1. }\\

\newpage

\subsection*{\frage{2}{3}}
Sei $ f \ : \ D_f \to \mathbb{R} $ eine positive Funktion zweier reeller Variablen, welche ein lokales Maximum beim Punkt $ P = (5,2)  $ hat.
Sei $ h  \ : \ D_f \to \mathbb{R} $ die Funktion definiert durch
\begin{align*}
	h(x,y) = e^{\sqrt{f(x,-y)}}.
\end{align*} 
Welche der folgenden Aussagen ist korrekt?
\renewcommand{\labelenumi}{(\alph{enumi})}
\begin{enumerate}
	\item Die Funktion $ h $ hat ein lokales Maximum beim Punkt $ Q = (-5,2). $
	\item Die Funktion $ h $ hat ein lokales Maximum beim Punkt $ Q = (5,-2). $
	\item Die Funktion $ h $ hat ein lokales Minimum beim Punkt $ Q = (-5,2). $
	\item Die Funktion $ h $ hat ein lokales Minimum beim Punkt $ Q = (5,-2). $
	\item Wir können die Extremstellen von $ h $ nicht analysieren, ohne mehr Informationen zur Funktion $ f $ zu haben.
\end{enumerate}
\ \\
\textbf{Lösung:}
\begin{mdframed}
	\underline{\textbf{Vorgehensweise:}}
	\renewcommand{\labelenumi}{\theenumi.}
	\begin{enumerate}
		\item .
	\end{enumerate}
\end{mdframed}
\underline{1. }\\



\newpage
\subsection*{\frage{3}{3}}
Sei $f$ eine stetige reellwertige Funktion auf einem abgeschlossenen Intervall $[a,b]$.\\
\\
Was ist eine der Aussagen des Hauptsatzes der Differential- und Integralrechnung?
\renewcommand{\labelenumi}{(\alph{enumi})}
\begin{enumerate}
	\item 
	$\int_{-\infty}^b f(x) \ dx = 
	\lim_{a \to -\infty} \int_{a}^b f(x) \ dx
	$.
	\item 
	$\int f(g(x)) g^\prime(x) \ dx
	= \int f(y) \ d(y)
	= F(g(x) ) + C
	$, wobei $F$ eine beliebige Stammfunktion von $f$ ist.
	\item 
	$\int u(x) v^\prime(x) \ dx = u(x) v(x)  - \int u^\prime(x) v(x) \ dx$.
	\item 
	$\int_a^b f(x) \ dx
	=\left[F(x)\right]_a^b= F(b) +F(a)
	$, wobei $F$ eine beliebige Stammfunktion von $f$ ist.
	\item 
	Sei die Funktion $F$ definiert als $F(x) = \int_a^x f(u) \ du$.
	Dann ist $F$ eine Stammfunktion von $f$.
\end{enumerate}
\ \\
\textbf{Lösung:}
\begin{mdframed}
\underline{\textbf{Vorgehensweise:}}
\renewcommand{\labelenumi}{\theenumi.}
\begin{enumerate}
\item 
\end{enumerate}
\end{mdframed}

\underline{1. }\\


\newpage

\subsection*{\frage{4}{3}}
Sei $ f $ eine gerade Funktion.
Für $a > 0$ ist $\int_{-a}^a f(x) \ dx$ gleich:
\renewcommand{\labelenumi}{(\alph{enumi})}
\begin{enumerate}
	\item 
	$ 0 $.
	\item
	$ 2 \int_a^0 f(x) \ dx $.
	\item
	$ -2  \int_a^0 f(x) \ dx$.
	\item
	$ 1 $.
	\item
	Keine der obigen Antworten ist korrekt.
\end{enumerate}
\ \\
\textbf{Lösung:}
\begin{mdframed}
\underline{\textbf{Vorgehensweise:}}
\renewcommand{\labelenumi}{\theenumi.}
\begin{enumerate}
\item Nutze die Eigenschaft der geraden Funktion.
\end{enumerate}
\end{mdframed}

\underline{1. Nutze die Eigenschaft der geraden Funktion }\\
Da $f$ eine gerade Funktion ist gilt $f(x) = f(-x)$ für alle $x \in D_f$.
Mit der Substitution $y = -x $ erhalten wir zunächst.
\begin{align*}
	\frac{dy}{dx} = -1 
	\ \Leftrightarrow \
	dx = - dy
\end{align*}
Damit gilt:
\begin{align*}
	\int \limits_{- a}^a f(x) \ dx
	&=
	\int \limits_{-a}^0 f(x) \ dx
	+ 
	\int \limits_{0}^a f(x) \ dx
	=
	\int \limits_{-a}^0 f(-x) \ dx
	+ 
	\int \limits_{0}^a f(x) \ dx\\
	&=
	\int \limits_{a}^0 -f(y) \ dy
	+ 
	\int \limits_{0}^a f(x) \ dx
	=
	-\int \limits_{a}^0 f(y) \ dy
	+ 
	\int \limits_{0}^a f(x) \ dx\\
	&=
	-\int \limits_{a}^0 f(x) \ dy
	- 
	\int \limits_{a}^0 f(x) \ dx
	= 
	-2 
	\int \limits_{a}^0 f(x) \ dx.
\end{align*}
Dabei haben wir auch verwendet, dass bei Vertauschen der Integrationsgrenzen das Vorzeichen wechselt.

Somit ist die Antwort (c) korrekt.\\
\\
Die schnellere Variante ist:
Da $f$ gerade (also achsensymmetrisch) ist folgt durch Vertauschen der Integrationsgrenzen
\begin{align*}
	\int \limits_{- a}^a f(x) \ dx
	=
	2 \int \limits_{0}^a f(x) \ dx
	= 
	- 2 \int \limits_{a}^0 f(x) \ dx.
\end{align*}


\newpage
\subsection*{\frage{5}{4}}
$A$ ist eine $(m \times n)$-dimensionale und invertierbare Matrix mit $m,n \in \mathbb{N}$.\\
\\
Es folgt:
\renewcommand{\labelenumi}{(\alph{enumi})}
\begin{enumerate}
	\item 
	Der Rang von $A$ ist $\mathrm{rg}(A) = m < n$.
	\item 
	Wenn $A$ eine Diagonalmatrix ist, dann gilt $\det(A) = \det(A^{-1})$.
	\item 
	Wenn $A$ eine idempotente Matrix ist, d.h. $A = A^2$, dann gilt $\det(A) = 1$.
	\item 
	Es existiert ein Vektor $\mathbf{x} \neq 0$ so, dass $A \mathbf{x} = 0$.
	\item 
	$A$ ist nicht symmetrisch  für $m=n$. 
\end{enumerate}
\ \\
\textbf{Lösung:}
\begin{mdframed}
\underline{\textbf{Vorgehensweise:}}
\renewcommand{\labelenumi}{\theenumi.}
\begin{enumerate}
\item 
\end{enumerate}
\end{mdframed}

\underline{1. }\\

\newpage

\subsection*{\frage{6}{3}}
Sei $\mathbf{u} $ ein Eigenvektor einer Matrix $A$ mit zugehörigem Eigenwert $\lambda$.
\\
Es folgt, dass 
\begin{align*}
	AA \mathbf{u}
\end{align*}
gleich:
\renewcommand{\labelenumi}{(\alph{enumi})}
\begin{enumerate}
	\item 
	$ \frac{1}{\lambda} \mathbf{u} $ ist.
	\item 
	$ \lambda \mathbf{u} $ ist.
	\item
	$ \lambda^2 \mathbf{u}$ ist.
	\item
	$ 2\lambda \mathbf{u} $ ist.
	\item 
	$ \lambda \mathbf{u}^2 $ ist.
\end{enumerate}
\ \\
\textbf{Lösung:}
\begin{mdframed}
\underline{\textbf{Vorgehensweise:}}
\renewcommand{\labelenumi}{\theenumi.}
\begin{enumerate}
\item Verwende die Definition des Eigenvektors.
\end{enumerate}
\end{mdframed}

\underline{1. Verwende die Definition des Eigenvektors}\\
Ein Vektor $\mathbf{u} \neq 0$ heißt Eigenvektor einer quadratischen Matrix $A$ zum Eigenwert $\lambda$, falls
\begin{align*}
	A \mathbf{u} = \lambda \mathbf{u}
\end{align*}
gilt.\\
\\
$\mathbf{u}$ ist ein Eigenvektor der Matrix $A$ mit Eigenwert $\lambda $.
Somit folgt:
\begin{align*}
	A A \mathbf{u} 
	=
	A \lambda \mathbf{u} 
	= 
	\lambda A \mathbf{u} 
	= 
	\lambda \lambda \mathbf{u}
	= 
	\lambda^2 \mathbf{u}.
\end{align*}
Damit ist Antwort (c) korrekt.



\newpage
\subsection*{\frage{7}{3}}
Seien $A$ und $B$ zwei reguläre $(n \times n)$-dimensionale Matrizen mit $\det(A) = \det(B) = n$.\\
Es folgt, dass $\det(2AA^{-1} B B^{-1} B A)$:
\renewcommand{\labelenumi}{(\alph{enumi})}
\begin{enumerate}
	\item 
	gleich $0$ ist.
	\item
	gleich $6n$ ist.
	\item
	gleich $n^6$ ist.
	\item
	gleich $2 n^2$ ist.
	\item
	gleich $2^n n^2$ ist.
\end{enumerate}
\ \\
\textbf{Lösung:}
\begin{mdframed}
\underline{\textbf{Vorgehensweise:}}
\renewcommand{\labelenumi}{\theenumi.}
\begin{enumerate}
\item Verwende die Rechenregeln für Determinanten.
\end{enumerate}
\end{mdframed}

\underline{1. Verwende die Rechenregeln für Determinanten}\\
Die korrekte Antwort erhalten wir, indem wir zunächst den Ausdruck innerhalb der Determinante vereinfachen. 
Darauf folgend verwenden wir die Rechenregeln
\begin{align*}
	\det(c A)&= c^n \det(A)\\
	\det(AB) &= \det(A) \det(B).
\end{align*}
Hiermit erhalten wir:
\begin{align*}
	\det(2AA^{-1} B B^{-1} B A)
	=
	\det(2I I B A)
	=
	\det(2B A)
	=
	2^n \det(BA) 
	=
	2^n \det(B) \det(A)
	=
	2^n n^2.
\end{align*}
Damit ist Antwort (d) korrekt.

\newpage

\subsection*{\frage{8}{3}}
Sei $ A $ eine $ (3 \times 1) $-Matrix mit $A = (a_1, a_2, a_3)^\top$, wobei $ a_i \in \mathbb{R}, \ i = 1,2,3$.\\
\\
Sei $ B $ eine $ (3 \times 1) $-Matrix mit $B = (b_1, b_2, b_3)^\top$, wobei $ b_i \in \mathbb{R}, \ i = 1,2,3$.\\
\\
Sei $ C $ eine $ (3 \times 1) $-Matrix mit $C = (c_1, c_2, c_3)^\top$, wobei $ c_i \in \mathbb{R}, \ i = 1,2,3$.\\

Wir definieren die Matrix $D$ durch die folgende Zusammensetzung der Spalten von $A$, $B$ und $C$:
\begin{align*}
	D = [A,B,C]
\end{align*}
Es folgt, dass:
\renewcommand{\labelenumi}{(\alph{enumi})}
\begin{enumerate}
	\item 
	$ \mathrm{rg}(D) = 3 $, wenn $a_1 = a_2$, $b_1 = b_2$ und $c_1 = c_2$.
	\item
	$ \mathrm{rg}(D) = 1 $, wenn $ a_i = 0 $ für alle $ i \in \{1,2,3\} $.
	\item
	$ \mathrm{rg}(D) \geq  2 $, wenn $ a_i \neq 0 $ für mindestens ein $ i \in \{1,2,3\} $.
	\item
	$ \mathrm{rg}(D) < 3 $, wenn $a_3 = 0$, $b_3 = 0$ und $a_1 b_2 = a_2 b_1$. 
	\item
	$ \mathrm{rg}(D) = 2 $, wenn $a_1 = a_2$, $b_1 = b_2$ und $c_1 = c_2$.
	\item 
	Keine der obigen Antworten ist korrekt.
\end{enumerate}
\ \\
\textbf{Lösung:}
\begin{mdframed}
\underline{\textbf{Vorgehensweise:}}
\renewcommand{\labelenumi}{\theenumi.}
\begin{enumerate}
\item Verwende die Definition des Ranges.
\end{enumerate}
\end{mdframed}

\underline{1. Verwende die Definition des Ranges}\\



\newpage
\subsection*{\frage{9}{3}}
Sei $ A $ eine $ (5 \times 3) $-dimensionale Matrix mit Rang $ 2 $ und $ B $ eine andere Matrix mit den Dimensionen $ (5 \times 2) $ mit Rang $ 2 $.\\
\\
Wir definieren die Matrix $ C $ durch die folgende Zusammensetzung der Spalten von $ A $ und $ B $:
\begin{align*}
	C = [A, B].
\end{align*}
Es folgt:
\renewcommand{\labelenumi}{(\alph{enumi})}
\begin{enumerate}
	\item 
	$ \det(C) = 4$.
	\item
	$ \mathrm{rg}(C) < 4 $.
	\item
	$ C$ hat vollen Rang.
	\item
	Die Inverse von $C$ existiert nicht.
	\item
	$\det(C) = \det(A) \det(B)$.
\end{enumerate}
\ \\
\textbf{Lösung:}
\begin{mdframed}
	\underline{\textbf{Vorgehensweise:}}
	\renewcommand{\labelenumi}{\theenumi.}
	\begin{enumerate}
		\item Verwende die Definition des Ranges.
	\end{enumerate}
\end{mdframed}

\underline{1. Verwende die Definition des Ranges}\\


\newpage
\subsection*{\frage{10}{3}}
Sei $A$ eine $(n \times n)$-dimensionale, nicht-invertierbare Matrix.\\
\\
Welche der folgenden Aussagen trifft auf die Matrix $A$ zu?
\renewcommand{\labelenumi}{(\alph{enumi})}
\begin{enumerate}
	\item 
	$A$ muss negative Eigenwerte haben.
	\item
	Alle Eigenwerte sind positiv.
	\item
	$\det(A)$ ist negativ.
	\item
	$0$ ist ein Eigenwert von $A$.
	\item 
	$\lambda_1 \cdot \lambda_2 \cdot \dots \cdot \lambda_n > 0 $, wobei $\lambda_i$ für $i = 1,2,\dots,n$ die Eigenwerte von $A$ sind.
\end{enumerate}
\ \\
\textbf{Lösung:}
\begin{mdframed}
	\underline{\textbf{Vorgehensweise:}}
	\renewcommand{\labelenumi}{\theenumi.}
	\begin{enumerate}
		\item Nutze den Zusammenhang der Determinante zu den Eigenwerten.
	\end{enumerate}
\end{mdframed}

\underline{1. Nutze den Zusammenhang der Determinante zu den Eigenwerten}\\
Die quadratische Matrix $A$ ist nicht invertierbar. Deswegen gilt
\begin{align*}
	\det(A) = \prod \limits_{i=1}^n \lambda_i = 0.
\end{align*}
Hierbei sind $\lambda_i$ für $i = 1, ..., n$ die Eigenwerte von $A$. Diese sind nicht notwendigerweise verschieden.
Da $\det(A) = 0$ muss mindestens einer dieser Eigenwerte $0$ sein. \\
\\
Damit ist die Antwort (d) korrekt.\\
\\
Für die Antwort (a) ist
\begin{align*}
	A = 
	\begin{pmatrix}
		1 & 0\\
		0 & 0
	\end{pmatrix}
\end{align*}
ein Gegenbeispiel. Antwort (b),(c) und (e) widerspricht $\det(A) = 0$.
 