\vspace{1cm}
\fancyhead[C]{\normalsize\textbf{$\qquad$ Teil I: Offene Aufgaben}}
\renewcommand{\labelenumi}{\theenumi.}
\section*{Aufgabe 1 (40 Punkte)}
\vspace{0.4cm}
\subsection*{\aufgabe{a}{12}}
Estis lang ersehnter Traum ist endlich wahr geworden
- 
sie hat sich für den Ironman Hawaii qualifiziert.
Allerdings weiss sie, dass sie ihr Training optimieren muss, um ihre bestmögliche Leistung erbringen zu können.
Esti beschliesst, als Steuerungsparameter ihren $VO_2max$-Wert zu nutzen.
Dieser misst die maximale Menge an Sauerstoff, die ihr Körper während der Belastung nutzen kann, und stellt somit eine hilfreiche Grösse dar, um die Verbesserung ihres Fitnesslevels abzubilden.
Sie möchte die optimale Verteilung ihrer Trainingszeit zwischen
Schwimmen, Radfahren und Laufen finden, um ihren $VO_2max$-Wert zu maximieren.\\
Estis derzeitiger $VO_2max$-Wert beträgt $48$ und ihre Trainingszeit ist auf $15$ Stunden pro Woche begrenzt.
Sei $x$ die Anzahl an Stunden pro Woche, die sie für Schwimmen und Radfahren zusammen aufwendet und $y$ die Anzahl an Stunden pro Woche, die sie mit Laufen verbringt.
Esti folgt der Empfehlung ihres Trainers und trainiert pro Woche viermal so viel Radfahren wie Schwimmen. Ausserdem nimmt sie an, dass sich der Trainingseffekt bezüglich ihres, am Tag des Ironman-Events gemessenen, $VO_2max$-Werts wie folgt modellieren lässt:
\begin{align*}
	f(x,y) = 1.8 x^{\frac{2}{3}} y^{\frac{1}{3}}.
\end{align*}
\begin{enumerate}
	\item[\textbf{(a1)}]
	Formulieren Sie das Optimierungsproblem, das Esti lösen muss, um ihren $VO_2max$-Wert am Tag des Events zu optimieren.
	\item[\textbf{(a2)}] 
	Lösen Sie das Optimierungsproblem und bestimmen Sie die optimale Anzahl an Stunden pro Woche, die Esti für Schwimmen, Radfahren und Laufen aufwenden sollte, sowie den $VO_2max$-Wert am Tag des Ironman-Events, wenn sie diesem Trainingsplan folgt.\\
	\\
	\textit{Anmerkung: Es ist nicht notwendig, die hinreichende Bedingung für einen Extremwert zu überprüfen.}
\end{enumerate}
 
\textbf{Lösung:}
\begin{mdframed}
\underline{\textbf{Vorgehensweise:}}
\renewcommand{\labelenumi}{\theenumi.}
\begin{enumerate}
\item[\textbf{(a1)}] Formuliere das Optimierungsproblem.
\item[\textbf{(a2)}] Wende die Lagrangemethode an.
\begin{enumerate}
	\item[1.] Stelle die Lagrangefunktion und die Lagrangebedingungen auf. 
	\item[2.] Löse die Lagrangebedingungen auf.
\end{enumerate}
\end{enumerate}
\end{mdframed}
Bevor wir uns in die Aufgabe stürzen, müssen wir den mathematischen Rahmen aufbauen.
Das Ziel von Esti den $VO_2max$-Wert zu maximieren deutet auf ein Optimierungsproblem hin. Bei diesen Problemen müssen wir prüfen, ob Nebenbedingungen vorliegt. Nebenbedingungen schränkten die möglichen Variablenwerte ein.
Da Esti $15$ Stunden pro Woche trainieren sind die Stunden für das Radfahren/Schwimmen $x$ und die Stunden für das Laufen $y$ durch
\begin{align*}
	x + y = 15
\end{align*} 
eingeschränkt.
\newpage

\underline{\textbf{(a1)} Formuliere das Optimierungsproblem }\\
Das Ziel von Esti ist, dass der $VO_2max$-Wert zum Ironman-Tag maximal ist.
Wegen Estis aktuellem $VO_2max$-Wert von $48$ und der Modellierung des Trainingseffekts durch $f(x,y) = 1.8 x^{\frac{2}{3}} y^{\frac{1}{3}}$ erhalten wir das folgende Optimierungsproblem:
\begin{align*}
	\max_{x,y} V(x,y)
	=
	48 + 1.8 x^{\frac{2}{3}} y^{\frac{1}{3}},
	\ \text{unter der Bedingung} \ x + y = 5 \ \Leftrightarrow \ \varphi(x,y) := x + y - 15 = 0.
\end{align*}

\underline{\textbf{(a2)} 1. Stelle die Lagrangefunktion und die Lagrangebedingungen auf}\\
Die Lagrangefunktion $L(x,y, \lambda)$ ist gegeben durch
\begin{align*}
	L(x,y, \lambda) 
	=
	V(x,y) + \lambda \varphi(x,y)
	= 
	48 + 1.8 x^{\frac{2}{3}} y^{\frac{1}{3}} + \lambda(x+y-15).
\end{align*}
Die partiellen Ableitungen sind gegeben durch:
\begin{align*}
	&\frac{\partial L }{\partial x}(x,y,\lambda) = 
	1.8 \frac{2}{3} x^{\frac{2}{3} - 1 } y^{\frac{1}{3}} + \lambda
	= 
	\frac{18}{10} \cdot \frac{2}{3} x^{-\frac{1}{3}} y^{\frac{1}{3}} + \lambda
	= 
	\frac{6}{5}  x^{-\frac{1}{3}} y^{\frac{1}{3}} + \lambda
	=
	= 
	1.2  x^{-\frac{1}{3}} y^{\frac{1}{3}} + \lambda
	\\
	&\frac{\partial L }{\partial y}(x,y,\lambda) 
	=
	1.8 \frac{1}{3} x^{\frac{2}{3}  } y^{\frac{1}{3}- 1} + \lambda
	=
	\frac{6}{10} x^{\frac{2}{3}  } y^{-\frac{2}{3}} + \lambda
	=
	0.6 x^{\frac{2}{3}  } y^{-\frac{2}{3}} + \lambda
	\\
	&\frac{\partial L }{\partial \lambda}(x,y,\lambda) = x + y -15
\end{align*}
Hieraus ergeben sich die Lagrangebedingungen:
\begin{align*}
	\mathrm{(I)}& \
	\frac{\partial L }{\partial x}(x,y,\lambda) = 0
	\ \Leftrightarrow \
	1.2  x^{-\frac{1}{3}} y^{\frac{1}{3}} + \lambda = 0
	\\
	\mathrm{(II)}& \
	\frac{\partial L }{\partial y}(x,y,\lambda) = 0
	\ \Leftrightarrow \
	0.6 x^{\frac{2}{3}  } y^{-\frac{2}{3}} + \lambda = 0
	\\
	\mathrm{(III)}& \
	\frac{\partial L }{\partial \lambda}(x,y,\lambda) = 0
	\ \Leftrightarrow \
	 x + y -15 = 0.
\end{align*}
\ \\
\underline{\textbf{(a2)} 2. Löse die Lagrangebedingungen auf}\\
Durch Gleichsetzen der Gleichungen (I) und (II) erhält man 
\begin{align*}
	1.2  x^{-\frac{1}{3}} y^{\frac{1}{3}} + \lambda
	=
	0.6 x^{\frac{2}{3}  } y^{-\frac{2}{3}} + \lambda
	\ &\Leftrightarrow \
	\frac{12}{10} x^{-\frac{1}{3}} y^{\frac{1}{3}}
	=
	\frac{6}{10} x^{\frac{2}{3}  } y^{-\frac{2}{3}} \\
	\ &\Leftrightarrow \
	\frac{12}{10} \cdot \frac{10}{6} x^{-\frac{1}{3}} y^{\frac{1}{3}}
	= 
	x^{\frac{2}{3}  } y^{-\frac{2}{3}} 
	\ \Leftrightarrow \
	2  y^{\frac{1}{3}}
	= 
	x^{\frac{2}{3} + \frac{1}{3} } y^{-\frac{2}{3}} \\
	\ &\Leftrightarrow \
	2  y
	= 
	x.
\end{align*}
Mit der Gleichung (III) folgt somit
\begin{align*}
	x + y = = 2y + y = 3y =  15
	\ \Leftrightarrow \
	y = 5
	\ \Rightarrow \ 2y = 10 = x.
\end{align*}
Damit gilt $x = 10 $ (Stunden Schwimmen und Radfahren) und $y= 5$ (Stunden Laufen).
Da Esti laut Aufgabenstellen viermal mehr Rad fährt als schwimmt, 
Aus $x = 10$ folgt, dass Esti $2 $ Stunden schwimmt und $8$ Stunden Rad fährt.
Dies gilt, da Esti laut Aufgabenstellen viermal mehr Rad fährt als schwimmt.
Der $VO_2max$-Wert zum Ironman-Tag ist für $5$ Stunden Laufen, $8$ Stunden Radfahren und $2$ Stunden Schwimmen gegeben durch:
\begin{align*}
	V(10,5)
	= 
	48 + 1.8 10^{\frac{2}{3} } 5^{\frac{1}{3}}
	=
	62.29.
\end{align*}
\newpage

\subsection*{\aufgabe{b}{14}}
Vier Werkstücke $P_1$, $P_2$, $P_3$ und $P_4$ müssen in einem gegebenen Produktionsprozess jeweils drei Maschinen $M_1$, $M_2$ und $M_3$ durchlaufen.
Die für die einzelnen Werkstücke pro Einheit benötigten Bearbeitungszeiten in Minuten sind in der folgenden Tabelle angegeben: 
\begin{table}[H]
	\centering
	%
	\begin{tabular}{c | c c c c}
		$ $  & $P_1$  &  $P_2$ &  $P_3$ & $P_4$ \\ 
		\hline
		$ M_1 $ & $ 10 $ & $ 20 $ & $ 20 $ & $15$  \\ 
		$ M_2 $ & $ 10 $ & $ 30 $ & $ 16 $ & $5$ \\
		$ M_3 $ & $ 20 $ & $ 20 $ & $ 10 $ & $25$
	\end{tabular}%
\end{table}
Jede Maschine läuft genau $8$ Stunden pro Tag.\\
\\
Der Nettogewinn in Schweizer Franken für die vier Werkstücke beträgt pro Einheit der Reihe nach
\begin{align*}
	q_1 = 5; \quad
	q_2=  10; \quad
	q_3=12; \quad 
	q_4=m.
\end{align*}
Der angestrebte Nettogewinn aus der Gesamtproduktion beträgt $250$ Schweizer Franken pro Tag.
\begin{enumerate}
	\item[\textbf{(b1)}]
	Beschreiben Sie die Beziehung zwischen Produktionszeiten, Laufzeit je Maschine pro Tag, Nettogewinn und der Anzahl der produzierten Werkstücke pro Tag in einem Gleichungssystem.
	
	\item[\textbf{(b2)}] 
	Für welche(s) $m > 0$ existiert ein realisierbarer Produktionsplan?
	\item[\textbf{(b3)}]
	Bestimmen Sie den Produktionsplan, d.h. die Anzahl produzierter Einheiten jedes Werkstücks unter Berücksichtigung der verfügbaren Kapazitäten der Maschinen und des angestrebten Nettogewinns, für $m = 5$.
\end{enumerate}
\ \\
\textbf{Lösung:}
\begin{mdframed}
\underline{\textbf{Vorgehensweise:}}
\renewcommand{\labelenumi}{\theenumi.}
\begin{enumerate}
\item[\textbf{(b1)}] Formuliere das lineare Gleichungssystem.
\item[\textbf{(b2)}] Beurteile die Lösbarkeit des linearen Gleichungsystems.
\begin{enumerate}
	\item[1.] Bringe die Matrix in eine obere Dreiecksform
	\item[2.] Wende bekannte Kriterien für die Lösbarkeit an.
\end{enumerate}
\item[\textbf{(b3)}] Löse das lineare Gleichungssystem für $m = 5$.
\end{enumerate}
\end{mdframed}

\underline{\textbf{(b1)} Formuliere das lineare Gleichungssystem}\\
In der Aufgabe wird ein Produktionsprozess von vier Werkstücken.
Dieser Prozess ist durch die Laufzeit der Maschinen von $8$ Stunden und dem angestrebten Nettogewinn begrenzt.
Mit $x_i$ bezeichnen wir die pro Werktag produzierten Werkstücke $P_i$ (von allen vier Maschinen). Hiermit erhalten wir durch
\begin{align*}
	10 x_1 + 20 x_2 + 20 x_3 + 15 x_4
\end{align*}
die Gesamtzahl der Minuten welche die Maschine $M_1$ benötigt um $x_1,x_2, x_3, x_4$ Teile der Werkstücke $P_1, P_2 , P_3, P_4$ zu produzieren. Da die drei Maschinen für $8$ Stunden laufen ($8 \cdot 60 = 480$ Minuten), ergibt sich die folgende Einschränkung für Maschine $M_1$:
\begin{align*}
	10 x_1 + 20 x_2 + 20 x_3 + 15 x_4  =480.
\end{align*}
\newpage
Für die beiden anderen Maschinen erhalten wir analog:
\begin{align*}
	&10 x_1 + 30 x_2 + 16 x_3 +  \ 5 x_4  =480,\\
	&20 x_1 + 20 x_2 + 10 x_3 + 25 x_4  =480.
\end{align*}
Durch $x_i q_i$ ist der Nettogewinn der produzierten Einheiten des Werkstücks $P_i$ gegeben. Deswegen erhalten wir für den Nettogewinn die Gleichung
\begin{align*}
	q_1 x_1 + q_2 x_2 + q_3 x_3 + q_4 x_4 = 
	5 x_1 + 10 x_2 + 12 x_3 + m  = 250 \ \text{Schweizer Franken}.
\end{align*}
Insgesamt konnten wir aus der Aufgabenstellung das folgenden lineare Gleichungssystem extrahieren:
\begin{align*}
	\begin{cases}
		10 x_1 + 20 x_2 + 20 x_3 + 15 x_4  &=480 \\
		10 x_1 + 30 x_2 + 16 x_3 +  \ 5 x_4  &=480\\
		20 x_1 + 20 x_2 + 10 x_3 + 25 x_4  &=480\\
		\ 5 x_1 + 10 x_2 + 12 x_3 + m  &= 250
	\end{cases}
\end{align*}
Überführt in die erweiterte Koeffizientenmatrix erhalten wir:
\begin{align*}
(A | \mathbf{b}) =	\bordermatrix{ & x_1 & x_2 & x_3 & x_4 & &  \cr
		& 10 & 20 & 20 &  15 & \BAR & 480 \cr
		& 10 & 30 & 16& 5 &\BAR & 480 \cr
		& 20 & 20 & 10 & 25 &  \BAR & 480  \cr 
		& 5 & 10 & 12 & m & \BAR & 250}.
\end{align*}
Die erste bis vierte Spalte repräsentiert die erste bis vierte Variable. Diese Kennzeichnung lassen wir im weiteren Verlauf weg.\\
\\
\underline{\textbf{(b2)} 1. Bringe die Matrix in eine obere Dreiecksform}\\
Durch elementare Zeilenumformungen erhalten wir aus der erweiterten Koeffizientenmatrix $(A | \mathbf{b})$:
\begin{align*}
	&\begin{gmatrix}[p]
		10 & 20 & 20 &  15 & \BAR & 480\\
		10 & 30 & 16& 5 &\BAR & 480 \\
		20 & 20 & 10 & 25 &  \BAR & 480  \\
		5 & 10 & 12 & m & \BAR & 250
		\rowops
		\add[ \cdot (-1)]{0}{1}
		\add[ \cdot (-2)]{0}{2}
		\mult{3}{ \cdot 2}
	\end{gmatrix}
	\leadsto
	\begin{gmatrix}[p]
		10 & 20 & 20 &  15 & \BAR & 480\\
		0 & 10 & -4 & -10 &\BAR & 0 \\
		0 & -20 & -30 & -5 &  \BAR & -480  \\
		10 &  20 & 24 & 2m & \BAR & 500
		\rowops
		\add[ \cdot (-1)]{0}{3}
	\end{gmatrix}\\
	\leadsto
	&\begin{gmatrix}[p]
		10 & 20 & 20 &  15 & \BAR & 480\\
		0 & 10 & -4 & -10 &\BAR & 0 \\
		0 & -20 & -30 & -5 &  \BAR & -480  \\
		0 &  0 & 4 & 2m - 15 & \BAR & 20
		\rowops
		\add[ \cdot 2]{1}{2}
	\end{gmatrix}
	\leadsto
	\begin{gmatrix}[p]
		10 & 20 & 20 &  15 & \BAR & 480\\
		0 & 10 & -4 & -10 &\BAR & 0 \\
		0 & 0 & -38 & -25 &  \BAR & -480  \\
		0 &  0 & 4 & 2m - 15 & \BAR & 20
		\rowops
		\swap{2}{3} 
	\end{gmatrix}
	\\
	\leadsto
	&\begin{gmatrix}[p]
		10 & 20 & 20 &  15 & \BAR & 480\\
		0 & 10 & -4 & -10 &\BAR & 0 \\
		0 &  0 & 4 & 2m - 15 & \BAR & 20\\
		0 & 0 & -38 & -25 &  \BAR & -480  		
		\rowops
		\mult{2}{ \cdot \left( \frac{1}{4}\right)}
		\mult{3}{ \cdot \left(- \frac{1}{10}\right)}
	\end{gmatrix}
	\leadsto
	\begin{gmatrix}[p]
		10 & 20 & 20 &  15 & \BAR & 480\\
		0 & 10 & -4 & -10 &\BAR & 0 \\
		0 &  0 & 1 & \frac{m}{2} - \frac{15}{4} & \BAR & 5\\
		0 & 0 & 3.8 & 2.5 &  \BAR & 48		
		\rowops
		\add[ \cdot (-3.8)]{2}{3}
	\end{gmatrix}	\\
	\leadsto
	&\begin{gmatrix}[p]
		10 & 20 & 20 &  15 & \BAR & 480\\
		0 & 10 & -4 & -10 &\BAR & 0 \\
		0 &  0 & 1 & \frac{m}{2} - \frac{15}{4} & \BAR & 5\\
		0 & 0 & 0 & 2.5 - 3.8 \left(\frac{m}{2} - \frac{15}{4} \right)  &  \BAR & 48	- 3.8 \cdot 5	
		\rowops
		\add[ \cdot (-3.8)]{2}{3}
	\end{gmatrix} \\
	\leadsto
	&\begin{gmatrix}[p]
		10 & 20 & 20 &  15 & \BAR & 480\\
		0 & 10 & -4 & -10 &\BAR & 0 \\
		0 &  0 & 1 & \frac{m}{2} - \frac{15}{4} & \BAR & 5\\
		0 & 0 & 0 & 16.75 - 1.9m  &  \BAR & 29
		\rowops
		\mult{0}{ \cdot  \frac{1}{10}}
		\mult{1}{ \cdot \frac{1}{10}}	
	\end{gmatrix}
	\leadsto
	\begin{gmatrix}[p]
		1 & 2 & 2 &  1.5 & \BAR & 48\\
		0 & 1 & -0.4 & -1 &\BAR & 0 \\
		0 &  0 & 1 & \frac{m}{2} - \frac{15}{4} & \BAR & 5\\
		0 & 0 & 0 & 16.75 - 1.9m  &  \BAR & 29	
	\end{gmatrix}
\end{align*}

\underline{\textbf{(b2)} 2. Wende bekannte Kriterien für die Lösbarkeit an}\\
Wir haben mithilfe von elementaren Zeilenumformungen die Zeilenstufenform erhalten:
\begin{align*}
	(A | \mathbf{b}) \leadsto
	\dots
	\leadsto
	(\tilde{A} | \mathbf{\tilde{b}})
	=
	\begin{gmatrix}[p]
		1 & 2 & 2 &  1.5 & \BAR & 48\\
		0 & 1 & -0.4 & -1 &\BAR & 0 \\
		0 &  0 & 1 & \frac{m}{2} - \frac{15}{4} & \BAR & 5\\
		0 & 0 & 0 & 16.75 - 1.9m  &  \BAR & 29	
	\end{gmatrix}.
\end{align*}
Es gilt 
\begin{align*}
	 \det(\tilde{A})
	= 
	1 \cdot 1 \cdot 1 \cdot ( 16.75 - 1.9m)
	=
	16.75 - 1.9m
	,
\end{align*}
da $\tilde{A}$ eine obere Dreiecksform besitzt. Wegen $\det(A) = C \cdot \det(\tilde{A}), \ C \neq 0$ ist $A$ genau dann singulär, wenn 
\begin{align*}
	\det(\tilde{A}) = 16.75 - 1.9m = 0
	\ \Leftrightarrow \
	1.9 m = 16.75
	\ \Leftrightarrow \
	m = \frac{16.75}{1.9} = \frac{\frac{1675}{100}}{\frac{19}{10}}
	=
	\frac{1675 }{10 \cdot 19}
	=
	\frac{335 }{5 \cdot 19}
	=
	\frac{335 }{38}
\end{align*}
gilt. Damit ist $A$ für $m \neq \frac{335 }{38}$ invertierbar und 
\begin{align*}
	A \mathbf{x} = \mathbf{b}
\end{align*}
besitzt eine eindeutige Lösung $ \mathbf{x}$.
Für $m = \frac{335 }{38}$ existiert wegen  
\begin{align*}
	\mathrm{rg}(A) = \mathrm{rg}(\tilde{A})  = 3  
	<  
	\mathrm{rg}(\tilde{A}|\mathbf{\tilde{b}})= 	\mathrm{rg}(A|\mathbf{b}) = 4
\end{align*}
keine Lösung zu $A \mathbf{x} = \mathbf{b}$.\\
\\
\textit{Alternative Eindeutigkeit:}\\
Für $m \neq \frac{335 }{38}$ gilt $\mathrm{rg}(A) = 4$ ($A$ hat vollen Rang). Damit ist $A$ invertierbar und die Lösung zu $A \mathbf{x} = \mathbf{b}$ eindeutig.\\
\\
\underline{\textbf{(b3)} Löse das lineare Gleichungssystem für $m = 5$}\\
Mit der Zeilenstufenform aus der vorherigen Teilaufgabe erhalten wir für $m = 5$:
\begin{align*}
	\begin{gmatrix}[p]
		1 & 2 & 2 &  1.5 & \BAR & 48\\
		0 & 1 & -0.4 & -1 &\BAR & 0 \\
		0 &  0 & 1 & \frac{5}{2} - \frac{15}{4} & \BAR & 5\\
		0 & 0 & 0 & 16.75 - 1.9\cdot 5  &  \BAR & 29	
	\end{gmatrix}
	=
	&\begin{gmatrix}[p]
		1 & 2 & 2 &  1.5 & \BAR & 48\\
		0 & 1 & -0.4 & -1 &\BAR & 0 \\
		0 &  0 & 1 & \frac{10}{4} - \frac{15}{4} & \BAR & 5\\
		0 & 0 & 0 & 16.75 - 9.5 &  \BAR & 29	
	\end{gmatrix}\\
		=
	&\begin{gmatrix}[p]
		1 & 2 & 2 &  1.5 & \BAR & 48\\
		0 & 1 & -0.4 & -1 &\BAR & 0 \\
		0 &  0 & 1 & -1.25 & \BAR & 5\\
		0 & 0 & 0 & 7.25 &  \BAR & 29
		\rowops
		\mult{0}{ \cdot  2}
		\mult{1}{ \cdot 5}	
		\mult{2}{ \cdot 4}	
		\mult{3}{ \cdot 4}	
	\end{gmatrix}\\
	\leadsto
	&\begin{gmatrix}[p]
		2 & 4 & 4 &  3 & \BAR & 96\\
		0 & 5 & -2 & -5 &\BAR & 0 \\
		0 &  0 & 4 & -5 & \BAR & 20\\
		0 & 0 & 0 & 29 &  \BAR & 4 \cdot 29	
	\end{gmatrix}.
\end{align*}
\newpage
Damit folgt durch Rückwärtseinsetzen:
\begin{align*}
	29 x_4 &= 4 \cdot 29 \ \Leftrightarrow \ x_4 = \frac{4 \cdot 29}{29} = 4\\
	4 x_3 -5 x_4 &= 20 
	\ \Leftrightarrow \ 
	x_3  = \frac{1}{4} (20 + 5 x_4) = \frac{1}{4} \cdot 40  =  10\\
	5 x_2 -2 x_3 - 5 x_4 &= 0 
	\ \Leftrightarrow \ x_2 = 
	\frac{1}{5}(2 x_3 + 5 x_4) = 8\\
	2 x_1 + 4x_2 +4 x_3 + 3 x_4 &= 96
	\ \Leftrightarrow \
	x_1 = \frac{1}{2} (96 - 4 x_2 - 4 x_3 -3 x_4) = \frac{1}{2} \cdot 12 = 6
\end{align*}
Zusammengefasst erhalten wir für $m = 5$ den Produktionsplan 
\begin{align*}
	x_1 = 6, \quad x_2 = 8, \quad x_3 = 10, \quad x_4 = 4.
\end{align*}


\newpage
\subsection*{\aufgabe{c}{14}}
Nach der letzten Preiserhöhung bei Strom möchten Sie in Solarpanele investieren und den erzeugten Strom verkaufen. 
Sie zahlen $10'000$ Schweizer Franken für den Kauf und die
Installation der Solarpanele.
Unter Berücksichtigung des kontinuierlich schlechter werdenden Wirkungsgrads nehmen Sie konservativ an, dass sich die Leistungsabgabe der Solarpanele, gemessen in Kilowattstunden pro Jahr, mit der Zeit wie folgt entwickelt:
\begin{align*}
	f(t) = 10'000 e^{-0.1 t},
\end{align*}
wobei $t$ in Jahren gemessen wird.
Sie gehen davon aus, dass die Lebensdauer der Solarpanele $20$ Jahre beträgt. Nach diesen $20$ Jahren müssen Sie die Solarpanele zu einem Preis von $10'000$ Schweizer Franken entsorgen. Ausserdem nehmen Sie an, dass sich der Marktpreis in
Schweizer Franken für eine Kilowattstunde mit der Zeit wie folgt entwickelt:
\begin{align*}
	p(t)
	=
	0.3 e^{0.01t}.
\end{align*}
Derzeit sind die Zinssätze eher hoch, aber Sie sind sich sicher, dass sie nach einer Weile wieder niedriger werden. Sie gehen folglich für die nächsten $5$ Jahre von einer kontinuierlichen
Verzinsung zu einem Satz von $4 \%$ p.a. und anschliessend zu einem Satz von $2\%$ p.a. aus.

\begin{enumerate}
	\item[\textbf{(c1)}] 
	Bestimmen Sie die insgesamt erzeugte Energiemenge.
	\item[\textbf{(c2)}]
	Berechnen Sie den Nettobarwert Ihrer Investition.
\end{enumerate}
\ \\
\textbf{Lösung:}
\begin{mdframed}
\underline{\textbf{Vorgehensweise:}}
\begin{enumerate}
\item[\textbf{(c1)}] Bestimme die Leistungsabgabe über $20$ Jahre (Integration).
\item[\textbf{(c2)}] Bestimme den Nettobarwert.
\end{enumerate}
\end{mdframed}

\underline{\textbf{(c1)} Bestimme die Leistungsabgabe über $20$ Jahre (Integration)}\\
Die Leistungsabgabe zum Zeitpunkt $t$ ist 
\begin{align*}
	f(t) = 10'000 e^{-0.1 t}.
\end{align*}
Da wir an der insgesamt erzeugten Energiemenge über $20$ Jahre interessiert sind bestimmen wir durch
\begin{align*}
	\int 
	\limits_{0}^{20}
	f(t)
	\ dt 
	= 
	\int 
	\limits_{0}^{20}
	10'000 e^{-0.1 t}
	\ dt 
	= 
	\left[-\frac{10'000}{0.1} e^{-0.1 t}\right]_0^20
	= 
	\left[-100'000 e^{-0.1 t}\right]_0^20
	= 
	86'466.47 \ \mathrm{(kWh)}
\end{align*}
die Fläche unter $f$ von $0$ bis $20$.
Damit werden in den nächsten $20$ Jahren $86'466.47 $ kWh erzeugt.\\
\\
\newpage
\underline{\textbf{(c2)} Bestimme den Nettobarwert}\\
Der Nettobarwert ($\mathrm{NPV}$) der Investition ist die Differenz zwischen den Barwerten des verkauften Stroms und der entstehenden Kosten.
Die entstehenden Kosten setzen sich dabei aus der Anfangsinvestition und den Kosten für das Recycling zusammen.
Da sich der Zinssatz nach $5$ Jahren ändert ergibt sich für die beiden Barwerte:
\begin{align*}
	\mathrm{NPV}_{\textit{verk. Strom}}
	&=
	\int \limits_{0}^5 
	10'000 e^{-0.1 t} 0.3 e^{0.01t} e^{-0.04t} \ dt
	+
	e^{-5 \cdot 0.04}
	\int \limits_{5}^{20}
	10'000 e^{-0.1 t} 0.3 e^{0.01t} e^{-0.02t} \ dt\\
	&=
	\int \limits_{0}^5 
	3'000 e^{-0.13 t}  \ dt
	+
	e^{-5 \cdot 0.04}
	\int \limits_{5}^{20}
	3'000 e^{-0.11 t} \ dt\\
	\mathrm{NPV}_{\textit{Kosten}}
	&=
	10'000 + 10000 e^{-15 \cdot 0.02 - 5 \cdot 0.04}
	= 
	10'000 ( 1 + e^{-15 \cdot 0.02 - 5 \cdot 0.04}). 
\end{align*}
Insgesamt erhalten wir:
\begin{align*}
	\mathrm{NPV} &= 
	\mathrm{NPV}_{\textit{verk. Strom}} - \mathrm{NPV}_{\textit{Kosten}}\\
	&=
	\int \limits_{0}^5 
	3'000 e^{-0.13 t}  \ dt
	+
	e^{-5 \cdot 0.04}
	\int \limits_{5}^{20}
	3'000 e^{-0.11 t} \ dt
	-
	10'000 ( 1 + e^{-15 \cdot 0.02 - 5 \cdot 0.04})\\
	&=
	\left[-\frac{3'000}{0.13} e^{-0.13}\right]_0^5
	+
	e^{-5 \cdot 0.04}
	\left[-\frac{3'000}{0.11} e^{-0.113}\right]_5^{20}
	-
	10'000 ( 1 + e^{-0.5})\\
	&=
	5'373.
\end{align*}

\newpage

