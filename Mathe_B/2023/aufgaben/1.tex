\vspace{1cm}
\fancyhead[C]{\normalsize\textbf{$\qquad$ Teil I: Offene Aufgaben}}
\renewcommand{\labelenumi}{\theenumi.}
\section*{Aufgabe 1 (40 Punkte)}
\vspace{0.4cm}
\subsection*{\aufgabe{a}{12}}
Estis lang ersehnter Traum ist endlich wahr geworden
- 
sie hat sich für den Ironman Hawaii qualifiziert.
Allerdings weiss sie, dass sie ihr Training optimieren muss, um ihre bestmögliche Leistung erbringen zu können.
Esti beschliesst, als Steuerungsparameter ihren $VO_2max$-Wert zu nutzen.
Dieser misst die maximale Menge an Sauerstoff, die ihr Körper während der Belastung nutzen kann, und stellt somit eine hilfreiche Grösse dar, um die Verbesserung ihres Fitnesslevels abzubilden.
Sie möchte die optimale Verteilung ihrer Trainingszeit zwischen
Schwimmen, Radfahren und Laufen finden, um ihren $VO_2max$-Wert zu maximieren.\\
Estis derzeitiger $VO_2max$-Wert beträgt $48$ und ihre Trainingszeit ist auf $15$ Stunden pro Woche begrenzt.
Sei $x$ die Anzahl an Stunden pro Woche, die sie für Schwimmen und Radfahren zusammen aufwendet und $y$ die Anzhal an Stunden pro Woche, die sie mit Laufen verbringt.
Esti folgt der Empfehlung ihres Trainers und trainiert pro Woche viermal so viel Radfahren wie Schwimmen. Ausserdem nimmt sie an, dass sich der Trainingseffekt bezüglich ihres, am Tag des Ironman-Events gemessenen, $VO_2max$-Werts wie folgt modellieren lässt:
\begin{align*}
	f(x,y) = 1.8 x^{\frac{2}{3}} y^{\frac{1}{3}}.
\end{align*}
\begin{enumerate}
	\item[\textbf{(a1)}]
	Formulieren Sie das Optimierungsproblem, das Esti lösen muss, um ihren $VO_2max$-Wert am Tag des Events zu optimieren.
	\item[\textbf{(a2)}] 
	Lösen Sie das Optimierungsproblem und bestimmen Sie die optimale Anzahl an Stunden pro Woche, die Esti für Schwimmen, Radfahren und Laufen aufwenden sollte, sowie den $VO_2max$-Wert am Tag des Ironman-Events, wenn sie diesem Trainingsplan folgt.\\
	\\
	\textit{Anmerkung: Es ist nicht notwendig, die hinreichende Bedingung für einen Extremwert zu überprüfen.}
\end{enumerate}
\ \\
\textbf{Lösung:}
\begin{mdframed}
\underline{\textbf{Vorgehensweise:}}
\renewcommand{\labelenumi}{\theenumi.}
\begin{enumerate}
\item[\textbf{(a1)}] 
\item[\textbf{(a2)}] 
\end{enumerate}
\end{mdframed}
\ \\
\underline{\textbf{(a1)} }\\


\newpage

\subsection*{\aufgabe{b}{14}}
Vier Werkstücke $P_1$, $P_2$, $P_3$ und $P_4$ müssen in einem gegebenen Produktionsprozess jeweils drei Maschinen $M_1$, $M_2$ und $M_3$ durchlaufen.
Die für die einzelnen Werkstücke pro Einheit benötigten Bearbeitungszeiten in Minuten sind in der folgenden Tabelle angegeben: 
\begin{table}[H]
	\centering
	%
	\begin{tabular}{c | c c c c}
		$ $  & $P_1$  &  $P_2$ &  $P_3$ & $P_4$ \\ 
		\hline
		$ M_1 $ & $ 10 $ & $ 20 $ & $ 20 $ & $15$  \\ 
		$ M_2 $ & $ 10 $ & $ 30 $ & $ 16 $ & $5$ \\
		$ M_3 $ & $ 20 $ & $ 20 $ & $ 10 $ & $25$
	\end{tabular}%
\end{table}
Jede Maschine läuft genau $8$ Stunden pro Tag.\\
\\
Der Nettogewinn in Schweizer Franken für die vier Werkstücke beträgt pro Einheit der Reihe nach
\begin{align*}
	q_1 = 5; \quad
	q_2=  10; \quad
	q_3=12; \quad 
	q_4=m.
\end{align*}
Der angestrebte Nettogewinn aus der Gesamtproduktion beträgt $250$ Schweizer Franken pro Tag.
\begin{enumerate}
	\item[\textbf{(b1)}]
	Beschreiben Sie die Beziehung zwischen Produktionszeiten, Laufzeit je Maschine pro Tag, Nettogewinn und der Anzahl der produzierten Werkstücke pro Tag in einem Gleichungssystem.
	
	\item[\textbf{(b2)}] 
	Für welche(s) $m > 0$ existiert ein realisierbarer Produktionsplan?
	\item[\textbf{(b3)}]
	Bestimmen Sie den Produktionsplan, d.h. die Anzahl produzierter Einheiten jedes Werkstücks unter Berücksichtigung der verfügbaren Kapazitäten der Maschinen und des angestrebten Nettogewinns, für $m = 5$.
\end{enumerate}
\ \\
\textbf{Lösung:}
\begin{mdframed}
\underline{\textbf{Vorgehensweise:}}
\renewcommand{\labelenumi}{\theenumi.}
\begin{enumerate}
\item[\textbf{(b1)}] 
\item[\textbf{(b2)}] 
\begin{enumerate}
	\item[1.] .
	\item[2.] 
\end{enumerate}
\end{enumerate}
\end{mdframed}


\newpage
\subsection*{\aufgabe{c}{14}}
Nach der letzten Preiserhöhung bei Strom möchten Sie in Solarpanele investieren und den erzeugten Strom verkaufen. 
Sie zahlen $10'000$ Schweizer Franken für den Kauf und die
Installation der Solarpanele.
Unter Berücksichtigung des kontinuierlich schlechter werdenden Wirkungsgrads nehmen Sie konservativ an, dass sich die Leistungsabgabe der Solarpanele, gemessen in Kilowattstunden pro Jahr, mit der Zeit wie folgt entwickelt:
\begin{align*}
	f(t) = 10'0000 e^{-0.1 t},
\end{align*}
wobei $t$ in Jahren gemessen wird.
Sie gehen davon aus, dass die Lebensdauer der Solarpanele $20$ Jahre beträgt. Nach diesen $20$ Jahren müssen Sie die Solarpanele zu einem Preis von $10'000$ Schweizer Franken entsorgen. Ausserdem nehmen Sie an, dass sich der Marktpreis in
Schweizer Franken für eine Kilowattstunde mit der Zeit wie folgt entwickelt:
\begin{align*}
	p(t)
	=
	0.3 e^{0.01t}.
\end{align*}
Derzeit sind die Zinssätze eher hoch, aber Sie sind sich sicher, dass sie nach einer Weile wieder niedriger werden. Sie gehen folglich für die nächsten $5$ Jahre von einer kontinuierlichen
Verzinsung zu einem Satz von $4 \%$ p.a. und anschliessend zu einem Satz von $2\%$ p.a. aus.

\begin{enumerate}
	\item[\textbf{(c1)}] 
	Bestimmen Sie die insgesamt erzeugte Energiemenge.
	\item[\textbf{(c2)}]
	Berechnen Sie den Nettobarwert Ihrer Investition.
\end{enumerate}
\ \\
\textbf{Lösung:}
\begin{mdframed}
\underline{\textbf{Vorgehensweise:}}
\begin{enumerate}
\item[\textbf{(c1)}] .
\item[\textbf{(c2)}]
\begin{enumerate}
	\item[1.] 
\end{enumerate}
\end{enumerate}
\end{mdframed}




\newpage

