%\newcommand{\ein}[2]{(#1) (#2 Punkte)}


\begin{Large}
\textbf{Teil I: Offene Aufgaben (36 Punkte)}
\end{Large}
\\
\\
\\
\textbf{Allgemeine Anweisungen für offene Fragen:}
\\
\renewcommand{\labelenumi}{(\roman{enumi})}
\begin{enumerate}
\item
Ihre Antworten müssen alle Rechenschritte enthalten,
diese müssen klar ersichtlich sein.
Verwendung korrekter mathematischer Notation wird erwartet
und fliesst in die Bewertung ein.

\item
Ihre Antworten zu den jeweiligen Teilaufgaben müssen in den dafür vorgesehenen Platz geschrie-
ben werden. Sollte dieser Platz nicht ausreichen, setzen Sie Ihre Antwort auf der Rückseite oder
dem separat zur Verfügung gestellten Papier fort. Verweisen Sie in solchen Fällen ausdrücklich
auf Ihre Fortsetzung. Bitte schreiben Sie zudem Ihren Vor- und Nachnamen auf jeden separaten
Lösungsbogen.

\item
Es werden nur Antworten im dafür vorgesehenen Platz bewertet. Antworten auf der Rückseite
oder separatem Papier werden nur bei einem vorhandenen und klaren Verweis darauf bewertet.

\item
Die Teilaufgaben werden mit den jeweils oben auf der Seite angegebenen Punkten bewertet.

\item
Ihre endgültige Lösung jeder Teilaufgabe darf nur eine einzige Version enthalten.

\item
Zwischenrechnungen und Notizen müssen auf einem getrennten Blatt gemacht werden. Diese
Blätter müssen, deutlich als Entwurf gekennzeichnet, ebenfalls abgegeben werden.
\end{enumerate}

\newpage
\section*{\hfil Aufgaben \hfil}
\vspace{1cm}
\section*{Aufgabe 1 (36 Punkte)}
\vspace{0.4cm}
\subsection*{\aufgabe{a}{8}}
Ein Investment generiert zum Zeitpunkt $ t $, für $ t \in [0,10], $ den stetigen Cashflow $ B(t) = a \ t + 10 $,
wobei $ a > 0  $.
Die Verzinsung erfolgt kontinuierlich zum Zinssatz $ i = 5 \% $.
Bestimmen Sie den Parameter $ a $ so, dass der Barwert des Investments $ 1'000 $ CHF beträgt.
\\
\\
\subsection*{\aufgabe{b}{8}}
Die folgende Tabelle beschreibt die jährlichen Payoffs dreier Wertpapiere zu identischem Ausgangspreis, abhängig von der jeweiligen konjunkturellen Lage:
\begin{table}[H]
	\centering
	%
		\begin{tabular}{l c c c}
			\hline
			Konjunktur & Aktie 1  &  Aktie 2 &  Aktie 3 \\ \hline
			Expansion & $ 1.5 $ & $ 3 $ & $ m $  \\ 
			wirtschaftliche Stabilität & $ 1.5 $ & $ 2 $ & $ 0.5 $ \\ 
			Rezession & $ 1.5 $ & $ 0.5 $ & $ 1.5 $  \\ \hline
		\end{tabular}%
	
\end{table}
Ein Investor möchte die drei Wertpapiere linear zu folgendem Auszahlungsschema kombinieren:
\begin{table}[H]
	\centering
	%
	\begin{tabular}{l c }
		\hline
		Konjunktur & Payoff des Investors \\ \hline
		Expansion & $ 2m $  \\ 
		wirtschaftliche Stabilität & $ 1.0 $  \\ 
		Rezession & $ 0.5 $   \\ \hline
	\end{tabular}%
	
\end{table}
Bestimmen Sie mit Hilfe des \textit{Gauss Verfahrens} alle möglichen Parameterwerte $ m $, für die das gewünschte Auszahlungsschema durch eine Kombination der drei Aktien möglich ist.
 \\
\\
\subsection*{\aufgabe{c}{10}}
Die (fixen) Entwicklungskosten für ein neues Produkt betragen $ 20'000 $ (CHF).
Jede Einheit des Produkts kann zu einem Preis von $ p $ (CHF) verkauft werden, 
die Produktionskosten pro Einheit betragen $ 2 $ (CHF).
Sie entscheiden sich dazu, Ihr Produkt mit einer Marketingkampagne zu bewerben.
Der Erfolg der Kampagne hängt von dem Betrag $ a $ (CHF),
der für die Werbung ausgegeben wird, sowie dem Preis $ p $ (CHF) Ihres Produkts ab.
In der Tat bestimmen Sie , dass Sie, bei einem Werbeaufwand von $ a $ (CHF) und einem Verkaufspreis von $ p $ (CHF),
\begin{align*}
3'000 + 4 \sqrt{a} - 20 p
\end{align*}
Einheiten Ihres Produktes verkaufen.\\
Bestimmen Sie $ a $ und $ p $ so, dass Ihr Gewinn (d.h. Verkaufserlöse abzüglich Kosten, inklusive der Kosten für Werbung) maximal wird.
\ \\
\subsection*{\aufgabe{d}{10}}
Ein Rechteck $ R $ habe seine Ecken in den Punkten $ (0,0),(x,0),(0,y), $ und $ (x,y) $, mit $ x,y > 0 $.
Außerdem sei die Distanz zwischen dem Punkt $ (x,y)  $ und dem Punkt $ (a,0) $ gleich 4, wobei $ a \in (4,8) $.\\
Bestimmen Sie mögliche Punkte $ (x,y) $, sodass die Fläche von $ R $ ein Extremum annimmt (also maximal oder minimal wird).\\
\textbf{Bemerkung:} 
\begin{enumerate}[label=(\arabic*)]
	\item Es ist nicht nötig zu überprüfen, ob die gefundenen potentiellen Kandidaten für ein Extremum tatsächlich einen maximalen oder minimalen Flächeninhalt von $ R $ erzeugen.
	\item Die Distanz zwischen zwei Punkten $ P_1 = (x_1,y_1) $ und $ P_2 = ( x_2,y_2) $ entspricht\\
	$ d = \sqrt{(x_1 - x_2)^2 + (y_1 - y_2)^2} $.
	Demnach gilt: $ d^2 = (x_1 -x_2)^2 + (y_1 - y_2)^2 $.
	\item  Für diese Aufgabe müssen Sie die Terme der Lösung(en) nicht vereinfachen.
\end{enumerate}

\newpage


\fancyhead[C]{\normalsize\textbf{$\qquad$ Teil II: Multiple-Choice}}
\begin{Large}
\textbf{Teil II: Multiple-Choice-Fragen (64 Punkte)}
\end{Large}
\\
\\
\\
\textbf{Allgemeine Anweisungen für Multiple-Choice-Fragen:}
\\
\renewcommand{\labelenumi}{(\roman{enumi})}
\begin{enumerate}
\item
Die Antworten auf die Multiple-Choice-Fragen müssen im dafür vorgesehenen Antwortbogen ein-
getragen werden. Es werden ausschliesslich Antworten auf diesem Antwortbogen bewertet. Der
Platz unter den Fragen ist nur für Notizen vorgesehen und wird nicht korrigiert.

\item
Jede Frage hat nur eine richtige Antwort. Es muss also auch jeweils nur eine Antwort angekreuzt
werden.

\item
Falls mehrere Antworten angekreuzt sind, wird die Antwort mit 0 Punkten bewertet, auch wenn
die korrekte Antwort unter den angekreuzten ist.

\item
Bitte lesen Sie die Fragen sorgfältig.

\end{enumerate}
\newpage
\section*{Aufgabe 2 (32 Punkte)}
\vspace{0.4cm}
\subsection*{\frage{1}{3}}
Die Funktion $ f(x,y)  = x$ hat unter der Nebenbedingung $\varphi(x,y) = \frac{x^2}{36} + \frac{(y-3)^2}{16} = 1 $ ihr Maximum im Punkt
\renewcommand{\labelenumi}{(\alph{enumi})}
\begin{enumerate}
\item $ P = (-6,3) $.
\item $ P = (5,-3) $.
\item $ P = (0,7) $.
\item $ P = (6,3) $.
\end{enumerate}
\ \\
\subsection*{\frage{2}{3}}
Eine zweimal differenzierbare Funktion $ f  $ hat ein lokales Maximum im Punkt $ (x_0,y_0) $.
Sei $ g  $ die Funktion definiert durch $ g(x,y) = - f(-x,-y)  $ und $ D_g = D_f $. Dann gilt:

\renewcommand{\labelenumi}{(\alph{enumi})}
\begin{enumerate}
\item $ g $ hat ein lokales Maximum in $ (x_0,y_0) $.
\item $ g $ hat ein lokales Minimum in $ (x_0,y_0) $.
\item
$ g $ hat ein lokales Maximum in $ (-x_0,-y_0) $.
\item
$ g $ hat ein lokales Minimum in $ (-x_0,-y_0) $.
\end{enumerate}
\ \\
\subsection*{\frage{3}{4}}
Die Funktion $ f $ in zwei Variablen hat unter der Nebenbedingung $ \varphi(x,y) = x^2 + 3 y - 7 = 0 $ ein lokales Maximum im Punkt $ (x_0,y_0) = (1,2) $.\\
\\
Die Steigung der Tangente  an die Niveaulinie von $ f $ in $ (x_0,y_0) = (1,2) $ hat den Wert:
\renewcommand{\labelenumi}{(\alph{enumi})}
\begin{enumerate}
\item 
$-\frac{3}{2}$.
\item 
$-\frac{2}{3}$.
\item 
$\frac{3}{2}$.
\item
$\frac{2}{3}$.
\end{enumerate}
\ \\
\subsection*{\frage{4}{2}}
Für eine stetige Funktion $ f $ und $ a,x \in D_f $ mit $ a \leq x  $ ist die Integralfunktion $ I $ definiert als $ I(x) = \int_a^x f(t) \ dt $. Es gilt:
\renewcommand{\labelenumi}{(\alph{enumi})}
\begin{enumerate}
	\item 
	$ I $ ist eine Stammfunktion von $ f $.
	\item
	$ f^\prime(x) = I(x) $.
	\item
	$ I^\prime(x) = f(x) - f(a) $.
	\item
	$ I $ ist nicht differenzierbar
\end{enumerate}
\ \\
\subsection*{\frage{5}{2}}
Sei $ F $ eine Stammfunktion von $ f $ und $ g $ eine differenzierbare Funktion.
Es folgt:
\renewcommand{\labelenumi}{(\alph{enumi})}
\begin{enumerate}
\item 
$ \int f(g(x)) dx = F(x) +C, \quad C \in \mathbb{R} $.
\item 
$ \int f(g(x)) dx = F(g(x)) +C, \quad C \in \mathbb{R} $.
\item 
$ \int f(g(x)) f^\prime(x) dx = F(g(x)) +C, \quad C \in \mathbb{R} $.
\item 
$ \int f(g(x)) g^\prime(x) dx = F(g(x)) +C, \quad C \in \mathbb{R} $.
\end{enumerate}
\ \\
\subsection*{\frage{6}{3}}
Das unbestimmte Integral
\begin{align*}
\int \left[ 6x+ (2 x^2 + 1) e^{x^2} \right] dx
\end{align*}
ist
\renewcommand{\labelenumi}{(\alph{enumi})}
\begin{enumerate}
\item 
$ x^2 + x \ e^{x^2} + C, \quad C \in \mathbb{R} $.
\item 
$ 3x  + 2x  \ e^{x^2} + C, \quad C \in \mathbb{R} $.
\item
$ 3x^2  + 2x  \ e^{x^2} + C, \quad C \in \mathbb{R} $.
\item
$ 3x^2  + x  \ e^{x^2} + C, \quad C \in \mathbb{R} $.
\end{enumerate}
\ \\
\subsection*{\frage{7}{3}}
Gegeben sei die Funktion $ f $ definiert durch
\begin{align*}
f(x) = 
\begin{cases}
a x^2 + \frac{1}{2} &\ \textrm{für } 0 \leq x \leq 1\\
\quad 0 &\qquad \textrm{sonst}
\end{cases}.
\end{align*}
$ f $ ist eine Dichtefunktion für
\renewcommand{\labelenumi}{(\alph{enumi})}
\begin{enumerate}
\item 
$ a = \frac{1}{2} $.
\item
$ a = \frac{3}{2} $.
\item
$ a = \frac{5}{2} $.
\item
Für kein $ a \in \mathbb{R} $.
\end{enumerate}
\ \\
\subsection*{\frage{8}{4}}
Gegeben sei die Funktion
\begin{align*}
f(x) = 
\begin{cases}
a x + \frac{1}{16} &\ \textrm{für } 0 \leq x \leq 8\\
\quad 0 &\qquad \textrm{sonst}
\end{cases}.
\end{align*}
Der Parameter $ a $ wird so gewählt, dass $ f $ die Dichtefunktion einer stetigen Zufallsvariablen $ X $ ist. Der Erwartungswert $ \mathbb{E}[X] $ von $ X $ ist:
\renewcommand{\labelenumi}{(\alph{enumi})}
\begin{enumerate}
	\item 
	$ \mathbb{E}[X] = \frac{25}{16} $.
	\item
	$ \mathbb{E}[X] = \frac{14}{3} $.
	
	\item
	$ \mathbb{E}[X] = \frac{-5}{3} $.
	\item
	$ \mathbb{E}[X] = \frac{7}{3} a $.
\end{enumerate}
\ \\
\subsection*{\frage{9}{2}}
Die Vektoren $ \textbf{a} $ und $ \textbf{b} $ seien orthogonal und $ \textbf{a}  $ habe die Länge $ \| \textbf{a} \| = 3 $.
Dann gilt:
\renewcommand{\labelenumi}{(\alph{enumi})}
\begin{enumerate}
	\item 
	$ \textbf{a} \cdot ( \textbf{a} + \textbf{b}) = 0 $.
	\item
	$ \textbf{a} \cdot ( \textbf{a} + \textbf{b}) = 3 $.
	
	\item
$ \textbf{a} \cdot ( \textbf{a} + \textbf{b}) = 9 $.
	\item
	$ \textbf{a} \cdot ( \textbf{a} + \textbf{b}) $ kann ohne Wissen über die Komponenten von $ \textbf{a} $ und $ \textbf{b} $ nicht bestimmt werden.
\end{enumerate}
\ \\
\subsection*{\frage{10}{2}}
Gegeben seien die Vektoren:
\begin{align*}
\textbf{a}
=
\begin{pmatrix}
1 \\ 1 \\ -1
\end{pmatrix},
\textbf{b}
=
\begin{pmatrix}
0 \\ 1 \\ 3
\end{pmatrix},
\textbf{c}
=
\begin{pmatrix}
-1 \\ 0 \\ 2
\end{pmatrix},
\textbf{d}
=
\begin{pmatrix}
1 \\ 4 \\ 8
\end{pmatrix},
\textbf{e}
=
\begin{pmatrix}
1 \\ 1 \\ 1
\end{pmatrix}
\end{align*}
Welches der folgenden Systeme ist eine Basis des $ \mathbb{R}^3 $?
\renewcommand{\labelenumi}{(\alph{enumi})}
\begin{enumerate}
	\item 
	$ \lbrace \textbf{a}, \textbf{b} , \textbf{c} \rbrace  $.
	\item
	$ \lbrace \textbf{a}, \textbf{b} , \textbf{d} \rbrace  $.
	
	\item
	$ \lbrace \textbf{b}, \textbf{c} , \textbf{e} \rbrace  $.
	\item
$ \lbrace \textbf{a}, \textbf{b} , \textbf{c},\textbf{d}, \textbf{e} \rbrace  $.
\end{enumerate}
\ \\
\subsection*{\frage{11}{2}}
$ A $ sei eine $ 7 \times 5 $ Matrix. Das System von linearen Gleichungen $ A \textbf{x} = \textbf{b} $ habe unendliche viele Lösungen und der Lösungsraum habe die Dimension $ 3 $. Dann gilt:
\renewcommand{\labelenumi}{(\alph{enumi})}
\begin{enumerate}
	\item 
	$ \mathrm{rg}(A)< \mathrm{rg}(A;\textbf{b} ) =3 $.
	\item
	$ \mathrm{rg}(A)= \mathrm{rg}(A;\textbf{b} ) =2 $.
	
	\item
	$ \mathrm{rg}(A)< \mathrm{rg}(A;\textbf{b} ) =2 $.
	\item
	$ \mathrm{rg}(A)= \mathrm{rg}(A;\textbf{b} ) =3 $.
\end{enumerate}
\ \\
\subsection*{\frage{12}{2}}
$ A $ und $ B $ seien reguläre Matrizen. Ausserdem sei $ A $ symmetrisch.
Der Ausdruck
\begin{align*}
B^\top (AB)^\top (B^{-1} A^{-1})^\top B (AB)^{-1}
\end{align*}
entspricht:
\begin{enumerate}
	\item 
	$ (A^\top B^\top)^{-1} $.
	\item
	$ (B^{-1} A)^\top $.
	
	\item
	$ (A^{-1} B )^\top $
	\item
	Keinem der obigen Terme.
\end{enumerate}
\newpage
\section*{Aufgabe 3 (32 Punkte)}
\vspace{0.4cm}

\subsection*{\frage{1}{4}}
$ X $ sei eine stetige Zufallsvariable mit Dichtefunktion
\begin{align*}
f(x) = 
\begin{cases}
\frac{2}{3} x + \sqrt{x} &\ \textrm{für } 0 \leq x \leq 1\\
\quad 0 &\qquad \textrm{sonst}
\end{cases}
\end{align*}
und Erwartungswert
\begin{align*}
\mathbb{E}[X] = \frac{28}{45}.
\end{align*}
Dann hat das Integral $ \int_0^1 (x+1) f(x) \ dx $ den Wert:
\renewcommand{\labelenumi}{(\alph{enumi})}
\begin{enumerate}
\item 
$\frac{28}{45}$.
\item
$\frac{73}{45}$.
\item
$\frac{12}{45}$.
\item
$\frac{52}{45}$.
\end{enumerate}
\ \\
\subsection*{\frage{2}{4}}
Für welchen Wert von $ t \in \mathbb{R} $ sind die Vektoren $ \textbf{u} = \begin{pmatrix}
t \\ t-1 \\ -6
\end{pmatrix} $ und $ \textbf{v} = \begin{pmatrix}
t+5 \\ -1 \\ 1
\end{pmatrix} $ orthogonal und die Länge von $ \textbf{u} $ gleich $ \sqrt{37} $?
\renewcommand{\labelenumi}{(\alph{enumi})}
\begin{enumerate}
	\item 
	$t = -5$.
	\item
	$t = 0$.
	\item
	$ t= 1 $.
	\item
	Für kein $ t \in \mathbb{R} $.	
\end{enumerate}

\newpage
\subsection*{\frage{3}{3}}
Die $3 \times 4$ Matrix
\begin{align*}
A
=
\begin{pmatrix}
1 & 2  & 1 & -1  \\
9 & 5  & 2 & 2 \\
7 & 1  & 0 &  4  
\end{pmatrix}
\end{align*}
\renewcommand{\labelenumi}{(\alph{enumi})}
\begin{enumerate}
\item 
hat Rang $1$.
\item
hat Rang $2$.
\item
hat Rang $3$.
\item
hat Rang $4$.
\end{enumerate}
\ \\
\subsection*{\frage{4}{5}}
Gegeben sei die $ 3 \times 3 $ Matrix
\begin{align*}
A =
\begin{pmatrix}
1 & 0 & 1 \\
0 & -1 & 0 \\
1 & 0 &0
\end{pmatrix}.
\end{align*}
Es gilt:
\renewcommand{\labelenumi}{(\alph{enumi})}
\begin{enumerate}
\item 
$A^{-1}
= 
\begin{pmatrix}
1 & 1 & -1  \\
0 & -1 & 0 \\
0 & 0 & 1
\end{pmatrix}$.
\item 
$A^{-1}
= 
\begin{pmatrix}
0 & 0 & 1  \\
0 & -1 & 0 \\
1 & 0 & -1
\end{pmatrix}$.
\item
$A^{-1}
= 
\begin{pmatrix}
0 & 0 & -1  \\
1 & 1 & 1 \\
0 & 1 & 0
\end{pmatrix}$.
\item
$ A $ ist singulär.
\end{enumerate}
\ \\
\subsection*{\frage{5}{4}}
Sei
\begin{align*}
A
=
\begin{pmatrix}
1 & 2 & 1\\
0 & 1 & 0\\
1 & -1 & 1
\end{pmatrix}.
\end{align*}
Die Matrix $ A^2 $ hat den Eigenwert:
\renewcommand{\labelenumi}{(\alph{enumi})}
\begin{enumerate}
\item 
$ 2 $.
\item
$ 4 $.
\item
$ 6 $.
\item
$ 8 $.
\end{enumerate}
\ \\
\subsection*{\frage{6}{4}}
Das Anfangswertproblem
\begin{align*}
&y_{k+1} -(1+a) y_k = 2 a, \ \textrm{wobei } a \neq -1, a \neq 0,\\
&y_0 = 2
\end{align*}
hat die Lösung
\renewcommand{\labelenumi}{(\alph{enumi})}
\begin{enumerate}
	\item 
	$ y_k = -4 ( 1+a)^k $.
	\item
	$ y_k = 2 ( 1+a)^k  -1$.
	\item
	$ y_k = 4 ( 1+a)^k -2 $.
	\item
	$ y_k = 8 ( 1+a)^k -3$.
\end{enumerate}
\ \\
\subsection*{\frage{7}{3}}
Die allgemeine Lösung der Differenzengleichung
\begin{align*}
3 (y_{k+1}   -y_k  )+ 5 = 2 y_{k+1} -y_k + 12 
\end{align*}
ist
\renewcommand{\labelenumi}{(\alph{enumi})}
\begin{enumerate}
\item
oszillierend und konvergent.
\item
oszillierend und divergent.	
\item 
monoton und konvergent.
\item
monoton und divergent.
\end{enumerate}
\ \\
\subsection*{\frage{8}{5}}
Die allgemeine Lösung der linearen Differenzengleichung
\begin{align*}
2 (a + 2) y_{k+1} - 2 y_k + 2 (a^2 - 4) = 0, \quad k = 0,1,2,...,
\end{align*}
mit $ a \in \mathbb{R} \setminus \{-2,-1  \} $, konvergiert monoton gegen $ 0 $ genau dann, wenn
\renewcommand{\labelenumi}{(\alph{enumi})}
\begin{enumerate}
	\item 
	$ a > -2 $.
	\item
	$a > -1$.
	\item
	$ a = 2 $.
	\item
	$ a = 1 $.
\end{enumerate}