\fancyhead[C]{\normalsize\textbf{$\qquad$ Teil II: Multiple-Choice}}
\section*{Aufgabe 2 (32 Punkte)}
\vspace{0.4cm}
\subsection*{\frage{1}{3}}
Die Funktion $ f(x,y)  = x$ hat unter der Nebenbedingung $\varphi(x,y) = \frac{x^2}{36} + \frac{(y-3)^2}{16} = 1 $ ihr Maximum im Punkt
\renewcommand{\labelenumi}{(\alph{enumi})}
\begin{enumerate}
	\item $ P = (-6,3) $.
	\item $ P = (5,-3) $.
	\item $ P = (0,7) $.
	\item $ P = (6,3) $.
\end{enumerate}
\ \\
\textbf{Lösung:}
\begin{mdframed}
\underline{\textbf{Vorgehensweise:}}
\renewcommand{\labelenumi}{\theenumi.}
\begin{enumerate}
\item Bestimme das in der Nebenbedingung beschriebene Objekt.
\item Bestimme die richtige Antwort.
\end{enumerate}
\end{mdframed}

\underline{1. Bestimme das in der Nebenbedingung beschriebene Objekt}\\
Wegen 
\begin{align*}
\varphi(x,y) = \frac{x^2}{36} + \frac{(y-3)^2}{16} = 
\frac{x^2}{6^2} + \frac{(y-3)^2}{4^2} = 1
\end{align*}
beschreibt die Nebenbedingung eine Ellipse mit dem Mittelpunkt $ (0,3) $ und den Halbachsen $ a = 6 $ bezüglich der $ x $-Achse und $ b = 4 $ bezüglich der $ y $-Achse.\\
\\
\underline{2. Bestimme die richtige Antwort}\\
Da $ f(x,y) = x $ erhalten wir das Maximum unter der Nebenbedingung am Endpunkt der Halbachse bezüglich $ x $. Dieser liegt bei $ (6,3) $. Das heißt, dass $ (6,3) $ das Maximum ist.\\

%wenn wir die Halbachse bezüglich $ x $ von dem Mittelpunkt $ (0,3) $ ausgehen. 

Damit ist die Antwort (d) korrekt.
\newpage

\subsection*{\frage{2}{3}}
Eine zweimal differenzierbare Funktion $ f  $ hat ein lokales Maximum im Punkt $ (x_0,y_0) $.
Sei $ g  $ die Funktion definiert durch $ g(x,y) = - f(-x,-y)  $ und $ D_g = D_f $. Dann gilt:

\renewcommand{\labelenumi}{(\alph{enumi})}
\begin{enumerate}
	\item $ g $ hat ein lokales Maximum in $ (x_0,y_0) $.
	\item $ g $ hat ein lokales Minimum in $ (x_0,y_0) $.
	\item
	$ g $ hat ein lokales Maximum in $ (-x_0,-y_0) $.
	\item
	$ g $ hat ein lokales Minimum in $ (-x_0,-y_0) $.
\end{enumerate}
\ \\
\textbf{Lösung:}
\begin{mdframed}
	\underline{\textbf{Vorgehensweise:}}
	\renewcommand{\labelenumi}{\theenumi.}
	\begin{enumerate}
		\item Bestimme einen stationären Punkt von $ g $.
		\item Bestimme die Art des stationären Punkts.
	\end{enumerate}
\end{mdframed}
\underline{1. Bestimme einen stationären Punkt von $ g $}\\
Die partiellen Ableitungen von $ g $ sind mit der Kettenregel gegeben durch:
\begin{align*}
g_x(x,y) &=  f_x(-x,-y)\\
g_y(x,y) &=  f_y(-x,-y)
\end{align*}
Wegen 
\begin{align*}
g_x(-x_0,-y_0) = f_x(x_0,y_0) = 0\\
g_y(-x_0,-y_0) = f_y(x_0,y_0) = 0
\end{align*}
ist $ (-x_0,-y_0) $ ein stationärer Punkt.\\
\\
\underline{2. Bestimme die Art des stationären Punkts}\\
Da $ (x_0,y_0)  $ ein lokales Maximum von $ f $ ist, ist $ (-x_0,-y_0) $ ein lokales Minimum  von $ g $.
Dies folgt wegen
\begin{align*}
g_{xx}(-x_0,-y_0) =  - \underbrace{f_{xx}(x_0,y_0)}_{< 0 } > 0\\
g_{yy}(-x_0,-y_0) =  - \underbrace{f_{yy}(x_0,y_0)}_{< 0 } > 0
\end{align*}
und
\begin{align*}
g_{xx}(-x_0,-y_0) g_{yy}(-x_0,-y_0) - (g_{xy}(-x_0,-y_0))^2
=
(-1)^2f_{xx}(x_0,y_0)f_{yy}(x_0,y_0) - (f_{yy}(x_0,y_0))^2 > 0. 
\end{align*}
\ \\
Damit ist die Antwort (d) korrekt.
\newpage
\subsection*{\frage{3}{4}}
Die Funktion $ f $ in zwei Variablen hat unter der Nebenbedingung $ \varphi(x,y) = x^2 + 3 y - 7 = 0 $ ein lokales Maximum im Punkt $ (x_0,y_0) = (1,2) $.\\
\\
Die Steigung der Tangente  an die Niveaulinie von $ f $ in $ (x_0,y_0) = (1,2) $ hat den Wert:
\renewcommand{\labelenumi}{(\alph{enumi})}
\begin{enumerate}
	\item 
	$-\frac{3}{2}$.
	\item 
	$-\frac{2}{3}$.
	\item 
	$\frac{3}{2}$.
	\item
	$\frac{2}{3}$.
\end{enumerate}
\ \\
\textbf{Lösung:}
\begin{mdframed}
\underline{\textbf{Vorgehensweise:}}
\renewcommand{\labelenumi}{\theenumi.}
\begin{enumerate}
\item Verwende den Satz über implizite Funktionen.
\item Verwende das lokale Maximum von $ f $.
\end{enumerate}
\end{mdframed}

\underline{1. Verwende den Satz über implizite Funktionen}\\
Wir betrachten die Nebenbedingung $ \varphi $. Für diese gilt:
\begin{align*}
\varphi_x(x,y) = 2x \ &\Rightarrow \ \varphi_y(x_0,y_0) = 2\\
\varphi_y(x,y) = 3 \ &\Rightarrow \ \varphi_y(x_0,y_0) = 3.
\end{align*}
Damit können wir den Satz über implizite Funktionen anwenden. Dieser liefert uns eine stetig differenzierbare Funktion $ y : (x_0- \delta, x_0 + \delta) \to \mathbb{R} $ mit $ y(x_0) = y_0 $, sodass
\begin{align*}
\varphi(\underbrace{x, y(x)}_{c(x):=}) = \varphi(c(x)) =  0 
\end{align*}
für alle $ x \in (x_0- \delta, x_0 + \delta)   $ gilt.
Damit finden wir ein kleines Intervall um $ x_0 $, worauf die $ y $-Variable nach $ x $ aufgelöst werden kann.
Dies soll durch die Notation $ y(x) $ deutlich gemacht werden. 
Wegen 
\begin{align*}
c^\prime(x) = 
\begin{pmatrix}
1 \\ y^\prime(x)
\end{pmatrix}
\end{align*}
ist die Steigung der Tangente an die Niveaulinie $ \varphi(x,y) = 0 $ in $ (x_0,y_0) = (1,2) $ durch $ y^\prime(1)  $ gegeben.
Außerdem gilt
\begin{align*}
\frac{\mathrm{d}}{\mathrm{dx}}
(\varphi(x, y(x)) )
=
\varphi_x(x,y(x)) + y^{\prime}(x) \varphi_y(x,y(x)) = 0
\ \Leftrightarrow \
y^\prime(x)
=
-\frac{\varphi_x(x,y(x))}{\varphi_y(x,y(x))}
\end{align*}
für alle $ x \in (x_0 - \delta,x_0 + \delta) $. Damit erhalten wir:
\begin{align*}
y^\prime(1) = - \frac{\varphi_x(1,2)}{\varphi_y(1,2)} = - \frac{2}{3}.
\end{align*}

\newpage
\underline{2. Verwende das lokale Maximum von $ f $}\\
Da $ f $ ein lokales Maximum in $ (x_0,y_0) $ unter der Nebenbedingung $ \varphi(x,y) = 0 $ hat und $ c(x) $ die Nebenbedingung für $ x \in (x_0- \delta, x_0 + \delta) $ erfüllt erhalten wir:
\begin{align*}
 \frac{\mathrm{d}}{\mathrm{dx}} (f(c(x))) =
&  \textbf{grad} f(c(x)) \cdot c^\prime(x)\\
\ \Rightarrow \
&0 =\textbf{grad} f(c(1)) \cdot c^\prime(1)
=
\textbf{grad} f(1,2) \cdot c^\prime(1) 
=
f_x(1,2) + y^\prime(1) f_y(1,2)\\
\ \Leftrightarrow \
&y^\prime(1) = - \frac{f_x(1,2)}{f_y(1,2)}.
\end{align*}
Wegen 
\begin{align*}
y^\prime(1)= - \frac{f_x(1,2)}{f_y(1,2)} = - \frac{\varphi_x(1,2)}{\varphi_y(1,2)} 
\end{align*}
sind $ \textbf{grad} f(1,2)  $ und $ \textbf{grad} \varphi(1,2)  $ linear abhängig.
Da diese jeweils orthogonal zu den Tangenten der Niveaulinien sind, ist die Steigung der Niveaulinie von $ f $ im Punkt $ (x_0,y_0) = (1,2) $ gegeben durch $ -\frac{2}{3} $.\\
\\
Damit ist die Antwort (b) korrekt.
%ist die Steigung der Niveaulinie von $ f $ im Punkt $ (x_0,y_0) = (1,2) $ gegeben durch
%\begin{align*}
%y^\prime(1)= - \frac{f_x(1,2)}{f_y(1,2)} = - \frac{\varphi_x(1,2)}{\varphi_y(1,2)} = - \frac{2}{3}.
%\end{align*}

\newpage

\subsection*{\frage{4}{2}}
Für eine stetige Funktion $ f $ und $ a,x \in D_f $ mit $ a \leq x  $ ist die Integralfunktion $ I $ definiert als $ I(x) = \int_a^x f(t) \ dt $. Es gilt:
\renewcommand{\labelenumi}{(\alph{enumi})}
\begin{enumerate}
	\item 
	$ I $ ist eine Stammfunktion von $ f $.
	\item
	$ f^\prime(x) = I(x) $.
	\item
	$ I^\prime(x) = f(x) - f(a) $.
	\item
	$ I $ ist nicht differenzierbar
\end{enumerate}
\ \\
\textbf{Lösung:}
\begin{mdframed}
\underline{\textbf{Vorgehensweise:}}
\renewcommand{\labelenumi}{\theenumi.}
\begin{enumerate}
\item Nehme eine bekannte Stammfunktion $ F $ von $ f $ an.
\end{enumerate}
\end{mdframed}

\underline{1. Nehme eine bekannte Stammfunktion $ F $ von $ f $ an}\\
Angenommen $ F $ sei eine Stammfunktion von $ f $.
Dann gilt:
\begin{align*}
I(x)
=
\int \limits_{a}^x f(t) \ dt
=
F(t) \bigg|_a^x
=
F(x) \underbrace{- F(a)}_{ = + C}
= F(x) + C.
\end{align*}
Die Integralfunktion $ I(x) $ unterscheidet sich nur um eine Konstante von $ F $.
Damit erhalten wir $ I^\prime = F^\prime =f $.\\
\\
Also ist die Antwort (a) korrekt. 
\newpage
\subsection*{\frage{5}{2}}
Sei $ F $ eine Stammfunktion von $ f $ und $ g $ eine differenzierbare Funktion.
Es folgt:
\renewcommand{\labelenumi}{(\alph{enumi})}
\begin{enumerate}
	\item 
	$ \int f(g(x)) dx = F(x) +C, \quad C \in \mathbb{R} $.
	\item 
	$ \int f(g(x)) dx = F(g(x)) +C, \quad C \in \mathbb{R} $.
	\item 
	$ \int f(g(x)) f^\prime(x) dx = F(g(x)) +C, \quad C \in \mathbb{R} $.
	\item 
	$ \int f(g(x)) g^\prime(x) dx = F(g(x)) +C, \quad C \in \mathbb{R} $.
\end{enumerate}
\ \\
\textbf{Lösung:}
\begin{mdframed}
\underline{\textbf{Vorgehensweise:}}
\renewcommand{\labelenumi}{\theenumi.}
\begin{enumerate}
\item Finde die korrekte Antwort durch Differenzieren .
\end{enumerate}
\end{mdframed}

\underline{1. Finde die korrekte Antwort durch Differenzieren}\\
Da $ F $ eine Stammfunktion von $ f $ ist, gilt $ F^\prime = f $.
Das Differenzieren des Integrals liefert uns den Integranden.
Wir werden beide Seiten der Antwortmöglichkeiten differenzieren.
Damit erhalten wir
\begin{align*}
f(g(x)) = F^\prime(x) = f(x).
\end{align*}
Mit $ g(x) \neq x $ haben wir ein Gegenbeispiel gefunden, da dann $ f(g(x)) $ nicht $ f(x) $ entspricht. Damit ist (a) falsch.\\
\\
Differenzieren der Möglichkeit (b) liefert mit der Kettenregel:
\begin{align*}
f(g(x)) = F^\prime(g(x)) g^\prime(x) = f(g(x)) g^\prime(x).
\end{align*}
Ein Gegenbeispiel erhält man durch $ f(x) = g(x) = c \neq 0 $.
Dies erkennen wir an dem Widerspruch:
\begin{align*}
c = \underbrace{c}_{F^\prime} \cdot \underbrace{0}_{g^\prime}.
\end{align*}
Damit ist (b) falsch.
Dies ist ebenso ein Gegenbeispiel zur (c).
\\
Wegen
\begin{align*}
\left(F(g(x))\right)^\prime = 
F^\prime (g(x)) \cdot g^\prime(x)
= 
f(g(x)) \cdot g^\prime(x)
\end{align*}
entspricht die Ableitung von $ F(g(x)) $ dem Integranden in Möglichkeit (d).
Dies ist gerade die Substitutionsregel für Integrale.\\
\\
Also ist die Antwort (d) korrekt.


\newpage

\subsection*{\frage{6}{3}}
Das unbestimmte Integral
\begin{align*}
\int \left[ 6x+ (2 x^2 + 1) e^{x^2} \right] dx
\end{align*}
ist
\renewcommand{\labelenumi}{(\alph{enumi})}
\begin{enumerate}
	\item 
	$ x^2 + x \ e^{x^2} + C, \quad C \in \mathbb{R} $.
	\item 
	$ 3x  + 2x  \ e^{x^2} + C, \quad C \in \mathbb{R} $.
	\item
	$ 3x^2  + 2x  \ e^{x^2} + C, \quad C \in \mathbb{R} $.
	\item
	$ 3x^2  + x  \ e^{x^2} + C, \quad C \in \mathbb{R} $.
\end{enumerate}
\ \\
\textbf{Lösung:}
\begin{mdframed}
\underline{\textbf{Vorgehensweise:}}
\renewcommand{\labelenumi}{\theenumi.}
\begin{enumerate}
\item Bestimme die korrekte Antwort durch Differenzieren.
\end{enumerate}
\end{mdframed}

\underline{1. Bestimme die korrekte Antwort durch Differenzieren}\\
Die Antwortmöglichkeiten bestehen alle aus zwei Summanden. Die Konstante $ C $ können wir ignorieren, da diese bei der Ableitung wegfällt.\\
\\
Wir definieren die Funktion $ h(x) = f(x) + g(x) $, wobei $ f $ und $ g $ differenzierbar sind.
Dann gilt für die Ableitung:
\begin{align*}
h^\prime (x) = (f(x)+g(x))^\prime
= 
f^\prime(x)+ g^\prime(x).
\end{align*}
Das bedeutet, dass wir die Summanden in den Antwortmöglichkeiten separat untersuchen können.\\
\\
Wir betrachten zuerst den Term $ x e^{x^2} $, welcher in allen Antworten vorkommt.
Für diesen gilt mit der Produkt-und Kettenregel:
\begin{align*}
\frac{\mathrm{d}}{\mathrm{dx}}(x e^{x^2})
=
1 \cdot e^{x^2} + x \ 2x \ e^{x^2}
= 
(2x^2 + 1) e^{x^2}.
\end{align*}
Nach diesem Resultat sind die Antworten (b) und (c) falsch.
Wegen
\begin{align*}
\frac{\mathrm{d}}{\mathrm{dx}} (3x^2) = 6x
\end{align*}
ist die Antwort (a) falsch.\\
\\
Also ist die Antwort (d) korrekt.\\
\\
Dies erkennen wir auch mit der Ketten-und Produktregel:
\begin{align*}
\left(3x^2  + x  \ e^{x^2} + C\right)^\prime
=
6x + e^{x^2} + 2x^2 x e^{x^2}
=
6x (2x^2 +1 ) e^{x^2}.
\end{align*}
\newpage
\subsection*{\frage{7}{3}}
Gegeben sei die Funktion $ f $ definiert durch
\begin{align*}
f(x) = 
\begin{cases}
a x^2 + \frac{1}{2} &\ \textrm{für } 0 \leq x \leq 1\\
\quad 0 &\qquad \textrm{sonst}
\end{cases}.
\end{align*}
$ f $ ist eine Dichtefunktion für
\renewcommand{\labelenumi}{(\alph{enumi})}
\begin{enumerate}
	\item 
	$ a = \frac{1}{2} $.
	\item
	$ a = \frac{3}{2} $.
	\item
	$ a = \frac{5}{2} $.
	\item
	Für kein $ a \in \mathbb{R} $.
\end{enumerate}
\ \\
\textbf{Lösung:}
\begin{mdframed}
\underline{\textbf{Vorgehensweise:}}
\renewcommand{\labelenumi}{\theenumi.}
\begin{enumerate}
\item Gebe das Kriterium für eine Dichtefunktion an.
\item Wende das Kriterium an, um das passende $ a $ zu finden.
\end{enumerate}
\end{mdframed}

\underline{1. Gebe das Kriterium für eine Dichtefunktion an}\\
Die Funktion $ f $ ist eine Dichtefunktion, falls
\begin{align*}
\int \limits_{- \infty}^\infty f(x) \ dx = 1
\end{align*}
gilt.\\
\\
\underline{2. Wende das Kriterium an, um das passende $ a $ zu finden}\\
Demnach erhalten wir
\begin{align*}
\int \limits_{- \infty}^\infty f(x) \ dx
=
\int \limits_{0}^1 ax^2 + \frac{1}{2} \ dx 
=
\frac{a}{3}x^3 + \frac{1}{2}x \bigg|_0^1
&=
\frac{a}{3} + \frac{1}{2} - \left( \frac{a}{3}0^3 + \frac{1}{2}0 \right) =
\frac{a}{3} + \frac{1}{2}\\
 = 1
\ \Leftrightarrow \
\frac{a}{3} &= \frac{1}{2}
\ \Leftrightarrow \
a = \frac{3}{2}
\end{align*}
\ \\
Also ist Antwort (b) korrekt.

\newpage

\subsection*{\frage{8}{4}}
Gegeben sei die Funktion
\begin{align*}
f(x) = 
\begin{cases}
a x + \frac{1}{16} &\ \textrm{für } 0 \leq x \leq 8\\
\quad 0 &\qquad \textrm{sonst}
\end{cases}.
\end{align*}
Der Parameter $ a $ wird so gewählt, dass $ f $ die Dichtefunktion einer stetigen Zufallsvariablen $ X $ ist. Der Erwartungswert $ \mathbb{E}[X] $ von $ X $ ist:
\renewcommand{\labelenumi}{(\alph{enumi})}
\begin{enumerate}
\item 
$ \mathbb{E}[X] = \frac{25}{16} $.
\item
$ \mathbb{E}[X] = \frac{14}{3} $.

\item
$ \mathbb{E}[X] = \frac{-5}{3} $.
\item
$ \mathbb{E}[X] = \frac{7}{3} a $.
\end{enumerate}
\ \\
\textbf{Lösung:}
\begin{mdframed}
\underline{\textbf{Vorgehensweise:}}
\renewcommand{\labelenumi}{\theenumi.}
\begin{enumerate}
\item Gebe das Kriterium einer Dichtefunktion an.
\item Bestimme den Parameter $ a $ um den Erwartungswert zu berechnen.
\end{enumerate}
\end{mdframed}

\underline{1. Gebe das Kriterium einer Dichtefunktion an}\\
Damit $ f $ eine Dichtefunktion ist, muss
\begin{align*}
\int \limits_{- \infty}^\infty f(x) \ dx
= 
1
\end{align*}
erfüllt sein.\\
\\

\underline{2. Bestimme den Parameter $ a $ um den Erwartungswert zu berechnen}\\
Da $ f(x)  \neq 0$ für $ 0 \leq x \leq 8 $ ist, erhalten wir:
\begin{align*}
\int \limits_{- \infty}^\infty f(x) \ dx
= 
\int \limits_0^8 ax + \frac{1}{16} \ dx
&=
\frac{a}{2}x^2 + \frac{1}{16}x \bigg|_0^8
=
\frac{a}{2} \cdot 64 + \frac{1}{2}
- \left( \frac{a}{2}0^2 + \frac{1}{16}0 \right)\\
 = 1 
\ \Leftrightarrow \
32 a &= \frac{1}{2}
\ \Leftrightarrow \ 
a = \frac{1}{64}
\end{align*}
Da $ f $ die Dichtefunktion der Zufallsvariablen $ X $ ist gilt:
\begin{align*}
\mathbb{E}[X]
&=
\int \limits_{- \infty}^\infty x f(x) \ dx
=
\int \limits_{0}^8
x \left(\frac{1}{64}x  + \frac{1}{16}\right) \ dx
= 
\int \limits_{0}^8 
\frac{1}{64}x^2 + \frac{1}{16} x \ dx
= 
\frac{1}{64 \cdot 3}x^3 + \frac{1}{32} x^2 \bigg|_0^8\\
&= 
\frac{8^3}{64 \cdot 3}  + \frac{1}{32} \cdot 64 - \left( \frac{0^3}{64 \cdot 3}  + \frac{1}{32} \cdot 0^2 \right)
=
\frac{8}{ 3}  + 2
=
\frac{ 8 }{3}  + \frac{6}{3}
=
\frac{ 14}{3}. 
\end{align*}
\ \\
Also ist die Antwort (b) korrekt.

\newpage
\subsection*{\frage{9}{2}}
Die Vektoren $ \textbf{a} $ und $ \textbf{b} $ seien orthogonal und $ \textbf{a}  $ habe die Länge $ \| \textbf{a} \| = 3 $.
Dann gilt:
\renewcommand{\labelenumi}{(\alph{enumi})}
\begin{enumerate}
	\item 
	$ \textbf{a} \cdot ( \textbf{a} + \textbf{b}) = 0 $.
	\item
	$ \textbf{a} \cdot ( \textbf{a} + \textbf{b}) = 3 $.
	
	\item
	$ \textbf{a} \cdot ( \textbf{a} + \textbf{b}) = 9 $.
	\item
	$ \textbf{a} \cdot ( \textbf{a} + \textbf{b}) $ kann ohne Wissen über die Komponenten von $ \textbf{a} $ und $ \textbf{b} $ nicht bestimmt werden.
\end{enumerate}
\ \\
\textbf{Lösung:}
\begin{mdframed}
	\underline{\textbf{Vorgehensweise:}}
	\renewcommand{\labelenumi}{\theenumi.}
	\begin{enumerate}
		\item Gebe den Zusammenhang zwischen Skalarprodukt und Länge eines Vektors an.
		\item Verwende die Orthogonalität.
	\end{enumerate}
\end{mdframed}

\underline{1. Gebe den Zusammenhang zwischen Skalarprodukt und Länge eines Vektors an}\\
Für die Länge des Vektors $ \textbf{a} = \begin{pmatrix} a_1, & a_2, & ... & a_n  
\end{pmatrix}$ gilt:
\begin{align*}
\| a \| = \sqrt{a_1^2 + a_2^2+...+a_n^2}
= \sqrt{\textbf{a} \cdot \textbf{a}} = 3
\ \Rightarrow \
\textbf{a} \cdot \textbf{a} = 3^2 = 9.
\end{align*}
Hierbei sind $ a_1 $ bis $ a_n $ die einzelnen Komponenten des Vektors $ \textbf{a} $.\\
\\
\underline{2. Verwende die Orthogonalität}\\
Da $ \textbf{a} $ und $ \textbf{b} $ orthogonal sind gilt $ \textbf{a}\cdot \textbf{b} = 0 $ und wir erhalten:
\begin{align*}
\textbf{a} \cdot ( \textbf{a} + \textbf{b})
= \textbf{a} \cdot \textbf{a} +\textbf{a} \cdot \textbf{b}
= 9 + 0 = 9.
\end{align*}
\ \\
Damit ist Antwort (c) korrekt.
\newpage
\subsection*{\frage{10}{2}}
Gegeben seien die Vektoren:
\begin{align*}
\textbf{a}
=
\begin{pmatrix}
1 \\ 1 \\ -1
\end{pmatrix},
\textbf{b}
=
\begin{pmatrix}
0 \\ 1 \\ 3
\end{pmatrix},
\textbf{c}
=
\begin{pmatrix}
-1 \\ 0 \\ 2
\end{pmatrix},
\textbf{d}
=
\begin{pmatrix}
1 \\ 4 \\ 8
\end{pmatrix},
\textbf{e}
=
\begin{pmatrix}
1 \\ 1 \\ 1
\end{pmatrix}
\end{align*}
Welches der folgenden Systeme ist eine Basis des $ \mathbb{R}^3 $?
\renewcommand{\labelenumi}{(\alph{enumi})}
\begin{enumerate}
	\item 
	$ \lbrace \textbf{a}, \textbf{b} , \textbf{c} \rbrace  $.
	\item
	$ \lbrace \textbf{a}, \textbf{b} , \textbf{d} \rbrace  $.
	
	\item
	$ \lbrace \textbf{b}, \textbf{c} , \textbf{e} \rbrace  $.
	\item
	$ \lbrace \textbf{a}, \textbf{b} , \textbf{c},\textbf{d}, \textbf{e} \rbrace  $.
\end{enumerate}
\ \\
\textbf{Lösung:}
\begin{mdframed}
	\underline{\textbf{Vorgehensweise:}}
	\renewcommand{\labelenumi}{\theenumi.}
	\begin{enumerate}
		\item Schließe falsche Antworten durch Linearkombinationen bzw. die Determinante aus.
	\end{enumerate}
\end{mdframed}

\underline{1. Schließe falsche Antworten durch Linearkombinationen bzw. die Determinante aus}\\
Unter einer Basis verstehen wir ein System von linear unabhängigen Vektoren.
Wir können die Antwort (d) direkt verwerfen, da es keine fünf linear unabhängige Vektoren in einem dreidimensionalen Raum geben kann.\\
\\
Wir werden verwenden, dass drei Vektoren in $ \mathbb{R}^3 $ genau dann eine Basis bilden, wenn diese linear unabhängig sind.
Das heißt die Matrix bestehend aus den Vektoren ist regulär. Dies ist äquivalent dazu, dass die Determinate ungleich null ist.
Für die falschen Antworten werden wir jedoch die Linearkombinationen genauer betrachten.\\
\\
Wegen
\begin{align*}
1 \cdot  \textbf{a} + 3 \cdot \textbf{b}
=
1 \cdot \begin{pmatrix}
1 \\ 1 \\ -1
\end{pmatrix}
+ 3 \cdot
\begin{pmatrix}
0 \\ 1 \\ 3
\end{pmatrix}
=
\begin{pmatrix}
1 \\ 4 \\ 8
\end{pmatrix} =\textbf{d}
\end{align*}
ist das System $ \lbrace \textbf{a}, \textbf{b} , \textbf{d} \rbrace  $ linear abhängig und Antwort (b) ist falsch.
Dies lässt sich auch durch
\begin{align*}
\det(\textbf{a}, \textbf{b}, \textbf{d} ) = 0
\end{align*}
feststellen.\\
\\
Durch identisches Vorgehen gilt:
\begin{align*}
1 \cdot \textbf{b} + (-1) \cdot \textbf{c} = 
1 \cdot \begin{pmatrix}
0 \\ 1 \\3
\end{pmatrix}
+ (-1) \cdot 
\begin{pmatrix}
-1 \\ 0 \\ 2
\end{pmatrix}
=
\begin{pmatrix}
1 \\ 1 \\ 1
\end{pmatrix}
= \textbf{e}.
\end{align*}
Also ist die Antwort (c) falsch.
Dies lässt sich auch durch
\begin{align*}
\det(\textbf{b}, \textbf{c}, \textbf{e} ) = 0
\end{align*}
feststellen.\\
\\
Übrig bleibt die korrekte Antwort (a).
Dies wollen wir noch verifizieren. Hierfür berechnen wir die Determinante, indem wir nach der ersten Zeile entwickeln.
Dies ist eine geschickte Wahl, da sich eine Null in der ersten Zeile befindet. Es gilt:
\begin{align*}
\det (\textbf{a},\textbf{b}, \textbf{c})
=
\det \begin{pmatrix}
1 & 0 & -1\\
1 & 1 & 0 \\
-1 & 3 & 2 
\end{pmatrix}
=
1 \cdot
\begin{pmatrix}
1 & 0 \\
3 & 2
\end{pmatrix}
+ (-1) \cdot 
\begin{pmatrix}
1 & 1 \\
-1 & 3
\end{pmatrix}
=
2 - 4 = -2 \neq 0.
\end{align*}
Damit ist das System $ \lbrace \textbf{a}, \textbf{b} , \textbf{c} \rbrace  $ linear unabhängig und somit eine Basis des $ \mathbb{R}^3 $.\\
\\
Also ist die Antwort (a) korrekt.
\newpage

\subsection*{\frage{11}{2}}
$ A $ sei eine $ 7 \times 5 $ Matrix. Das System von linearen Gleichungen $ A \textbf{x} = \textbf{b} $ habe unendliche viele Lösungen und der Lösungsraum habe die Dimension $ 3 $. Dann gilt:
\renewcommand{\labelenumi}{(\alph{enumi})}
\begin{enumerate}
	\item 
	$ \mathrm{rg}(A)< \mathrm{rg}(A;\textbf{b} ) =3 $.
	\item
	$ \mathrm{rg}(A)= \mathrm{rg}(A;\textbf{b} ) =2 $.
	
	\item
	$ \mathrm{rg}(A)< \mathrm{rg}(A;\textbf{b} ) =2 $.
	\item
	$ \mathrm{rg}(A)= \mathrm{rg}(A;\textbf{b} ) =3 $.
\end{enumerate}
\ \\
\textbf{Lösung:}
\begin{mdframed}
	\underline{\textbf{Vorgehensweise:}}
	\renewcommand{\labelenumi}{\theenumi.}
	\begin{enumerate}
		\item Wende die Dimensionsformel für den Lösungsraum an.
	\end{enumerate}
\end{mdframed}

\underline{1. Wende die Dimensionsformel für den Lösungsraum an}\\
Die Dimensionsformel für den Lösungsraum $ L $ ist durch 
\begin{align*}
\dim \ L = n - \mathrm{rg}(A )
\end{align*}
gegeben. 
Hierbei ist $ n $ die Anzahl der Variablen.
Da $ A $ eine $ 7 \times 5 $ Matrix ist, besitzt das lineare Gleichungssystem
$ A \textbf{x} = \textbf{b} $ fünf Variablen.
Das heißt $ n = 5 $.
Insgesamt erhalten wir
\begin{align*}
3 = \dim \ L = 5 - \mathrm{rg}(A)
\ \Leftrightarrow \
\mathrm{rg}(A) = 5 - \dim \ L = 5 -3 = 2.
\end{align*}
Da das lineare Gleichungssystem $ A \textbf{x} = \textbf{b} $ lösbar ist, erhalten wir
\begin{align*}
2=\mathrm{rg}(A)= \mathrm{rg}(A;\textbf{b} ).
\end{align*}
\ \\
Damit ist Antwort (b) korrekt.
\newpage
\subsection*{\frage{12}{2}}
$ A $ und $ B $ seien reguläre Matrizen. Ausserdem sei $ A $ symmetrisch.
Der Ausdruck
\begin{align*}
B^\top (AB)^\top (B^{-1} A^{-1})^\top B (AB)^{-1}
\end{align*}
entspricht:
\begin{enumerate}
	\item 
	$ (A^\top B^\top)^{-1} $.
	\item
	$ (B^{-1} A)^\top $.
	
	\item
	$ (A^{-1} B )^\top $
	\item
	Keinem der obigen Terme.
\end{enumerate}
\ \\
\textbf{Lösung:}
\begin{mdframed}
	\underline{\textbf{Vorgehensweise:}}
	\renewcommand{\labelenumi}{\theenumi.}
	\begin{enumerate}
		\item Wende die Rechenregeln für Matrizen an.
	\end{enumerate}
\end{mdframed}

\underline{1. Wende die Rechenregeln für Matrizen an}\\
Wir werden die folgenden Rechenregeln verwenden:
\begin{align*}
(A B)^\top &= B^\top A^\top\\
(A B)^{-1} &= B^{-1} A^{-1}.
\end{align*}
Mit diesen Regeln erhalten wir:
\begin{align*}
B^\top (AB)^\top (B^{-1} A^{-1})^\top B (AB)^{-1}
&=
B^\top B^\top A^\top (A^{-1})^\top (B^{-1})^\top B B^{-1} A^{-1}
=
B^\top B^\top I (B^{-1})^\top I A^{-1}\\
&=
B^\top B^\top  (B^\top)^{-1}  A^{-1}
=
B^\top  A^{-1}
\overset{A \ \textrm{symmetrisch}}{=}
B^\top (A^{-1})^\top\\
&=
(A^{-1} B)^\top.
\end{align*}
\ \\
Damit ist die Antwort (c) korrekt.