\section*{Aufgabe 3 (32 Punkte)}
\vspace{0.4cm}
\subsection*{\frage{1}{4}}
$ X $ sei eine stetige Zufallsvariable mit Dichtefunktion
\begin{align*}
f(x) = 
\begin{cases}
\frac{2}{3} x + \sqrt{x} &\ \textrm{für } 0 \leq x \leq 1\\
\quad 0 &\qquad \textrm{sonst}
\end{cases}
\end{align*}
und Erwartungswert
\begin{align*}
\mathbb{E}[X] = \frac{28}{45}.
\end{align*}
Dann hat das Integral $ \int_0^1 (x+1) f(x) \ dx $ den Wert:
\renewcommand{\labelenumi}{(\alph{enumi})}
\begin{enumerate}
	\item 
	$\frac{28}{45}$.
	\item
	$\frac{73}{45}$.
	\item
	$\frac{12}{45}$.
	\item
	$\frac{52}{45}$.
\end{enumerate}
\ \\
\textbf{Lösung:}
\begin{mdframed}
\underline{\textbf{Vorgehensweise:}}
\renewcommand{\labelenumi}{\theenumi.}
\begin{enumerate}
\item Bestimme mit der Definition der Dichtefunktion und dem Erwartungswert die korrekte Antwort 
\end{enumerate}
\end{mdframed}

\underline{1. Bestimme mit der Definition der Dichtefunktion und dem Erwartungswert die korrekte Antwort}\\
Da $ f $ eine Dichtefunktion ist, erfüllt diese 
\begin{align*}
\int \limits_{- \infty}^\infty f(x) \ dx
=
\int \limits_{0}^1 f(x) \ dx = 1.
\end{align*}
Das erste Gleichzeichen erhalten wir , da $ f(x) $ für $ x < 0  $ und $ x > 1 $ gleich null ist.
Da die Zufallsvariable $ X $ eine Dichtefunktion $ f $ besitzt, ist der Erwartungswert gerade das Integral über $ x f(x) $. Das heißt für den gegebenen Erwartungswert:
\begin{align*}
\mathbb{E}[X] = 
\int \limits_{- \infty}^\infty x f(x) \ dx = 
\int \limits_{0}^1 x f(x) \ dx
= \frac{28}{45}.
\end{align*}
Nun betrachten wir das gewünschte Integral:
\begin{align*}
\int \limits_0^1 (x+1) f(x) \ dx
=
\underbrace{\int \limits_0^1 x  f(x) \ dx}_{\mathbb{E}[X] =}
 + 
\underbrace{\int \limits_0^1  f(x) \ dx}_{= 1} 
=
\frac{28}{45} + 1 = 
\frac{28+ 45}{45} = \frac{73}{45}.
 \end{align*}
\ \\
Damit ist Antwort (b) korrekt.
\newpage

\subsection*{\frage{2}{4}}
Für welchen Wert von $ t \in \mathbb{R} $ sind die Vektoren $ \textbf{u} = \begin{pmatrix}
t \\ t-1 \\ -6
\end{pmatrix} $ und $ \textbf{v} = \begin{pmatrix}
t+5 \\ -1 \\ 1
\end{pmatrix} $ orthogonal und die Länge von $ \textbf{u} $ gleich $ \sqrt{37} $?
\renewcommand{\labelenumi}{(\alph{enumi})}
\begin{enumerate}
	\item 
	$t = -5$.
	\item
	$t = 0$.
	\item
	$ t= 1 $.
	\item
	Für kein $ t \in \mathbb{R} $.	
\end{enumerate}
\ \\
\textbf{Lösung:}
\begin{mdframed}
\underline{\textbf{Vorgehensweise:}}
\renewcommand{\labelenumi}{\theenumi.}
\begin{enumerate}
\item Bestimme $ t $ so, dass die Vektoren $ \textbf{u} $ und $ \textbf{v} $ orthogonal sind.
\item Bestimme die Länge von  $ \textbf{u} $ und finde die korrekte Antwort.
\end{enumerate}
\end{mdframed}

\underline{1. Bestimme $ t $ so, dass die Vektoren $ \textbf{u} $ und $ \textbf{v} $ orthogonal sind}\\
Die Vektoren $ \textbf{u} $ und $ \textbf{v} $ sind orthogonal, wenn das Skalarprodukt gleich Null ist. Es gilt:
\begin{align*}
\textbf{u} \cdot \textbf{v}
=
\begin{pmatrix}
t \\ t-1 \\ -6
\end{pmatrix}
\cdot
\begin{pmatrix}
t+5 \\ -1 \\ 1
\end{pmatrix} 
&=
t\cdot(t+5) + (t-1)\cdot(-1)- 6 \cdot 1
=
t^2 + 5 t + -t +1 +(-6)\\
&=
t^2 +4t -5
=
0.
\end{align*}
Mit der Mitternachts-Formel erhalten wir:
\begin{align*}
t_{\nicefrac{1}{2}} = 
\frac{-4 \pm \sqrt{4^2 + 4 \cdot 1 \cdot  5}}{2}
=
\frac{-4 \pm \sqrt{36}}{2}
=
\frac{-4 \pm 6}{2}
\ \Rightarrow \
t_1 = 1, \ t_2 = -5.
\end{align*}
Damit können wir die Antworten (a) und (c) in Betracht ziehen.\\
\\
\underline{2. Bestimme die Länge von  $ \textbf{u} $ und finde die korrekte Antwort}\\
Die Länge des Vektors $ \textbf{u} $ ist gegeben durch
\begin{align*}
\| \textbf{u} \|
= \sqrt{t^2 + (t-1)^2 + (-6)^2}
= \sqrt{t^2 + (t-1)^2 + 36}.
\end{align*}
Wir setzen jetzt $ t_1 = 1$ bzw. $t_2 = -5 $.
Durch 
\begin{align*}
\| \textbf{u} \| = 
\sqrt{1^2 + (1-1)^2 + (-6)^2}
=
\sqrt{1 + 0 + 36 } 
=
\sqrt{37}
\end{align*}
erkennen wir, dass die Antwort (c) korrekt ist.\\
\\
Damit müssen wir $ t_2  $ nicht mehr überprüfen, da dies auch eine Antwortmöglichkeit ist.\\
\\
Damit ist die Antwort (c) korrekt.

\newpage
\subsection*{\frage{3}{3}}
Die $3 \times 4$ Matrix
\begin{align*}
A
=
\begin{pmatrix}
1 & 2  & 1 & -1  \\
9 & 5  & 2 & 2 \\
7 & 1  & 0 &  4  
\end{pmatrix}
\end{align*}
\renewcommand{\labelenumi}{(\alph{enumi})}
\begin{enumerate}
	\item 
	hat Rang $1$.
	\item
	hat Rang $2$.
	\item
	hat Rang $3$.
	\item
	hat Rang $4$.
\end{enumerate}
\ \\
\textbf{Lösung:}
\begin{mdframed}
\underline{\textbf{Vorgehensweise:}}
\renewcommand{\labelenumi}{\theenumi.}
\begin{enumerate}
\item Wende das Gauß-Verfahren an.
\end{enumerate}
\end{mdframed}
%\allowdisplaybreaks
\underline{1. Wende das Gauß-Verfahren an}\\
Durch das Gauß-Verfahren erhalten wir:
\begin{align*}
\begin{gmatrix}[p]
1 & 2  & 1 & -1  \\
9 & 5  & 2 & 2 \\
7 & 1  & 0 &  4  
\rowops
\add[(-9)]{0}{1}
\add[(-7)]{0}{2}
\end{gmatrix}
\leadsto
\begin{gmatrix}[p]
1 & 2  & 1 & -1  \\
0 & -13  & -7 & 11 \\
0 & -13  & -7 &  11  
\rowops
\add[(-1)]{1}{2}
\end{gmatrix}
\begin{gmatrix}[p]
1 & 2  & 1 & -1  \\
0 & -13  & -7 & 11 \\
0 & 0  & 0 &  0  
\end{gmatrix}.
\end{align*}
Wir können keine weitere Nullzeile durch elementare Zeilenoperationen erhalten.
Demzufolge ist der Rang der Matrix $ A $ gleich $ 2 $.\\
\\
Damit ist Antwort (b) korrekt.\\
\\
\textit{Bemerkung:}\\
Der Rang entspricht der Anzahl der Zeilen, welche keine Nullzeilen sind.
\newpage
\subsection*{\frage{4}{5}}
Gegeben sei die $ 3 \times 3 $ Matrix
\begin{align*}
A =
\begin{pmatrix}
1 & 0 & 1 \\
0 & -1 & 0 \\
1 & 0 &0
\end{pmatrix}.
\end{align*}
Es gilt:
\renewcommand{\labelenumi}{(\alph{enumi})}
\begin{enumerate}
	\item 
	$A^{-1}
	= 
	\begin{pmatrix}
	1 & 1 & -1  \\
	0 & -1 & 0 \\
	0 & 0 & 1
	\end{pmatrix}$.
	\item 
	$A^{-1}
	= 
	\begin{pmatrix}
	0 & 0 & 1  \\
	0 & -1 & 0 \\
	1 & 0 & -1
	\end{pmatrix}$.
	\item
	$A^{-1}
	= 
	\begin{pmatrix}
	0 & 0 & -1  \\
	1 & 1 & 1 \\
	0 & 1 & 0
	\end{pmatrix}$.
	\item
	$ A $ ist singulär.
\end{enumerate}
\ \\
\textbf{Lösung:}
\begin{mdframed}
\underline{\textbf{Vorgehensweise:}}
\renewcommand{\labelenumi}{\theenumi.}
\begin{enumerate}
\item 
Bestimme die Determinante der Matrix.
\item
Finde durch Ausschließen die korrekte Antwort.
\end{enumerate}
\end{mdframed}

\underline{1. Bestimme die Determinante der Matrix}\\
Durch Entwicklung nach der dritten Spalte erhalten wir:
\begin{align*}
\det(A) = 
\det
\begin{pmatrix}
1 & 0 & 1 \\
0 & -1 & 0 \\
1 & 0 &0
\end{pmatrix} 
= 1 \cdot \det \begin{pmatrix}
0 & -1\\
1 & 0
\end{pmatrix}
=
0 \cdot 0 - 1 \cdot (-1)
 = 1 .
\end{align*}
Damit ist die Matrix $ A $ regulär.\\
\\
\underline{2. Finde durch Ausschließen die korrekte Antwort}\\
Wir machen uns zunutze, dass
\begin{align*}
A \cdot A^{-1} = A^{-1} \cdot A = I
\end{align*}
gelten muss. Hierbei ist $ I $ die Einheitsmatrix.
Zuerst betrachten wir $ A^{-1} $ aus (a) und überprüfen, ob $ A^{-1} \cdot A = I $ gilt.
Durch Multiplikation der ersten Zeile von $ A^{-1} $ mit der erste Spalte von $ A $ erhalten wir:
\begin{align*}
\begin{pmatrix}
1 & 1 & -1
\end{pmatrix} \cdot 
\begin{pmatrix}
1 \\ 0 \\ 1
\end{pmatrix}
=
1 \cdot 1 + 1 \cdot 0 + 1 \cdot (-1)
=
0
\neq 1. 
\end{align*}
Damit ist Antwort (a) falsch. Nun untersuchen wir $ A^{-1} $ aus der Möglichkeit (c) und überprüfen auch, ob $ A^{-1} \cdot A = I $ gilt.
Hier multiplizieren wir die zweite Zeile von $ A^{-1} $ mit der zweiten Spalten von $ A $ und erhalten:
\begin{align*}
\begin{pmatrix}
1& 1 & 1
\end{pmatrix} \cdot 
\begin{pmatrix}
0 \\ -1 \\ 0
\end{pmatrix}
=
1 \cdot 0 + 1 \cdot (-1) + 1 \cdot 0
=
-1
\neq 1. 
\end{align*}
Damit ist Antwort (c) falsch.\\
\\
Also ist Antwort (b) korrekt.\\
\\
\textit{Bemerkung:}\\
Der einfachste und aufwendigste Weg wäre die inverse Matrix von $ A $ durch 
\begin{align*}
(A | I ) \leadsto ... \leadsto (I | A^{-1})
\end{align*} 
mit elementaren Zeilenumformungen zu bestimmen.
\newpage


\subsection*{\frage{5}{4}}
Sei
\begin{align*}
A
=
\begin{pmatrix}
1 & 2 & 1\\
0 & 1 & 0\\
1 & -1 & 1
\end{pmatrix}.
\end{align*}
Die Matrix $ A^2 $ hat den Eigenwert:
\renewcommand{\labelenumi}{(\alph{enumi})}
\begin{enumerate}
	\item 
	$ 2 $.
	\item
	$ 4 $.
	\item
	$ 6 $.
	\item
	$ 8 $.
\end{enumerate}
\ \\
\textbf{Lösung:}
\begin{mdframed}
\underline{\textbf{Vorgehensweise:}}
\renewcommand{\labelenumi}{\theenumi.}
\begin{enumerate}
\item Überlege dir, was du über die Eigenwerte von $ A^2 $ aussagen kannst.
\item Bestimme einen/die Eigenwert/e der Matrix $ A $.
\end{enumerate}
\end{mdframed}

\underline{1. Überlege dir, was du über die Eigenwerte von $ A^2 $ aussagen kannst}\\
Ein Vektor $ \textbf{v} \neq 0 $ heißt Eigenvektor zum Eigenwert $ \lambda $ der Matrix $ A $, falls
\begin{align*}
A\textbf{v} = \lambda \textbf{v} 
\end{align*}
gilt.
%Also $ A \textbf{v} $ ergibt ein Vielfaches von $ \textbf{v} $.
Hieraus folgt:
\begin{align*}
A^2 A \textbf{v} = A (A \textbf{v}) = A (\lambda  \textbf{v})
= \lambda A  \textbf{v} = 
\lambda^2 \textbf{v}.
\end{align*}
Demnach gilt: Falls $ \lambda $ ein Eigenwert von $ A $ ist, so ist $ \lambda^2 $ ein Eigenwert von $ A^2 $.\\
\\
\underline{2. Bestimme einen/die Eigenwert/e der Matrix $ A $}\\
Wir analysieren die erste und dritte Zeile der Matrix $ A $.
Beide enthalten eine $ 1 $ als ersten und dritten Eintrag.
Die zweite Zeile hingegen hat an diesen Einträgen eine $ 0 $. Damit könnte
\begin{align*}
\begin{pmatrix}
1 \\ 0 \\ 1
\end{pmatrix}
\end{align*}
ein Eigenvektor sein. Dies werden wir nun überprüfen:
\begin{align*}
A \cdot \begin{pmatrix}
1 \\ 0 \\ 1
\end{pmatrix}
=
\begin{pmatrix}
1 & 2 & 1\\
0 & 1 & 0\\
1 & -1 & 1
\end{pmatrix}
\cdot 
\begin{pmatrix}
1 \\ 0 \\ 1
\end{pmatrix}
=
\begin{pmatrix}
1 \cdot 1 + 2 \cdot 0 + 1 \cdot 1\\
1 \cdot 0  +  1 \cdot 0 +  1 \cdot 0\\
1 \cdot 1 + (-1) \cdot 0 + 1 \cdot 1
\end{pmatrix}
= 
\begin{pmatrix}
2 \\ 0 \\ 2
\end{pmatrix}
= 2 \cdot \begin{pmatrix}
1 \\ 0 \\ 1
\end{pmatrix}.
\end{align*}
Damit ist $ 2 $ ein Eigenwert der Matrix $ A $ und somit $ 2^2 = 4 $ einer von $ A^2 $.\\
\\
Damit ist die Antwort (b) korrekt.\\
\\
\textit{Bemerkung:}\\
Alternativ lassen sich auch die Nullstellen des charakteristischen Polynoms bestimmen. Hierfür erhalten wir mit Entwicklung nach der zweiten Zeile:
\begin{align*}
\det (A - \lambda I )
&= 
\det \begin{pmatrix}
1- \lambda & 2 & 1\\
0 & 1- \lambda & 0\\
1 & -1 & 1- \lambda
\end{pmatrix}
= (1- \lambda) \det \begin{pmatrix}
1- \lambda & 1\\
1 & 1 - \lambda
\end{pmatrix}\\
&= 
(1- \lambda) \cdot \left((1- \lambda)^2 - 1 \cdot 1\right)
=
(1- \lambda) \cdot ( 1 -2\lambda + \lambda^2 - 1)
=
\lambda \cdot (1 - \lambda) \cdot (\lambda - 2)
=0\\
\ \Leftrightarrow \
&\lambda_1 = 0, \lambda_2 = 1, \lambda_3 = 2.
\end{align*}
Wir haben nach der zweiten Zeile entwickelt, da dort zwei Nullen auftreten.
$ A $ besitzt die Eigenwerte $ 0 $, $ 1 $, $ 2 $ und $ A^2 $ somit $ 0 $, $ 1 $, $ 4 $.



\newpage

\subsection*{\frage{6}{4}}
Das Anfangswertproblem
\begin{align*}
&y_{k+1} -(1+a) y_k = 2 a, \ \textrm{wobei } a \neq -1, a \neq 0,\\
&y_0 = 2
\end{align*}
hat die Lösung
\renewcommand{\labelenumi}{(\alph{enumi})}
\begin{enumerate}
	\item 
	$ y_k = -4 ( 1+a)^k $.
	\item
	$ y_k = 2 ( 1+a)^k  -1$.
	\item
	$ y_k = 4 ( 1+a)^k -2 $.
	\item
	$ y_k = 8 ( 1+a)^k -3$.
\end{enumerate}
\ \\
\textbf{Lösung:}
\begin{mdframed}
\underline{\textbf{Vorgehensweise:}}
\renewcommand{\labelenumi}{\theenumi.}
\begin{enumerate}
\item  Bringe die Differenzengleichung in Normalform.
\end{enumerate}
\end{mdframed}

\underline{1. Bringe die Differenzengleichung in Normalform}\\
Die Normalform der Differenzengleichung erhalten wir durch
\begin{align*}
y_{k+1} -(1+a) y_k = 2a
\ \Leftrightarrow \
y_{k+1}
= (1+a ) y_k + 2a.
\end{align*}
Also ist $ A = (1+a) $ und $ B = 2a $.
Die allgemeine Lösung ist dann durch
\begin{align*}
y_k = A^k(y_0 - y^\ast) + y^\ast = (1+a)^k(y_0 - y^\ast) + y^\ast
\end{align*}
mit $ y^\ast = \frac{B}{1 - A} =  \frac{2a}{1-(1+a)} = -2 $ gegeben.
Eingesetzt liefert dies mit der Anfangsbedingung $ y_0 = 2 $:
\begin{align*}
y_k = 4 (1+a)^k -2.
\end{align*}
\ \\
Damit ist Antwort (c) korrekt.\\
\\
\textit{Bemerkung:}\\
Die Antworten (a), (b) und (d) erfüllen $ y_0 =2 $ nicht.\\
%Die Antwort (a) erfüllt wegen
%\begin{align*}
%y_{k+1} -(1+a) y_k = 2 (1+a)^{k+1} - (1+a) 2 (1+a)^k
%= 0 \neq a
%\end{align*}
%die Differenzengleichung nicht.




\newpage



\subsection*{\frage{7}{3}}
Die allgemeine Lösung der Differenzengleichung
\begin{align*}
3 (y_{k+1}   -y_k  )+ 5 = 2 y_{k+1} -y_k + 12 
\end{align*}
ist
\renewcommand{\labelenumi}{(\alph{enumi})}
\begin{enumerate}
	\item
	oszillierend und konvergent.
	\item
	oszillierend und divergent.	
	\item 
	monoton und konvergent.
	\item
	monoton und divergent.
\end{enumerate}
\ \\
\textbf{Lösung:}
\begin{mdframed}
\underline{\textbf{Vorgehensweise:}}
\renewcommand{\labelenumi}{\theenumi.}
\begin{enumerate}
\item Bestimme die Normalform.
\end{enumerate}
\end{mdframed}

\underline{1. Bestimme die Normalform}\\
Die Normalform einer Differenzengleichung ist durch
\begin{align*}
y_{k+1} = A y_{k} + B
\end{align*}
mit $ A, B \in \mathbb{R} $ gegeben.
Diese erhalten wir durch
\begin{align*}
3 (y_{k+1}   -y_k  )+ 5 = 2 y_{k+1} -y_k + 12 
\ \Leftrightarrow \
3 y_{k+1} - 3 y_k + 5  = 2 y_{k+1} - y_k +12 
\ \Leftrightarrow \
y_{k+1} = 2 y_k + 7.
\end{align*}
Also ist $ A = 2$ und $ B= 7 $. \\
\\
Eine Differenzengleichung in Normalenform
\begin{align*}
y_{k+1} = A y_{k} + B
\end{align*}
konvergiert für $ |A| < 1 $ und divergiert für $ |A| > 1 $.
Für $ A> 0 $ ist das Verhalten monoton und für $  A < 0 $ oszillierend.\\
\\ 
Wegen $ |A |> 1  $ und $ A > 0  $ ist die allgemeine Lösung monoton und divergent.\\
\\
Damit ist Antwort (d) korrekt.

\newpage

\subsection*{\frage{8}{5}}
Die allgemeine Lösung der linearen Differenzengleichung
\begin{align*}
2 (a + 2) y_{k+1} - 2 y_k + 2 (a^2 - 4) = 0, \quad k = 0,1,2,...,
\end{align*}
mit $ a \in \mathbb{R} \setminus \{-2,-1  \} $, konvergiert monoton gegen $ 0 $ genau dann, wenn
\renewcommand{\labelenumi}{(\alph{enumi})}
\begin{enumerate}
	\item 
	$ a > -2 $.
	\item
	$a > -1$.
	\item
	$ a = 2 $.
	\item
	$ a = 1 $.
\end{enumerate}
\ \\
\textbf{Lösung:}
\begin{mdframed}
\underline{\textbf{Vorgehensweise:}}
\renewcommand{\labelenumi}{\theenumi.}
\begin{enumerate}
\item Verwende die Konvergenz gegen $ 0 $, um die Antwort zu finden.
\end{enumerate}
\end{mdframed}

\underline{1. Verwende die Konvergenz gegen $ 0 $, um die Antwort zu finden}\\
Sei $ y_k $ eine Lösung von
\begin{align*}
2 (a + 2) y_{k+1} - 2 y_k + 2 (a^2 - 4) = 0,
\end{align*}
wobei $ y_k $ monoton gegen $ 0 $ konvergiert.
Das heißt insbesondere $ y_k \to 0  $ und $ y_{k+1} \to 0 $ für $ k \to \infty $.
Hiermit folgt:
\begin{align*}
2 (a + 2) y_{k+1} &- 2 y_k + 2 (a^2 - 4) = 0\\
&\quad\\
&\downarrow \quad k \to \infty\\
&\quad\\
0 &- 0  + 2(a^2 -4) = 0.
\end{align*}
Wenn $ y_k  $ gegen $ 0 $ konvergiert muss $ a^2 - 4 = 0 $ erfüllt sein.
Dies gilt für $ a = \pm 2 $. In der Aufgabenstellung ist $ a = -2 $ ausgeschlossen. Deshalb ist $ a = 2 $ korrekt.\\
\\
Also ist die Antwort (c) korrekt.\\
\\
\textit{Bemerkung:}\\
Die Lösung lässt sich auch über die Normalform bestimmen.
Diese erhalten wir durch
\begin{align*}
2 (a + 2) y_{k+1} - 2 y_k + 2 (a^2 - 4) = 0
\ \Leftrightarrow \
y_{k+1} &= \frac{2 y_k}{2(a+2)} - \frac{2 (a^2 + 4)}{2(a+2)}\\
&=
\frac{1}{(a+2)} y_k - \frac{(a +2)(a-2)}{(a+2)}\\
&=
\frac{1}{a+2} y_k - (a-2)
\end{align*}
mit $ A = \frac{1}{a+2} $ und $ B = -(a-2) $.
Die Lösung kann nur gegen $ 0 $ konvergieren, wenn $ B = 0 $ ist.
Dies ist für $ a= 2 $ der Fall und wird auch an der Lösungsformel sichtbar:
\begin{align*}
y_k = A^k (y_0 - y^\ast) + y^\ast
= \left(\frac{1}{a+2}\right)^k \left(y_0 - \frac{-(a-2)}{\frac{1}{a+2}}\right) + \frac{-(a-2)}{\frac{1}{a+2}}
\ \overset{k \to \infty, \ | A | < 1 }{\rightarrow} \ \frac{-(a-2)}{\frac{1}{a+2}}.
\end{align*}
