\section*{Aufgabe 3 (30 Punkte)}
\vspace{0.4cm}
\subsection*{\frage{1}{4}}
Gegeben ist die Funktion $ f(x,y) = e^{-\frac{1}{3} x^3 + x +y^2 } $.
Welche der folgenden Aussagen ist wahr?
\renewcommand{\labelenumi}{(\alph{enumi})}
\begin{enumerate}
	\item 
	$ P = (1,0) $ ist ein lokales Minimum von $ f $.
	\item
	$ P = (1,0) $ ist ein lokales Maximum von $ f $.
	\item
	$ P = (0,0) $ ist ein Sattelpunkt von $ f $.
	\item 
	$ P = (-1,0) $ ist ein lokales Minimum von $ f $.
	\item
	$ P = (-1,0) $ ist ein lokales Maximum von $ f $.
\end{enumerate}
\ \\
\textbf{Lösung:}
\begin{mdframed}
\underline{\textbf{Vorgehensweise:}}
\renewcommand{\labelenumi}{\theenumi.}
\begin{enumerate}
\item Verwende die strenge Monotonie der Exponentialfunktion.
\end{enumerate}
\end{mdframed}

\underline{1. Verwende die strenge Monotonie der Exponentialfunktion}\\
Die Funktion $ f $ ist von der Form $ f(x,y) = e^{g(x,y)} $.
Die lokalen Extrema von $ g $ entsprechen den von $ f $, da die Exponentialfunktion streng monoton wachsend ist.
In unseren Fall ist $ g $ durch
\begin{align*}
	g(x,y) 
	=
	-\frac{1}{3} x^3 + x +y^2  
\end{align*}
gegeben. Die partiellen Ableitungen erster Ordnung sind:
\begin{align*}
	g_x(x,y)
	&=
	-x^2 + 1 = 1 - x^2\\
	g_y(x,y)
	&=
	2 y.
\end{align*}
Für $ \mathrm{grad}\ g(x,y) = \textbf{0} $ muss
\begin{align*}
	1 - x^2 = 0 \ &\Leftrightarrow \ x = \pm 1\\
	2y = 0 \ &\Leftrightarrow \ y = 0
\end{align*}
gelten. Damit sind $ (1,0) $ und $ (-1,0) $ Kandidaten für lokale Extrema.
Die partiellen Ableitungen zweiter Ordnung sind:
\begin{align*}
	g_{xx }(x,y) &= -2x\\
	g_{yy}(x,y) &= 2\\
	g_{xy}(x,y) &= 0.
\end{align*}
Wegen 
\begin{align*}
	g_{xx }(1,0) g_{yy }(1,0) - (g_{xy }(1,0))^2
	=
	-2 \cdot 2 - 0 < 0
\end{align*}
liegt an $ (1,0) $ ein Sattelpunkt vor.
Wegen 
\begin{align*}
	g_{xx}(-1,0) &> 0 \\
	g_{yy}(-1,0) &> 0\\
	g_{xx}(-1,0) g_{yy}(-1,0) &> 0
\end{align*}
liegt an $ (-1,0) $ ein Minimum vor.\\
\\
Damit ist die Antwort (d) korrekt.
%\\
%\\
%\textit{Alternativer Lösungsweg:}\\


 
\newpage

\subsection*{\frage{2}{3}}
Gegeben ist die Funktion $ f $ mit 
\begin{align*}
	f^\prime(x) = \frac{2 e^{2x}}{e^{2x} + 1} \quad \textrm{mit } \quad f(0) = \ln(2)
\end{align*}
Es folgt:
\renewcommand{\labelenumi}{(\alph{enumi})}
\begin{enumerate}
	\item 
	$f(x) = \ln(e^{2x} +1) $.
	\item
	$f(x) = \ln(e^{2x} +1) + \ln(2) $.
	\item
	$f(x) = \ln(e^{x} +e^{-x}) $.
	\item
	$f(x) = \ln(e^{x} +e^{-x}) + \ln(2) $.	
	\item
	$f(x) = 2 \ln(e^{2x} +1) - \ln(2) $.
\end{enumerate}
\ \\
\textbf{Lösung:}
\begin{mdframed}
\underline{\textbf{Vorgehensweise:}}
\renewcommand{\labelenumi}{\theenumi.}
\begin{enumerate}
\item Bestimme die richtige Lösung mit dem Eliminationsverfahren.
\end{enumerate}
\end{mdframed}

\underline{1. Bestimme die richtige Lösung mit dem Eliminationsverfahren}\\
Es gilt 
\begin{align*}
	\ln(e^{2\cdot 0} +1) &= \ln ( 1+ 1 ) = \ln (2)\\
	\ln(e^{0} + e^{-0})&= \ln(2).
\end{align*}
Damit können wir die Antwortmöglichkeiten (b) und (d) auschließen.
Wir betrachten nun folgende Ableitungen:
\begin{align*}
	\frac{\mathrm{d}}{\mathrm{dx}}
	\left(\ln(e^{2x} +1)\right)
	&=
	\frac{1}{e^{2x} +1} \cdot (e^{2x} +1)^\prime
	=
	\frac{1}{e^{2x} +1} \cdot 2 e^{2x}
	=
	\frac{ 2 e^{2x}}{e^{2x} +1}\\
	\frac{\mathrm{d}}{\mathrm{dx}}
	\left(\ln(e^{x} +e^{-x})\right)
	&=
	\frac{1}{e^{x} +e^{-x}} \cdot (e^{x} +e^{-x})^\prime
	=
	\frac{1}{e^{x} +e^{-x}} \cdot (e^x - e^{-x})
	=
	\frac{e^x - e^{-x}}{e^{x} +e^{-x}}.
\end{align*}
Also sind die Antwortmöglichkeiten (c) und (e) falsch.\\
\\
Damit ist die Antwort (a) korrekt.   

\newpage
\subsection*{\frage{3}{4}}
Die stetige Zufallsvariable $ X $ habe die Dichtefunktion
\begin{align*}
	f(x) =
	\begin{cases}
		ax^2 + bx^3 \quad &\textrm{falls } 0 \leq x \leq 1\\
		\quad  \  0 \ \ &\textrm{sonst}
	\end{cases}
\end{align*}
und den Erwartungswert $ \mathbb{E}[X] = 0.75 $.
\renewcommand{\labelenumi}{(\alph{enumi})}
\begin{enumerate}
	\item 
	$a= 3,b=0$.
	\item
	$a= \frac{3}{2},b=2$.
	\item
	$a= -1,b=4$.
	\item
	$a= 0,b=2$.
	\item
	$a= \frac{12}{5},b=\frac{3}{4}$.	
\end{enumerate}
\ \\
\textbf{Lösung:}
\begin{mdframed}
\underline{\textbf{Vorgehensweise:}}
\renewcommand{\labelenumi}{\theenumi.}
\begin{enumerate}
\item Verwende die Definition einer Dichtefunktion und des Erwartungswerts.
\end{enumerate}
\end{mdframed}
%\allowdisplaybreaks
\underline{1. Verwende die Definition einer Dichtefunktion und des Erwartungswerts}\\
Wir verwenden die Definitionen der Dichtefunktion und des Erwartungswerts um ein lineares Gleichungssystem aufzustellen.
Da $ f $ eine Dichtefunktion ist, muss diese $ \int_\mathbb{R} f(x) \td{x} = 1 $ erfüllen.
Dies führt zu der der Gleichung:
\begin{align*}
	\int \limits_{\mathbb{R}} f(x) \td{x}
	&=
	\int \limits_0^1 a x^2 + bx^3 \td{x}
	=
	\frac{a}{3}x^3 + \frac{b}{4} x^4 \bigg|_{0}^1\\
	&=
	\frac{a}{3} + \frac{b}{4} = 1
	\ \Leftrightarrow \
	4 a + 3 b = 12. 
\end{align*}
Wenn $ f $ die Dichtefunktion der Zufallsvariablen $ X $ ist, gilt für deren Erwartungswert:
\begin{align*}
	\mathbb{E}[X]
	= 
	\int \limits_{\mathbb{R}} x f(x) \td{x}.
\end{align*}
%Da $ f $ zudem die Dichtefunktion der Zufallsvariablen $ X $ ist, gilt für deren Erwartungswert:
Deswegen folgt:
\begin{align*}
	\mathbb{E}[X]
	&= 
	\int \limits_{\mathbb{R}} x f(x) \td{x}
	= 
	\int \limits_0^1
	x\left(ax^2 + b x^3\right) \td{x}
	=
	\int \limits_0^1
	ax^3 + b x^4\td{x}
	=
	\frac{a}{4}x^4 + \frac{b}{5} x^5 \bigg|_{0}^1\\
	&=
	\frac{a}{4} + \frac{b}{5} = 0.75 = \frac{3}{4}
	\ \Leftrightarrow \
	5 a + 4 b = 15.
\end{align*}
Mit diesen beiden Gleichungen erhalten wir das lineare Gleichungssystem:
\begin{align*}
	\begin{pmatrix}
		4 & 3 & \BAR & 12\\
		5 & 4 & \BAR & 15
	\end{pmatrix}
	\leadsto
	\begin{pmatrix}
		20 & 15 & \BAR & 60\\
		20 & 16 & \BAR & 60
	\end{pmatrix}
	\leadsto
	\begin{pmatrix}
		20 & 15 & \BAR & 60\\
		0 & 1 & \BAR & 0
	\end{pmatrix}
	\leadsto
	\begin{pmatrix}
		20 & 0 & \BAR & 60\\
		0 & 1 & \BAR & 0
	\end{pmatrix}
	\leadsto
	\begin{pmatrix}
		1 & 0 & \BAR & 3\\
		0 & 1 & \BAR & 0
	\end{pmatrix}.
\end{align*}
Also hat das LGS die Lösung $ a = 3 $ und $ b = 0 $.\\
\\
Damit ist die Antwort (a) korrekt.
\newpage

\subsection*{\frage{4}{4}}
Gegeben ist die Funktion 
\begin{align*}
	f(x,y) = 2 \ln(x^2 +y^2 ) + (3-a) \ln(y^2), \qquad x,y > 0.
\end{align*}
Für welche Werte $ a \in \mathbb{R} $ ist der Gradient der Funktion $ f $ an der Stelle $ (1,1)  $ orthogonal zum Vektor $ \textbf{n} = \begin{pmatrix}2 \\ 1 \end{pmatrix} $?
\renewcommand{\labelenumi}{(\alph{enumi})}
\begin{enumerate}
	\item 
	$ a= 3 $.
	\item 
	$ a= 6 $.
	\item
	$ a= 9 $.
	\item
	$ a= 12 $.
	\item
	Für kein $ a \in \mathbb{R} $.
\end{enumerate}
\ \\
\textbf{Lösung:}
\begin{mdframed}
\underline{\textbf{Vorgehensweise:}}
\renewcommand{\labelenumi}{\theenumi.}
\begin{enumerate}
\item Bestimme den Gradienten der Funktion.
\end{enumerate}
\end{mdframed}

\underline{1. Bestimme den Gradienten der Funktion}\\
Die partiellen Ableitungen von $ f $ sind durch
\begin{align*}
	f_x(x,y) 
	&= 
	2\frac{2x}{x^2 +y^2}
	=
	\frac{4x}{x^2 +y^2}
	\\
	f_y(x,y)
	&=
	2\frac{2y}{x^2 +y^2}
	+
	(3-a) \frac{2y}{y^2} 
	=
	\frac{4y}{x^2 +y^2}+
	(3-a) \frac{2}{y} 
\end{align*}
gegeben. Damit ist der Gradient an der Stelle $ (1,1) $:
\begin{align*}
	\mathrm{grad} f(1,1)
	=
	\begin{pmatrix}
		\frac{4}{2}\\
		2 + (3-a)\cdot 2
	\end{pmatrix}
	=
	\begin{pmatrix}
		2\\
		8 - 2a
	\end{pmatrix}.
\end{align*}
Dieser ist orthogonal zu $ \textbf{n} $, falls gilt:
\begin{align*}
	\begin{pmatrix}
		2\\
		8 - 2 a
	\end{pmatrix}
	\cdot
	\begin{pmatrix}
		2\\
		1
	\end{pmatrix}
	=
	2 \cdot 2 +  (8-2a) \cdot 1
	=
	4 +8  - 2a 
	=12 - 2a = 0
	\ \Leftrightarrow \
	a = 6.
\end{align*}
Damit ist die Antwort (b) korrekt.
\newpage

\subsection*{\frage{5}{4}}
Gegeben ist die Matrix
\begin{align*}
	A =
	\begin{pmatrix}
		2 & 0 & 1 & 8 \\
		-1 & 2 & 1 & 1 \\
		1 & 2 & 1 &  0 \\
		0 & 0 & 1 & 9
	\end{pmatrix}.
\end{align*}
$ A $ hat
\renewcommand{\labelenumi}{(\alph{enumi})}
\begin{enumerate}
	\item 
	Rang $ 1 $.
	\item 
	Rang $ 2 $.
	\item
	Rang $ 3 $.
	\item
	Rang $ 4 $.
	\item
	Rang $ 5 $.
\end{enumerate}
\ \\
\textbf{Lösung:}
\begin{mdframed}
\underline{\textbf{Vorgehensweise:}}
\renewcommand{\labelenumi}{\theenumi.}
\begin{enumerate}
\item Verwende den Gaußalgorithmus.
\end{enumerate}
\end{mdframed}

\underline{1. Verwende den Gaußalgorithmus}\\
Wir wenden elementare Zeilenoperationen an, um den Rang abzulesen:
\begin{align*}
	A 
	&=
	\begin{gmatrix}[p]
		2 & 0 & 1 & 8 \\
		-1 & 2 & 1 & 1 \\
		1 & 2 & 1 &  0 \\
		0 & 0 & 1 & 9
		\rowops
		\swap{0}{1}	
	\end{gmatrix}
	\leadsto
	\begin{gmatrix}[p]
		-1 & 2 & 1 & 1 \\
		2 & 0 & 1 & 8 \\
		1 & 2 & 1 &  0 \\
		0 & 0 & 1 & 9
		\rowops
		\add[ \cdot 2]{0}{1}
		\add[\cdot 1]{0}{2}	
	\end{gmatrix}\\
	&\leadsto
	\begin{gmatrix}[p]
		-1 & 2 & 1 & 1 \\
		0 & 4 & 3 & 10 \\
		0 & 4 & 2 &  1 \\
		0 & 0 & 1 & 9
		\rowops
		\add[ \cdot (-1)]{1}{2}	
	\end{gmatrix}
	\leadsto
	\begin{gmatrix}[p]
		-1 & 2 & 1 & 1 \\
		0 & 4 & 3 & 10 \\
		0 & 0 & -1 &  -9 \\
		0 & 0 & 1 & 9
		\rowops
		\add[ \cdot 1]{2}{3}	
	\end{gmatrix}\\
	&\leadsto
	\begin{gmatrix}[p]
		-1 & 2 & 1 & 1 \\
		0 & 4 & 3 & 10 \\
		0 & 0 & -1 &  -9 \\
		0 & 0 & 0 & 0
		\rowops	
	\end{gmatrix}.
\end{align*}
Wir erhalten drei linear unabhängige Zeilen, womit $ \mathrm{rg}(A) = 3 $ gilt.\\
\\
Damit ist die Antwort (c) korrekt.
%\underline{1. Überlege dir, welche Möglichkeiten keine Eigenvektoren sein können}\\
%Wir kürzen mit $ \textbf{x}_a $, $ \textbf{x}_b $, $ \textbf{x}_c $ und $ \textbf{x}_d $ die Vektoren von (a) bis (d) ab.\\
%\\
%Ein Vektor $ \textbf{x} \neq 0 $ heißt Eigenvektor zum Eigenwert $ \lambda $, falls
%\begin{align*}
%M \textbf{x} = \lambda \textbf{x}
%\end{align*}
%gilt. Das heißt $ M \textbf{x} $ muss ein Vielfaches von $ \textbf{x} $ sein.
%Ansonsten ist $ \textbf{x} $ kein Eigenvektor.\\
%\\
%Wegen 
%\begin{align*}
%M \textbf{x}_b = \begin{pmatrix}
%0 \\ 1 \\ 1
%\end{pmatrix}
%\end{align*}
%kann $ \textbf{x}_b $ kein Eigenvektor sein. 
%\newpage
%Ebenso gilt:
%\begin{align*}
%M \textbf{x}_d = \begin{pmatrix}
%-2 \\ 4 \\ 4
%\end{pmatrix}.
%\end{align*}
%Also kann $ \textbf{x}_d $ auch kein Eigenvektor sein.\\
%\\
%\underline{2. Bestimme die richtige Aussage}\\
%Es bleiben die Möglichkeiten (a) und (c).
%Wir erhalten:
%\begin{align*}
%M \textbf{x}_a = \begin{pmatrix}
%2 \\ 0 \\-2
%\end{pmatrix}
%= 2 \cdot \textbf{x}_a\\
%M \textbf{x}_c = 
%\begin{pmatrix}
%-2 \\ 1 \\ 1
%\end{pmatrix}
%= 
%1 \cdot \textbf{x}_c.
%\end{align*}
%Also ist $ \textbf{x}_a $ ein Eigenvektor zum Eigenwert $ 2 $ und $ \textbf{x}_c $ ein Eigenvektor zum Eigenwert $ 1 $.\\
%\\
%Damit ist die Antwort (c) korrekt.
\newpage

\subsection*{\frage{6}{4}}
Gegeben ist die Matrix
\begin{align*}
	A =
	\begin{pmatrix}
		0 & -1 & 0 \\
		1 & 0 & 0 \\
		1 & 1 & m
	\end{pmatrix}
\end{align*}
und ihre Inverse
\begin{align*}
	A^{-1} =
	\begin{pmatrix}
		0 & 1 & 0 \\
		-1 & 0 & 0 \\
		0.5 & -0.5 &0.5
	\end{pmatrix}
\end{align*}
Es folgt:
\renewcommand{\labelenumi}{(\alph{enumi})}
\begin{enumerate}
	\item 
	$ A $ ist singulär für alle $ m \in \mathbb{R} $.
	\item 
	$ m = -1 $.
	\item
	$ m = 0 $.
	\item
	$ m = 1 $.
	\item
	$ m = 2 $.
\end{enumerate}
\ \\
\textbf{Lösung:}
\begin{mdframed}
\underline{\textbf{Vorgehensweise:}}
\renewcommand{\labelenumi}{\theenumi.}
\begin{enumerate}
\item Bestimme das Produkt der Matrix mit ihrer Inversen.
\end{enumerate}
\end{mdframed}

\underline{1. Bestimme das Produkt der Matrix mit ihrer Inversen}\\
Das Matrixprodukt von $ A $ und $ A^{-1} $ ist:
\begin{align*}
	A \cdot A^{-1}
	=
	\begin{pmatrix}
		0 & -1 & 0 \\
		1 & 0 & 0 \\
		1 & 1 & m
	\end{pmatrix}
	\cdot
	\begin{pmatrix}
		0 & 1 & 0 \\
		-1 & 0 & 0 \\
		0.5 & -0.5 &0.5
	\end{pmatrix}
	=
	\begin{pmatrix}
		(-1) \cdot (-1) & 0 & 0\\
		0 & 1 & 0\\
		-1 + \frac{m}{2} & 1 - \frac{m}{2} & \frac{m}{2}
	\end{pmatrix}.
\end{align*}
Nun gilt 
\begin{align*}
	\begin{pmatrix}
		(-1) \cdot (-1) & 0 & 0\\
		0 & 1 & 0\\
		-1 + \frac{m}{2} & 1 - \frac{m}{2} & \frac{m}{2}
	\end{pmatrix}
	=
	\begin{pmatrix}
		1 & 0 & 0 \\
		0 & 1 & 0 \\
		0 & 0 & 1
	\end{pmatrix}
\end{align*}
für $ m = 2 $.\\
\\
Damit ist die Antwort (e) korrekt.






\newpage



\subsection*{\frage{7}{4}}
Gegeben ist die Matrix
\begin{align*}
	A =
	\begin{pmatrix}
		4 & 1 & 1 \\
		1 & 4 & 1 \\
		1 & 1 & 4
	\end{pmatrix}.
\end{align*}
Die Matrix $ B = A^3 $ hat den Eigenwert
\renewcommand{\labelenumi}{(\alph{enumi})}
\begin{enumerate}
	\item
	$ 64 $.
	\item
	$ 27 $.	
	\item 
	$ 8 $.
	\item
	$ 5 $.
	\item
	$ 1 $.
\end{enumerate}
\ \\
\textbf{Lösung:}
\begin{mdframed}
\underline{\textbf{Vorgehensweise:}}
\renewcommand{\labelenumi}{\theenumi.}
\begin{enumerate}
\item Verwende die Definition der Eigenwerte.
\end{enumerate}
\end{mdframed}

\underline{1. Verwende die Definition der Eigenwerte}\\
Ein Vektor $ \textbf{x} \neq \textbf{0} $ heißt Eigenvektor zum Eigenwert $ \lambda $ der Matrix $ A $, falls
\begin{align*}
	A \textbf{x} = \lambda \cdot \textbf{x} 
\end{align*}
gilt. Hieraus folgt:
\begin{align*}
	B \textbf{x} = A^3  \textbf{x}
	= A^2 ( A \textbf{x} ) 
	= \lambda A^2 \textbf{x}
	= \lambda^2 A \textbf{x}
	= \lambda^3\textbf{x}.
\end{align*}
Damit gilt: Wenn $ \lambda $ ein Eigenwert von $ A $ ist, ist $ \lambda^3 $ ein Eigenwert von $ B $.
Umgekehrt gilt: Ist $ \mu $ ein Eigenwert von $ B $, so ist $ \sqrt[3]{\mu} $ ein Eigenwert von $ A $.
Um die Frage zu beantworten, benötigen wir die Eigenwerte von $ A $. Hierfür zerlegen wir das charakteristische Polynom in Linearfaktoren:
\begin{align*}
	\det(A - \lambda I)
	&=
	\begin{vmatrix}
		4- \lambda & 1 & 1 \\
		1 & 4- \lambda & 1\\
		1 & 1 & 4 - \lambda
	\end{vmatrix}
=
	(4 - \lambda) 
	\begin{vmatrix}
		4- \lambda & 1\\
		 1 & 4 - \lambda
	\end{vmatrix}
	-
	\begin{vmatrix}
		1 & 1\\
		1 & 4 - \lambda
	\end{vmatrix}
	+
	\begin{vmatrix}
		1 & 1\\
		4 - \lambda &  1
	\end{vmatrix}\\
	&=
	(4-\lambda)^3 - (4- \lambda)
	-(4-\lambda) + 1 
	+1 - (4-\lambda )\\
	&=
	(4-\lambda)( (4-\lambda)^2 - 1) + 2 ( 1 - (4- \lambda))\\
	&=
	(4 - \lambda)(15 - 8 \lambda + \lambda^2) + 2 (\lambda -3)\\
	&=
	(4 - \lambda)(\lambda - 5)(\lambda - 3) + 2 ( \lambda -3)\\
	&=
	(\lambda - 3) 
	\left(
	(4 - \lambda)(\lambda - 5) +2
	\right)\\
	&=
	(\lambda - 3) 
	\left(
	4 \lambda - 18 - \lambda^2 + 5 \lambda
	\right)\\
	&=
	-(\lambda - 3) 
	\left(
	\lambda^2 - 9 \lambda + 18
	\right)\\
	&=
	(\lambda-3)^2(\lambda-6).
\end{align*}
Damit lassen sich die  Eigenwerte $ \lambda_1 = 3 $ und $ \lambda_2 = 6 $ von $ A $ ablesen.
Also besitzt $ B $ die Eigenwerte $ \lambda_1^3 = 27 $ und $ \lambda_2^3 = 216 $.\\
\\
Somit ist Antwort (b) korrekt.

\newpage

\subsection*{\frage{8}{3}}
Die allgemeine Lösung der  Differenzengleichung
\begin{align*}
	\frac{1}{3} y_k - \frac{3}{4} y_{k+1} + \frac{1}{27} y_k  = \frac{1}{9} y_k + 2 \quad (k = 0,1,2,...)
\end{align*}
ist
\renewcommand{\labelenumi}{(\alph{enumi})}
\begin{enumerate}
	\item
	monoton und konvergent.
	\item
	monoton und divergent.	
	\item 
	oszillierend und konvergent.
	\item
	oszillierend und divergent.
\end{enumerate}
\ \\
\textbf{Lösung:}
\begin{mdframed}
\underline{\textbf{Vorgehensweise:}}
\renewcommand{\labelenumi}{\theenumi.}
\begin{enumerate}
\item Forme in die Normalform um.
\end{enumerate}
\end{mdframed}

\underline{1. Forme in die Normalform um}\\
Die Normalform der Differenzengleichung erhalten wir durch:
\begin{align*}
	&\frac{1}{3} y_k - \frac{3}{4} y_{k+1} + \frac{1}{27} y_k  = \frac{1}{9} y_k + 2\\
	\ \Leftrightarrow \
	- &\frac{3}{4} y_{k+1}
	=
	\left(
	\frac{1}{9} - \frac{1}{3} - \frac{1}{27} 
	\right)
	\cdot 
	y_k + 2
	=
	\left(
	\frac{3}{27} - \frac{9}{27} - \frac{1}{27} 
	\right)
	\cdot 
	y_k + 2
	=
	-\frac{7}{27} y_k +2\\
	\ \Leftrightarrow \
	&\ 
	y_{k+1} = -\frac{4}{3} \cdot
	\left(-\frac{7}{27} y_k +2\right)
	=
	\frac{28}{81} y_k -\frac{8}{3}.
\end{align*}
Hierbei ist $ A = \frac{28}{81} $  und $ B= - \frac{8}{3} $.
Für das Konvergenzverhalten ist $ A $ relevant.
Wegen $ 0 < A < 1 $ ist $ |A| < 1 $ und $ A > 0 $ erfüllt.
Die erste Eigenschaft sorgt für die Konvergenz und die Zweite für die Monotonie.\\
\\
Damit ist die Antwort (a) korrekt.
