\fancyhead[C]{\normalsize\textbf{$\qquad$ Teil II: Multiple-Choice}}
\section*{Aufgabe 2 (33 Punkte)}
\vspace{0.4cm}
\subsection*{\frage{1}{3}}
Unter der Nebenbedingung $ \varphi(x,y) = \frac{(x-4)^2}{16} + \frac{(y-3)^2}{49} -1 = 0 $ hat die Funktion $ f(x,y) = 2 x + 1  $ ein Minimum in welchem Punkt?
\renewcommand{\labelenumi}{(\alph{enumi})}
\begin{enumerate}
	\item $ P= (0,3) $.
	\item $ P= (4,10) $.
	\item $ P= (-5,3) $.
	\item $ P= (4,-4) $.
	\item Keiner der oben gegebenen Punkte ist ein Minimum der Funktion $ f $ unter der Nebenbedingung $ \varphi(x,y)= 0 $.
\end{enumerate}\ \\
\textbf{Lösung:}
\begin{mdframed}
\underline{\textbf{Vorgehensweise:}}
\renewcommand{\labelenumi}{\theenumi.}
\begin{enumerate}
\item Überlege dir, welche Punkte die Nebenbedingung erfüllen.
\item Nutze die Funktion (Darstellung einer Ellipse).
\end{enumerate}
\end{mdframed}

\underline{1. Überlege dir, welche Punkte die Nebenbedingung erfüllen}\\
Wir kennzeichnen die Punkte durch $ P_a, P_b , P_c$ und $ P_d $.
Durch Einsetzen in die Nebenbedingung erhalten wir:
\begin{align*}
	\varphi(P_a) 
	&= 
	\frac{(0-4)^2}{16} + \frac{(3-3)^2}{49} -1
	=
	\frac{16}{16} + 0 - 1 = 0\\
	\varphi(P_b)
	&=
	\frac{(4-4)^2}{16} + \frac{(10-3)^2}{49} -1
	=
	0 + \frac{49}{49} -1 = 0\\
	\varphi(P_c)
	&=
	\frac{(-5-4)^2}{16} + \frac{(3-3)^2}{49} -1
	=
	\frac{81}{16} - 1 \neq 0\\
	\varphi(P_d)
	&=
	\frac{(4-4)^2}{16} + \frac{(-4-3)^2}{49} -1
	=
	\frac{49}{49 } -1 = 0.
\end{align*}
Damit erfüllen alle Punkte außer $ P_c $ die Nebenbedingung.\\
\\
\underline{2. Nutze die Funktion (Darstellung einer Ellipse)}\\
Die Funktion $ f(x,y) = 2x + 1 $ ist unabhängig von der $ y $- Koordinate und linear in der $ x $-Koordinate.
Außerdem ist $ 2x + 1$ streng monoton wachsend.
Die Nebenbedingung
\begin{align*}
	\frac{(x-4)^2}{16} + \frac{(y-3)^2}{49} -1 = 0 
	\ \Leftrightarrow \
	\frac{(x-4)^2}{16} + \frac{(y-3)^2}{49} = 1
\end{align*}
beschreibt eine Ellipse mit dem Mittelpunkt $ (4,3) $. 
Den kleinsten Wert in der $ x $-Koordinate hat diese gerade in dem Punkt $ P_a = (0,3) $.\\
\\
Damit ist die Antwort (a) korrekt. 


\newpage

\subsection*{\frage{2}{3}}
Die Funktion $ f $ hat ein lokales Minimum im Punkt $ (x_0,y_0) $.
Sei $ g $ die Funktion definiert als $ g(x,y) = e^{-f(-x,-y)} $ mit Definitionsgebiet $ D_g = D_f $.\\
Welche der folgenden Aussagen ist wahr?
\renewcommand{\labelenumi}{(\alph{enumi})}
\begin{enumerate}
	\item $ g $ hat ein lokales Maximum im Punkt $ (-x_0,-y_0) $.
	\item $ g $ hat ein lokales Minimum im Punkt $ (-x_0,-y_0) $.
	\item $ g $ hat ein lokales Maximum im Punkt $ (x_0,y_0) $.
	\item $ g $ hat ein lokales Minimum im Punkt $ (x_0,y_0) $.
\end{enumerate}
\ \\
\textbf{Lösung:}
\begin{mdframed}
	\underline{\textbf{Vorgehensweise:}}
	\renewcommand{\labelenumi}{\theenumi.}
	\begin{enumerate}
		\item Löse die Aufgabe anschaulich.
	\end{enumerate}
\end{mdframed}
\underline{1. Löse die Aufgabe anschaulich}\\
Uns ist bekannt, dass $ f $ ein Minimum an der Stelle $ (x_0,y_0) $ besitzt.
Damit liegt für $ h_1(x,y) := f(-x,-y)$ ein Minimum an der Stelle $ (-x_0,-y_0) $ vor.
Wenn wir noch ein negatives Vorzeichen vorschalten ist $ (-x_0,-y_0) $ ein lokales Maximum der Funktion $ h_2(x,y) := -f(-x,-y) $.
Da die Exponentialfunktion streng monoton wachsend ist, bleibt die lokale Maximumseigenschaft in $ (-x_0,-y_0) $ für $ g $ erhalten.\\
\\
Damit ist die Antwort (a) korrekt.\\
\\
\textit{Alternativer Lösungsweg:}\\
Da $ f $ ein lokales Minimum an der Stelle $ (x_0,y_0) $ besitzt, ist die notwendige Bedingung 
\begin{align*}
	f_x(x_0,y_0) =0\\
	f_y(x_0,y_0) =0
\end{align*}
erfüllt. Für $ g $ erhalten wir mit der Kettenregel die partiellen Ableitungen:
\begin{align*}
	g_x(x,y) &= (-1) (-1) f_x(-x,-y) e^{-f(-x,-y)}
	=
	f_x(-x,-y) e^{-f(-x,-y)}\\
	g_y(x,y) &=
	f_y(-x,-y) e^{-f(-x,-y)}.
\end{align*}
Also können wir die Antworten (c) und (d) auschließen.
Für diese müssten auch $ f_x(-x_0,-y_0) = f_x(-x_0,-y_0) = 0 $ gelten. 
Wegen des lokalen Minimums ist die hinreichende Bedingung 
\begin{align*}
	&f_{xx}(x_0,y_0) >0\\
	&f_{yy}(x_0,y_0) > 0\\
	&f_{xx}(x_0,y_0)f_{yy}(x_0,y_0) - (f_{xy}(x_0,y_0))^2 >0
\end{align*}
erfüllt. Die zweifachen partiellen Ableitungen von $ g $ sind gegeben durch:
\begin{align*}
	g_{xx}(x,y) 
	&=
	-f_{xx}(-x,-y) e^{-f(-x,-y)}
	+ (f_x(-x,-y) )^2 e^{-f(-x,-y)}\\
	g_{xx}(x,y) 
	&=
	-f_{yy}(-x,-y) e^{-f(-x,-y)}
	+ (f_y(-x,-y) )^2 e^{-f(-x,-y)}\\
	g_{xy}(x,y)
	&=
	-f_{xy}(-x,-y) e^{-f(-x,-y)} + f_x(-x,-y) f_y(-x,-y) e^{-f(-x,-y)}
\end{align*}
Durch Einsetzen von $ (-x_0,-y_0) $ folgt:
\begin{align*}
	g_{xx}(-x_0,-y_0) 
	&=
	-f_{xx}(x_0,x_0) e^{-f(x_0,x_0)}
	+ (f_x(x_0,y_0) )^2 e^{-f(x_0,x_0)}
	= -f_{xx}(x_0,x_0) e^{-f(x_0,x_0)} < 0\\
	g_{yy}(-x_0,-y_0) 
	&=
	-f_{yy}(x_0,x_0) e^{-f(x_0,x_0)}
	+ (f_y(x_0,y_0) )^2 e^{-f(x_0,x_0)}
	= -f_{yy}(x_0,x_0) e^{-f(x_0,x_0)} < 0\\
	g_{xy}(-x_0,-y_0)
	&=
	-f_{xy}(x_0,x_0) e^{-f(x_0,x_0)} + f_x(x_0,x_0) f_y(x_0,x_0) e^{-f(x_0,x_0)}
	=
	-f_{xy}(x_0,x_0) e^{-f(x_0,x_0)}\\
	&\Rightarrow
	g_{xx}(-x_0,-y_0) g_{yy}(-x_0,-y_0) - (g_{xy}(-x_0,-y_0))^2\\
	&=
	(-f_{xx}(x_0,x_0)) (-f_{yy}(x_0,x_0)) e^{-2f(x_0,x_0)}
	-
	(-f_{xy}(x_0,x_0))^2 e^{-2f(x_0,x_0)}\\
	&=
	(f_{xx}(x_0,x_0)f_{yy}(x_0,x_0) - (f_{xy}(x_0,x_0))^2) e^{-2f(x_0,x_0)}
	> 0
\end{align*}
Damit besitzt $ g $ ein lokales Maximum an der Stelle $ (-x_0,-y_0) $.
\newpage
\subsection*{\frage{3}{4}}
Die Reihe $ \sum_{k=1}^\infty \frac{m}{k^4} $, wobei $ m > 0 $, konvergiert gegen $ a(m) \in \mathbb{R} $, wobei $ a(m) $ von dem Parameter $ m $ abhängt.\\
\\
Es gilt:
\renewcommand{\labelenumi}{(\alph{enumi})}
\begin{enumerate}
	\item 
	$ a(m) > \frac{4m}{3} $ für alle $ m \geq 1 $.
	\item 
	$ a(m) < \frac{4m}{3} $ für alle $ m  $.
	\item 
	$ a(m) < m $ für alle $ m  $.
	\item
	$ a(m) < \frac{m}{2} $ für alle $ m  $.
\end{enumerate}
\ \\
\textbf{Lösung:}
\begin{mdframed}
\underline{\textbf{Vorgehensweise:}}
\renewcommand{\labelenumi}{\theenumi.}
\begin{enumerate}
\item Verwende das Integralvergleichskriterium.
\end{enumerate}
\end{mdframed}

\underline{1. Verwende das Integralvergleichskriterium}\\
Das Integralvergleichskriterium besagt:
\begin{align*}
	\sum \limits_{k=1}^\infty f(k) \ \textrm{konvergent}
	\ \Leftrightarrow \
	\int \limits_1^\infty f(x) \td{x} < \infty.
\end{align*}
In unserem Fall gilt $ f (x) = \frac{1}{x^4} $. Wenn $ \sum_{k=1}^\infty f(k) $ konvergiert, gilt die Abschätzung:
\begin{align*}
	\sum \limits_{k=2}^\infty f(k) 
	<
	\int \limits_1^\infty f(x) \td{x}
	<
	\sum \limits_{k=1}^\infty f(k). 
\end{align*}
Wir erhalten:
\begin{align*}
	a(m)
	&=
	\sum \limits_{k=1}^\infty \frac{m}{k^4}
	=
	m \sum \limits_{k=1}^\infty \frac{1}{k^4}
	=
	m\left(1 + \sum \limits_{k=2}^\infty \frac{1}{k^4} \right)
	<
	m \left(1 + \int \limits_1^\infty \frac{1}{x^4} \td{x} \right)
	=
	m \left(
	1 -  \frac{1}{3x^3} \bigg|_1^\infty
	\right)\\
	&=
	m 
	\left(1  + \frac{1}{3} \right)
	=
 	\frac{4}{3}	m.
\end{align*}
Damit ist die Antwort (b) korrekt.

\newpage

\subsection*{\frage{4}{2}}
Sei $ f $ eine stetige Funktion $ a, x \in D_f $ mit $ a < x $.
Sei $ g $ definiert als $ g(x) = \int_a^x f(t) dt $.
Welche der folgenden Aussagen ist wahr?  
\renewcommand{\labelenumi}{(\alph{enumi})}
\begin{enumerate}
	\item 
	$ f^\prime(x) = g(x) $.
	\item
	$ g^\prime(x) = f(x) - f(a) $.
	\item
	$ g \circ f  $ ist eine Stammfunktion von $ f $.
	\item
	$ \lim_{\Delta \to 0} \frac{g(x+ \Delta x) - g(x)}{\Delta x} = f(x) $.
	\item
	Keine der oben gegebenen Antwortmöglichkeiten ist korrekt.
\end{enumerate}
\ \\
\textbf{Lösung:}
\begin{mdframed}
\underline{\textbf{Vorgehensweise:}}
\renewcommand{\labelenumi}{\theenumi.}
\begin{enumerate}
\item Verwende den Hauptsatz der Differential -und Integralrechnung(HDI).
\end{enumerate}
\end{mdframed}

\underline{1. Verwende den Hauptsatz der Differential -und Integralrechnung(HDI)}\\
%Wir kürzen den Hauptsatz durch HDI ab.
Nach dem HDI ist $ g(x) = \int_a^x f(t) dt $ eine Stammfunktion von $ f $.
Damit gilt $ g^\prime(x) = f(x) $ und die Antwortmöglichkeiten (a) , (b) und (c) sind im Allgemeinen falsch.
Mit der Definition der Ableitung folgt:
\begin{align*}
	g^\prime(x) 
	=
	\lim\limits_{\Delta \to 0} \frac{g(x+ \Delta x) - g(x)}{\Delta x} 
	= f(x). 
\end{align*}
Damit ist die Antwort (d) korrekt.
 

\newpage
\subsection*{\frage{5}{3}}
Gegeben ist die Funktion
\begin{align*}
	f(x) =
	\begin{cases}
		\frac{1}{2 x^3} + 2 x c e^{-c x^2}, \ &x \geq 1\\
		\qquad	\ 0 , \ &x < 1
	\end{cases}
\end{align*}
Dann ist $ f $ eine Dichtefunktion
\renewcommand{\labelenumi}{(\alph{enumi})}
\begin{enumerate}
	\item 
	für alle $ c \in \mathbb{R} $.
	\item 
	für kein $ c \in \mathbb{R} $.
	\item 
	für alle $ c > 0 $.
	\item 
	für $ c = \ln(0.75) $.
	\item
	für $ c = \ln\left(\frac{4}{3}\right) $.
\end{enumerate}
\ \\
\textbf{Lösung:}
\begin{mdframed}
\underline{\textbf{Vorgehensweise:}}
\renewcommand{\labelenumi}{\theenumi.}
\begin{enumerate}
\item Verwende die Defintion einer Dichtefunktion.
\end{enumerate}
\end{mdframed}

\underline{1. Verwende die Defintion einer Dichtefunktion}\\
Eine Funktion $ f : \mathbb{R} \to \mathbb{R} $ heißt Dichtefunktion, falls $ g $ nichtnegativ ($ f(x) \geq 0 $ für alle $ x \in \mathbb{R} $) und 
\begin{align*}
	\int \limits_\mathbb{R} f(x) \td{x} = 1
\end{align*}
gilt.
Die erste Eigenschaft ist für das $ f $  aus der Aufgabenstellung erfüllt.
Wir suchen nun das $ c $, sodass auch die zweite Eigenschaft gilt.
Für das Integral über $ f $ gilt:
\begin{align*}
	\int \limits_\mathbb{R} f(x) \td{x}
	&=
	\int \limits_1^\infty f(x) \td{x}
	=
	\int \limits_1^\infty \frac{1}{2x^3} + 2x c e^{-cx^2} \td{x}
	=
	\int \limits_1^\infty \frac{1}{2x^3} \td{x} 
	+ 
	\int \limits_1^\infty 2 c x e^{-cx^2} \td{x}\\
	&=
	-\frac{1}{4x^2} \bigg|_1^\infty
	-
	e^{-cx^2} \bigg|_1^\infty 
	=
	\frac{1}{4} + e^{-c}.
\end{align*}
Damit $ f $ eine Dichtefunktion ist, muss gelten:
\begin{align*}
	\frac{1}{4} + e^{-c} = 1
	\ \Leftrightarrow \
	e^{-c} = \frac{3}{4}
	\ \Leftrightarrow \
	-c = \ln \left(\frac{3}{4}\right)
	\ \Leftrightarrow \
	c = - \ln \left(\frac{3}{4}\right) = \ln \left(\frac{4}{3}\right).
\end{align*}
Für $ c  = \ln \left(\frac{4}{3}\right)$ ist $ f $ also eine Dichtefunktion.\\
\\
Damit ist Antwort (e) korrekt.
 

\newpage

\subsection*{\frage{6}{3}}
$ A $ sei eine $ (n \times m )- $dimensionale Matrix, und sei $ B $ eine $ (p \times q)- $dimensionale Matrix.\\
Wenn die Matrizen $ C = BA $ und $ D = AB $ existieren und es gilt $ C = D $, dann folgt daraus:
\renewcommand{\labelenumi}{(\alph{enumi})}
\begin{enumerate}
	\item 
	$ n= m = p = q $.
	\item 
	$ n = q $ und $ m= p $ und $ m \neq q $.
	\item
	$ m= p $ und $ q = n$ und $ n \neq p $.
	\item
	$ n = p $ und $ m= q $ und $ n \neq m $.
\end{enumerate}
\ \\
\textbf{Lösung:}
\begin{mdframed}
\underline{\textbf{Vorgehensweise:}}
\renewcommand{\labelenumi}{\theenumi.}
\begin{enumerate}
\item Verwende den Zusammenhang zwischen Zeilen und Spalten bei einem Matrixprodukt.
\end{enumerate}
\end{mdframed}

\underline{1. Verwende den Zusammenhang zwischen Zeilen und Spalten bei einem Matrixprodukt}\\
Wir wissen, dass die Matrizen $ C = BA $ und $ D = AB $ existieren.
Da $ A $ eine $ (n \times m) $-Matrix und $ B $ eine $ (p \times q) $-Matrix ist, erhalten wir:
\begin{align*}
	C = BA \ \textrm{existiert} \ 
	&\Rightarrow 
	\
	q = n\\
	D = AB \ \textrm{existiert} \ 
	&\Rightarrow 
	\
	p = q.
\end{align*}
Hierbei ist $ C $ eine $ (p \times m) $-Matrix und $ B  $ eine $(n \times q)$-Matrix.
Wegen $ C = D $ gilt auch $ n = p $ und $ q = m $.
Insgesamt folgt dann $ n = m = p = q $.\\
\\
Damit ist die Antwort (a) korrekt. 


\newpage
\subsection*{\frage{7}{2}}
Eine $ (4 \times 5)- $dimensionale Matrix $ A $ hat den Rang $ 3 $.
$ B $ ist eine $ (8 \times 5)- $dimensionale Matrix, in welcher jede Zeile von $ A $ genau zweimal vorkommt. Dann gilt:
\renewcommand{\labelenumi}{(\alph{enumi})}
\begin{enumerate}
	\item 
	$ \mathrm{rg}(B) = 1 $.
	\item
	$ \mathrm{rg}(B) = 3 $.
	\item
	$ \mathrm{rg}(B) = 4 $.
	\item
	$ \mathrm{rg}(B) = 5 $.
	\item
	$ \mathrm{rg}(B) = 6 $.
	\item
	$ \mathrm{rg}(B) = 8 $.
\end{enumerate}
\ \\
\textbf{Lösung:}
\begin{mdframed}
\underline{\textbf{Vorgehensweise:}}
\renewcommand{\labelenumi}{\theenumi.}
\begin{enumerate}
\item Verwende die Definition des Ranges einer Matrix.
\end{enumerate}
\end{mdframed}

\underline{1. Verwende die Definition des Ranges einer Matrix}\\
Die Matrix $ A $ hat den Rang $ 3 $.
Damit sind $ 3  $ Zeilen der Matrix $ A $ zueinander unabhängig.
Also erreichen wir mithilfe elementarer Zeilenumformungen:
\begin{align*}
	A
	\leadsto
	\begin{pmatrix}
		1 & \ast & \ast & \ast &\ast \\
		0 & 1 &  \ast & \ast &\ast\\
		0 & 0 & 1 & \ast &\ast\\
		0 & 0 & 0 & 0 &0
	\end{pmatrix}.
\end{align*}
Hierbei sind mit $ \ast $ beliebige Werte gekennzeichnet.
Nun kommen in $ B $ die Zeilen von $ A $ genau zweimal vor.
Somit folgt durch elementare Zeilenumformungen:
\begin{align*}
	B 
	\leadsto
	\begin{pmatrix}
		A\\
		A
	\end{pmatrix}
	\leadsto
	\begin{pmatrix}
		A\\
		0
	\end{pmatrix}
	\leadsto
	\begin{pmatrix}
		1 & \ast & \ast & \ast &\ast \\
		0 & 1 &  \ast & \ast &\ast\\
		0 & 0 & 1 & \ast &\ast\\
		0 & 0 & 0 & 0 &0\\
		0 & 0 & 0 & 0 &0\\
		0 & 0 & 0 & 0 &0\\
		0 & 0 & 0 & 0 &0\\
		0 & 0 & 0 & 0 &0.
	\end{pmatrix}
	\ \Rightarrow \
	\mathrm{rg}(B) = 3.
\end{align*}
Wir sortieren $ B $ zuerst so um, dass die Matrix $  A $ untereinander steht.
Darauffolgend eliminieren wir die Kopie von $ A $ und sehen, dass $ A $ den Rang $ 3 $ hat.\\
\\
Also ist die Antwort (b) korrekt.   


\newpage

\subsection*{\frage{8}{3}}
Seien $ \textbf{a}_1, \textbf{a}_2 , \textbf{a}_3 $, und $ \textbf{b} $ $ m $-dimensionale Vektoren und $ m \geq 2 $.
Der Vektor $ \textbf{b} $ ist eine Linearkombination der Vektoren $  \textbf{a}_1, \textbf{a}_2  $ und $ \textbf{a}_3 $ genau dann, wenn gilt:
\renewcommand{\labelenumi}{(\alph{enumi})}
\begin{enumerate}
	\item 
	$ \mathrm{rg}\left([ \textbf{a}_1, \textbf{a}_2 , \textbf{a}_3 ,\textbf{b}] \right) = 3 $.
	\item
	$ \mathrm{rg}\left([ \textbf{a}_1, \textbf{a}_2 , \textbf{a}_3 ,\textbf{b}] \right) =m $.
	
	\item
	$ \det\left([ \textbf{a}_1, \textbf{a}_2 , \textbf{a}_3] \right) = 0 $.
	\item
	$ \mathrm{rg}\left([ \textbf{a}_1, \textbf{a}_2 , \textbf{a}_3 ,\textbf{b}] \right) = \mathrm{rg}\left([ \textbf{a}_1, \textbf{a}_2 , \textbf{a}_3]  \right) = 3 $.
	\item
	$ \mathrm{rg}\left([ \textbf{a}_1, \textbf{a}_2 , \textbf{a}_3 ,\textbf{b}] \right) = \mathrm{rg}\left([ \textbf{a}_1, \textbf{a}_2 , \textbf{a}_3] \right) $.
\end{enumerate}
\ \\
\textbf{Lösung:}
\begin{mdframed}
\underline{\textbf{Vorgehensweise:}}
\renewcommand{\labelenumi}{\theenumi.}
\begin{enumerate}
\item Verwende die Lösbarkeit eines linearen Gleichungssystems.
\end{enumerate}
\end{mdframed}

\underline{1. Verwende die Lösbarkeit eines linearen Gleichungssystems}\\
Wir fassen die Vektoren $ \textbf{a}_1 $, $ \textbf{a}_2 $ und $ \textbf{a}_3 $ in der Matrix $ A $ zusammen. 
Das heißt : $ A = [\textbf{a}_1,\textbf{a}_2,\textbf{a}_3] $.
Der Vektor $ \textbf{b} $ ist eine Linearkombination der drei Vektoren, falls das LGS
\begin{align*}
	A \textbf{x} = 
	A \begin{pmatrix}
		x_1\\ x_2 \\ x_3
	\end{pmatrix}
	= \textbf{b}
\end{align*}
lösbar ist.
Diese Lösbarkeit ist genau dann erfüllt, wenn 
\begin{align*}
	\mathrm{rg}
	[A, \textbf{b}]
	=
	\mathrm{rg}(A)
\end{align*}
gilt. Dies entspricht der Antwort (e).\\
\\
Damit ist die Antwort (e) korrekt.\\
\\
\textit{Alternative Lösung:}\\
Es lassen sich auch Gegenbeispiele für die Antwortmöglichkeiten (a) - (d) geben.
$ m \geq 2 $ ist beliebig. Da (c) nur für $ m = 3 $ definiert ist, können wir diese Antwort direkt ausschließen.
Für $ m = 3 $ lässt sich hier aber auch ein konkretes Gegenbeispiel angeben(Warum?).
Für $ m = 4 $ würde durch (b)  die lineare Unabhängigkeit der vier Vektoren folgen.
In (a) könnte $ \textbf{b} $ linear unabhängig von $ \textbf{a}_1 $, $ \textbf{a}_2  $ und $ \textbf{a}_3 $, falls $ \mathrm{rg}\left([ \textbf{a}_1, \textbf{a}_2 , \textbf{a}_3]  \right) = 2 $ gilt.
Die Antwort (d) ist für $ m=2 $ nicht möglich.
Damit bleibt die Antwort (e) übrig.

\newpage
\subsection*{\frage{9}{3}}
Welche der folgenden Aussagen ist \textbf{nicht wahr}?
\renewcommand{\labelenumi}{(\alph{enumi})}
\begin{enumerate}
	\item 
	Eine quadratische Matrix $ A $ ist singulär genau dann, wenn sie nicht regulär ist.
	\item
	Eine quadratische Matrix $ A $ ist singulär genau dann, wenn das System ihrer Zeilenvektoren linear abhängig ist.
	
	\item
	Eine quadratische Matrix $ A $ ist singulär genau dann, wenn $ \lambda = 0 $ kein Eigenwert von ihr ist.
	
	\item
	Eine quadratische Matrix $ A $ ist singulär genau dann, wenn $\det(A) = 0 $.
\end{enumerate}
\ \\
\textbf{Lösung:}
\begin{mdframed}
	\underline{\textbf{Vorgehensweise:}}
	\renewcommand{\labelenumi}{\theenumi.}
	\begin{enumerate}
		\item Bestimme die korrekte Antwort durch das Ausschlussverfahren. 
	\end{enumerate}
\end{mdframed}

\underline{1. Bestimme die korrekte Antwort durch das Ausschlussverfahren}\\
Eine Matrix $ A $ ist regulär genau dann, wenn $ \det(A) \neq  0 $ ist.
Eine Matrix heißt singulär, wenn $ A $ nicht regulär ist. Also ist die Antwort (a) wahr.
%Die Antwort (a) ist die Definition einer singulären Matrix und damit wahr.
Insbesondere ist dann auch die Antwort (d) wahr.
Nun ist $ \det(A) $ das Produkt der Eigenwerte von $ A $.

Damit ist die Antwort (c) nicht wahr. Wenn $ \lambda = 0 $ kein Eigenwert von $ A $ ist, gilt $ \det(A) \neq 0 $.
Also wäre $ A $ gleichzeitig regulär und singulär, was nicht möglich ist.\\
\\
Damit ist die Antwort (c) korrekt.
\\
\\
Eine Matrix $ A $ heißt regulär, genau dann wenn das System ihrer Zeilenvektoren linear unabhängig ist.
Die Antwort (b) ist das Gegenteil hiervon und somit auch wahr.
\newpage
\subsection*{\frage{10}{4}}
Sei $ \{y_k\}_{k=0,1,2,...} $ eine Folge, welche die Differenzengleichung
\begin{align*}
	y_{k+1} = A y_k + B,
\end{align*}
mit $ A \in (0,1), B\in \mathbb{R} $ erfüllt. Sei $ y^\star = \frac{B}{1-A} $.\\
\\
Welche der folgenden Aussagen ist wahr?
\renewcommand{\labelenumi}{(\alph{enumi})}
\begin{enumerate}
	\item 
	$ \{y_k\}_{k=0,1,2,...}$ divergiert.
	\item
	$ \{y_k\}_{k=0,1,2,...}$ konvergiert und $ y_k > y^\star  $ für alle $ k = 0,1,2,... $.
	
	\item
	$ \{y_k\}_{k=0,1,2,...}$ konvergiert und $ y_k <y^\star  $ für alle $ k = 0,1,2,... $.
	\item
	Keine der oben gegebenen Antwortmöglichkeiten ist im Allgemeinen korrekt.
	
\end{enumerate}
\ \\
\textbf{Lösung:}
\begin{mdframed}
	\underline{\textbf{Vorgehensweise:}}
	\renewcommand{\labelenumi}{\theenumi.}
	\begin{enumerate}
		\item Bestimme die richtige Lösung mit dem Eliminationsverfahren. 
	\end{enumerate}
\end{mdframed}

\underline{1. Bestimme die richtige Lösung mit dem Eliminationsverfahren}\\
Die Differenzengleichung ist in Normalform.
Wegen $ |A| < 1 $ konvergiert die Folge gegen $ y^\star $.
Damit ist Antwort (a) falsch.
Wir setzen $ B = 0 $. Dann konvergiert $ y_k $ gegen $ 0 $ und es gilt $ y_k = A^k y_0 $.
Damit hängt das Vorzeichen von der Anfangsbedingung $ y_0 $ ab.
Für $ y_0 < 0  $ ist (b) falsch und für $ y_0 > 0 $ ist (c) falsch.
Also bleibt Antwort (d) übrig.\\
\\
Somit ist die Antwort (d) korrekt.
