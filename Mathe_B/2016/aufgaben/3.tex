\fancyhead[C]{\normalsize\textbf{$\qquad$ Teil II: Multiple-Choice}}
\section*{Aufgabe 3 (22 Punkte)}
\vspace{0.4cm}
\subsection*{\frage{1}{3}}
Der Punkt $P \ = \left( -1, \frac{5}{2} \right)$ ist ein Maximum der Funktion $f$ unter der Nebenbedingung $\varphi(x,y) = 3x +2y -2 = 0$.
Dann gilt:
\renewcommand{\labelenumi}{(\alph{enumi})}
\begin{enumerate}
\item Der Punkt $P \ = \left( -1, \frac{5}{2} \right)$ ist ein Maximum der Funktion $f$ unter der Nebenbedingung $\varphi(x,y) = 4x +2y -1 = 0$.
\item Der Punkt $P \ = \left( -1, \frac{5}{2} \right)$ ist ein Maximum der Funktion $f$ unter der Nebenbedingung $\varphi(x,y) = 5x +2y -1 = 0$.
\item Der Punkt $P \ = \left( -1, \frac{5}{2} \right)$ ist ein Minimum der Funktion $f$ unter der Nebenbedingung $\varphi(x,y) = 4x +2y -1 = 0$.
\item keine der obigen Antworten ist im Allgemeinen richtig.
\end{enumerate}
\ \\
\textbf{Lösung:}
\begin{mdframed}
\underline{\textbf{Vorgehensweise:}}
\renewcommand{\labelenumi}{\theenumi.}
\begin{enumerate}
\item Überlege dir, welche Antworten ausgeschlossen werden können.
\end{enumerate}
\end{mdframed}

\underline{1. Überlege dir, welche Antworten ausgeschlossen werden können}\\
Wegen 
\begin{align*}
\varphi\left( -1, \frac{5}{2} \right)
= 5 \cdot (-1)  + 2 \cdot \frac{5}{2} -2 = -5 + 5 -2 = -2
\end{align*}
ist die Nebenbedingung von (b) nicht erfüllt.
Bei (a) und (c) sind die Nebenbedingung erfüllt.
Jedoch sind diese Aussagen im Allgemeinen falsch, 
denn das Erfüllen der Nebenbedingung lässt nicht auf einen Extrempunkt von $f$ unter dieser schließen.\\
\\
Damit ist die Antwort (d) korrekt.

\newpage

\subsection*{\frage{2}{4}}
Die Funktion $f$ hat folgende Eigenschaften:
\renewcommand{\labelenumi}{(\roman{enumi})}
\begin{enumerate}
\item $f(x) \geq -3 $ für $x \in [0,1],$ und
\item $\int_0^1 f(x) dx = 3$.
\end{enumerate}
Dann gilt:
\renewcommand{\labelenumi}{(\alph{enumi})}
\begin{enumerate}
\item $g_1  =  \frac{1}{3} f $ ist eine Dichtefunktion auf $[0,1]$.
\item $g_2  =   f +3 $ ist eine Dichtefunktion auf $[0,1]$.
\item
$g_3  =  \frac{1}{6} f +3 $ ist eine Dichtefunktion auf $[0,1]$.
\item
$g_4  =  \frac{1}{6} f + \frac{1}{2} $ ist eine Dichtefunktion auf $[0,1]$.
\end{enumerate}
\ \\
Wir wissen, dass
\begin{align*}
g_1(x) = \frac{1}{3} f(x) \geq \frac{1}{3} (-3) = -1
\end{align*}
gilt.
Damit ist Antwort (a) falsch.\\
Für $g_2$ ist
\begin{align*}
g_2(x) = f(x) + 3 \geq -3 +3 = 0
\end{align*}
erfüllt. Jedoch ist die zweite Bedingung wegen
\begin{align*}
\int \limits_0^1 g_2(x) \ dx 
= 
\int \limits_0^1 f(x) + 3 \ dx
=
\int \limits_0^1 f(x) \ dx + \int \limits_0^1 3 \ dx
=
3 + 3 = 6
\end{align*}
nicht erfüllt.\\
Für $g_3$ ist wegen 
\begin{align*}
g_3(x) = \frac{1}{6} f + 3 \geq \frac{1}{6} \cdot (-3) +3
= -\frac{1}{2} +3 = \frac{5}{2}
\end{align*}
die erste Bedingung erfüllt.
Jedoch gilt
\begin{align*}
\int \limits_0^1 g_3(x) \ dx
= 
\frac{1}{6} \int \limits_0^1 f(x) \ dx + \int \limits_0^1 3 \ dx
= 
\frac{1}{6} \cdot 3 +  3 = \frac{1}{2} + 3 = \frac{7}{2}
\end{align*}
für die zweite Bedingung.
Also wissen wir schon durch das Ausschlußprinzip, dass Antwort (d) korrekt ist.
Die Funktion $g_4$ erfüllt beide Bedingungen:
\begin{align*}
g_4(x) &= \frac{1}{6} f(x) + \frac{1}{2}
= - \frac{1}{2} + \frac{1}{2} = 0\\
\int \limits_0^1 g_4(x) \ dx &=
\frac{1}{6} \int \limits_0^1 f(x) \ dx + \int \limits_0^1 \frac{1}{2} \ dx
= \frac{1}{2} + \frac{1}{2} = 1
\end{align*}
\ \\
Die Antwort (d) ist korrekt.

\newpage

\subsection*{\frage{3}{2}}
$A = (a_{ij})$ ist eine $4 \times 5$-Matrix vom Rang $4$.
Dann gilt:
\renewcommand{\labelenumi}{(\alph{enumi})}
\begin{enumerate}
\item 
alle $3 \times 3$ Untermatrizen von $A$ sind regulär.
\item 
alle $3 \times 3$ Untermatrizen von $A$ sind singulär.
\item 
alle $4 \times 4$ Untermatrizen von $A$ sind regulär.
\item
es existiert mindestens eine reguläre $4 \times 4$ Untermatrix von $A$.
\end{enumerate}
\ \\
\textbf{Lösung:}
\begin{mdframed}
\underline{\textbf{Vorgehensweise:}}
\renewcommand{\labelenumi}{\theenumi.}
\begin{enumerate}
\item Überlege dir, was der Rang bezüglich der Spalten bedeutet und bestimme die richtige Antwort.
\end{enumerate}
\end{mdframed}

\underline{1. Überlege dir, was der Rang bezüglich der Spalten bedeutet und bestimme die richtige Antwort}\\
Der Rang $4$ der Matrix $A$ bedeutet, dass die Matrix vier linear unabhängige Spalten besitzt.
Also existiert eine reguläre $4 \times 4$ Untermatrix in $A$.
Für (a),(b) und (c) können wir mit
\begin{align*}
A
= 
\begin{pmatrix}
1 & 1 & 0 & 0 & 0\\
0 & 0 & 1 & 0 &0 \\
0 & 0 & 0 & 1 & 0\\
0 & 0 & 0 & 0 & 1
\end{pmatrix}
\end{align*}
ein Gegenbeispiel angeben.
Wir sehen, dass die letzten vier Spalten linear unabhängig sind. 
Dementsprechend haben wir mit
\begin{align*}
\begin{pmatrix}
 1 & 0 & 0 & 0\\
 0 & 1 & 0 &0 \\
 0 & 0 & 1 & 0\\
 0 & 0 & 0 & 1
\end{pmatrix}
\end{align*}
eine reguläre $4 \times 4 $ Untermatrix.
Durch 
\begin{align*}
&\begin{pmatrix}
1 & 0 & 0 \\
0 & 1 & 0 \\
0 & 0 & 1
\end{pmatrix}\\
&\begin{pmatrix}
1 & 1 & 0 \\
0 & 0 & 1 \\
0 & 0 & 0
\end{pmatrix}
\end{align*}
haben wir eine reguläre und eine singuläre $3 \times 3 $ Untermatrix.
Eine Matrix $A$ heißt regulär, falls wir eine Matrix $A^{-1}$ finden, so dass 
\begin{align*}
A \cdot A^{-1} = A^{-1} \cdot A = I
\end{align*}
gilt.
Andernfalls ist $A$ singulär.
Analog können wir mit
\begin{align*}
\begin{pmatrix}
1 & 1 & 0 & 0\\
0 & 0 & 1 & 0  \\
0 & 0 & 0 & 1 \\
0 & 0 & 0 & 0 
\end{pmatrix}
\end{align*}
eine singuläre $4 \times 4 $ Matrix angeben.\\
\\
Also ist Antwort (d) korrekt.

\newpage

\subsection*{\frage{4}{2}}
$A$ und $B$ sind quadratische $4 \times 4$ Matrizen mit
$\det(A) = 1$ und $\det(B) = -1$.
Sei $C$ die Matrix definiert durch 
$C \ = \ A^{-1}B^2 A^2 B^{-1}$ und $\textbf{b} \in \mathbb{R}^4$.
\renewcommand{\labelenumi}{(\alph{enumi})}
\begin{enumerate}
\item 
Das System von linearen Gleichungen $C \textbf{x} = \textbf{b}$ hat unendlich viele Lösungen.
\item
Das System von linearen Gleichungen $C \textbf{x} = \textbf{b}$ hat keine Lösungen.
\item
Das System von linearen Gleichungen $C \textbf{x} = \textbf{b}$ hat eine eindeutige Lösung.
\item
Das System von linearen Gleichungen $C \textbf{x} = \textbf{b}$ hat abhängig von $A$ und $B$ unendlich viele Lösungen, keine Lösung, oder eine eindeutige Lösung.
\end{enumerate}
\ \\
\textbf{Lösung:}
\begin{mdframed}
\underline{\textbf{Vorgehensweise:}}
\renewcommand{\labelenumi}{\theenumi.}
\begin{enumerate}
\item Überlege dir, was man über das Produkt von regulären Matrizen sagen kann und bestimme die Antwort.
\end{enumerate}
\end{mdframed}

\underline{1. Überlege dir, was man über das Produkt von regulären Matrizen sagen kann und bestimme die Antwort}\\
Wir wissen, dass 
\begin{align*}
\det(A) \neq 0, \quad \det(B) \neq 0
\end{align*}
für reguläre Matrizen gilt.
Wegen
\begin{align*}
\det(A \cdot B) = \det(A) \cdot \det(B) \neq 0
\end{align*}
sind die Produkte von regulären Matrizen auch regulär.
Sei nun $\det(A) = 1$ und $\det(B) = -1$.
Mit obiger Argumentation können wir direkt sagen, dass $C$ regulär ist.
Damit folgt sofort, dass Antwort (c) korrekt ist.
Durch 
\begin{align*}
\det(C) &= \det( A^{-1}B^2 A^2 B^{-1})
= \det(A^{-1}) \det(B^2) \det(A^2) \det(B^{-1})\\
&= \frac{1}{\det(A)}  \det(B)^2 \det(A)^2 \frac{1}{\det(B)}
= 1 \cdot (-1)^2 \cdot 1^2 \cdot \frac{1}{-1}
= -1
\end{align*}
können wir dies auch von Hand nachrechnen.\\
\\
Also ist Antwort (c) korrekt.

\newpage

\subsection*{\frage{5}{3}}
Das System von $3$-dimensionalen Vektoren
$\lbrace \textbf{u}_1, \textbf{u}_2, \textbf{u}_3 \rbrace$
ist linear abhängig.
Sei $A = [\textbf{u}_1, \textbf{u}_2, \textbf{u}_3]$ die Matrix mit Spaltenvektoren $\textbf{u}_1$, $\textbf{u}_2$ und $\textbf{u}_3$. 
\renewcommand{\labelenumi}{(\alph{enumi})}
\begin{enumerate}
\item 
$A^n$ ist regulär für alle $n \in \mathbb{N}$.
\item
$A^n$ ist singulär für alle $n \in \mathbb{N}$.
\item
$A^n$ ist regulär für $n$ ungerade und singulär für $n$ gerade.
\item
$A^n$ ist singulär für $n$ ungerade und regulär für $n$ gerade.
\end{enumerate}
\ \\
\textbf{Lösung:}
\begin{mdframed}
\underline{\textbf{Vorgehensweise:}}
\renewcommand{\labelenumi}{\theenumi.}
\begin{enumerate}
\item Überlege dir, ob $A$ regulär oder singulär ist und löse damit die Aufgabe.
\end{enumerate}
\end{mdframed}

\underline{1. Überlege dir, ob $A$ regulär oder singulär ist und löse damit die Aufgabe}\\
Eine quadratische Matrix ist genau dann regulär, wenn die Spalten linear unabhängig sind.
Da die Spalten von $A$ linear abhängig sind, ist $A$ singulär.
Damit gilt auch $\det(A) = 0$.
Aus dem Zusammenhang
\begin{align*}
\det\left(A^n\right) = \det(A)^n = 0^n = 0
\end{align*}
erhalten wir die Singularität von $A^n$ für alle $n \in \mathbb{N}$.

\newpage

\subsection*{\frage{6}{2}}
Für das System von linearen Gleichungen $A \textbf{x} = \textbf{b}$ gilt:
$\text{rg}(A) = 4$ und $\text{rg}(A,\textbf{b}) = 4$, wobei $A$ eine $5 \times 6$ Matrix ist.
\renewcommand{\labelenumi}{(\alph{enumi})}
\begin{enumerate}
\item 
Das System hat keine Lösung.
\item
Das System hat genau eine Lösung.
\item
Das System hat unendlich viele Lösungen und der Lösungsraum hat Dimension $1$.
\item
Das System hat unendlich viele Lösungen und der Lösungsraum hat Dimension $2$.
\end{enumerate}
\ \\
\textbf{Lösung:}
\begin{mdframed}
\underline{\textbf{Vorgehensweise:}}
\renewcommand{\labelenumi}{\theenumi.}
\begin{enumerate}
\item Überlege dir, wann ein lineares Gleichungssystem lösbar ist.
\item Bestimme die korrekte Antwort.
\end{enumerate}
\end{mdframed}

\underline{1. Überlege dir, wann ein lineares Gleichungssystem lösbar ist}\\
Ein lineares Gleichungssystem $ A x  = b$ ist genau dann lösbar, wenn
\begin{align*}
\mathrm{rg}(A) = \mathrm{(A| b) }
\end{align*}
gilt.
Dies ist in der Aufgabenstellung gegeben.
Dementsprechend können wir Antwort (a) ausschließen.\\
\\

\underline{2. Bestimme die korrekte Antwort}\\
Falls $Ax = b$ lösbar ist, gilt
\begin{align*}
\dim L = n - \mathrm{rg}(A)
\end{align*}
für eine $m \times n $ Matrix.
Hierbei bezeichnen wir mit $L$ die Lösungsmenge des linearen Gleichungssystems.
In unserem Fall ist $n = 6$ und $\mathrm{rg}(A)  =4$:
Wegen
\begin{align*}
\dim L = 6 - 4 = 2
\end{align*}
ist Antwort (d) korrekt.


\newpage
\subsection*{\frage{7}{3}}
$A$ ist eine quadratische Matrix und $\lambda = 0 $ einer ihrer Eigenwerte.

\renewcommand{\labelenumi}{(\alph{enumi})}
\begin{enumerate}
\item 
Da aus $A \textbf{x} = \textbf{0}$ folgt, dass $\textbf{x } = 0$, hat $A$ keinen zum Eigenwert $ \lambda = 0$ gehörenden Eigenvektor.
\item
$A$ hat einen eindeutigen zum Eigenwert $\lambda = 0$ gehörenden Eigenvektor.
\item
$A$ hat unendlich viele zum Eigenwert $\lambda = 0$ gehörende Eigenvektoren.
\item
Wie viele Eigenvektoren zum Eigenwert $\lambda = 0$ existieren, hängt von der Matrix $A$ ab.
\end{enumerate}
\ \\
\textbf{Lösung:}
\begin{mdframed}
\underline{\textbf{Vorgehensweise:}}
\renewcommand{\labelenumi}{\theenumi.}
\begin{enumerate}
\item Rufe dir die Definition des Eigenwerts in Erinnerung.
\item
Bestimme die korrekte Antwort.
\end{enumerate}
\end{mdframed}

\underline{1. Rufe dir die Definition des Eigenwerts in Erinnerung}\\
Ein Vektor $v \neq 0$ heißt Eigenvektor zum Eigenwert $\lambda$, falls
\begin{align*}
Av = \lambda v
\end{align*}
erfüllt ist.
Dies können wir durch
\begin{align*}
Av = \lambda v 
\Leftrightarrow
Av - \lambda v = (A - \lambda I) v = 0
\end{align*}
umformen.
Hierbei ist $I$ die Einheitsmatrix.
Wegen $v \neq 0 $ sind auch beliebige Vielfache von $v$ wieder Eigenvektoren zum Eigenwert $\lambda$.
Dies können wir durch
\begin{align*}
A (\alpha v ) 
= 
\alpha Av 
= \alpha \lambda v
= \lambda (\alpha v) 
\end{align*}
für $\alpha \in \mathbb{R} \setminus \lbrace 0 \rbrace$ veranschaulichen.
Damit können wir allgemein sagen, dass es zu einem Eigenwert $\lambda$ immer unendlich viele Eigenvektoren gibt.\\
\\

\underline{2. Bestimme die korrekte Antwort}\\
Im Prinzip haben wir die korrekte Antwort (c) im ersten Abschnitt gefunden.
Ein anderer Weg ist es, die Determinante zu betrachten.
Da $\lambda = 0 $ ein Eigenwert ist, gilt
\begin{align*}
\det(A) = 0.
\end{align*}
Damit ist $A$ singulär und das System
\begin{align*}
A x = 0
\end{align*}
besitzt unendliche viele Lösungen.
Diese Lösungen sind gerade die Eigenvektoren.\\
\\
Damit ist Antwort (c) korrekt.

\newpage

\subsection*{\frage{8}{3}}
Ein dynamisches Model für die Variablen $\textbf{u}_t = (x_t , y_t)^\top \neq \textbf{0}$ erfüllt die Gleichung
$\textbf{u}_{t+1} = A \textbf{u}_t$ für
\begin{align*}
A = 
\begin{pmatrix}
1 & -1\\
2 & 1
\end{pmatrix}
\end{align*}
für $t = 0,1,\dots$ \\
\\
Die Gleichgewichtsbedingung $\textbf{u}_{t+1} = \lambda \textbf{u}_t$ für alle $t = 0,1,\dots$ mit $\lambda \in \mathbb{R}$
\renewcommand{\labelenumi}{(\alph{enumi})}
\begin{enumerate}
\item 
kann nur erfüllt sein für $\lambda = 0$.
\item
kann nur erfüllt sein für $\lambda = 1$.
\item
kann nur erfüllt sein für $\lambda = 2$.
\item
kann nie erfüllt sein.
\end{enumerate}
\ \\
\textbf{Lösung:}
\begin{mdframed}
\underline{\textbf{Vorgehensweise:}}
\renewcommand{\labelenumi}{\theenumi.}
\begin{enumerate}
\item Überlege dir, wie man die Gleichgewichtsbedingung auf Eigenwerte zurückführen kann.
\item
Bestimme die korrekte Antwort.
\end{enumerate}
\end{mdframed}

\underline{1. Überlege dir, wie man die Gleichgewichtsbedingung auf Eigenwerte zurückführen kann}\\
Es ist
\begin{align*}
\textbf{u}_{t+1}= A \textbf{u}_t
\end{align*}
gegeben.
Damit entspricht die Gleichgewichtsbedingung
\begin{align*}
\textbf{u}_{t+1} = \lambda \textbf{u}_t 
\Rightarrow
A \textbf{u}_t = \lambda \textbf{u}_t 
\end{align*}
dem Eigenwertproblem.
\\
\\
\underline{2. Bestimme die korrekte Antwort}\\
Um die korrekte Antwort zu finden, untersuchen wir die Eigenwerte der Matrix $A$.
Es gilt:
\begin{align*}
\det(A - \lambda I) 
= 
\left| 
\begin{pmatrix}
1 - \lambda & -1 \\
2 	& 1 - \lambda
\end{pmatrix}
\right|
=
(1- \lambda)^2  + 2 
\end{align*}
Wir sehen nun, dass
\begin{align*}
\det(A - \lambda I) > 0
\end{align*}
für alle $\lambda \in \mathbb{R}$.
Damit besitzt $A$ keine reellen Eigenwerte.
\\
\\
Also ist Antwort (d) korrekt.