\vspace{1cm}
\fancyhead[C]{\normalsize\textbf{$\qquad$ Teil I: Offene Aufgaben}}
\renewcommand{\labelenumi}{\theenumi.}
\section*{Aufgabe 1 (25 Punkte)}
\vspace{0.4cm}
\subsection*{\aufgabe{a}{6}}
Ein Konsument \textit{maximiert} seine Nutzenfunktion $u(c_1,c_2)$ in den Einheiten $c_1$ und $c_2$
der Güter 1 und 2 definiert durch:
\begin{equation*}
u \ : \ \mathbb{R}_+ \times \mathbb{R}_+ \to \mathbb{R},
\quad (c_1,c_2) \mapsto u(c_1,c_2) = c_1^{0.6} c_2^{0.4}
\end{equation*}
über die Wahl des Konsumbündels $(c_1^\star,c_2^\star)$.
Die Preise der Güter 1 und 2 sind $p_1 \ = \ 3$ beziehungsweise $p_2 \ = \ 4$,
und das Budget, welches \textit{vollständig} genutzt wird, beträgt $e = 15$.
\\
Bestimmen Sie die stationären Punkte des Maximierungsproblems des Konsumenten, 
d.h., Kandidaten für das optimale Konsumbündel $(c_1^\star,c_2^\star)$.
\\
\\
\textbf{Hinweis:}\\ 
Eine Abklärung, ob es sich bei den stationären Punkten tatsächlich um Maxima handelt, wird nicht verlangt.
\\
\\
\textbf{Lösung:}
\begin{mdframed}
\underline{\textbf{Vorgehensweise:}}
\renewcommand{\labelenumi}{\theenumi.}
\begin{enumerate}
\item Formuliere das Optimierungsproblem mathematisch.
\item Löse die Aufgabe mithilfe der Lagrange Methode.
\item Alternative Rechenmethode.
\end{enumerate}
\end{mdframed}
\underline{1. Formuliere das Optimierungsproblem mathematisch}\\
Unser Ziel ist es, die Funktion
\begin{equation*}
u(c_1,c_2) = c_1^{0.6}c_2^{0.4}
\end{equation*}
unter der Nebenbedingung
\begin{equation*}
p_1 c_1 + p_2 c_2 = e \Rightarrow
\varphi(c_1,c_2) = p_1 c_1 + p_2 c_2 - e = 3 c_1 + 4 c_2 - 15 = 0
\end{equation*}
zu maximieren.
\\
\\
\underline{2. Löse die Aufgabe mithilfe der Lagrange Methode}\\
Zunächst bietet es sich bei Extremwertaufgaben unter Nebenbedingungen an, die Lagrange Methode zu verwenden
Die Nebenbedingung
\begin{equation*}
\varphi(c_1,c_2) = 3 c_1 + 4 c_2 - 15 = 0
\end{equation*}
besitzt nur lineare Terme.
Das heißt, wir können diese explizit nach $c_1$ bzw. $c_2$ auflösen.
Durch diese Umformung erhalten wir aus $u$ eine Funktion, welche von einer Variablen abhängt.
Beide Methoden sind in etwa gleich schwer, jedoch funktioniert die zweite Methode nicht immer.
\\
\\ 
Wir definieren durch
\begin{equation*}
F(c_1,c_2,\lambda) = u(c_1,c_2) + \lambda \varphi(c_1,c_2)
\end{equation*}
die Lagrange-Funktion.
Die notwendige Bedingung für Extremstellen unter unserer Nebenbedingung ist,
dass die partiellen Ableitungen der Lagrange-Funktion gleichzeitig Null sind.
Durch
\begin{align*}
F_{c_1}(c_1,c_2,\lambda) &= 0.6 c_1^{-0.4} c_2^{0.4} + 3 \lambda\\
F_{c_2}(c_1,c_2, \lambda) &= 0.4 c_1^{0.6} c_2^{-0.6} + 4 \lambda \\
F_{\lambda}(c_1,c_2,\lambda) &=\varphi(c_1,c_2) = 3 c_1 + 4 c_2 -15 
\end{align*}
erhalten wir die partiellen Ableitungen.
Wir müssen nun das Gleichungssystem
\newcounter{eq}
\setcounter{eq}{0}
\begin{align}
0.6 c_1^{-0.4} c_2^{0.4} + 3 \lambda = 0  \tag{\text{\Roman{eq}}}\stepcounter{eq}\\
0.4 c_1^{0.6} c_2^{-0.6} + 4 \lambda = 0  \tag{\text{\Roman{eq}}}\stepcounter{eq}\\
3 c_1 + 4 c_2 -15 = 0 \tag{\text{\Roman{eq}}}\stepcounter{eq}
\end{align}
lösen. 
Die Gleichungen (I) und (II) liefern uns durch
\begin{align}
0.6 c_1^{-0.4} c_2^{0.4} + 3 \lambda &= 0 
\Leftrightarrow 
3 \lambda = -0.6 c_1^{-0.4} c_2^{0.4} 
\Leftrightarrow
3 \lambda= -0.6 \frac{c_2^ {0.4}}{c_1^{0.4}}
\Leftrightarrow
3 \lambda= -0.6 \left( \frac{c_2}{c_1} \right)^{0.4}
\tag{\text{\Roman{eq}}}\stepcounter{eq}\\
0.4 c_1^{0.6} c_2^{-0.6} + 4 \lambda &= 0 
\Leftrightarrow 
4 \lambda = - 0.4 c_1^{0.6} c_2^{-0.6} 
\Leftrightarrow
4 \lambda = -0.4 = \frac{c_1^{0.6}}{c_2^{0.6}}
\Leftrightarrow
4 \lambda=- 0.4 \left( \frac{c_1}{c_2} \right)^{0.6}
\tag{\text{\Roman{eq}}}\stepcounter{eq}    
\end{align}
zwei neue Gleichungen.
Wegen $c_1, c_2 > 0 $ sehen wir, dass $\lambda \neq 0 $ gilt.
Also können wir (IV) durch (V) teilen.
Damit gilt
\begin{equation*}
\begin{split}
\frac{3 \lambda }{4 \lambda} 
&= \frac{-0.6 \left( \frac{c_2}{c_1} \right)^{0.4}}{- 0.4 \left( \frac{c_1}{c_2} \right)^{0.6}}
= \frac{3}{2} \cdot \frac{c_2^{0.4} c_2^{0.6}}{c_1^{0.6} c_1^{0.4}}
= \frac{3}{2} \frac{c_2}{c_1}\\
\Rightarrow
\frac{3 \lambda }{4 \lambda} &= \frac{3}{2} \frac{c_2}{c_1}
\Leftrightarrow
\frac{3  }{4 } = \frac{3}{2} \frac{c_2}{c_1}
\Leftrightarrow
6 c_1 = 12 c_2
\Leftrightarrow
c_1 = 2 c_2,
\end{split}
\end{equation*}
womit durch Gleichung (III)
\begin{equation*}
\begin{split}
3 c_1 + 4 c_2 - 15 &= 3 (2c_2) + 4 c_2 -15 = 10 c_2  -15 = 0 \\
\Leftrightarrow
10 c_2 &= 15 
\Leftrightarrow
c_2 = \frac{15}{10} = \frac{3}{2} = 1.5  
\end{split}
\end{equation*}
folgt.
Durch $c_1 = 2 c_2 $ wissen wir auch $c_1 = 3$.
Damit ist 
\begin{align*}
(c_1^\star, c_2^\star) = ( 3, 1.5)
\end{align*}
der einzige Kandidat für eine Extremstelle unter der Nebenbedingung $\varphi$.\\
\\
\underline{3. Alternative Rechenmethode}\\
Für die alternative Methode formen wir die Nebenbedingung $\varphi$ nach einer Variablen um
und substituieren diese in $u$.
Durch die Umformung 
\begin{equation*}
\begin{split}
\varphi(c_1,c_2) &= p_1 c_1 + p_2 c_2 -e = 3 c_1 + 4 c_2 - 15 = 0\\
\Leftrightarrow
3 c_1 &= 15 - 4 c_2 \\
\Leftrightarrow
c_1 &= 5 - \frac{4}{3} c_2
\end{split}
\end{equation*}
können wir $c_1 $ durch $c_2$ darstellen.
Damit erhalten wir mit 
\begin{equation}
U(c_2) = u( c_1 , c_2) 
= u\left( 5 - \frac{4}{3}c_2,c_2 \right) 
=\left( 5 - \frac{4}{3}c_2 \right)^{0.6} c_2^{0.4}
\end{equation}
eine Funktion, welche nur von der Variablen $c_2$ abhängt.
Wir müssen nun die Extrempunkte von $U$ finden. 
Hierfür lösen wir $U^\prime(c_2) = 0 $.
Zunächst gilt
\begin{equation*}
\begin{split}
U^\prime(c_2)
&= 0.6 \left( 5 - \frac{4}{3} c_2 \right)^{-0.4} \left( -\frac{4}{3} \right) c_2^{0.4}
+ 0.4 \left( 5 - \frac{4}{3} c_2 \right)^{0.6} c_2^{-0.6}\\
&= -\frac{4}{5} \left( 5 - \frac{4}{3} c_2 \right)^{-0.4}  c_2^{0.4}
+ \frac{2}{5} \left( 5 - \frac{4}{3} c_2 \right)^{0.6} c_2^{-0.6}\\
&= - \frac{4}{5} \left( \frac{c_2}{5 - \frac{4}{3} c_2} \right)^{0.4}
+ \frac{2}{5} \left( \frac{5 - \frac{4}{3} c_2}{c_2}  \right)^{0.6}\\
&= - \frac{4}{5} \left( \frac{c_2}{5 - \frac{4}{3} c_2} \right)^{0.4}
+ \frac{2}{5} \left( \frac{c_2}{5 - \frac{4}{3} c_2} \right)^{0.4} \frac{5 - \frac{4}{3} c_2}{c_2}\\
&=
\left( \frac{c_2}{5 - \frac{4}{3} c_2} \right)^{0.4} \left( -\frac{4}{5} + \frac{2}{5} \frac{5 - \frac{4}{3} c_2}{c_2} \right)
\end{split}
\end{equation*}
mithilfe der Produkt und Kettenregel.
Die weiteren Umformungen helfen uns $U^\prime(c_2) = 0$ zu lösen.
Wegen $c_2 \neq 0 $ erhalten wir durch
\begin{equation*}
\begin{split}
U^\prime(c_2) = 0 
&\Leftrightarrow
\left( \frac{c_2}{5 - \frac{4}{3} c_2} \right)^{0.4} \left( -\frac{4}{5} + \frac{2}{5} \frac{5 - \frac{4}{3} c_2}{c_2} \right) = 0\\
&\Leftrightarrow
\left( -\frac{4}{5} + \frac{2}{5} \frac{5 - \frac{4}{3} c_2}{c_2} \right) = 0\\
&\Leftrightarrow
-\frac{4}{5} c_2 \frac{2}{5}\left(5 -\frac{4}{3} c_2 \right) = 0 \\
&\Leftrightarrow
\left(- \frac{4}{5} - \frac{8}{15} \right) c_2 = -2\\
&\Leftrightarrow
\frac{20}{15} c_2 =  2 
\Leftrightarrow
c_2 = 1.5
\end{split}
\end{equation*}
das passende $c_2$.
Mit $c_1 = 5 - \frac{4}{3} c_2$ folgt auch $c_1 = 3$.
Damit ist auch bei dieser Methode
\begin{align*}
(c_1^\star , c_2^\star) = (3, 1.5)
\end{align*}
der einzige Kandidat für eine Extremstelle unter der Nebenbedingung $\varphi$.
\\
\\
Der Kandidat für das optimale Konsumbündel ist $(c_1^\star , c_2^\star) = (3, 1.5)$.

\newpage

\subsection*{\aufgabe{b}{9}}
Sei $f \ : \ \mathbb{R} \times (-5, \infty) \to \mathbb{R}$ eine Funktion zweier reeller Variablen definiert durch:
\begin{equation*}
f(x,y)\ = \ x^2 + 3 x y + 16 \ln(y+5).
\end{equation*}
Sei $g \ : \ R_f \to \mathbb{R}$ eine stetig differenzierbare Funktion einer reellen Variablen mit $g^\prime(x) > 0$
für alle $x \in R_f$, wobei $R_f$ der Wertebereich von $f$ ist.
Schließlich sei die Komposition $h$ gegeben als
\begin{equation*}
h \ : \ D_f \to \mathbb{R}, \quad (x,y) \mapsto h(x,y) = g(f(x,y)),
\end{equation*}
$D_f$ ist dabei das Definitionsgebiet von $f$.
\\
\\
Untersuchen Sie die Funktion $h$ auf stationäre Punkte, d.h., Maxima, Minima und Sattelpunkte.
\\
\\
\textbf{Hinweis:} \\
Dank der Eigenschaft von $g$ ist es möglich, das Problem in handhabbare Form zu bringen.\\

\textbf{Lösung:}
\begin{mdframed}
\underline{\textbf{Vorgehensweise:}}
\renewcommand{\labelenumi}{\theenumi.}
\begin{enumerate}
\item Verwende die gegebenen Eigenschaften, um das Problem zu strukturieren.
\item Finde die stationären Punkte von $h$.
\item Bestimme die Art des stationären Punkts.
\end{enumerate}
\end{mdframed}

\underline{1. Verwende die gegebenen Eigenschaften, um das Problem zu strukturieren}\\
Nach Voraussetzung ist $g^\prime(x) > 0$ für alle $x \in R_f$.
Die \textit{notwendige} Bedingung für stationäre Punkte ist, dass die partiellen Ableitungen von $h$ gleichzeitig Null sind.
Mathematisch können wir dies durch
\begin{align*}
h_x(x, y) = 0 \qquad 
h_y(x, y) = 0
\end{align*}
ausdrücken.
Die Punkte $(x, y)$, welche diese Bedingung erfüllen, nennen wir stationäre Punkte.
Wegen $g^\prime(x) > 0 $ gilt
\begin{align*}
h_x(x, y) &=  g^\prime(f(x,y)) \cdot f_x(x,y) = 0
\Leftrightarrow
f_x(x,y) = 0\\
h_y(x,y) &= g^\prime(f(x,y)) \cdot f_y(x,y) = 0 
\Leftrightarrow
f_y(x,y) = 0.
\end{align*}
Dies kann man sich mithilfe der Kettenregel überlegen.
Die Kettenregel findet Anwendung bei verketteten Funktionen sprich, wenn eine Funktion eine äußere und eine innere Funktion hat.
Dies ist bei 
\begin{align*}
h(x,y) = g(f(x,y)) 
\end{align*}
gegeben. Hierbei ist $g$ die äußere und $f$ die innere Funktion.
Die Kettenregel funktioniert hier wie im eindimensionalen Fall.
Wenn wir partiell nach $x$ ableiten, behandeln wir $y$ als eine Konstante.
Damit genügt es die stationären Punkte von $f$ zu bestimmen, da diese mit denen von $h$ übereinstimmen.
Wegen $g^\prime(x) > 0 $ können $h_x$ und $h_y$ nur Null ergeben, wenn $f_x$ bzw. $f_y$ Null ergeben.\\
\newpage
\underline{2. Finde die stationären Punkte von $h$}\\
Zunächst bestimmen wir durch
\begin{equation*}
\begin{split}
f_x(x,y) &= 2x + 3y \\
f_y(x,y) &= 3x + 16 (y+5)^{-1} = 3x + \frac{16}{y+5}
\end{split}
\end{equation*}
die ersten partiellen Ableitungen von $f$.
Diese müssen nun die \textit{notwendigen} Bedingungen
\begin{equation*}
f_x(x,y) = 0 \ \text{und} \ f_y(x,y) = 0
\end{equation*}
erfüllen. Dies führt zu dem Gleichungssystem
\begin{equation*}
\begin{split}
f_x(x,y) &= 2x + 3y = 0\\
f_y(x,y) &=  3x + \frac{16}{y+5} = 0.
\end{split}
\end{equation*}
Durch Umformen der ersten Gleichung erhalten wir mit
\begin{equation*}
2x + 3y = 0 
\Leftrightarrow
3y = -2x
\Leftrightarrow
y = -\frac{2}{3} x 
\end{equation*}
eine Darstellung für $y$.
Wir lösen die Gleichung durch Einsetzen in die Mitternachtsformel Mitternachtsformel:
\begin{equation*}
\begin{split}
&3x + \frac{16}{y+5} = 3x + \frac{16}{-\frac{2}{3}x+5} = 0\\
\Leftrightarrow
&3x \left( -\frac{2}{3}x+5 \right) + 16 = -2 x^2 + 15 x + 16 = 0\\
\Leftrightarrow
&x_{\nicefrac{1}{2}}= \frac{- 15 \pm \sqrt{15^2 + 4 \cdot 2 \cdot 16}}{- 2 \cdot 2} 
= \frac{15 \pm \sqrt{353}}{4}\\
\Rightarrow
&x_1 = \frac{15 + \sqrt{353}}{4}, \quad
x_2 = \frac{15 - \sqrt{353}}{4}
\end{split}
\end{equation*}
Wir setzen nun $x_1$ und $x_2$ in die Gleichung $y = -\frac{2}{3} x$ ein und bestimmen so die möglichen stationären Punkte:
\begin{equation*}
\begin{split}
x_1  = \frac{15 - \sqrt{353}}{4}
&\Rightarrow
y_1 = -\frac{2}{3} x_1 = \frac{-15 + \sqrt{353}}{6}
\Rightarrow
P_1 = \left( \frac{15 - \sqrt{353}}{4}, \frac{-15 + \sqrt{353}}{6} \right)
\\
x_2  = \frac{15 + \sqrt{353}}{4}
&\Rightarrow
y_2 = -\frac{2}{3} x_2 = \frac{-15 - \sqrt{353}}{6}
\Rightarrow
P_2 = \left( \frac{15 + \sqrt{353}}{4}, \frac{-15 - \sqrt{353}}{6} \right).
\end{split}
\end{equation*}
Wir können den Punkt $P_2$ direkt ausschließen, da dieser sich nicht in dem Definitionsgebiet von $f$ befindet.
Dies können wir an
\begin{align*}
\frac{-15 - \sqrt{353}}{6} < -5
\end{align*}
erkennen.
Unser stationärer Punkt ist also durch
\begin{align*}
P_1 = \left( \frac{15 - \sqrt{353}}{4}, \frac{-15 + \sqrt{353}}{6} \right)
\end{align*}
gegeben.\\

\newpage

\underline{3. Bestimme die Art des stationären Punkts}\\
Auch hier reicht es wieder aus $f$ zu untersuchen.
Zunächst betrachten wir die \textit{hinreichenden} Bedingungen für Maxima, Minima und Sattelpunkte.\\

\textbf{Hinreichende Bedingungen}\\
Sei $f : \mathbb{R}^2 \to \mathbb{R}$ gegeben und $(x_0,y_0) $ ein kritischer Punkt.
Dann ist
\renewcommand{\labelenumi}{\theenumi.}
\begin{enumerate}
\item
\index{Minimum}
$f(x_0,y_0)$ ein \textit{Minimum}, falls
\begin{align*}
f_{xx}(x_0,y_0) > 0, \ f_{yy}(x_0,y_0) > 0, \ \text{und} \ f_{xx}(x_0,y_0)f_{yy}(x_0,y_0) - (f_{xy}(x_0,y_0))^2  > 0.
\end{align*}

\item
\index{Maximum}
$f(x_0,y_0)$ ein \textit{Maximum}, falls
\begin{align*}
f_{xx}(x_0,y_0) < 0, \ f_{yy}(x_0,y_0) < 0, \ \text{und} \ f_{xx}(x_0,y_0)f_{yy}(x_0,y_0) - (f_{xy}(x_0,y_0))^2  > 0.
\end{align*}

\item
\index{Sattelpunkt}
$f(x_0,y_0)$ ein \textit{Sattelpunkt}, falls
\begin{align*}
f_{xx}(x_0,y_0)f_{yy}(x_0,y_0) - (f_{xy}(x_0,y_0))^2  < 0.
\end{align*}
\end{enumerate}

Wir bestimmen nun die zweiten partiellen Ableitungen von $f$.
Es gilt
\begin{equation*}
\begin{split}
f_{xx}(x,y)  &= 
\frac{\partial}{\partial \mathrm{x}} (2x + 3y) = 2
 \\
f_{xy}(x,y)  &=
= \frac{\partial}{\partial \mathrm{y}} (2x + 3y) 
= 3 \\
f_{yy}(x,y) &= 
\frac{\partial}{\partial \mathrm{y}} \left(3x + \frac{16}{y+5} \right)
=
\frac{\partial}{\partial \mathrm{y}} \left(3x + 16 \cdot (y+5)^{-1} \right)
= 16 \cdot (-1)\cdot(y+5)^{-2}
- \frac{16}{(y+5)^2}.
\end{split}
\end{equation*}
Wir sehen, dass $f_{xx}(x,y) > 0$ für alle $(x,y) \in D_f$ gilt. 
Es gilt auch $f_{yy}(x,y) < 0 $ für alle $(x,y) \in D_f$.
Damit ist 
\begin{equation*}
f_{xx}(x,y) f_{yy} (x,y) - (f_{xy}(x,y))^2 < 0
\end{equation*}
für alle $(x,y) \in D_f$ gegeben.
Somit ist $P_1 $ ein Sattelpunkt.

\newpage
\subsection*{\aufgabe{c}{6}}
Ein Investment Fonds generiert innerhalb der Zeitspanne von $t=0$ zu $T=12$ einen
stetigen Cashflow von $B(t) = 10 t + 5$.
Die Verzinsung erfolgt kontinuierlich zum Zinssatz $i = 5 \%$.
\\
\\
Bestimmen Sie den Nettobarwert $PV(0)$ zum Zeitpunkt $t = 0$ \textit{aller} Zahlungsströme,
die der Investment Fonds zwischen den Zeitpunkten $t = 0$ und $T = 12$ generiert.\\
\\
\textbf{Lösung:}
\begin{mdframed}
\underline{\textbf{Vorgehensweise:}}
\begin{enumerate}
\item Formuliere das Problem mathematisch.
\item Bestimme die nötige Stammfunktion mithilfe partieller Integration.
\item Berechne den Wert des Integrals.
\end{enumerate}
\end{mdframed}

\underline{1. Formuliere das Problem mathematisch}\\
Um den Nettobarwert $PV(0)$ zu bestimmen, müssen wir das Integral
\begin{equation*}
PV(0)
=
\int \limits_0^T B(t) e^{-i t} \ dt
=
\int \limits_0^{12} (10 t + 5) e^{-0.05 t} \ dt
\end{equation*}
berechnen.\\
\\
\underline{2. Bestimme die nötige Stammfunktion mithilfe partieller Integration}\\
%Zunächst betrachten wir die Produktregel und formen diese durch
%\begin{equation*}
%(u(t) v(t))^\prime = u^\prime(t) \cdot v(t) + u(t) \cdot v^\prime(t)
%\Leftrightarrow
%u(t) v^\prime(t) = (u(t)  v(t))^\prime - u^\prime(t) v(t)
%\end{equation*}
%um. Nun erhalten wir mit
%\begin{equation*}
%\int u(t) v^\prime(t) \ dt 
%= 
%\int (u(t)  v(t))^\prime \ dt - \int u^\prime(t) v(t) \ dt
%=
%u(t) v(t) - \int u^\prime(t) v(t) \ dt
%\end{equation*}
%die Regel für partielle Integration für unbestimmte Integrale.
%Für bestimmte Integrale geht die Umformung genauso.
%Dies war eine kurze Herleitung der partiellen Integration.
%Nun wenden wir diese Regel an.
Wir wählen
\begin{equation*}
\begin{split}
u^\prime(t)  =  e^{-0.05 t},  \quad
v(t)  = 10 t +5
\end{split}
\end{equation*}
für die partielle Integration.
Dies macht Sinn, denn $v$ wird nach einmal Ableiten konstant.
Es gilt 
\begin{equation*}
\begin{split}
u(t) = \frac{1}{-0.05} e^{-0.05t} 
=- \frac{100}{5} e^{-0.05t} = -20 e^{-0.05t}, \quad
v^\prime(t) = 10.
\end{split}
\end{equation*}
Mithilfe der Regel für partielle Integration erhalten wir
\begin{equation*}
\begin{split}
\int (10 t + 5) e^{-0.05 t} \ dt
&= 
(10t +5) (-20) e^{-0.05 t} ) - \int  10 (-20) e^{-0.05t} \ dt\\
&= -( 200 t + 100) e^{-0.05 t} + 200 \int e^{-0.05 t} \ dt\\
&=-(200 t + 100)  e^{-0.05t } + 200 (-20) e^{-0.05 t}  + C\\
&= -(200 t + 100) e^{-0.05 t } -4000 e^{-0.05t} + C \\
&= -(200 t +4100) e^{-0.05 t} + C
\end{split}
\end{equation*}
als Stammfunktion.\\

\newpage
\underline{3. Berechne den Wert des Integrals}\\
Da wir nun die Stammfunktion kennen, können wir das Integral direkt durch
\begin{equation*}
\int \limits_0^{12} (10t + 5) e^{-0.05 t} \ dt
=  \left[ -(200 t +4100) e^{-0.05 t} \right]_0^{12}
= -6500 e^{-0.6 }+4100 
\approx 532.70
\end{equation*}
berechnen.\\
\\
Der Nettobarwert beträgt ungefähr $532.70$.

\newpage
\subsection*{\aufgabe{d}{4}}
Berechnen Sie das unbestimmte Integral
\begin{equation*}
\int \frac{\sin(\ln(x)) \ \cos(\ln(x))}{x} dx.
\end{equation*}
\textbf{Lösung:}
\begin{mdframed}
\underline{\textbf{Vorgehensweise:}}
\begin{enumerate}
\item Finde eine geeignete Substitution und bestimme die Stammfunktion.
\end{enumerate}
\end{mdframed}

\underline{1. Finde eine geeignete Substitution und bestimme die Stammfunktion}\\
Es gilt
\begin{equation*}
\frac{\text{d}}{\text{d} x}
\sin(\ln(x))
= \frac{\cos(\ln(x))}{x}
\end{equation*}
mit der Kettenregel.
Demnach haben wir durch
\begin{equation*}
\int \frac{\sin(\ln(x)) \ \cos(\ln(x))}{x} dx
=
\int \sin(\ln(x)) \cdot \frac{ \ \cos(\ln(x))}{x} dx
=
\int \sin(\ln(x)) \ \frac{\text{d}}{\text{d} x} \sin(\ln(x)) dx
\end{equation*}
die geeignete Substitution 
\begin{equation*}
u = \sin(\ln(x))
\end{equation*}
gefunden.
Durch 
\begin{equation*}
\frac{\text{d} u }{\text{d} x} = \frac{\cos(\ln(x))}{x}
\Rightarrow
\text{d}  u =  \frac{\cos(\ln(x))}{x} \ \text{d} x  
\end{equation*}
erhalten wir mit
\begin{equation*}
\int \frac{\sin(\ln(x)) \ \cos(\ln(x))}{x} dx 
=
\int u \ du
= \frac{1}{2} u^2 + C
= \frac{1}{2} (\sin(\ln(x)))^2 + C
\end{equation*}
die gesuchte Stammfunktion.\\ \\
Eine andere Möglichkeit ist, zweimal zu substituieren.
Zuerst substituieren wir $v = \ln(x)$.
Mit 
\begin{equation*}
\frac{\text{d} v}{\text{d} x}
= \frac{1}{x}
\Rightarrow
\text{d} v = \frac{1}{x} \ \text{d} x
\end{equation*}
erhalten wir
\begin{equation*}
\int \frac{\sin(\ln(x)) \ \cos(\ln(x))}{x} \ dx
= \int \sin(v) \cos(v) \ dv.
\end{equation*}
Die zweite Substitution ist $w = \sin(v)$.
Für diese gilt
\begin{equation*}
\frac{\text{d} w}{\text{d} v} = \cos(v)
\Rightarrow
\text{d} w = \cos(v) \ \text{d} v
\end{equation*}
und es folgt
\begin{equation*}
\begin{split}
\int \frac{\sin(\ln(x)) \ \cos(\ln(x))}{x} \ dx
&= \int \sin(v) \cos(v) \ dv
= \int w \ dw\\
&= \frac{1}{2} w^2 + C
= \frac{1}{2} (\sin(v))^2
= \frac{1}{2} (\sin(\ln(x)))^2 +C
\end{split}
\end{equation*}
durch zweimaliges Zurücksubstituieren.\\
\\
Die gesuchte Stammfunktion ist
\begin{align*}
\frac{1}{2} (\sin(\ln(x)))^2 +C.
\end{align*}