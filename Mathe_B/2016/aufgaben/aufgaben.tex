%\newcommand{\ein}[2]{(#1) (#2 Punkte)}


\begin{Large}
\textbf{Teil I: Offene Aufgaben (50 Punkte)}
\end{Large}
\\
\\
\\
\textbf{Allgemeine Anweisungen für offene Fragen:}
\\
\renewcommand{\labelenumi}{(\roman{enumi})}
\begin{enumerate}
\item
Ihre Antworten müssen alle Rechenschritte enthalten,
diese müssen klar ersichtlich sein.
Verwendung korrekter mathematischer Notation wird erwartet
und fliesst in die Bewertung ein.

\item
Ihre Antworten zu den jeweiligen Teilaufgaben müssen in den dafür vorgesehenen Platz geschrie-
ben werden. Sollte dieser Platz nicht ausreichen, setzen Sie Ihre Antwort auf der Rückseite oder
dem separat zur Verfügung gestellten Papier fort. Verweisen Sie in solchen Fällen ausdrücklich
auf Ihre Fortsetzung. Bitte schreiben Sie zudem Ihren Vor- und Nachnamen auf jeden separaten
Lösungsbogen.

\item
Es werden nur Antworten im dafür vorgesehenen Platz bewertet. Antworten auf der Rückseite
oder separatem Papier werden nur bei einem vorhandenen und klaren Verweis darauf bewertet.

\item
Die Teilaufgaben werden mit den jeweils oben auf der Seite angegebenen Punkten bewertet.

\item
Ihre endgültige Lösung jeder Teilaufgabe darf nur eine einzige Version enthalten.

\item
Zwischenrechnungen und Notizen müssen auf einem getrennten Blatt gemacht werden. Diese
Blätter müssen, deutlich als Entwurf gekennzeichnet, ebenfalls abgegeben werden.
\end{enumerate}

\newpage
\section*{\hfil Aufgaben \hfil}
\vspace{1cm}
\section*{Aufgabe 1 (25 Punkte)}
\vspace{0.4cm}
\subsection*{\aufgabe{a}{6}}
Ein Konsument \textit{maximiert} seine Nutzenfunktion $u(c_1,c_2)$ in den Einheiten $c_1$ und $c_2$
der Güter 1 und 2 definiert durch:
\begin{equation*}
u \ : \ \mathbb{R}_+ \times \mathbb{R}_+ \to \mathbb{R},
\quad (c_1,c_2) \mapsto u(c_1,c_2) = c_1^{0.6} c_2^{0.4}
\end{equation*}
über die Wahl des Konsumbündels $(c_1^\star,c_2^\star)$.
Die Preise der Güter 1 und 2 sind $p_1 \ = \ 3$ beziehungsweise $p_2 \ = \ 4$,
und das Budget, welches \textit{vollständig} genutzt wird, beträgt $e = 15$.
\\
Bestimmen Sie die stationären Punkte des Maximierungsproblems des Konsumenten, 
d.h., Kandidaten für das optimale Konsumbündel $(c_1^\star,c_2^\star)$.
\\
\\
\textbf{Hinweis:}\\ 
Eine Abklärung, ob es sich bei den stationären Punkten tatsächlich um Maxima handelt, wird nicht verlangt.
\\
\\
\subsection*{\aufgabe{b}{9}}
Sei $f \ : \ \mathbb{R} \times (-5, \infty) \to \mathbb{R}$ eine Funktion zweier reeller Variablen definiert durch:
\begin{equation*}
f(x,y)\ = \ x^2 + 3 x y + 16 \ln(y+5).
\end{equation*}
Sei $g \ : \ R_f \to \mathbb{R}$ eine stetig differenzierbare Funktion einer reellen Variablen mit $g^\prime(x) > 0$
für alle $x \in R_f$, wobei $R_f$ der Wertebereich von $f$ ist.
Schließlich sei die Komposition $h$ gegeben als
\begin{equation*}
h \ : \ D_f \to \mathbb{R}, \quad (x,y) \mapsto h(x,y) = g(f(x,y)),
\end{equation*}
$D_f$ ist dabei das Definitionsgebiet von $f$.
\\
\\
Untersuchen Sie die Funktion $h$ auf stationäre Punkte, d.h., Maxima, Minima und Sattelpunkte.
\\
\\
\textbf{Hinweis:} \\
Dank der Eigenschaft von $g$ ist es möglich, das Problem in handhabbare Form zu bringen.
\\
\\
\subsection*{\aufgabe{c}{6}}
Ein Investment Fonds generiert innerhalb der Zeitspanne von $t=0$ zu $T=12$ einen
stetigen Cashflow von $B(t) = 10 t + 5$.
Die Verzinsung erfolgt kontinuierlich zum Zinssatz $i = 5 \%$.
\\
\\
Bestimmen Sie den Nettobarwert $PV(0)$ zum Zeitpunkt $t = 0$ \textit{aller} Zahlungsströme,
die der Investment Fonds zwischen den Zeitpunkten $t = 0$ und $T = 12$ generiert.
\\
\\
\subsection*{\aufgabe{d}{4}}
Berechnen Sie das unbestimmte Integral
\begin{equation*}
\int \frac{\sin(\ln(x)) \ \cos(\ln(x))}{x} dx.
\end{equation*}

\newpage
\section*{Aufgabe 2 (24 Punkte)}
\vspace{0.4cm}
\subsection*{\aufgabe{a}{3}}
Gegeben sei die Matrix
\begin{equation*}
A = 
\begin{pmatrix}
1 & t & 1 \\
2 & t & 0 \\
-1 & t & 2t
\end{pmatrix},
\end{equation*}
wobei $t \in \mathbb{R}$.
\\
\\
Für welche Werte von $t$ ist der Rang von $A$ gleich $3$?
\\
\\
\subsection*{\aufgabe{b}{6}}
Die folgende Tabelle beschreibt die jährlichen Payoffs zweier Wertpapiere zu identischem Ausgangspreis,
abhängig von der jeweiligen konjunkturellen Lage:
\begin{table}[H]
\centering
\begin{tabular}{lcc}
\hline 
Konjunktur & Aktie 1 & Aktie 2 \\ 
\hline 
Expansion & 1.5 & 3 \\ 
wirtschaftliche Stabilität & 1.5 & 2 \\ 
Rezession & 1.5 & 0.5 \\ 
\hline 
\end{tabular} 
\end{table}
Ermitteln Sie, ob das folgende Auszahlungsschema für den Investor möglich ist,
wenn er nur in die Aktien 1 und 2 investiert:
\begin{table}[H]
\centering
\begin{tabular}{lc}
\hline 
Konjunktur & Payoff des Investors \\ 
\hline 
Expansion & 1'500 \\ 
wirtschaftliche Stabilität & 2'000 \\ 
Rezession & 1'000 \\ 
\hline 
\end{tabular} 
\end{table}
\ \\
\subsection*{\aufgabe{c}{6}}
Gegeben sei die $4 \times 4$ Matrix
\begin{equation*}
A= 
\begin{pmatrix}
2 & 0 &0 & 0 \\
1 & 5 & 1  & -1\\
1 & 3 & 4s & 0 \\
0 & 1 & -1 & 0
\end{pmatrix},
\end{equation*}
wobei $s \in \mathbb{R}$.\\ \\
Ermitteln Sie $s$ so, dass $\lambda = 0$ ein Eigenwert von $A$ ist.
Berechnen Sie weiterhin für diesen Fall die Eigenvektoren von $A$.
\\
\\
\subsection*{\aufgabe{d}{4}}
Gegeben sei die Funktion zweier reeller Variablen 
$f \ : \ \mathbb{R}\times \mathbb{R} \to \mathbb{R}$ definiert durch:
\begin{equation*}
f(x,y) = e^{2x^2+y^3 +3x +3y}.
\end{equation*}
Bestimmen Sie die Gleichung (allgemeine Form) einer Ebene $\beta$ so,
dass der Vektor $\textbf{u} = (x_0,y_0,z_0)^\top$ orthogonal zu $\beta$ ist,
wobei $(x_0,y_0)^\top = \textbf{grad} f(0,0)$ und $z_0 = f(0,0)$ gilt.
\\
\\
\subsection*{\aufgabe{e}{6}}
Verwenden Sie das \textit{Gauss Verfahren}, um die Lösungsmenge des folgenden linearen Gleichungssystems zu bestimmen:
\begin{equation*}
\begin{split}
x_1 \ + \ 2 x_2 \ + \ 3 x_3 \ + \ 6 x_4 \ &= \ \ 5 \\
x_1 \ + \ 3 x_2 \ + \ 4 x_3 \ + \ 8 x_4 \ &= \ \ 7 \\
2 x_1 \ + \ \ x_2 \ + \ 3 x_3 \ + \ 4 x_4 \ &= \ -1
\end{split}
\end{equation*}
\newpage


\fancyhead[C]{\normalsize\textbf{$\qquad$ Teil II: Multiple-Choice}}
\begin{Large}
\textbf{Teil II: Multiple-Choice-Fragen (50 Punkte)}
\end{Large}
\\
\\
\\
\textbf{Allgemeine Anweisungen für Multiple-Choice-Fragen:}
\\
\renewcommand{\labelenumi}{(\roman{enumi})}
\begin{enumerate}
\item
Die Antworten auf die Multiple-Choice-Fragen müssen im dafür vorgesehenen Antwortbogen ein-
getragen werden. Es werden ausschliesslich Antworten auf diesem Antwortbogen bewertet. Der
Platz unter den Fragen ist nur für Notizen vorgesehen und wird nicht korrigiert.

\item
Jede Frage hat nur eine richtige Antwort. Es muss also auch jeweils nur eine Antwort angekreuzt
werden.

\item
Falls mehrere Antworten angekreuzt sind, wird die Antwort mit 0 Punkten bewertet, auch wenn
die korrekte Antwort unter den angekreuzten ist.

\item
Bitte lesen Sie die Fragen sorgfältig.

\end{enumerate}
\newpage
\section*{Aufgabe 3 (22 Punkte)}
\vspace{0.4cm}
\subsection*{\frage{1}{3}}
Der Punkt $P \ = \left( -1, \frac{5}{2} \right)$ ist ein Maximum der Funktion $f$ unter der Nebenbedingung $\varphi(x,y) = 3x +2y -2 = 0$.
Dann gilt:
\renewcommand{\labelenumi}{(\alph{enumi})}
\begin{enumerate}
\item Der Punkt $P \ = \left( -1, \frac{5}{2} \right)$ ist ein Maximum der Funktion $f$ unter der Nebenbedingung $\varphi(x,y) = 4x +2y -1 = 0$.
\item Der Punkt $P \ = \left( -1, \frac{5}{2} \right)$ ist ein Maximum der Funktion $f$ unter der Nebenbedingung $\varphi(x,y) = 5x +2y -1 = 0$.
\item Der Punkt $P \ = \left( -1, \frac{5}{2} \right)$ ist ein Minimum der Funktion $f$ unter der Nebenbedingung $\varphi(x,y) = 4x +2y -1 = 0$.
\item keine der obigen Antworten ist im Allgemeinen richtig.
\end{enumerate}
\ \\
\subsection*{\frage{2}{4}}
Die Funktion $f$ hat folgende Eigenschaften:
\renewcommand{\labelenumi}{(\roman{enumi})}
\begin{enumerate}
\item $f(x) \geq -3 $ für $x \in [0,1],$ und
\item $\int_0^1 f(x) dx = 3$.
\end{enumerate}
Dann gilt:
\renewcommand{\labelenumi}{(\alph{enumi})}
\begin{enumerate}
\item $g_1  =  \frac{1}{3} f $ ist eine Dichtefunktion auf $[0,1]$.
\item $g_2  =   f +3 $ ist eine Dichtefunktion auf $[0,1]$.
\item
$g_3  =  \frac{1}{6} f +3 $ ist eine Dichtefunktion auf $[0,1]$.
\item
$g_4  =  \frac{1}{6} f + \frac{1}{2} $ ist eine Dichtefunktion auf $[0,1]$.
\end{enumerate}
\ \\
\subsection*{\frage{3}{2}}
$A = (a_{ij})$ ist eine $4 \times 5$-Matrix vom Rang $4$.
Dann gilt:
\renewcommand{\labelenumi}{(\alph{enumi})}
\begin{enumerate}
\item 
alle $3 \times 3$ Untermatrizen von $A$ sind regulär.
\item 
alle $3 \times 3$ Untermatrizen von $A$ sind singulär.
\item 
alle $4 \times 4$ Untermatrizen von $A$ sind regulär.
\item
es existiert mindestens eine reguläre $4 \times 4$ Untermatrix von $A$.
\end{enumerate}
\ \\
\subsection*{\frage{4}{2}}
$A$ und $B$ sind quadratische $4 \times 4$ Matrizen mit
$\det(A) = 1$ und $\det(B) = -1$.
Sei $C$ die Matrix definiert durch 
$C \ = \ A^{-1}B^2 A^2 B^{-1}$ und $\textbf{b} \in \mathbb{R}^4$.
\renewcommand{\labelenumi}{(\alph{enumi})}
\begin{enumerate}
\item 
Das System von linearen Gleichungen $C \textbf{x} = \textbf{b}$ hat unendlich viele Lösungen.
\item
Das System von linearen Gleichungen $C \textbf{x} = \textbf{b}$ hat keine Lösungen.
\item
Das System von linearen Gleichungen $C \textbf{x} = \textbf{b}$ hat eine eindeutige Lösung.
\item
Das System von linearen Gleichungen $C \textbf{x} = \textbf{b}$ hat abhängig von $A$ und $B$ unendlich viele Lösungen, keine Lösung, oder eine eindeutige Lösung.
\end{enumerate}
\ \\
\subsection*{\frage{5}{3}}
Das System von $3$-dimensionalen Vektoren
$\lbrace \textbf{u}_1, \textbf{u}_2, \textbf{u}_3 \rbrace$
ist linear abhängig.
Sei $A = [\textbf{u}_1, \textbf{u}_2, \textbf{u}_3]$ die Matrix mit Spaltenvektoren $\textbf{u}_1$, $\textbf{u}_2$ und $\textbf{u}_3$. 
\renewcommand{\labelenumi}{(\alph{enumi})}
\begin{enumerate}
\item 
$A^n$ ist regulär für alle $n \in \mathbb{N}$.
\item
$A^n$ ist singulär für alle $n \in \mathbb{N}$.
\item
$A^n$ ist regulär für $n$ ungerade und singulär für $n$ gerade.
\item
$A^n$ ist singulär für $n$ ungerade und regulär für $n$ gerade.
\end{enumerate}
\ \\
\subsection*{\frage{6}{2}}
Für das System von linearen Gleichungen $A \textbf{x} = \textbf{b}$ gilt:
$\text{rg}(A) = 4$ und $\text{rg}(A,\textbf{b}) = 4$, wobei $A$ eine $5 \times 6$ Matrix ist.
\renewcommand{\labelenumi}{(\alph{enumi})}
\begin{enumerate}
\item 
Das System hat keine Lösung.
\item
Das System hat genau eine Lösung.
\item
Das System hat unendlich viele Lösungen und der Lösungsraum hat Dimension $1$.
\item
Das System hat unendlich viele Lösungen und der Lösungsraum hat Dimension $2$.
\end{enumerate}
\ \\
\subsection*{\frage{7}{3}}
$A$ ist eine quadratische Matrix und $\lambda = 0 $ einer ihrer Eigenwerte.
\renewcommand{\labelenumi}{(\alph{enumi})}
\begin{enumerate}
\item 
Da aus $A \textbf{x} = \textbf{0}$ folgt, dass $\textbf{x } = 0$, hat $A$ keinen zum Eigenwert $ \lambda = 0$ gehörenden Eigenvektor.
\item
$A$ hat einen eindeutigen zum Eigenwert $\lambda = 0$ gehörenden Eigenvektor.
\item
$A$ hat unendlich viele zum Eigenwert $\lambda = 0$ gehörende Eigenvektoren.
\item
Wie viele Eigenvektoren zum Eigenwert $\lambda = 0$ existieren, hängt von der Matrix $A$ ab.
\end{enumerate}
\ \\
\subsection*{\frage{8}{3}}
Ein dynamisches Model für die Variablen $\textbf{u}_t = (x_t , y_t)^\top \neq \textbf{0}$ erfüllt die Gleichung
$\textbf{u}_{t+1} = A \textbf{u}_t$ für
\begin{align*}
A = 
\begin{pmatrix}
1 & -1\\
2 & 1
\end{pmatrix}
\end{align*}
für $t = 0,1,\dots$ \\
\\
Die Gleichgewichtsbedingung $\textbf{u}_{t+1} = \lambda \textbf{u}_t$ für alle $t = 0,1,\dots$ mit $\lambda \in \mathbb{R}$
\renewcommand{\labelenumi}{(\alph{enumi})}
\begin{enumerate}
\item 
kann nur erfüllt sein für $\lambda = 0$.
\item
kann nur erfüllt sein für $\lambda = 1$.
\item
kann nur erfüllt sein für $\lambda = 2$.
\item
kann nie erfüllt sein.
\end{enumerate}


\newpage
\section*{Aufgabe 4 (28 Punkte)}
\vspace{0.4cm}

\subsection*{\frage{1}{3}}
Das unbestimmte Integral
\begin{align*}
\int \left[ 6 \ x \  +  \ (2 \ x^2 \ + \ 1 ) e^{x^2} \right] \ dx
\end{align*}
ist
\renewcommand{\labelenumi}{(\alph{enumi})}
\begin{enumerate}
\item 
$x^2 + x e^{x^2} + C, \quad C \in \mathbb{R}$.
\item
$3x + 2x e^{x^2} + C , \quad C \in \mathbb{R}$.
\item
$3x^2 + 2x e^{x^2} + C , \quad C \in \mathbb{R}$.
\item
$3x^2 + x e^{x^2} + C , \quad C \in \mathbb{R}$.
\end{enumerate}
\ \\
\subsection*{\frage{2}{3}}
Der Vektor
\begin{align*}
\textbf{x}
=
\begin{pmatrix}
1\\
2\\
t\\
4\\
3
\end{pmatrix}
\end{align*}
ist eine Eigenvektor der $5 \times 5$ Matrix $A$ zum Eigenwert $\lambda \neq 0$, wobei $t \in \mathbb{R}$.
Der Vektor
\begin{align*}
\textbf{y}
= 
\begin{pmatrix}
3\\
4\\
3\\
t\\ 
1
\end{pmatrix}
\end{align*}
ist orthogonal zum Vektor $A \textbf{x}$ für
\renewcommand{\labelenumi}{(\alph{enumi})}
\begin{enumerate}
\item 
$t = 1$.
\item
$t \in \lbrace 1, -2 \rbrace$.
\item
$t = -2$.
\item
Es gibt kein $t \in \mathbb{R}$, sodass $\textbf{y}$ orthogonal zu $A \textbf{x} $ ist.
\end{enumerate}
\ \\
\subsection*{\frage{3}{5}}
Die $5 \times 4$ Matrix
\begin{align*}
A
=
\begin{pmatrix}
1 & 0  & 1 & 2\\
1 & 1 & 5 & 7\\
0 & 1 & 2 & 1 \\
2 & 1 & 1 & 4 \\
0 & 1 & 1 & 0 
\end{pmatrix}
\end{align*}
\renewcommand{\labelenumi}{(\alph{enumi})}
\begin{enumerate}
\item 
hat Rang $2$.
\item
hat Rang $3$.
\item
hat Rang $4$.
\item
hat Rang $5$.
\end{enumerate}
\ \\
\subsection*{\frage{4}{4}}
Gegeben sei die $3 \times 3 $ Matrix
\begin{align*}
A
= 
\begin{pmatrix}
1 & 1 & 0\\
1 & 0 & 1 \\
1 & 1 & 1
\end{pmatrix}.
\end{align*}
Dann gilt:
\renewcommand{\labelenumi}{(\alph{enumi})}
\begin{enumerate}
\item 
$A^{-1}
= 
\begin{pmatrix}
1 & 1 & -1 \\
0 & -1 & 1\\
-1 & 0 & 1
\end{pmatrix}$.
\item
$A^{-1}
= 
\begin{pmatrix}
1 & 1 & -1 \\
1 & -1 & 1\\
-1 & 0 & 1
\end{pmatrix}$.
\item
$A^{-1}
= 
\begin{pmatrix}
1 & 1 & -1 \\
0 & -1 & 1\\
-1 & 1 & 1
\end{pmatrix}$.
\item
$A$ ist singulär.
\end{enumerate}

\newpage
\subsection*{\frage{5}{5}}
Die Matrix
\begin{align*}
A 
= 
\begin{pmatrix}
1 & 3 \\
3 & 1
\end{pmatrix}
\end{align*}
hat reellwertige Eigenwerte $\lambda_1$ und $\lambda_2$
mit zugehörigen Eigenvektoren $\textbf{v}_1$ und $\textbf{v}_2$.
Dann gilt:
\renewcommand{\labelenumi}{(\alph{enumi})}
\begin{enumerate}
\item 
$\lambda_1 \lambda_2 = 0$.
\item
$\lambda_1 +  \lambda_2 = 0$.
\item
$\textbf{v}_1^\top \textbf{v}_2  = 0 $.
\item
$\textbf{v}_1 +  \textbf{v}_2  = \textbf{0} $.
\end{enumerate}
\ \\
\subsection*{\frage{6}{2}}
Die Folge $\lbrace y_k \rbrace_{k \in \mathbb{N}_0}$ definiert durch
\begin{align*}
y_k =  4 - \left( \frac{1}{a}\right)^k
\end{align*}
mit $a \neq 0$ löst das Anfangswertproblem
\begin{align*}
4 y_{k+1} -y_k = 12 \ \text{und} \ y_0 = 3
\end{align*}
für
\renewcommand{\labelenumi}{(\alph{enumi})}
\begin{enumerate}
\item 
$a= 1$.
\item
$a= 2$.
\item
$a= 4$.
\item
Keine der vorangehenden Antworten ist richtig.
\end{enumerate}
\ \\
\subsection*{\frage{7}{2}}
Die allgemeine Lösung der Differenzengleichung
\begin{align*}
5 y_{k+1} + 6 y_k = 1 , \quad k = 0,1,2,...
\end{align*}
ist
\renewcommand{\labelenumi}{(\alph{enumi})}
\begin{enumerate}
\item 
monoton und konvergent.
\item
monoton und divergent.
\item
oszillierend und konvergent.
\item
oszillierend und divergent.
\end{enumerate}
\ \\
\subsection*{\frage{8}{4}}
Die allgemeine Lösung der Differenzengleichung
\begin{align*}
m \ y_{k+1} + y_k  \ = \ m^2, \quad k=0,1,2,...,
\end{align*}
wobei $m \in \mathbb{R} \setminus \lbrace 0 \rbrace$, ist monoton und konvergent genau dann, wenn
\renewcommand{\labelenumi}{(\alph{enumi})}
\begin{enumerate}
\item 
$m \in [-1,0)$.
\item
$m \in (0,1]$.
\item
$m < -1$.
\item
$m> 1$.
\end{enumerate}