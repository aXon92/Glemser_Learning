\section*{Aufgabe 2 (25 Punkte)}
\vspace{0.4cm}
\subsection*{\aufgabe{a}{3}}
Gegeben sei die Matrix
\begin{equation*}
A = 
\begin{pmatrix}
1 & t & 1 \\
2 & t & 0 \\
-1 & t & 2t
\end{pmatrix},
\end{equation*}
wobei $t \in \mathbb{R}$.
\\
\\
Für welche Werte von $t$ ist der Rang von $A$ gleich $3$?
\\
\\
\textbf{Lösung:}
\begin{mdframed}
\renewcommand{\labelenumi}{\theenumi.}
\underline{\textbf{Vorgehensweise:}}
\begin{enumerate}
\item Gebe eine Bedingung an, sodass eine quadratische Matrix den Rang $3$ besitzt.
\item Bestimme die Werte von $t$.
\end{enumerate}
\end{mdframed}
\underline{1. Gebe eine Bedingung an, sodass eine quadratische Matrix den Rang $3$ besitzt}\\
Für eine quadratische Matrix kennen wir den Zusammenhang:
\begin{align*}
\text{rg}(A) = 3 
\Leftrightarrow
A \ \text{ist regulär}
\Leftrightarrow
\det(A) \neq 0
\end{align*}
Wir müssen also nur die Determinate auf Nullstellen überprüfen.\\
\\
\underline{2. Bestimme die Werte von $t$}\\
Zunächst berechnen wir die Determinate von $A$.
Wir erhalten 
\begin{equation*}
\begin{split}
\det(A)
&= 
\left| 
\begin{pmatrix}
1 & t & 1 \\
2 & t & 0 \\
-1 & t & 2t
\end{pmatrix}
\right|\\
&=
1 \cdot t \cdot 2t +t \cdot 0 \cdot (-1) + 1 \cdot 2 \cdot t 
- (-1) \cdot t \cdot 1 - t \cdot 0  \cdot 1 - 2t \cdot 2 \cdot t\\
&=
2 t^2 + 2t + t - 4 t^2 
= -2 t^2 +3 t = -t ( 2 t - 3)
\end{split}
\end{equation*}
durch die Regel von Sarrus (Sauron).
Weiter gilt
\begin{equation*}
\det(A) = 0 
\Leftrightarrow
-t ( 2 t - 3) = 0 
\Leftrightarrow
t_1 = 0, \quad t_2 = \frac{3}{2}
\end{equation*}
für die Nullstellen der Determinante.
Wir wissen nun, dass 
\begin{align*}
\det(A) \neq 0 
\Leftrightarrow
\text{rg}(A) = 3 
\Leftrightarrow
t \in \mathbb{R} \setminus \left\lbrace 0, \frac{3}{2} \right\rbrace
\end{align*}
gilt.\\
\\
Die Matrix $A$ besitzt den Rang $3$ für alle $t \in \mathbb{R} \setminus \left\lbrace 0, \frac{3}{2} \right\rbrace$.

\newpage

\subsection*{\aufgabe{b}{6}}
Die folgende Tabelle beschreibt die jährlichen Payoffs zweier Wertpapiere zu identischem Ausgangspreis,
abhängig von der jeweiligen konjunkturellen Lage:
\begin{table}[H]
\centering
\begin{tabular}{lcc}
\hline 
Konjunktur & Aktie 1 & Aktie 2 \\ 
\hline 
Expansion & 1.5 & 3 \\ 
wirtschaftliche Stabilität & 1.5 & 2 \\ 
Rezession & 1.5 & 0.5 \\ 
\hline 
\end{tabular} 
\end{table}
Ermitteln Sie, ob das folgende Auszahlungsschema für den Investor möglich ist,
wenn er nur in die Aktien 1 und 2 investiert:
\begin{table}[H]
\centering
\begin{tabular}{lc}
\hline 
Konjunktur & Payoff des Investors \\ 
\hline 
Expansion & 1'500 \\ 
wirtschaftliche Stabilität & 2'000 \\ 
Rezession & 1'000 \\ 
\hline 
\end{tabular} 
\end{table}
\ \\
\textbf{Lösung:}
\begin{mdframed}
\underline{\textbf{Vorgehensweise:}}
\begin{enumerate}
\renewcommand{\labelenumi}{\theenumi.}
\item Gebe eine Bedingung an, sodass der Payoff des Investors realisiert werden kann.
\item Überprüfe diese Bedingung.
\end{enumerate}
\end{mdframed}
\underline{1. Gebe eine Bedingung an, sodass der Payoff des Investors realisiert werden kann}\\
Wir können den Payoff des Investors realisieren, falls wir $\lambda_1, \lambda_2$ finden, so dass
\begin{align*}
\begin{pmatrix}
1500\\
2000\\
1000\\
\end{pmatrix}
= \lambda_1 
\begin{pmatrix}
1.5\\
1.5\\
1.5
\end{pmatrix}
+ \lambda_2
\begin{pmatrix}
3\\
2\\ 
0.5
\end{pmatrix}
\end{align*}
gilt.
Die Payoff-Vektoren der Wertpapiere sind linear unabhängig.
Damit finden wir unsere $\lambda_1, \lambda_2$ nur, falls
\begin{align*}
\left\lbrace 
\begin{pmatrix}
1500\\
2000\\
1000\\
\end{pmatrix},
\begin{pmatrix}
1.5\\
1.5\\
1.5
\end{pmatrix},
\begin{pmatrix}
3\\
2\\ 
0.5
\end{pmatrix}
\right\rbrace
\end{align*}
linear abhängig ist.
Um uns die Rechnung zu vereinfachen, können wir die Vektoren mit geschickten Vielfachen austauschen.
Durch
\begin{align*}
\left\lbrace 
\begin{pmatrix}
3\\
4\\
2\\
\end{pmatrix},
\begin{pmatrix}
1\\
1\\
1
\end{pmatrix},
\begin{pmatrix}
6\\
4\\ 
1
\end{pmatrix}
\right\rbrace
\end{align*}
erhalten wir ein einfacheres System.
Wir wollen noch exemplarisch die Rechnung für den ersten Vektor durchführen:
\begin{align*}
\frac{1}{5} \cdot \frac{1}{100}
\begin{pmatrix}
1500\\
2000\\
1000\\
\end{pmatrix}
=
\frac{1}{5}
\begin{pmatrix}
15\\
20\\
10\\
\end{pmatrix}
=
\begin{pmatrix}
3\\
4\\
2\\
\end{pmatrix}
\end{align*}
Wenn die lineare Abhängigkeit nachgewiesen werden kann, wissen wir, dass
\begin{align*}
A
= 
\begin{pmatrix}
3 & 1 & 6 \\
4 & 1 & 4 \\
2 & 1 & 1
\end{pmatrix}
\end{align*}
singulär ist.\\
\\
\underline{2. Überprüfe diese Bedingung}\\
Die Matrix $A$ ist singulär genau dann, wenn
\begin{align*}
\det(A) = 0
\end{align*}
gilt.
Hier berechnen wir die Determinante
\begin{equation*}
\begin{split}
\det(A)
=
\left|
\begin{pmatrix}
3 & 1 & 6 \\
4 & 1 & 4 \\
2 & 1 & 1
\end{pmatrix}
\right|
=
3 + 8 + 24 - 12 - 12 -4
= 
7
\end{split}
\end{equation*}
mit der Regel von Sarrus.
Damit ist die Matrix $A$ regulär und das System
\begin{align*}
\left\lbrace 
\begin{pmatrix}
1500\\
2000\\
1000\\
\end{pmatrix},
\begin{pmatrix}
1.5\\
1.5\\
1.5
\end{pmatrix},
\begin{pmatrix}
3\\
2\\ 
0.5
\end{pmatrix}
\right\rbrace
\end{align*}
linear unabhängig.
Eine weitere Möglichkeit ist es, elementare Zeilenumformungen auf die Matrix $A$ anzuwenden.
Die Matrix $A$ ist regulär, wenn wir eine Zeilenstufenform erreichen.
Wir betrachten nun:
\begin{align*}
\begin{gmatrix}[p]
3 & 1 & 6 \\
4 & 1 & 4 \\
2 & 1 & 1
\rowops
\add[-1]{0}{1}
\add[-1]{0}{2}
\end{gmatrix}
\leadsto
&\begin{gmatrix}[p]
3 & 1 & 6 \\
1 & 0 & -2 \\
-1 & 0 & -5
\rowops
\add[-3]{1}{0}
\add[\cdot 1]{1}{2}
\end{gmatrix}\\
\leadsto
&\begin{gmatrix}[p]
0 & 1 & 12 \\
1 & 0 & -2 \\
0 & 0 & -7
\rowops
\mult{2}{\cdot ( -\frac{1}{7} )}
\end{gmatrix}\\
\leadsto
&\begin{gmatrix}[p]
0 & 1 & 12 \\
1 & 0 & -2 \\
0 & 0 & 1
\rowops
\add[\cdot 2]{2}{1}
\add[\cdot (12)]{2}{0}
\end{gmatrix}\\
\leadsto
&\begin{gmatrix}[p]
0 & 1 & 0 \\
1 & 0 & 0 \\
0 & 0 & 1
\rowops
\swap{0}{1}
\end{gmatrix}\\
\leadsto
&\begin{gmatrix}[p]
1 & 0 & 0 \\
0 & 1 & 0 \\
0 & 0 & 1
\end{gmatrix}\\
\end{align*}
Somit sehen wir auch hieran, dass die Matrix nicht singulär ist.
\\
\\
Das Auszahlungsschema des Investors lässt sich also nicht verwirklichen.


\newpage

\subsection*{\aufgabe{c}{6}}
Gegeben sei die $4 \times 4$ Matrix
\begin{equation*}
A= 
\begin{pmatrix}
2 & 0 &0 & 0 \\
1 & 5 & 1  & -1\\
1 & 3 & 4s & 0 \\
0 & 1 & -1 & 0
\end{pmatrix},
\end{equation*}
wobei $s \in \mathbb{R}$.\\ \\
Ermitteln Sie $s$ so, dass $\lambda = 0$ ein Eigenwert von $A$ ist.
Berechnen Sie weiterhin für diesen Fall die Eigenvektoren von $A$.
\\
\\
\textbf{Lösung:}
\begin{mdframed}
\underline{\textbf{Vorgehensweise:}}
\renewcommand{\labelenumi}{\theenumi.}
\begin{enumerate}
\item Rufe dir die Definition von Eigenwerten und Eigenvektoren in Erinnerung.
\item Bestimme $s$.
\item Bestimme die Eigenvektoren
\end{enumerate}
\end{mdframed}

\underline{1. Rufe dir die Definition von Eigenwerten und Eigenvektoren in Erinnerung}\\
Ein Vektor $v \neq 0$ heißt Eigenvektor zum Eigenwert $\lambda$,
falls
\begin{align*}
A v = \lambda v
\end{align*}
gilt.
Dies können wir zu 
\begin{align*}
A v - \lambda v = (A-\lambda I)v = 0
\end{align*}
umformen. 
Hierbei bezeichnet $I$ die Einheitsmatrix.
Da $v \neq 0$ ist, kann diese Gleichung nur erfüllt sein, wenn $A$ singulär ist.
Deswegen untersuchen wir
\begin{align*}
\det(A - \lambda I) 
\end{align*}
auf Nullstellen, um die Eigenwerte zu finden.
In der Aufgabe ist der Eigenwert vorgegeben. 
Damit $ \lambda = 0 $ ein Eigenwert ist, muss also
\begin{align*}
\det(A(s) - 0 I) = \det(A(s)) = 0
\end{align*}
gelten.\\
\\
\underline{2. Bestimme $s$}\\
Durch Entwicklung nach der ersten Zeile und der dritten Spalte erhalten wir
\begin{equation*}
\begin{split}
\det(A(s))&=
\left| 
\begin{pmatrix}
2 & 0 &0 & 0 \\
1 & 5 & 1  & -1\\
1 & 3 & 4s & 0 \\
0 & 1 & -1 & 0
\end{pmatrix}
\right|\\
&=
2 
\left| 
\begin{pmatrix}
 5 & 1  & -1\\
 3 & 4s & 0 \\
 1 & -1 & 0
\end{pmatrix}
\right|\\
&= 
2 \cdot (-1) 
\left| 
\begin{pmatrix}
  3 & 4s  \\
 1 & -1 
\end{pmatrix}
\right|\\
&= 2 \cdot(-1) ( -3 - 4 s)
= 2( 3 + 4s) 
= 6 + 8s 
\end{split}
\end{equation*}
die Determinante von $A$ in Abhängigkeit von $s$.
Damit erhalten wir durch
\begin{equation*}
\det(A(s)) = 8s +6 = 0
\Leftrightarrow
s = -\frac{6}{8} = -\frac{3}{4}
\end{equation*}
das gesuchte $s$.
Für $s = -\frac{3}{4}$ ist $\lambda = 0$ ein Eigenwert von $A$.
\\
\\
\underline{3. Bestimme die Eigenvektoren}\\
Um die Eigenvektoren zu bestimmen, müssen wir das lineare Gleichungssystem
\begin{align*}
(A - 0 I) v = A v = 0
\end{align*}
lösen.
Wir wenden also auf erweiterte Koeffizientenmatrix
\begin{align*}
(A \  | \ 0 ) =
\begin{pmatrix}[cccc|c]
2 & 0 &0 & 0 & 0 \\
1 & 5 & 1  & -1 & 0\\
1 & 3 & -3 & 0 & 0  \\
0 & 1 & -1 & 0 & 0
\end{pmatrix}
\end{align*}
elementare Zeilenumformungen an.
Da die rechte Seite nur der Nullvektor ist, können wir diese Spalte in der Matrix getrost ignorieren.
Durch
\begin{equation*}
\begin{split}
\begin{gmatrix}[p]
2 & 0 &0 & 0 \\
1 & 5 & 1  & -1\\
1 & 3 & -3 & 0 \\
0 & 1 & -1 & 0
\rowops
\add[-2]{0}{1}
\add[-2]{0}{2}
\end{gmatrix}
&\leadsto
\begin{gmatrix}[p]
2 & 0 &0 & 0 \\
0 & 5 & 1  & -1\\
0 & 3 & -3 & 0 \\
0 & 1 & -1 & 0
\rowops
\swap{1}{2}
\end{gmatrix}\\
&\leadsto
\begin{gmatrix}[p]
2 & 0 &0 & 0 \\
0 & 1& -1  & 0 \\
0 & 3 & -3 & 0 \\
0 & 5 & 1  & -1
\rowops
\add[-3]{1}{2}
\add[-5]{1}{3}
\end{gmatrix}\\
&\leadsto
\begin{gmatrix}[p]
2 & 0 &0 & 0 \\
0 & 1& -1  & 0 \\
0 & 0 & 0 & 0 \\
0 & 5 & 1  & -1
\rowops
\swap{2}{3}
\end{gmatrix}\\
&\leadsto
\begin{gmatrix}[p]
2 & 0 &0 & 0 \\
0 & 1& -1  & 0 \\
0 & 5 & 1  & -1 \\
0 & 0 & 0 & 0 
\rowops
\add[-5]{1}{2}
\end{gmatrix}
\\
&\leadsto
\begin{gmatrix}[p]
2 & 0 &0 & 0 \\
0 & 1& -1  &  0\\
0 & 0 & 6  & -1 \\
0 & 0 & 0 & 0 
\rowops
\add[-5]{1}{2}
\end{gmatrix}
\end{split}
\end{equation*}
erhalten wir das System
\begin{align*}
2 x_1 \ &= 0\\
x_2 \ - \ x_3 \ &= 0 \\
6x_3 \ -  \ x_4 \ &= 0
\end{align*}
mit frei wählbarem $x_4 \in \mathbb{R}$.
Durch Rückwärtseinsetzen erhalten wir
\begin{align*}
x_3 &= \frac{x_4}{6}\\
x_2 &=  x_3 =  \frac{x_4}{6}\\
x_1 &= 0,
\end{align*}
womit die Eigenvektoren durch
\begin{align*}
\textbf{x }  = t \cdot 
\begin{pmatrix}
0 \\
1 \\ 
1 \\
6
\end{pmatrix}
\end{align*}
für $t \in \mathbb{R} \setminus \lbrace 0 \rbrace$ gegeben sind.\\
\\
Beim Gaußverfahren von Hand ist es empfehlenswert Brüche zu vermeiden.
Man reduziert dabei die Wahrscheinlichkeit für Rechnenfehler sehr.

\newpage

\subsection*{\aufgabe{d}{4}}
Gegeben sei die Funktion zweier reeller Variablen 
$f \ : \ \mathbb{R}\times \mathbb{R} \to \mathbb{R}$ definiert durch:
\begin{equation*}
f(x,y) = e^{2x^2+y^3 +3x +3y}.
\end{equation*}
Bestimmen Sie die Gleichung (allgemeine Form) einer Ebene $\beta$ so,
dass der Vektor $\textbf{u} = (x_0,y_0,z_0)^\top$ orthogonal zu $\beta$ ist,
wobei $(x_0,y_0)^\top = \textbf{grad} f(0,0)$ und $z_0 = f(0,0)$ gilt.
\\
\\
\textbf{Lösung:}
\begin{mdframed}
\underline{\textbf{Vorgehensweise:}}
\renewcommand{\labelenumi}{\theenumi.}
\begin{enumerate}
\item Stelle eine passende Ebenengleichung auf und berechne die fehlenden Parameter.
\end{enumerate}
\end{mdframed}

\underline{1. Stelle eine passende Ebenengleichung auf und berechne die fehlenden Parameter}\\
Da der Vektor $\textbf{u} = (x_0,y_0,z_0)^\top$ orthogonal auf $\beta$ ist, ist $\textbf{u}$ ein Normalenvektor.
Damit können wir $\beta$ durch
\begin{align*}
\beta \ : \ x_0 x + y_0 y + z_0 z + d  = 0, \qquad d \in \mathbb{R}
\end{align*}
beschreiben.
Wir müssen also nur noch $\text{u}$ bestimmen.
Es gilt  
\begin{align*}
\textbf{grad} f(x,y) = e^{2x^2+y^3 +3x +3y} 
\begin{pmatrix}
4 x+ 3\\
3 y^2 + 3 
\end{pmatrix}
\end{align*}
für den Gradienten von $f$.
Durch
\begin{align*}
\textbf{grad} f(0 ,0)&=
e^0
\begin{pmatrix}
3\\
3
\end{pmatrix}
=
\begin{pmatrix}
3\\
3
\end{pmatrix} \\
f(0,0) &= e^0 = 1
\end{align*}
erhalten wir $(x_0,y_0,z_0)^\top = ( 3, 3,1)^\top$.
Damit haben wir durch
\begin{align*}
\beta \ : \ 3 x + 3 y + 1 z + d, \qquad d \in \mathbb{R}
\end{align*}
die allgemeine Form von $\beta$ gegeben.
\newpage

\subsection*{\aufgabe{e}{6}}
Verwenden Sie das \textit{Gauss Verfahren}, um die Lösungsmenge des folgenden linearen Gleichungssystems zu bestimmen:
\begin{equation*}
\begin{split}
x_1 \ + \ 2 x_2 \ + \ 3 x_3 \ + \ 6 x_4 \ &= \ \ 5 \\
x_1 \ + \ 3 x_2 \ + \ 4 x_3 \ + \ 8 x_4 \ &= \ \ 7 \\
2 x_1 \ + \ \ x_2 \ + \ 3 x_3 \ + \ 4 x_4 \ &= \ -1
\end{split}
\end{equation*}
\\
\textbf{Lösung:}
\begin{mdframed}
\underline{\textbf{Vorgehensweise:}}
\renewcommand{\labelenumi}{\theenumi.}
\begin{enumerate}
\item Stelle die erweiterte Koeffizientenmatrix auf und löse das System.
\end{enumerate}
\end{mdframed}


\underline{1. Stelle die erweiterte Koeffizientenmatrix auf und löse das System}\\
Die erweiterte Koeffizientenmatrix ist durch
\begin{align*}
(A \ |  \ b)
=
\begin{gmatrix}[p]
1 & 2  & 3 & 6 & \BAR & 5\\
1 & 3 & 4 & 8 &\BAR & 7 \\
2 & 1 & 3 & 4 &\BAR & -1
\end{gmatrix}
\end{align*}
gegeben.
Durch die Anwendung des Gauss Verfahren erhalten wir 
\begin{equation*}
\begin{split}
\begin{gmatrix}[p]
1 & 2  & 3 & 6 & \BAR & 5\\
1 & 3 & 4 & 8 &\BAR & 7 \\
2 & 1 & 3 & 4 &\BAR & -1
\rowops
\add[-1]{0}{1}
\add[-2]{0}{2}
\end{gmatrix}
&\leadsto
\begin{gmatrix}[p]
1 & 2  & 3 & 6 & \BAR & 5\\
0 & 1 & 1 & 2 &\BAR & 2 \\
0 & -3 & -3 & -8 &\BAR & -11
\rowops
\add[3]{1}{2}
\end{gmatrix}\\
&\leadsto
\begin{gmatrix}[p]
1 & 2  & 3 & 6 & \BAR & 5\\
0 & 1 & 1 & 2 &\BAR & 2 \\
0 & 0 & 0 & -2 &\BAR & -5
\rowops
\add[1]{2}{1}
\add[3]{2}{0}
\end{gmatrix}\\
&\leadsto
\begin{gmatrix}[p]
1 & 2  & 3 & 0 & \BAR & -10\\
0 & 1 & 1 & 0 &\BAR & -3 \\
0 & 0 & 0 & -2 &\BAR & -5
\rowops
\mult{2}{\cdot ( -1 )}
\end{gmatrix}\\
&\leadsto
\begin{gmatrix}[p]
1 & 2  & 3 & 0 & \BAR & -10\\
0 & 1 & 1 & 0 &\BAR & -3 \\
0 & 0 & 0 & 2 &\BAR & 5
\end{gmatrix}
\end{split}
\end{equation*}
das System
\begin{align*}
1x_1 \ + \ 2 x_2 \ + \ 3 x_3 \ = \ -10\\
x_2 \ + \ x_3 \ =  \ -3\\
2 x_4 \ = \ 5
\end{align*}
mit frei wählbarem $x_3 \in \mathbb{R}$.
Durch Einsetzen folgt:
\begin{align*}
x_4 &= \frac{5}{2}\\
x_2 &= -3 -x_3 \\
x_1 &= -2 x_2 - 3 x_3 - 10  = -2 (-3 - x_3)  - 3 x_3 -10 
= 6 + 2 x_3 - 3 x_3 - 10
= - x_3 - 4
\end{align*}
Damit können wir durch
\begin{align*}
\left\lbrace
\textbf{x} =
\begin{pmatrix}
x_1 \\
x_2 \\
x_3 \\ 
x_4
\end{pmatrix}
\ : \ 
\textbf{x} 
=
\begin{pmatrix}
-3\\
-2\\
0\\
\frac{5}{2}
\end{pmatrix}
+ t \cdot 
\begin{pmatrix}
-1\\
-1\\
1\\
0\\
\end{pmatrix},
\ t \in \mathbb{R}
\right\rbrace
\end{align*}
die Lösungsmenge beschreiben.