\section*{Aufgabe 4 (25 Punkte)}
\vspace{0.4cm}
\subsection*{\frage{1}{3}}
Das unbestimmte Integral
\begin{align*}
\int \left[ 6 \ x \  +  \ (2 \ x^2 \ + \ 1 ) e^{x^2} \right] \ dx
\end{align*}
ist
\renewcommand{\labelenumi}{(\alph{enumi})}
\begin{enumerate}
\item 
$x^2 + x e^{x^2} + C, \quad C \in \mathbb{R}$.
\item
$3x + 2x e^{x^2} + C , \quad C \in \mathbb{R}$.
\item
$3x^2 + 2x e^{x^2} + C , \quad C \in \mathbb{R}$.
\item
$3x^2 + x e^{x^2} + C , \quad C \in \mathbb{R}$.
\end{enumerate}
\ \\
\textbf{Lösung:}
\begin{mdframed}
\underline{\textbf{Vorgehensweise:}}
\renewcommand{\labelenumi}{\theenumi.}
\begin{enumerate}
\item Überlege dir, was für die Stammfunktionen gelten muss und finde die korrekte Antwort.
\end{enumerate}
\end{mdframed}

\underline{1. Überlege dir, was für die Stammfunktionen gelten muss und finde die korrekte Antwort}\\
Wir suchen eine Funktion $f$, welche eine der Funktionen von (a) - (d) ist.
Wir wissen, dass
\begin{align*}
f^\prime(x) = 6x +   (2  x^2 + 1 ) e^{x^2}
\end{align*}
erfüllt sein muss.
Wir sehen an den ersten Summanden von (a) und (b), dass wir diese ausschließen können.
Dies können wir aufgrund von
\begin{align*}
\frac{\mathrm{d}}{\mathrm{d} x} x^2  &= 2x\\
\frac{\mathrm{d}}{\mathrm{d} x} 3x &= 3
\end{align*}
machen.
Weiter gilt
\begin{align*}
\frac{\mathrm{d}}{\mathrm{d} x} x e^{x^2}
= 1 \cdot e^{x^2} + x (2x e^{x^2})
= e^{x^2} + 2x^2 e^{x^2}
= ( 1+ 2x^2) e^{x^2}
\end{align*}
mit der Produkt und Kettenregel,womit wir die korrekte Antwort (d) erhalten.

\newpage

\subsection*{\frage{2}{3}}
Der Vektor
\begin{align*}
\textbf{x}
=
\begin{pmatrix}
1\\
2\\
t\\
4\\
3
\end{pmatrix}
\end{align*}
ist eine Eigenvektor der $5 \times 5$ Matrix $A$ zum Eigenwert $\lambda \neq 0$, wobei $t \in \mathbb{R}$.
Der Vektor
\begin{align*}
\textbf{y}
= 
\begin{pmatrix}
3\\
4\\
3\\
t\\ 
1
\end{pmatrix}
\end{align*}
ist orthogonal zum Vektor $A \textbf{x}$ für
\renewcommand{\labelenumi}{(\alph{enumi})}
\begin{enumerate}
\item 
$t = 1$.
\item
$t \in \lbrace 1, -2 \rbrace$.
\item
$t = -2$.
\item
Es gibt kein $t \in \mathbb{R}$, sodass $\textbf{y}$ orthogonal zu $A \textbf{x} $ ist.
\end{enumerate}
\ \\
\textbf{Lösung:}
\begin{mdframed}
\underline{\textbf{Vorgehensweise:}}
\renewcommand{\labelenumi}{\theenumi.}
\begin{enumerate}
\item Überlege dir, worauf du aufgrund der Eigenvektoreigenschaft schließen kannst.
\item 
Rufe dir die Definition von Orthogonalität in Erinnerung und finde die korrekte Antwort.
\end{enumerate}
\end{mdframed}

\underline{1. Überlege dir, worauf du aufgrund der Eigenvektoreigenschaft schließen kannst}\\
Der Vektor $\textbf{x}$ ist Eigenvektor zu dem Eigenwert $\lambda \neq 0$.
Damit gilt
\begin{align*}
A \textbf{x} = \lambda x,
\end{align*}
womit $A \textbf{x}$ nur ein skalares Vielfaches von $\textbf{x}$ ist.\\
\\
\underline{2. Rufe dir die Definition von Orthogonalität in Erinnerung und finde die korrekte Antwort}\\
Zwei Vektoren $\textbf{u}, \textbf{v} \neq \textbf{0}$
sind Orthogonal, falls das Skalarprodukt
\begin{align*}
\textbf{u} \cdot \textbf{v} = 0 
\end{align*}
erfüllt.
Für skalare Vielfache gilt 
\begin{align*}
(\alpha \textbf{u}) \cdot \textbf{ v} 
= 
\alpha \ (\textbf{u} \cdot \textbf{ v} )
\end{align*}
für $\alpha \in \mathbb{R}$.
Dementsprechend müssen wir für die Orthogonalität das Skalarprodukt von $\textbf{x}$ und $\textbf{y}$  betrachten.
Es gilt:
\begin{align*}
\textbf{x} \cdot \textbf{y} 
= 
\begin{pmatrix}
1\\
2\\
t\\
4\\
3
\end{pmatrix}
\cdot 
\begin{pmatrix}
3\\
4\\
3\\
t\\ 
1
\end{pmatrix}
= 3 + 8 + 3t + 4t + 3 
=14 + 7t 
\end{align*}
Man erkennt durch schnelles Nachrechnen, dass
\begin{align*}
\textbf{x} \cdot \textbf{y} 
=14 + 7t 
= 0
\Leftrightarrow
14 = -7t
\Leftrightarrow
t = -2
\end{align*}
gelten muss.\\
\\
Damit ist Antwort (c) korrekt.

\newpage
\subsection*{\frage{3}{5}}
Die $5 \times 4$ Matrix
\begin{align*}
A
=
\begin{pmatrix}
1 & 0  & 1 & 2\\
1 & 1 & 5 & 7\\
0 & 1 & 2 & 1 \\
2 & 1 & 1 & 4 \\
0 & 1 & 1 & 0 
\end{pmatrix}
\end{align*}
\renewcommand{\labelenumi}{(\alph{enumi})}
\begin{enumerate}
\item 
hat Rang $2$.
\item
hat Rang $3$.
\item
hat Rang $4$.
\item
hat Rang $5$.
\end{enumerate}
\ \\
\textbf{Lösung:}
\begin{mdframed}
\underline{\textbf{Vorgehensweise:}}
\renewcommand{\labelenumi}{\theenumi.}
\begin{enumerate}
\item Verwende das Gaußverfahren, um den Rang zu bestimmen.
\item 
Alternativer Lösungweg.
\end{enumerate}
\end{mdframed}
\allowdisplaybreaks
\underline{1. Verwende das Gauß-Verfahren, um den Rang zu bestimmen}\\
Wir wenden das Gauß-Verfahren an:
\begin{align*}
&\begin{gmatrix}[p]
1 & 0  & 1 & 2\\
1 & 1 & 5 & 7\\
0 & 1 & 2 & 1 \\
2 & 1 & 1 & 4 \\
0 & 1 & 1 & 0 
\rowops
\add[-1]{0}{1}
\add[-2]{0}{3}
\end{gmatrix}\\
\leadsto
&\begin{gmatrix}[p]
1 & 0  & 1 & 2\\
0 & 1 & 4 & 5\\
0 & 1 & 2 & 1 \\
0 & 1 & -1 & 0\\
0 & 1 & 1 & 0 
\rowops
\add[-1]{1}{2}
\add[-1]{1}{3}
\add[-1]{1}{4}
\end{gmatrix}\\
\leadsto
&\begin{gmatrix}[p]
1 & 0  & 1 & 2\\
0 & 1 & 5 & 5\\
0 & 0 & -2 & -4 \\
0 & 0 & -5 & -5\\
0 & 0 & -3 & -5
\rowops
\add[\frac{1}{2}]{2}{0}
\add[1]{3}{1}
\end{gmatrix}\\
\leadsto
&\begin{gmatrix}[p]
1 & 0  & 0 & 0\\
0 & 1 & 0 & 0\\
0 & 0 & -2 & -4 \\
0 & 0 & -5 & -5\\
0 & 0 & -3 & -5 
\rowops
\mult{3}{\cdot ( -\frac{1}{5} )}
\end{gmatrix}\\
\leadsto
&\begin{gmatrix}[p]
1 & 0  & 0 & 0\\
0 & 1 & 0 & 0\\
0 & 0 & -2 & -4 \\
0 & 0 & 1 & 1\\
0 & 0 & -3 & -5 
\rowops
\add[2]{3}{2}
\add[3]{3}{4}
\end{gmatrix}\\
\leadsto
&\begin{gmatrix}[p]
1 & 0  & 0 & 0\\
0 & 1 & 0 & 0\\
0 & 0 & 0 & -2 \\
0 & 0 & 1 & 1\\
0 & 0 & 0 & -2 
\rowops
\mult{2}{\cdot ( -\frac{1}{2} )}
\end{gmatrix}\\
\leadsto
&\begin{gmatrix}[p]
1 & 0  & 0 & 0\\
0 & 1 & 0 & 0\\
0 & 0 & 0 & 1 \\
0 & 0 & 1 & 1\\
0 & 0 & 0 & -2 
\rowops
\add[\cdot(-1)]{2}{3}
\add[\cdot 2]{2}{4}
\end{gmatrix}\\
\leadsto
&\begin{gmatrix}[p]
1 & 0  & 0 & 0\\
0 & 1 & 0 & 0\\
0 & 0 & 0 & 1 \\
0 & 0 & 1 & 0\\
0 & 0 & 0 & 0 
\rowops
\swap{2}{3}
\end{gmatrix}\\
\leadsto
&\begin{gmatrix}[p]
1 & 0  & 0 & 0\\
0 & 1 & 0 & 0\\
0 & 0 & 1 & 0 \\
0 & 0 & 0 & 1\\
0 & 0 & 0 & 0 
\rowops
\end{gmatrix}
\end{align*}
Damit ist der Rang der Matrix gleich $4$.\\
\\
\underline{2. Alternativer Lösungweg}\\
Der Rang einer Matrix ist gleich der Anzahl linear unabhängiger Spalten bzw. Zeilen.
Die Anzahl linear unabhängiger Spalten nennen wir Spaltenrang und die Anzahl linear unabhängiger Zeilenrang.
Spalten -und Zeilenrang sind bei jeder Matrix gleich.
Dies sieht man auch an
\begin{align*}
\mathrm{rg}(A) = \mathrm{rg}(A^\top)
\end{align*}
für jede Matrix $A$.
Sollten wir also eine reguläre $4 \times 4$ Untermatrix von $A$ finden, so ist der Rang von $A$ gleich $4$.
Wegen
\begin{align*}
\begin{gmatrix}[v]
1 & 0  & 1 & 2\\
1 & 1 & 5 & 7\\
0 & 1 & 2 & 1 \\
2 & 1 & 1 & 4
\rowops
\add[\cdot (-1)]{0}{1} 
\add[\cdot (-2)]{0}{3}
\end{gmatrix}
&= 
\begin{gmatrix}[v]
1 & 0  & 1 & 2\\
0 & 1 & 4 & 5\\
0 & 1 & 2 & 1 \\
0 & 1 & -1 & 0 
\end{gmatrix}
\\
&=
\begin{gmatrix}[v]
 1 & 4 & 5\\
 1 & 2 & 1 \\
 1 & -1 & 0 
\end{gmatrix}
\\
&=
\begin{gmatrix}[v]
 1 & 4 & 5\\
 0 & -2 & -4 \\
 0 & -5 & -5 
\end{gmatrix}\\
&=
\begin{gmatrix}[v]
  -2 & -4 \\
 -5 & -5 
\end{gmatrix}
= 10 - 20 = -10 \neq 0
\end{align*}
haben wir eine reguläre $4 \times 4 $ Untermatrix von $A$ gefunden.
Damit ist der Rang von $A$ mindestens $4$.
Wir haben $5$ Zeilen mit jeweils $4$ Einträgen.
Demnach sind diese linear abhängig, womit der Rang nicht $5$ sein kann.\\
\\
Die Antwort (c) ist korrekt.

\newpage
\subsection*{\frage{4}{4}}
Gegeben sei die $3 \times 3 $ Matrix
\begin{align*}
A
= 
\begin{pmatrix}
1 & 1 & 0\\
1 & 0 & 1 \\
1 & 1 & 1
\end{pmatrix}.
\end{align*}
Dann gilt:
\renewcommand{\labelenumi}{(\alph{enumi})}
\begin{enumerate}
\item 
$A^{-1}
= 
\begin{pmatrix}
1 & 1 & -1 \\
0 & -1 & 1\\
-1 & 0 & 1
\end{pmatrix}$.
\item
$A^{-1}
= 
\begin{pmatrix}
1 & 1 & -1 \\
1 & -1 & 1\\
-1 & 0 & 1
\end{pmatrix}$.
\item
$A^{-1}
= 
\begin{pmatrix}
1 & 1 & -1 \\
0 & -1 & 1\\
-1 & 1 & 1
\end{pmatrix}$.
\item
$A$ ist singulär.
\end{enumerate}
\ \\
\textbf{Lösung:}
\begin{mdframed}
\underline{\textbf{Vorgehensweise:}}
\renewcommand{\labelenumi}{\theenumi.}
\begin{enumerate}
\item Überlege dir, welche Antworten falsch sind und gebe die richtige Antwort an.

\end{enumerate}
\end{mdframed}

\underline{1. Überlege dir, welche Antworten falsch sind und gebe die richtige Antwort an}\\
Zunächst gilt 
\begin{align*}
\det(A) 
&= 
\begin{gmatrix}[v]
1 & 1 & 0\\
1 & 0 & 1 \\
1 & 1 & 1
\rowops
\add[\cdot (-1)]{0}{1}
\add[\cdot (-1)]{0}{2}
\end{gmatrix}
=
\begin{gmatrix}[v]
1 & 1 & 0\\
0 & -1 & 1 \\
0 & 0 & 1
\end{gmatrix}\\
&=
1 \cdot
\begin{gmatrix}[v]
 -1 & 1 \\
 0 & 1
\end{gmatrix} 
- 0\cdot
\begin{gmatrix}[v]
 1 & 0 \\
 0 & 1
\end{gmatrix}
+
0 \cdot
\begin{gmatrix}[v]
 1 & 0 \\
 -1 & 1
\end{gmatrix}
= 1 \cdot (-1) \cdot 1
= -1,
\end{align*}
womit $A$ regulär ist.
Damit ist Antwort (d) falsch.
Wir untersuchen nun nacheinander die Antworten (b) und (c). Uns ist bekannt, dass der Zusammenhang
\begin{align*}
A \cdot A^{-1} = 
\begin{pmatrix}
1 & 0 & 0 \\
0 & 1 & 0 \\
0 & 0 & 1
\end{pmatrix}
\end{align*}
gelten muss. Wir überprüfen bei den Matrizen von (b) und (c), ob unpassende Einträge vorliegen.
Für Antwort (b) betrachten wir die erste Zeile von $A$ mal die erste Spalte von $A^{-1}$.
Wir erhalten
\begin{align*}
( 1 , 1 , 0 ) \cdot 
\begin{pmatrix}
1\\
1\\
-1
\end{pmatrix}
= 
1 \cdot 1 + 1 \cdot 1 + 0 \cdot -1 
= 2
\end{align*}
für den ersten Diagonaleintrag.
Somit ist Antwort (b) falsch.\\ \\
Für Antwort (c) wählen wir  die dritte Zeile von $A$ mal die zweite Spalte $A^{-1}$.
Dies ergibt den zweiten Eintrag der dritten Zeile von $A \cdot A^{-1}$. Dieser sollte gleich $0$ sein.
Wegen
\begin{align*}
(1,1,1) \cdot
\begin{pmatrix}
1\\
-1 \\
1
\end{pmatrix}
= 1 \cdot 1 + 1 \cdot (-1) + 1 \cdot 1
= 1 \neq 0
\end{align*}
ist auch Antwort (c) falsch.\\
\\
Es gibt noch zwei weitere Lösungsmöglichkeiten.
Die Erste wäre (a)-(c) nachzurechnen. 
Dies ist aber zeitaufwendig und fehleranfällig.
Die Zweite wäre das Gauß-Verfahren durchzurechnen.
Die angegebene Methode lässt sich jedoch bis auf die Determinante ohne schriftliche Rechnungen lösen, was zeitsparend ist.

\newpage

\subsection*{\frage{5}{5}}
Die Matrix
\begin{align*}
A 
= 
\begin{pmatrix}
1 & 3 \\
3 & 1
\end{pmatrix}
\end{align*}
hat reellwertige Eigenwerte $\lambda_1$ und $\lambda_2$
mit zugehörigen Eigenvektoren $\textbf{v}_1$ und $\textbf{v}_2$.
Dann gilt:
\renewcommand{\labelenumi}{(\alph{enumi})}
\begin{enumerate}
\item 
$\lambda_1 \lambda_2 = 0$.
\item
$\lambda_1 +  \lambda_2 = 0$.
\item
$\textbf{v}_1^\top \textbf{v}_2  = 0 $.
\item
$\textbf{v}_1 +  \textbf{v}_2  = \textbf{0} $.
\end{enumerate}
\ \\
\textbf{Lösung:}
\begin{mdframed}
\underline{\textbf{Vorgehensweise:}}
\renewcommand{\labelenumi}{\theenumi.}
\begin{enumerate}
\item Bestimme die Eigenwerte von $A$.
\item Bestimme Eigenvektoren zu den zugehörigen Eigenwerten.
\item Bestimme die korrekte Antwort.
\end{enumerate}
\end{mdframed}

\underline{1. Bestimme die Eigenwerte von $A$}\\
Durch Lösen von
\begin{align*}
\det(A - \lambda I ) = 0
\end{align*}
erhalten wir die Eigenwerte zu $A$.
Es gilt:
\begin{align*}
\det(A - \lambda I ) =
\left|
\begin{pmatrix}
1 - \lambda & 3 \\
3 & 1 - \lambda
\end{pmatrix}
\right|
=
(1- \lambda)^2 - 9
\end{align*}
Durch
\begin{align*}
&(1- \lambda)^2 - 9 = 0 \\
\Leftrightarrow
&(1- \lambda)^2 = 9 \\
\Leftrightarrow
&\ 1 - \lambda = \pm 3\\
\Leftrightarrow
&\ \lambda_1 = -2 , \ \ \lambda_2 = 4
\end{align*}
erhalten wir die Eigenwerte $\lambda_1$ und $\lambda_2$.\\
Wir sehen, dass die Antworten (a) und (b) falsch sind.\\
\\
\underline{2. Bestimme Eigenvektoren zu den zugehörigen Eigenwerten}\\
Zuerst bestimmen wir einen Eigenvektor $\textbf{v}_1$ zu $\lambda_1$.
Hierfür lösen wir das lineare Gleichungssystem
\begin{align*}
(A- \lambda_1 I ) \textbf{v} = \textbf{0}.
\end{align*}
Mit 
\begin{align*}
A- \lambda_1 I
=
\begin{pmatrix}
3 & 3 \\
3 & 3
\end{pmatrix} 
\leadsto
\begin{pmatrix}
3 & 3 \\
0 & 0
\end{pmatrix}
\leadsto
\begin{pmatrix}
1 & 1 \\
0 & 0
\end{pmatrix}
\end{align*}
erhalten wir die frei wählbare Variable $x_2$ und die 
Gleichung
\begin{align*}
x_1 + x_2 = 0
\Leftrightarrow
x_1 = -x_2.
\end{align*}
Damit können wir die Eigenvektoren zu $\lambda_1$ durch
\begin{align*}
\alpha 
\begin{pmatrix}
-1 \\
1
\end{pmatrix}
, \ \alpha \in \mathbb{R}
\end{align*}
angeben.
Der Nullvektor der erweiterten Koeffizientenmatrix kann hierbei ignoriert werden.
Da ein Eigenvektor ausreicht, können wir 
\begin{align*}
\textbf{v}_1 = 
\begin{pmatrix}
-1 \\
1
\end{pmatrix}
\end{align*}
setzen.
Wir erhalten analog durch
\begin{align*}
(A - \lambda_2 I) 
= 
\begin{pmatrix}
-3 & 3 \\
3 & -3
\end{pmatrix}
\leadsto
\begin{pmatrix}
-1 & 1 \\
0 & 0
\end{pmatrix}
\end{align*}
die Gleichung
\begin{align*}
-x_1 + x_2 = 0
\Leftrightarrow
x_1 = x_2.
\end{align*}
Damit können wir den zweiten Eigenvektor durch
\begin{align*}
\textbf{v}_2 = 
\begin{pmatrix}
1 \\
1
\end{pmatrix}
\end{align*}
angeben.\\
\\
\underline{3. Bestimme die korrekte Antwort}\\
Wir sehen wegen
\begin{align*}
\textbf{v}_1 + \textbf{v}_2
=
\begin{pmatrix}
0 \\
2
\end{pmatrix}
\end{align*}
Antwort (d) nicht erfüllt ist.
Des Weiteren ist wegen
\begin{align*}
\textbf{v}_1^\top \cdot \textbf{v}_2
=
( -1 , 1) \cdot 
\begin{pmatrix}
1\\
1
\end{pmatrix}
=
-1 + 1 = 0
\end{align*}
Antwort (c) korrekt.
Die Vektoren $\textbf{v}_1$ und $\textbf{v}_2$ sind also orthogonal zueinander.\\
Man könnte die Aufgabe auch schneller lösen: 
Die Matrix $A$ ist symmetrisch, d.h. es gilt
\begin{align*}
A = A^\top.
\end{align*}
Eigenvektoren zu verschiedenen Eigenwerten sind bei symmetrischen Matrizen immer orthogonal.\\
\\
Die Antwort (c) ist korrekt.

\newpage

\subsection*{\frage{6}{2}}
Die Folge $\lbrace y_k \rbrace_{k \in \mathbb{N}_0}$ definiert durch
\begin{align*}
y_k =  4 - \left( \frac{1}{a}\right)^k
\end{align*}
mit $a \neq 0$ löst das Anfangswertproblem
\begin{align*}
4 y_{k+1} -y_k = 12 \ \text{und} \ y_0 = 3
\end{align*}
für
\renewcommand{\labelenumi}{(\alph{enumi})}
\begin{enumerate}
\item 
$a= 1$.
\item
$a= 2$.
\item
$a= 4$.
\item
Keine der vorangehenden Antworten ist richtig.
\end{enumerate}
\ \\
\textbf{Lösung:}
\begin{mdframed}
\underline{\textbf{Vorgehensweise:}}
\renewcommand{\labelenumi}{\theenumi.}
\begin{enumerate}
\item Gebe die Normalform der Differenzengleichung an.
\item Finde die allgemeine Lösung der Differenzengleichung und löse die Aufgabe.
\end{enumerate}
\end{mdframed}

\underline{1. Gebe die Normalform der Differenzengleichung an}\\
In unserem Fall erhalten wir diese durch
\begin{align*}
4 y_{k+1} -y_k = 12 \ 
\Leftrightarrow \
4 y_{k+1} = y_k + 12
\ \Leftrightarrow \
y_{k+1} = \frac{1}{4} y_k + 3
\end{align*}
mit $A = \frac{1}{4} $ und $B = 3$.\\
\\
\underline{2. Finde die allgemeine Lösung der Differenzengleichung und löse die Aufgabe}\\
Die allgemeine Lösung einer Differenzengleichung erster Ordnung ist durch
\begin{align*}
y_k = A^k (y_0 - y^\star)  + y^\star
\end{align*}
mit
\begin{align*}
y^\star 
= \frac{B}{1 -A }, \quad A \neq 1 
\end{align*}
gegeben.
Es gilt:
\begin{align*}
y^\star 
= \frac{B}{1 -A }
= \frac{3}{1 - \frac{1}{4} }
= \frac{3}{\frac{3}{4}}
= \frac{3}{1} \cdot \frac{4}{3} = 4.
\end{align*}
Die allgemeine Lösung ist dann
\begin{align*}
y_k = A^k (y_0 - y^\star)  + y^\star
= \frac{1}{4^k} ( y_0 - 4 ) +4
=
\left( \frac{1}{4} \right)^k(y_0 -4) +4.
\end{align*}
Mit $y_0 = 3$ erhalten wir durch
\begin{align*}
y_k = \left( \frac{1}{4} \right)^k ( 3 - 4 ) +4
=  4 - \left( \frac{1}{4} \right)^k
\end{align*}
das passende $a = 4$.
Dies können wir durch Vergleichen mit der Aufgabenstellung erkennen.\\
\\
Damit ist Antwort (c) korrekt.

\newpage



\subsection*{\frage{7}{2}}
Die allgemeine Lösung der Differenzengleichung
\begin{align*}
5 y_{k+1} + 6 y_k = 1 , \quad k = 0,1,2,...
\end{align*}
ist
\renewcommand{\labelenumi}{(\alph{enumi})}
\begin{enumerate}
\item 
monoton und konvergent.
\item
monoton und divergent.
\item
oszillierend und konvergent.
\item
oszillierend und divergent.
\end{enumerate}
\ \\
\textbf{Lösung:}
\begin{mdframed}
\underline{\textbf{Vorgehensweise:}}
\renewcommand{\labelenumi}{\theenumi.}
\begin{enumerate}
\item Gebe die Normalform und allgemeine Lösung der Differenzengleichung an.
\item 
Bestimme die relevanten Eigenschaften.
\end{enumerate}
\end{mdframed}

\underline{1. Gebe die Normalform und allgemeine Lösung der Differenzengleichung an}\\
Wie in der vorigen Aufgabe erhalten wir durch
\begin{align*}
5 y_{k+1} + 6 y_k = 1
\ \Leftrightarrow \
5 y_{k+1} = - 6 y_k + 1
\ \Leftrightarrow \
y_{k+1} = - \frac{6}{5} y_k  + \frac{1}{5}
\end{align*}
die Normalform mit $A =  - \frac{6}{5} $ und $B = \frac{1}{5}$.
Mit 
\begin{align*}
y^\star = 
\frac{B}{1-A} 
= 
\frac{\frac{1}{5}}{1 - \left( -\frac{6}{5} \right)} 
=
\frac{\frac{1}{5}}{\frac{11}{5}}
= \frac{1}{5} \cdot \frac{5}{11} = \frac{1}{11}
\end{align*}
erhalten wir die allgemeine Lösung
\begin{align*}
y_k = A^k (y_0 - y^\star)  + y^\star
= \left( - \frac{6}{5} \right)^k \left ( y_0 - \frac{1}{11}\right) + \frac{1}{11}.
\end{align*}
\newpage
\underline{2. Bestimme die relevanten Eigenschaften}\\
Wir wissen, dass $A\neq 1$ ist.
Damit können die Fälle
\begin{align*}
A > 0 &\rightarrow \ \text{Lösung monoton}\\
A < 0 &\rightarrow \ \text{Lösung oszillierend}\\
|A| < 1  &\rightarrow \ \text{Lösung konvergent}\\
|A| > 1  &\rightarrow \ \text{Lösung divergent}
\end{align*}
eintreten.
Wegen $A = -\frac{6}{5} < 0$ wissen wir, dass die allgemeine Lösung oszillierend ist.
Zudem gilt
$
|A| = \frac{6}{5} > 1,
$
womit die allgemeine Lösung divergent ist.
Durch
\begin{align*}
\lim \limits_{k \to \infty} |A^k | 
=\lim \limits_{k \to \infty} \left| \left( - \frac{6}{5} \right)^k \right| = \infty
\end{align*}
können wir die Divergenz veranschaulichen und an
\begin{align*}
A^k = \left( - \frac{6}{5} \right)^k
= (-1)^k \cdot\left(  \frac{6}{5} \right)^k
\end{align*}
sehen wir, dass die Lösung oszilliert.
\\
\\
Somit ist Antwort (d) korrekt.
\newpage

\subsection*{\frage{8}{4}}
Die allgemeine Lösung der Differenzengleichung
\begin{align*}
m \ y_{k+1} + y_k  \ = \ m^2, \quad k=0,1,2,...,
\end{align*}
wobei $m \in \mathbb{R} \setminus \lbrace 0 \rbrace$, ist monoton und konvergent genau dann, wenn
\renewcommand{\labelenumi}{(\alph{enumi})}
\begin{enumerate}
\item 
$m \in [-1,0)$.
\item
$m \in (0,1]$.
\item
$m < -1$.
\item
$m> 1$.
\end{enumerate}
\ \\
\textbf{Lösung:}
\begin{mdframed}
\underline{\textbf{Vorgehensweise:}}
\renewcommand{\labelenumi}{\theenumi.}
\begin{enumerate}
\item Gebe die Normalform und allgemeine Lösung der Differenzengleichung an.
\item 
Argumentiere mithilfe den bekannten Eigenschaften.
\end{enumerate}
\end{mdframed}

\underline{1. Gebe die Normalform an }\\
Wie in den vorigen Aufgaben erhalten wir mit
\begin{align*}
m  y_{k+1} + y_k   =  m^2
 \ \Leftrightarrow \
m y_{k+1} = - y_k + m^2
\ \Leftrightarrow \
y_{k+1} = - \frac{1}{m} y_k + m 
\end{align*}
die Normalform mit $A = - \frac{1}{m}$ und $B = m$.
\\
\\
\underline{2. Argumentiere mithilfe den bekannten Eigenschaften}\\
Wir wissen, dass die allgemeine Lösung genau dann monoton und konvergent ist, wenn
\begin{align*}
A > 0  \ \text{und} \ |A| \leq 1
\end{align*}
erfüllt ist.
Wegen $m \in \mathbb{R} \setminus \lbrace 0 \rbrace$ ist $B=0$ unmöglich.
Daraus folgt
\begin{align*}
0 < - \frac{1}{m} \leq 1. 
\end{align*}
Wegen 
\begin{align*}
-\frac{1}{m} > 0 
\end{align*}
wissen wir direkt, dass $m < 0 $ ist.
Weiter gilt:
\begin{align*}
- \frac{1}{m} < 1 
\ \Leftrightarrow \
\frac{1}{m} >-1
 \ \Leftrightarrow \
m < -1
\end{align*}
\\

Somit ist Antwort (c) korrekt.
