\section*{Aufgabe 3 (32 Punkte)}
\vspace{0.4cm}
\subsection*{\frage{1}{4}}
Gegeben ist eine Funktion $ f $ mit 
\begin{align*}
f^\prime(x) = \frac{e^x - e^{-x}}{e^x + e^{-x}} \quad \textrm{und} \quad f(0 )= 0.
\end{align*}
Dann muss gelten:
\renewcommand{\labelenumi}{(\alph{enumi})}
\begin{enumerate}
	\item 
	$ f(x) = \ln(e^{2x} + 1) + 2 $.
	\item
	$ f(x) = \ln(e^{2x} + 1) + x  - \ln(2)$.
	\item
	$ f(x) = \ln(e^{x} + e^{-x})  - \ln(2)$.
	\item
	$ f(x) = \ln(e^{x} + e^{-x})  +2$.
\end{enumerate}
\ \\
\textbf{Lösung:}
\begin{mdframed}
\underline{\textbf{Vorgehensweise:}}
\renewcommand{\labelenumi}{\theenumi.}
\begin{enumerate}
\item Bestimme durch Ausschließen die korrekte Antwort.
\end{enumerate}
\end{mdframed}

\underline{1. Bestimme durch Ausschließen die korrekte Antwort}\\
Zunächst bestimmen wir die Ableitung von $ \ln(e^{2x} + 1) $.
Hierfür gilt mit der Kettenregel:
\begin{align*}
\left(\ln(e^{2x} + 1)\right)^\prime
=
\frac{1}{e^{2x} + 1} 2e^{2x}
=\frac{2e^{2x}}{e^{2x} + 1}.
\end{align*}
Damit können wir direkt die Antworten (a) und (b) ausschließen.
Nun betrachten wir die Ableitung von $ \ln(e^x + e^{-x}) $.
Der Term $ - \ln(2) $ fällt bei Antwort (c) und der Term $ +2 $ bei Antwort (d) durch das Differenzieren weg.
Für diese gilt
\begin{align*}
\left(\ln(e^x + e^{-x}) \right)^\prime
=
\frac{1}{e^x + e^{-x}}(e^x - e^{-x})
=
\frac{e^x - e^{-x}}{e^x + e^{-x}}.
\end{align*}
Wegen $ \ln(e^0 + e^{0}) = \ln(1 + 1) = \ln(2) $ ist die Bedingung $ f(0 ) = 0 $ nur für die Antwort (c) erfüllt.\\
\\
Damit ist Antwort (c) korrekt.\\
\\
%\textit{Bemerkung:}\\
%Die korrekte Antwort lässt sich auch durch Einsetzen der $ 0 $ und Differenzieren aller Möglichkeiten finden.


 
\newpage

\subsection*{\frage{2}{5}}
Die stetige Zufallsvariable $ X $ hat die Dichtefunktion
\begin{align*}
f(x) =
\begin{cases}
ax + bx^2 \quad &\textrm{falls } 0 \leq x \leq 1\\
\quad  \  0 \ \ &\textrm{sonst}
\end{cases}
\end{align*}
und den Erwartungswert $ \mathbb{E}[X] = \frac{17}{24} $.\\
\\
Dann gilt:
\renewcommand{\labelenumi}{(\alph{enumi})}
\begin{enumerate}
	\item 
	$a= 6,b=-6$.
	\item
	$a= 1,b=\frac{3}{2}$.
	\item
	$a= 6,b=-5$.
	\item
	$a= \frac{3}{2},b=-\frac{1}{2}$.	
\end{enumerate}
\ \\
\textbf{Lösung:}
\begin{mdframed}
\underline{\textbf{Vorgehensweise:}}
\renewcommand{\labelenumi}{\theenumi.}
\begin{enumerate}
\item Nutze die Bedingung für eine Dichtefunktion.
\item Bestimme den Erwartungswert von $ X $.
\item Ermittle die korrekte Antwort.
\end{enumerate}
\end{mdframed}

\underline{1. Nutze die Bedingung für eine Dichtefunktion}\\
Die Funktion $ f $ ist eine Dichtefunktion, falls
\begin{align*}
\int \limits_{- \infty}^\infty f(x) \ dx
= 1
\end{align*}
erfüllt ist. Da das $ f $ in der Aufgabenstellung eine Dichtefunktion ist, heißt das, dass
\begin{align*}
\int \limits_{- \infty}^\infty f(x) \ dx
=
\int \limits_{0}^1 ax + bx^2 \ dx
=
 \frac{a}{2} x^2 + \frac{b}{3} x^3 \bigg|_0^1
=
\frac{a}{2}1^2 + \frac{b}{3} 1^3 - \left( \frac{a}{2}0^2 + \frac{b}{3} 0^3\right)
= \frac{a}{2} + \frac{b}{3} = 1
\end{align*}
erfüllt sein muss. Damit ergibt mit
\begin{align*}
\frac{a}{2} + \frac{b}{3} = 1
\ \Leftrightarrow \
3 a + 2 b = 6
\end{align*}
eine bruchfreie Bedingung.\\
\\
\newpage
\underline{2. Bestimme den Erwartungswert von $ X $}\\
Da die Zufallsvariable die Dichtefunktion $ f $ besitzt, gilt für den Erwartungswert:
\begin{align*}
\mathbb{E}[X]
&=
\int \limits_{- \infty}^\infty x f(x) \ dx
=
\int \limits_{0}^1 x(ax + bx^2) \ dx
=
\int \limits_{0}^1 ax^2 + bx^3 \ dx
=
\frac{a}{3} x^3 + \frac{b}{4} x^3 \bigg|_0^1 \\
&=
\frac{a}{3} 1^3 + \frac{b}{4} 1^3 - 
\left( \frac{a}{3} 0^3 + \frac{b}{4} 0^3\right)
=
\frac{a}{3}  + \frac{b}{4}
= \frac{17}{24}.
\end{align*}
Heraus erhalten wir:
\begin{align*}
\frac{a}{3}  + \frac{b}{4}
= \frac{17}{24}
\ \Leftrightarrow \
8 a + 6 b = 17.
\end{align*}
\ \\
\underline{3. Ermittle die korrekte Antwort}\\
Insgesamt erhalten wir die Bedingungen:
\begin{align*}
3 a + 2 b &= 6\\
8 a + 6 b &= 17.
\end{align*}
Dies ist ein lineares Gleichungssystem mit den Variablen $ a $ und $ b $.
Wenn wir dieses System in eine erweiterte Koeffizientenmatrix übertragen und das  Gauß-Verfahren anwenden, erhalten wir:
\begin{align*}
\begin{gmatrix}[p]
3 & 2 & \BAR & 6\\
8 & 6 &  \BAR & 17
\rowops
\add[-3]{0}{1}
\end{gmatrix}
\leadsto
\begin{gmatrix}[p]
3 & 2 & \BAR & 6\\
-1 & 0 &  \BAR & -1
\rowops
\add[3]{1}{0}
\end{gmatrix}
\leadsto
\begin{gmatrix}[p]
0 & 2 & \BAR & 3\\
-1 & 0 &  \BAR & -1
\rowops
\mult{0}{\cdot \frac{1}{2}}
\mult{1}{\cdot (-1)}
\end{gmatrix}
\leadsto
\begin{gmatrix}[p]
0 & 1 & \BAR & \frac{3}{2}\\
1 & 0 &  \BAR & 1
\end{gmatrix}.
\end{align*}
Damit ist die Lösung des linearen Gleichungssystems $ a = 1 $ und $ b = \frac{3}{2} $.\\
\\
Also ist Antwort (b) korrekt\\
\\
\textit{Bemerkung:}\\
Das Finden der Antwort lässt sich durch Ausschließen beschleunigen.
Damit ist gemeint, dass das lineare Gleichungssystem nicht gelöst werden muss.
Die Bedingung für die Dichtefunktion schließt die Antworten (c) und (d) direkt aus.
Die korrekte Antwort finden wir nun, indem wir die restlichen Kombinationen in die Gleichung für den Erwartungswert von $ X $ einsetzen.



\newpage
\subsection*{\frage{3}{4}}
Gegeben ist die Funktion
\begin{align*}
f(x,y) = 3 \ln(x^2 + y^2) - (3-a) \ln(y).
\end{align*}
Für welche $ a \in \mathbb{R} $ ist die Länge des Gradienten an der Stelle  $ (x_0,y_0) = (1,1) $ gleich $ 5 $?
\renewcommand{\labelenumi}{(\alph{enumi})}
\begin{enumerate}
	\item 
	$ a = \pm 4 $.
	\item
	$ a = \pm 2 $.
	\item
	$ a = \pm \sqrt{2} $.
	\item
	Es gibt kein solches $ a \in \mathbb{R} $.
\end{enumerate}
\ \\
\textbf{Lösung:}
\begin{mdframed}
\underline{\textbf{Vorgehensweise:}}
\renewcommand{\labelenumi}{\theenumi.}
\begin{enumerate}
\item Bestimme den Gradienten.
\item Bestimme die Länge des Gradienten und die passenden $ a $.
\end{enumerate}
\end{mdframed}
%\allowdisplaybreaks
\underline{1. Bestimme den Gradienten}\\
Zuerst bestimmen wir die partiellen Ableitungen von $ f $:
\begin{align*}
f_x(x,y) &= 3 \cdot \frac{1}{x^2 + y^2} \cdot  2 x = \frac{6x}{x^2 + y^2} 
\ \Rightarrow \
f_x(1,1) = \frac{6 \cdot 1}{1+1} = \frac{6}{2} = 3
\\
f_y(x,y) &= \frac{6y}{x^2 + y^2} - (3-a) \frac{1}{y}
\ \Rightarrow \
f_y(1,1)
= \frac{6 \cdot 1}{1 + 1}- (3-a) \frac{1}{1}
=
3-3 + a = a
.
\end{align*}
Der Gradient ist durch
\begin{align*}
\textbf{grad} f(x,y) = 
\begin{pmatrix}
f_x(x,y)\\
f_y(x,y)
\end{pmatrix}
\ \Rightarrow \
\textbf{grad} f(1,1) = 
\begin{pmatrix}
f_x(1,1)\\
f_y(1,1)
\end{pmatrix}
= 
\begin{pmatrix}
3 \\ a
\end{pmatrix}
\end{align*}
gegeben.\\
\\
\underline{2. Bestimme die Länge des Gradienten und die passenden $ a $}\\
 Für die Länge muss 
\begin{align*}
\left\| \begin{pmatrix}
3 \\ a
\end{pmatrix} 
\right\| = \sqrt{3^2 + a^2} = 5
\ \Leftrightarrow \
9 + a^2 = 25 
\ \Leftrightarrow \
a^2 = 16 
\ \Leftrightarrow \
a = \pm 4
\end{align*}
erfüllt sein.\\
\\
Damit ist die Antwort (a) korrekt.
\newpage
\subsection*{\frage{4}{4}}
Gegeben ist die Matrix
\begin{align*}
A =
\begin{pmatrix}
2 & 4 & 6 & 8 \\
-1 & 2 & 1 & 4 \\
3 & 10 & 13 &  20 \\
4 & 0 & 4 & 0
\end{pmatrix}.
\end{align*}
$ A $ hat
\renewcommand{\labelenumi}{(\alph{enumi})}
\begin{enumerate}
	\item 
	den Rang $ 1 $.
	\item 
	den Rang $ 2 $.
	\item
	den Rang $ 3 $.
	\item
	den Rang $ 4 $.
\end{enumerate}
\ \\
\textbf{Lösung:}
\begin{mdframed}
\underline{\textbf{Vorgehensweise:}}
\renewcommand{\labelenumi}{\theenumi.}
\begin{enumerate}
\item 
Wende das Gauß-Verfahren an und bestimme den Rang.
\end{enumerate}
\end{mdframed}

\underline{1. Wende das Gauß-Verfahren an und bestimme den Rang}\\
Durch Anwendung des Gauß-Verfahrens erhalten wir:
\begin{align*}
\begin{gmatrix}[p]
2 & 4 & 6 & 8 \\
-1 & 2 & 1 & 4 \\
3 & 10 & 13 &  20 \\
4 & 0 & 4 & 0
\rowops
\mult{0}{\cdot \frac{1}{2}}
\end{gmatrix}
&\leadsto
\begin{gmatrix}[p]
1 & 2 & 3 & 4 \\
-1 & 2 & 1 & 4 \\
3 & 10 & 13 &  20 \\
4 & 0 & 4 & 0
\rowops
\add[\cdot 1]{0}{1}
\add[\cdot (-3)]{0}{2}
\add[\cdot (-4)]{0}{3}
\end{gmatrix}
\leadsto
\begin{gmatrix}[p]
1 & 2 & 3 & 4 \\
0 & 4 & 4 & 8 \\
0 & 4 & 4 &  8 \\
0 & -8 & -8 & -16
\rowops
\mult{3}{\cdot \left( - \frac{1}{2}\right)}
\end{gmatrix}\\
&\leadsto
\begin{gmatrix}[p]
1 & 2 & 3 & 4 \\
0 & 4 & 4 & 8 \\
0 & 4 & 4 &  8 \\
0 & 4 & 4 & 8
\rowops
\add[\cdot (-1)]{1}{2}
\add[\cdot (-1)]{1}{3}
\end{gmatrix}
\leadsto
\begin{gmatrix}[p]
1 & 2 & 3 & 4 \\
0 & 4 & 4 & 8 \\
0 & 0 & 0 &  0 \\
0 & 0 & 0 & 0
\end{gmatrix}.
\end{align*}
Damit liefert das Gauß-Verfahren zwei Nullzeilen.
Folglich hat die Matrix den Rang $ 2 $.\\
\\
Also ist Antwort (b) korrekt.

\newpage

\subsection*{\frage{5}{4}}
Gegeben ist die Matrix
\begin{align*}
M
=
\begin{pmatrix}
0 & 0 & -2\\
1 & 2 & 1\\
1 & 0 & 3
\end{pmatrix}.
\end{align*}
Welche der folgenden Aussagen ist richtig?
\renewcommand{\labelenumi}{(\alph{enumi})}
\begin{enumerate}
	\item 
	$ \textbf{x} = \begin{pmatrix}
	1 \\ 0 \\ -1
	\end{pmatrix} $ ist ein Eigenvektor von $ M $ zum Eigenwert $ \lambda = -2 $.
	\item
	$ \textbf{x} = \begin{pmatrix}
	1 \\ 0 \\ 0
	\end{pmatrix} $ ist ein Eigenvektor von $ M $ zum Eigenwert $ \lambda = 2 $.
	\item
	$ \textbf{x} = \begin{pmatrix}
	-2 \\ 1 \\ 1
	\end{pmatrix} $ ist ein Eigenvektor von $ M $ zum Eigenwert $ \lambda = 1 $.
	\item
	$ \textbf{x} = \begin{pmatrix}
	1 \\ 1 \\ 1
	\end{pmatrix} $ ist ein Eigenvektor von $ M $ zum Eigenwert $ \lambda = -1 $.
\end{enumerate}
\ \\
\textbf{Lösung:}
\begin{mdframed}
\underline{\textbf{Vorgehensweise:}}
\renewcommand{\labelenumi}{\theenumi.}
\begin{enumerate}
\item Überprüfe die Antwortmöglichkeiten.
\item Überlege dir, welche Möglichkeiten keine Eigenvektoren sein können.
\end{enumerate}
\end{mdframed}

\underline{1. Überprüfe die Antwortmöglichkeiten}\\
Ein Vektor $ \textbf{x} \neq 0 $ heißt Eigenvektor zum Eigenwert $ \lambda $, falls
\begin{align*}
M \textbf{x} = \lambda \textbf{x}
\end{align*}
gilt. Das heißt $ M \textbf{x} $ muss ein Vielfaches von $ \textbf{x} $ sein.
Ansonsten ist $ \textbf{x} $ kein Eigenvektor.\\
\\
\underline{2. Überlege dir, welche Möglichkeiten keine Eigenvektoren sein können}\\
Wir beginnen mit der Aussage (a). Es gilt:
\begin{align*}
M \cdot \textbf{x} = 
\begin{pmatrix}
0 & 0 & -2\\
1 & 2 & 1\\
1 & 0 & 3
\end{pmatrix}
\cdot 
\begin{pmatrix}
1 \\ 0 \\ -1
\end{pmatrix}
=
\begin{pmatrix}
2 \\ 0  \\-2
\end{pmatrix}
= 
2 \cdot 
\begin{pmatrix}
1 \\ 0 \\ -1
\end{pmatrix}
\neq
(-2) \cdot \begin{pmatrix}
1 \\ 0 \\ -1
\end{pmatrix}.
\end{align*}
Damit ist $ \textbf{x}  $ ein Eigenvektor zu dem Eigenwert $ 2 $. 
Aber nicht zu $ \lambda = -2 $.
Also ist die Antwort (a) falsch.\\
\\
Für die Aussage (b) betrachten wir:
\begin{align*}
M \cdot \textbf{x} = 
\begin{pmatrix}
0 & 0 & -2\\
1 & 2 & 1\\
1 & 0 & 3
\end{pmatrix}
\cdot 
\begin{pmatrix}
1 \\ 0 \\ 0
\end{pmatrix}
= 
\begin{pmatrix}
0 \\ 1 \\ 1
\end{pmatrix} \neq 2 \cdot \begin{pmatrix}
1 \\ 0 \\ 0
\end{pmatrix}.
\end{align*}
Damit ist $ \textbf{x} $ kein Eigenvektor und die Antwort (b) falsch.\\
\\
Wegen 
\begin{align*}
M \cdot \textbf{x} = 
\begin{pmatrix}
0 & 0 & -2\\
1 & 2 & 1\\
1 & 0 & 3
\end{pmatrix}
\cdot 
\begin{pmatrix}
-2 \\ 1 \\ 1
\end{pmatrix}
=
\begin{pmatrix}
-2 \\ 1 \\ 1
\end{pmatrix}
 = 
 1 \cdot \begin{pmatrix}
 -2 \\ 1 \\ 1
 \end{pmatrix}
\end{align*}
ist die Antwort (c) korrekt. Denn $ \textbf{x} $  ist ein Eigenvektor von $ M $ zum Eigenwert $ \lambda = 1 $.\\
\\
Zur Vollständigkeit noch die Aussage (d):
 \begin{align*}
 \begin{pmatrix}
 0 & 0 & -2\\
 1 & 2 & 1\\
 1 & 0 & 3
 \end{pmatrix}
 \cdot 
 \begin{pmatrix}
 1 \\ 1 \\ 1
 \end{pmatrix}
 =
  \begin{pmatrix}
-2 \\ 4 \\ 4
 \end{pmatrix}
 \neq
 (-1) \cdot  \begin{pmatrix}
 1 \\ 1 \\ 1
 \end{pmatrix}.
 \end{align*}
\ \\
Also ist die Antwort (c) korrekt.
%\underline{1. Überlege dir, welche Möglichkeiten keine Eigenvektoren sein können}\\
%Wir kürzen mit $ \textbf{x}_a $, $ \textbf{x}_b $, $ \textbf{x}_c $ und $ \textbf{x}_d $ die Vektoren von (a) bis (d) ab.\\
%\\
%Ein Vektor $ \textbf{x} \neq 0 $ heißt Eigenvektor zum Eigenwert $ \lambda $, falls
%\begin{align*}
%M \textbf{x} = \lambda \textbf{x}
%\end{align*}
%gilt. Das heißt $ M \textbf{x} $ muss ein Vielfaches von $ \textbf{x} $ sein.
%Ansonsten ist $ \textbf{x} $ kein Eigenvektor.\\
%\\
%Wegen 
%\begin{align*}
%M \textbf{x}_b = \begin{pmatrix}
%0 \\ 1 \\ 1
%\end{pmatrix}
%\end{align*}
%kann $ \textbf{x}_b $ kein Eigenvektor sein. 
%\newpage
%Ebenso gilt:
%\begin{align*}
%M \textbf{x}_d = \begin{pmatrix}
%-2 \\ 4 \\ 4
%\end{pmatrix}.
%\end{align*}
%Also kann $ \textbf{x}_d $ auch kein Eigenvektor sein.\\
%\\
%\underline{2. Bestimme die richtige Aussage}\\
%Es bleiben die Möglichkeiten (a) und (c).
%Wir erhalten:
%\begin{align*}
%M \textbf{x}_a = \begin{pmatrix}
%2 \\ 0 \\-2
%\end{pmatrix}
%= 2 \cdot \textbf{x}_a\\
%M \textbf{x}_c = 
%\begin{pmatrix}
%-2 \\ 1 \\ 1
%\end{pmatrix}
%= 
%1 \cdot \textbf{x}_c.
%\end{align*}
%Also ist $ \textbf{x}_a $ ein Eigenvektor zum Eigenwert $ 2 $ und $ \textbf{x}_c $ ein Eigenvektor zum Eigenwert $ 1 $.\\
%\\
%Damit ist die Antwort (c) korrekt.
\newpage

\subsection*{\frage{6}{3}}
Die Folge $ \{y_k\}_{k \in \mathbb{N}_0} $ mit $ y_k = 3 \cdot 2^k -1 $ löst die Differenzengleichung
\renewcommand{\labelenumi}{(\alph{enumi})}
\begin{enumerate}
	\item 
	$ 2 y_{k+1} - y_k = -1 $.
	\item
	$ y_{k+1} + 2 y_k  -1 = 0$.
	\item
	$ y_{k+1} + 2 y_k - 2 = 0 $.
	\item
	$ y_{k+1} - 2 y_k = 1$.
\end{enumerate}
\ \\
\textbf{Lösung:}
\begin{mdframed}
\underline{\textbf{Vorgehensweise:}}
\renewcommand{\labelenumi}{\theenumi.}
\begin{enumerate}
\item Untersuche die in den Antworten vorkommenden Folgenglieder.
\end{enumerate}
\end{mdframed}

\underline{1. Untersuche die in den Antworten vorkommenden Folgenglieder}\\
Die Antworten enthalten die Glieder $ y_k  $ und $ y_{k+1} $. Diese sind durch
\begin{align*}
y_k &= 3 \cdot 2^k - 1\\
y_{k+1} &= 3 \cdot 2^{k+1} -1
\end{align*}
gegeben. Wegen 
\begin{align*}
2 y_{k+1} - y_k = 
3 \cdot 2^{k+2} -2 - 3 \cdot 2^k +1
=
2^k (3\cdot 2^2 - 3) -1
=
2^k \cdot 9 - 1 \neq -1
\end{align*}
ist Antwort (a) falsch. In (b) - (d) ist der Term $ 2 y_k $ enthalten. Für diesen gilt
\begin{align*}
2 y_k =  2 \cdot 3 \cdot 2^k - 2 = 3 \cdot 2^{k+1} -2.
\end{align*}
Wegen
\begin{align*}
y_{k+1} + 2 y_k = 3 \cdot 2^{k+1} -1 + 3 \cdot 2^{k+1} -2
=
3 \cdot 2^{k+2 } -3
\end{align*}
können (b) und (c) nicht erfüllt sein. Übrig bleibt noch die korrekte Antwort (d). Dies wollen wir überprüfen:
\begin{align*}
y_{k+1} - 2 y_k =
3 \cdot 2^{k+1} -1 - (3 \cdot 2^{k+1} -2)
= 
1.
\end{align*}
\ \\
Also ist Antwort (d) korrekt.





\newpage



\subsection*{\frage{7}{3}}
Die allgemeine Lösung der Differenzengleichung
\begin{align*}
\frac{3}{4} y_k - \pi  e y_{k+1} + \frac{1}{8} y_k -3.2 = -2 y_k \quad (k=0,1,2,...)
\end{align*}
ist
\renewcommand{\labelenumi}{(\alph{enumi})}
\begin{enumerate}
	\item
	monoton und konvergent.
	\item
	monoton und divergent.	
	\item 
	oszillierend und konvergent.
	\item
	oszillierend und divergent.
\end{enumerate}
\ \\
\textbf{Lösung:}
\begin{mdframed}
\underline{\textbf{Vorgehensweise:}}
\renewcommand{\labelenumi}{\theenumi.}
\begin{enumerate}
\item Gebe die Normalform an.
\item Bestimme anhand der Eigenschaften von $ A $ die richtige Lösung.
\end{enumerate}
\end{mdframed}

\underline{1. Gebe die Normalform an}\\
Die Normalform einer Differenzengleichung ist durch
\begin{align*}
y_{k+1} = A y_{k} + B
\end{align*}
mit $ A, B \in \mathbb{R} $ gegeben.
Durch
\begin{align*}
\frac{3}{4} y_k - \pi  e y_{k+1} + \frac{1}{8} y_k -3.2 = -2 y_k 
&\ \Leftrightarrow \
\frac{6}{8} y_k + \frac{1}{8} y_k - \pi  e y_{k+1} -3.2 = -\frac{16}{8} y_k \\
&\ \Leftrightarrow  \
\frac{7}{8} y_k  - \pi  e y_{k+1} -3.2 = -\frac{16}{8} y_k \\
&\ \Leftrightarrow  \
- \pi  e y_{k+1} -3.2 =  -\frac{23}{8} y_k\\
&\ \Leftrightarrow  \
- \pi  e y_{k+1}  =  -\frac{23}{8} y_k + 3.2\\
&\ \Leftrightarrow  \
\pi  e y_{k+1}  =  \frac{23}{8} y_k - 3.2\\
& \ \Leftrightarrow \
 y_{k+1}  =  \underbrace{\frac{23}{8 \pi  e}}_{A :=} y_k - \underbrace{\frac{3.2}{\pi  e}}_{B:=}
\end{align*} 
erhalten wir die Normalform der Differenzengleichung.\\
\\
\underline{2. Bestimme anhand der Eigenschaften von $ A $ die richtige Lösung}\\
Eine Differenzengleichung in Normalenform
\begin{align*}
y_{k+1} = A y_{k} + B
\end{align*}
konvergiert für $ |A| < 1 $ und divergiert für $ |A| > 1 $.
Für $ A> 0 $ ist das Verhalten monoton und für $  A < 0 $ oszillierend.
Wegen $ 8 \pi e > 23 $ gilt $ 0 < A <1  $, also $ A > 0  $ und $ | A |< 1 $.
Damit ist die Lösung der Gleichung monoton und konvergent.\\
\\
Also ist die Antwort (a) korrekt.
\newpage

\subsection*{\frage{8}{5}}
Die allgemeine Lösung der linearen Differenzengleichung
\begin{align*}
(a+1) y_{k+1} - (3-a) y_k + 8 = 0, \quad k = 0,1,2,...,
\end{align*}
mit $ a \in \mathbb{R} \setminus \{-1,3  \} $ ist genau dann oszillierend und konvergent, wenn
\renewcommand{\labelenumi}{(\alph{enumi})}
\begin{enumerate}
	\item 
	$ -1 < a < 1 $.
	\item
	$ 1 < a < 3 $.
	\item
	$ a = -1 $ oder $ a > 3 $.
	\item
	Die allgemeine Lösung ist für kein $ a \in \mathbb{R} $ oszillierend und konvergent.
\end{enumerate}
\ \\
\textbf{Lösung:}
\begin{mdframed}
\underline{\textbf{Vorgehensweise:}}
\renewcommand{\labelenumi}{\theenumi.}
\begin{enumerate}
\item Gebe die Normalform an.
\item Suche die $ a $ so, dass die Lösung oszillierend und konvergent ist.
\end{enumerate}
\end{mdframed}

\underline{1. Gebe die Normalform an}\\
Die Normalform ist durch
\begin{align*}
(a+1) y_{k+1} - (3-a) y_k + 8 = 0
\ \Leftrightarrow \
(a+1) y_{k+1} = (3-a) y_k - 8
\ \Leftrightarrow \
y_{k+1} = \underbrace{\frac{3-a }{a+ 1}}_{A:=} y_k - \underbrace{\frac{8}{a+1}}_{B:=}
\end{align*}
gegeben.\\
\\
\underline{2. Suche die $ a $ so, dass die Lösung oszillierend und konvergent ist}\\
Die Lösung der Differenzengleichung ist oszillieren und konvergent, falls $ -1 < A < 0 $ gilt.
Wir müssen alle $ a \in \mathbb{R} \setminus \{-1,3\} $ finden, sodass
\begin{align*}
-1 < \frac{a-3}{a+1} < 0
\end{align*}
gilt. 
Hierfür müssen wir die Fälle $ a+1 > 0  \Leftrightarrow a > -1 $ und $ a+1 < 0 \Leftrightarrow a < -1 $ unterscheiden.
Dies ist wichtig, da die Multiplikation mit einer negativen Zahl die Richtung der Ungleichung verändert.
Zur Illustration die Multiplikation mit $ (-1) $:
\begin{align*}
1 < 2 \ \Leftrightarrow \ -1 > -2.
\end{align*}
Für den Fall $ a > -1 $ gilt:
\begin{align*}
-1 < \frac{a-3}{a+1} < 0 
&\ \Leftrightarrow \
-a - 1 < a - 3  < 0\\
&\ \Leftrightarrow \
-a + 2 < a < 3 \\
&\ \Leftrightarrow \
-a + 2 < a \ \wedge \ a < 3\\
&\ \Leftrightarrow \
 2 <  2 a \ \wedge \ a < 3\\
&\ \Leftrightarrow \
1 <   a \ \wedge \ a < 3 \\
&\ \Leftrightarrow \
1 <   a < 3 
\end{align*}
Es bleibt der Fall $ a <  -1  $ übrig:
\begin{align*}
-1 < \frac{a-3}{a+1} < 0 
&\ \Leftrightarrow \
-a - 1 > a - 3  > 0\\
&\ \Leftrightarrow \
-a + 2 > a   > 3\\
&\ \Leftrightarrow \\\
2 > 2 a  \ \wedge \ a   > 3
&\ \Leftrightarrow \
1  > a  \ \wedge \ a   > 3.
\end{align*}
Dies ist ein Widerspruch.\\
\\
Insgesamt ist die Lösung der Differenzengleichung für $ 1 < a < 3 $ oszillierend und konvergent.\\
\\
Damit ist die Antwort (b) korrekt.
